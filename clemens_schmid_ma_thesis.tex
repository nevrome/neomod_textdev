\documentclass[openany,twoside,twocolumn]{book}
\usepackage{lmodern}
\usepackage{amssymb,amsmath}
\usepackage{ifxetex,ifluatex}
\usepackage{fixltx2e} % provides \textsubscript
\ifnum 0\ifxetex 1\fi\ifluatex 1\fi=0 % if pdftex
  \usepackage[T1]{fontenc}
  \usepackage[utf8]{inputenc}
\else % if luatex or xelatex
  \ifxetex
    \usepackage{mathspec}
  \else
    \usepackage{fontspec}
  \fi
  \defaultfontfeatures{Ligatures=TeX,Scale=MatchLowercase}
\fi
% use upquote if available, for straight quotes in verbatim environments
\IfFileExists{upquote.sty}{\usepackage{upquote}}{}
% use microtype if available
\IfFileExists{microtype.sty}{%
\usepackage{microtype}
\UseMicrotypeSet[protrusion]{basicmath} % disable protrusion for tt fonts
}{}
\usepackage[left=2.5cm, right=2cm, top=2cm, bottom=2cm]{geometry}
\usepackage{hyperref}
\hypersetup{unicode=true,
            pdfborder={0 0 0},
            breaklinks=true}
\urlstyle{same}  % don't use monospace font for urls
\usepackage[style=authoryear]{biblatex}
\ExecuteBibliographyOptions{refsegment=chapter}
\addbibresource{references-cultural-evolution.bib}
\addbibresource{references-bronze-age-burials.bib}
\addbibresource{references-software-and-data-analysis.bib}
\usepackage{longtable,booktabs}
\usepackage{graphicx,grffile}
\makeatletter
\def\maxwidth{\ifdim\Gin@nat@width>\linewidth\linewidth\else\Gin@nat@width\fi}
\def\maxheight{\ifdim\Gin@nat@height>\textheight\textheight\else\Gin@nat@height\fi}
\makeatother
% Scale images if necessary, so that they will not overflow the page
% margins by default, and it is still possible to overwrite the defaults
% using explicit options in \includegraphics[width, height, ...]{}
\setkeys{Gin}{width=\maxwidth,height=\maxheight,keepaspectratio}
\IfFileExists{parskip.sty}{%
\usepackage{parskip}
}{% else
\setlength{\parindent}{0pt}
\setlength{\parskip}{6pt plus 2pt minus 1pt}
}
\setlength{\emergencystretch}{3em}  % prevent overfull lines
\providecommand{\tightlist}{%
  \setlength{\itemsep}{0pt}\setlength{\parskip}{0pt}}
\setcounter{secnumdepth}{5}

%%% Use protect on footnotes to avoid problems with footnotes in titles
\let\rmarkdownfootnote\footnote%
\def\footnote{\protect\rmarkdownfootnote}

%%% Change title format to be more compact
\usepackage{titling}

% Create subtitle command for use in maketitle
\newcommand{\subtitle}[1]{
  \posttitle{
    \begin{center}\large#1\end{center}
    }
}

\setlength{\droptitle}{-2em}

  \title{}
    \pretitle{\vspace{\droptitle}}
  \posttitle{}
    \author{}
    \preauthor{}\postauthor{}
    \date{}
    \predate{}\postdate{}
  
% turn off page header titles
\pagestyle{plain}
% change language to german
\usepackage[ngerman]{babel}
% turn off bibliography at the end of document
\let\pby\printbibliography
\renewcommand{\printbibliography}{}
% footnotes in tables
\usepackage{tablefootnote}
% promille symbol
\usepackage{textcomp}
% move all figures and tables to the end of the document
\usepackage[nomarkers,nolists,notables]{endfloat}
\AtBeginFigures{\setcounter{chapter}{0}}
\renewcommand{\figuresection}{Abbildungen}
% fancy headers
\usepackage{fancyhdr}
\pagestyle{fancy}
% section header formatting
\usepackage{titlesec}
\titleformat{\section}[display]{\normalfont\Large\bfseries}{\thesection}{0pt}{}
\titleformat{\subsection}[display]{\normalfont\large\bfseries}{\thesubsection}{0pt}{}
\titleformat{\subsubsection}[display]{\normalfont\normalsize\bfseries}{\thesubsubsection}{0pt}{}
\titleformat{\paragraph}[display]{\normalfont\normalsize\bfseries}{\theparagraph}{0pt}{}
\titleformat{\subparagraph}[display]{\normalfont\normalsize\bfseries}{\thesubparagraph}{0pt}{}
% figure rotation
\usepackage{pdflscape}
\DeclareDelayedFloatFlavor{landscape}{figure}
\usepackage{booktabs}
\usepackage{longtable}
\usepackage{array}
\usepackage{multirow}
\usepackage[table]{xcolor}
\usepackage{wrapfig}
\usepackage{float}
\usepackage{colortbl}
\usepackage{pdflscape}
\usepackage{tabu}
\usepackage{threeparttable}
\usepackage{threeparttablex}
\usepackage[normalem]{ulem}
\usepackage{makecell}

\begin{document}

\begin{titlepage}

  \vspace*{\fill}

    \begin{center}

        \Huge Ein computerbasiertes Cultural Evolution Modell zur Ausbreitungsdynamik europäisch-bronzezeitlicher Bestattungssitten

        \vspace{2cm}

        \huge Masterarbeit

        \large im Fach prähistorische und Historische Archäologie der Philosophischen Fakultät der Christian-Albrechts-Universität zu Kiel

        \vspace{2cm}

        \large vorlegt von \\
        \huge Clemens Schmid

    \end{center}

\vspace{4cm}

\large Erstgutachter: PD Dr. Oliver Nakoinz \\
\large Zweitgutachter: Dr. Martin Hinz

\vspace{1cm}

\large Kiel im September 2018

  \vspace*{\fill}

\end{titlepage}

\renewcommand{\chaptermark}[1]{\markboth{#1}{}}
\renewcommand{\sectionmark}[1]{\markright{\thesection\ #1}}
\fancyhf{}
\fancyhead[LE,RO]{\textbf{\thepage}}
\fancyhead[LO]{\textbf{\nouppercase{\rightmark}}}
\fancyhead[RE]{\textbf{\nouppercase{\leftmark}}}
\fancypagestyle{plain}{%
  \fancyhead{} % get rid of headers
  \renewcommand{\headrulewidth}{0pt} % and the line
}

\setcounter{tocdepth}{3}
\tableofcontents
\listoftables
\listoffigures

\parskip 4pt \setlength{\textfloatsep}{10pt plus 1.0pt minus 2.0pt}

\AtEndDocument{
  \begin{titlepage}

  \huge \textbf{Erklärung}

  \vspace{1.5cm}

  \large Hiermit erkläre ich, dass ich die vorliegende Arbeit selbstständig und ohne fremde Hilfe angefertigt und keine anderen als die angegebenen Quellen und Hilfsmittel verwendet habe.

  \vspace{1cm}

  \large Die eingereichte schriftliche Fassung der Arbeit entspricht der auf dem elektronischen Speichermedium.

  \vspace{1cm}

  \large Weiterhin versichere ich, dass diese Arbeit noch nicht als Abschlussarbeit an anderer Stelle vorgelegen hat.

  \vspace{2cm}

  \noindent\rule{8cm}{0.4pt}

\end{titlepage}

}

\hypertarget{intro}{%
\chapter{Einführung}\label{intro}}

Die vorliegende Master-Arbeit entstand 2017-2018 am Institut für Ur- und
Frühgeschichte der Christian-Albrechts-Universität zu Kiel unter
Betreuung von Priv.-Doz. Dr.~Oliver Nakoinz und Dr.~Martin Hinz. Sie ist
open source und voll reproduzierbar.

Ausgangspunkt der Überlegungen für diese Arbeit war die Frage, ob
kulturhistorische Transformations- und Ausbreitungsprozesse in einer
Modellimplementierung abgebildet werden können, die Ideen oder
Innovationen als handlungsfähige Agenten begreift. Grundlage dafür ist
Richard Dawkings Meme-Konzept, das unter dem Schlagwort \emph{Cultural
Evolution} Eingang in die archäologische Fachdiskussion gefunden hat.

Das Fallbeispiel, an dem dieser Modellansatz erprobt werden soll,
beschäftigt sich mit der Entwicklung von miteinander konkurrierenden
Bestattungssitten in der Europäischen Bronzezeit. Dabei werden die sich
gegenseitig jeweils weitestgehend ausschließenden Paare Brandbestattung
und Körperbestattung sowie Flachgrab und Hügelgrab betrachtet. Ein
Datensatz von X C14-datierten Gräbern mit Metainformationen bildet die
Phänomene in einem Zeitfenster von 2500 bis 500calBC in hoher Auflösung
ab. Das Modell wird mit diesem Datensatz gespeist bzw. abgeglichen.
Fragestellungen, die mithilfe des Modells potentiell beanwortet werden
können sind u.A.:

\begin{itemize}
\tightlist
\item
  X
\item
  Y
\item
  Z
\end{itemize}

\hypertarget{gliederung}{%
\section{Gliederung}\label{gliederung}}

Im zweiten Kapitel nach dieser Einführung wird die theoretische
Grundlage des Modells diskutiert. Dabei ist die \emph{Cultural Evolution
theory} die Perspektive aus der alle weiteren wesentlichen Aspekte
besehen werden sollen. Dazu gehören Fragen nach Wahrnehmung, Lernen und
Wissensvermittlung in prähistorischen Gesellschaften (\emph{Cognitive
Archaeology}), nach Mechanismen der Ausbreitung von Ideen
(\emph{Diffusion of Innovation}) und nach Interaktion komplexer,
selbständiger Entitäten in Computermodellen (\emph{Agent-based
Modelling}).

Das dritte Kapitel stellt das Fallbeispiel der Entwicklung
bronzezeitlicher Bestattungssitten vor. Die Betrachtung bezieht sich auf
ein großes Untersuchungsgebiet und einen langen Zeitraum. Entsprechend
werden die wichtigsten Trends aufgezeigt, ohne zu sehr auf regionale und
lokale Detailbeobachtungen eingehen zu können. Neben einer Darstellung
des Forschungsstands aus der Literatur greift dieses Kapitel nach einer
quellenkritischen Analyse auch auf oben beschriebenen Datensatz zurück,
um die Zusammenhänge zu visulisieren und abzugleichen. Ein letzter
Abschnitt reflektiert, inwiefern das Fallbeispiel vor dem Hintergrund
der vorangegangenen theoretischen Überlegungen verstanden werden kann
und welche Fragestellungen durch diesen Modellansatz beantwortet werden
können.

Das darauf folgende, vierte Kapitel präsentiert die
Modellimplementierung. Es erklärt einerseits kurz, wie die technische
Umsetzung des Modells gelöst wurde und andererseits, wie die
Zusammenhänge des Fallbeispiels durch die Software abgebildet werden.
Eine quantitative und qualitative Darstellung der Modelldurchläufe
schließt sich an.

Der Diskussion der Ergebnisse ist Kapitel fünf gewidmet. Dabei werden
Modelloutput und archäologisches Wissen über reale Zusammenhänge
verglichen und kontextualisiert. Die im dritten Kapitel eröffneten
Fragestellungen werden -- soweit möglich -- beantwortet oder
kommentiert.

Ein letztes, sechstes Kapitel fasst die theoretischen Grundlagen, die
Entwicklung der Bronzezeitlichen Bestattungssitten und die
Modellierungsergebnisse abschließend zusammen. Ein Ausblick eröffnet die
Perspektive für zukünftige Forschung und zeigt Möglichkeiten auf, wie
Modell und Vergleichsdatensatz für verschiedene Fragestellungen konkret
verbessert oder angepasst werden könnten.

\hypertarget{cultural-evolution}{%
\chapter{Cultural Evolution}\label{cultural-evolution}}

\hypertarget{definition-und-forschungsgeschichte}{%
\section{Definition und
Forschungsgeschichte}\label{definition-und-forschungsgeschichte}}

Die Grundaussage der \emph{Cultural evolution theory} ist, dass die
Prozesse der natürlichen Entwicklung von Spezies durch Evolution auch
bei der menschlichen Kulturentwicklung wirken. Mechanismen der Evolution
wie Selektion und Mutation wären entscheidend dafür, welche
Verhaltensweisen, Ideen und Innovationen sich langfristig durchsetzen
könnten. Entsprechend könnte biologische Terminologie und Modellbildung
zumindest eingeschränkt auch in anthropologischen Kontexten sinnvoll
eingesetzt werden.

Cultural evolution theory wird in der archäologischen Fachliteratur vor
allem als \emph{Darwinian Archaeology} oder \emph{Evolutionary
Archaeology} diskutiert. Daneben gab und gibt es in der
Forschungsgeschichte eine ganze Reihe weiterer Begriffe und Schulen, die
mit dem Evolutionsbegriff verknüpft sind. Das ist kein rein
archäologisches Forschungsgebiet: Unter anderem Verhaltensbiologie,
Neurologie, Genetik, Soziologie und alle Anthropologischen Fächer sind
betroffen und haben sich an dieser Diskussion beteiligt. Die Übertragung
biologisch-evolutiver Wirkmechanismen zur Erklärung menschlichen
Verhaltens war bereits Bestandteil der frühesten öffentlichen Debatte um
Charles Darwins (*1809 - †1882) Evolutionstheorie als seine
Standartwerke \emph{On the Origin of Species} \footnote{\textcite{Darwinoriginspeciesmeans1859}}
und \emph{The Descent of Man}\footnote{\textcite{Darwindescentmanselection1871}}
in der Fachwelt und Öffentlichkeit verarbeitet wurden\footnote{\textbf{Zitat!}}.

Die biologische Forschung ist nicht bei Charles Darwin stehen geblieben
sondern hat sich über die Korrekturen im \emph{Neo-Darwinismus} um 1890,
über die \emph{Synthetische Theorie der biologischen Evolution} um 1940
und die \emph{Erweiterte Synthetischen Theorie} Ende der 1990er
weiterentwickelt. Ende des 19. Jahrhunderts wurden wesentliche Aspekte
biologischen Evolutionstheorie noch kontrovers diskutiert\footnote{\textcite{bowler_evolution_1989},
  188-202.}. Insbesondere der Streit zwischen darwinistischer Evolution
durch Selektion und lamarkistischer Evolution durch Vererbung erworbener
Eigenschaften war nicht entschieden. Jean-Baptiste de Lamarck (*1744 -
†1829) war zwar weitestgehend überholt, aber sein Adaptionsgedanke lebte
in \emph{Neo-Lamarckismus}\footnote{\textcite{bowler_evolution_1989},
  236-247.} und \emph{Orthogenese}\footnote{\textcite{bowler_evolution_1989},
  247-250.} fort, die als Alternativen für den vor allem von August
Weismann (*1834 - †1914) und Alfred Russel Wallace (*1823 - †1913)
propagierten Neo-Darwinismus\footnote{\textcite{bowler_evolution_1989},
  251-260.} diskutiert wurde. Weismann vertrat einen dogmatischen
Selektionismus und führte mit der Keimplasmatheorie eine Erklärung für
Vererbung ein, die wichtige Aspekte der Genetik vorwegnahm und
lamarckistische Adaption ausschloss. Die frühe \emph{Genetik} ging
jedoch nicht aus darwinistischem Selektionismus hervor. Stattdessen
wurde die Wiederentdeckung der bereits von Gregor Mendel (*1822 - †1884)
1866 publizierten \emph{Mendelschen Vererbungsregeln} um 1900 vor allem
im Kontext der \emph{Saltationstheorie} diskutiert, die nicht Selektion,
sondern tiefgreifende, spontane Mutationen als Motor der Evolution
favorisierte\footnote{\textcite{bowler_evolution_1989}, 260-261.}. Ein
bekannter, streitbarer Vertreter dieser Schule war William Bateson
(*1861 - †1926). Er prägte den Begriff \emph{Genetik} und trug
maßgeblich zur Popularisierung der Mendelschen Regeln bei. Ihm entgegen
stand die ebenfalls noch junge Wissenschaft der \emph{Biometrie}, die
statistische Methoden zur Untersuchung von Populationen einführte und
die Bedeutung von Selektion hervorhob. Darwins Cousin Francis Galton
(*1822 - †1911) gilt als Vorreiter dieser Strömung, vertrat aber eine
fehlerhafte, inkohärente Vererbungslehre. Erst Nachfolgern wie Walter
Frank Raphael Weldon (*1860 - †1906) und Karl Pearson (*1857 - †1936)
gelang der Nachweis, dass Selektion zur nachhaltiger Veränderung in
Populationen führen kann\footnote{\textcite{bowler_evolution_1989},
  256-260.}. Die Debatte um den genauen Mechanismus der Evolution war
entscheidend für die Biologie im späten 19. und frühen 20. Jahrhundert.
Die Interdepenz von Mutation, Adaption und Selektion war noch nicht
verstanden. Est 1920\ldots{}\textbf{ausbauen} \footnote{\textcite{bowler_evolution_1989},
  268-273.}

Parallel zu den Entwicklungen in den Naturwissenschaften -- allerdings
mit allgemein geringen Wechselwirkungen -- wurde Evolutionstheorie auch
im wissenschaftichen Diskurs der Sozialwissenschaften reflektiert. Eine
erste wesentliche Spannungslinie, die hier betrachtet werden muss,
reicht von \emph{Evolutionismus} über \emph{Neoevolutionismus} hin zu
\emph{Kulturrelativismus} und \emph{Multilinearer Evolution}. Sie hat in
der archäologischen Theoriediskussion große Wirkung entfaltet und ist
untrennbar mit der Geschichte des Faches verknüpft.

Klassischer \emph{Evolutionismus} ist ein Überbegriff für die erste
Übertragung biologischer Evolutionsforschung auf die Kulturgeschichte.
Er betont den Aspekt des schrittweisen, kulturellen Aufstiegs und der
Zunahme organisatorischer Komplexität. Zivilisation hätte sich über
mehrere Fortschrittssstufen von einem primitiven Urzustand zur modernen
Industriegesellschaft weiterentwickelt. Die Beschreibung einer Kultur
kann vor diesem Hintergrund in sehr einfachen Begriffen und mit wenigen
Parametern erfolgen\footnote{Brockhaus (Zitat nachtragen)}. Bei der
ersten Formulierung Evolutionistischer Theorie hat Darwins Werk jedoch
nur eine untergeordnete Rolle gespielt. Protagonisten wie Herbert
Spencer (*1820 - †1903) und John Lubbok (*1834 - †1913) orientierten
sich stärker an Charles Lyell (*1797 - †1875), der in der ersten Hälfte
des 19. Jahrhunderts mit den geologischen Schlüsselprinzipien
\emph{Aktualismus} (rezente, natürliche Phänomene haben so auch in der
Vergangenheit stattgefunden) und \emph{Gradualismus} (geologischer
Wandel ist langsam und stetig) wesentliche Grundlagen für die
Evolutionsforschung gelegt hatte. Die Prinzipien gaben der
stratigraphischen Vergesellschaftung menschlicher Skelettüberreste mit
Pleistozänen Tierknochen eine neue Bedeutung, die eine auf breiter Front
\emph{vergleichende Methode} rechtfertigte. Damit wurden
vorgeschichtliche Gesellschaften dem Vergleich mit `primitiven',
rezenten Gesellschaften zugänglich. Evolutionismus konzentrierte sich
nicht auf Mechanismen der Evolution wie Mutation und Selektion, sondern
griff ein dem Kapitalismus entlehntes Konzept von Wettbewerb und
Weiterentwicklung der Kulturen auf, das durch Vergleich mit rezenten
Gesellschaften und deren Organisationsgrad versteh- und kategorisierbar
geworden war. Die Evolutionisten bildeten keine kohärente Schule.
Stattdessen wurde eine Gruppe von Individuen -- maßgeblich Lewis Henry
Morgan (*1818 - †1881), Herbert Spencer, John Ferguson McLennan (*1827 -
†1881), Edward Burnett Tylor (*1832 - †1917) und John Wesley Powell
(*1834 - †1902) -- abschätzig von Gegnern mit diesem Begriff belegt. Dem
Evolutionismus wurde vorgeworfen, die Aussagekraft materieller Kultur
über die soziale Organisation vorgeschichtlicher Gesellschaften
positivistisch überbewertet zu haben. \emph{Konjekturalgeschichte} und
\emph{vergleichende Methode} hätten zu einer Perspektive unlinearer
Entwicklung geführt, die durch Stufengliederung der
Menschheitsgeschichte kulturelle Vielfalt unangemessen reduziert und
durch die Konzentration auf progressive Entwicklungsabläufe zu falschen
ethnologischen Beobachtungen geführt habe\footnote{\textcite{petermann_geschichte_2004},
  464-474, 734.}. Zuletzt wäre die vorgenommene Abgrenzung von
Entwicklungsstadien mit einer Teleologisierung auf die moderne,
westliche Gesellschaft verbunden und damit Grundlage einer
Rechtfertigung von Rassismus, Eurozentrismus und Imperialismus. Damit
wurde der Begriff \emph{Sozialdarwinismus} assoziiert\footnote{\textcite{ShennanGenesmemeshuman2002},
  11.}.

\emph{Sozialdarwinismus} ist ebenso wie Evolutionismus keine koherente
wissenschaftstheoretische Schule, sondern eine polemische Zuschreibung
wissenschaftlicher, ideologischer und politischer Gegner. Die heftige
Kontroverse, die rund um Evolutionstheorie in der zweiten Hälfte des 19.
Jahrhunderts entstand, wurde von Propagandisten wie Thomas Henry Huxley
(*1825 - †1895) (``Darwin's Bulldog'') oder, im deutschsprachigen Raum,
Ernst Haeckel (*1834 - †1919) getragen. Die Erkenntnisse hatten
Konsequenzen für fundamentale weltanschauliche Fragen -- entsprechend
wurde die Diskussion von der Presse aufgegriffenen und einer breiten
Öffentlichkeit präsentiert. Das hatte starke, oft unagemessene
Vereinfachung der Themenstellung zufolge. Die Reduktion von
Evolutionstheorie auf griffige Phrasen wie ``survival of the fittest''
und ``natural selection'' wirkte sich schließlich auch auf den Diskurs
in den Sozialwissenschaften aus. Spencer entwickelt in seinem Hauptwerk
\emph{The Principles of Sociology}\footnote{\textcite{SpencerHerbertSpencerPrinciples1898}}
das Narrativ eines evolutionären Kampf ums Dasein, der nur in den
jüngsten Phasen der Menschheitsgeschichte von Altruismus begleitet
wird\footnote{\textcite{petermann_geschichte_2004}, 501-510.}. Diese
sozialphilosophische Theorie fällt im Klima der fortgeschrittenen
Industrialisierung und deren Konkurrenzgesellschaft auf fruchtbaren
Boden. Noch heute wirkt der Gedanke eines Überlebendskampfs im
marktwirtschaftlichen Geschehen nach und hat sich etwa über christliche
Prädestinationslehre zu jenem traditionell amerikanischen Topos
stabilisiert, der sich politisch gegen staatliche Eingriffe ins
Wirtschaftssystem und für individuelle, zwischenmenschliche Solidarität
ausspricht. Spencer beeinflusste eine ganze Reihe amerikanischer
Ethnologen und Soziologen\footnote{\textcite{smith_cultural_1992}, 62.},
darunter William Graham Sumner (*1840 - †1910), Lester Frank Ward (*1841
- †1913) und Franklin Henry Giddings (*1855 - †1931). Sie teilten
Spencers Verständnis biosozialer Evolution und deren
empirisch-positivistischer Erforschbarkeit, jeder repräsentiert
gleichermaßen aber gegensätzliche Ansichten darüber, wie stark die
evolutiven Prozesse menschliche Gesellschaften determinieren. Europas
Sozialdarwinisten waren keine Spencerianer, dafür aber umso stärker
Theorien radikal-biologischen und rassistischen Existenzkampfs
verpflichtet. Zu nennen sind unter anderem Gustav Ratzenhofer (*1842 -
†1904), Jakov Novicov (*1849 - †1912), Michelangelo Vaccaro (*1854 -
†1937) und besonders der jüdisch-polnische Jurist und Soziologe Ludwig
Gumplowicz (*1838 - †1909), der mit seinem wissenschaftlichen Rassismus
in einer Rede im September 1933 von Adolf Hitler fast wörtlich zitiert
wurde\footnote{\textcite{petermann_geschichte_2004}, 511-524.}:

\begin{quote}
Nie und nirgends sind Staaten anders entstanden als durch Unterwerfung
fremder Stämme seitens eines oder mehrerer verbündeter oder geeinigter
Stämme.

-- \autocite{GumplowiczGrundrissSoziologie1885}
\end{quote}

Ein wichtiger Antrieb für Sozialdarwinistische Theorie war die
biometrische Forschung von Galton, der intellektuelle Fähigkeit als eine
maßgeblich biologisch vererbbare Eigenschaft beschrieb. Ethnische
Herkunft hielt er in einer Form von Rassenlehre für das entscheidende
Kriterium für die Intelligenz eines Individuums. Er sprach sich in
dieser Konsequent für bewusste Zuchtwahl beim Menschen aus und prägte
den Begriff \emph{Eugenik}\footnote{\textcite{bowler_evolution_1989},
  256-257.}.

Kritiker des Evolutionismus in der ersten Hälfte des zwanzigsten
Jahrhunders waren Vertreter der britischen \emph{Social Anthropology},
deutscher \emph{Kulturgeschichte} und vor allem der von Franz Boas
(*1858 - †1942) etablierten, amerikanischen \emph{Kulturanthropologie}.
Die Gemeinsamkeit dieser Schulen und Strömungen liegt an ihrem
traditionellen Fokus auf den jeweiligen naturräumlichen, historischen
und soziopolitischen Kontext einer kulturellen Ausprägung. Boas war
Jude, absolvierte ein naturwissenschaftliches Studium in Deutschland und
emigrierte nach seiner Zuwendung zur Ethnologie in die USA. Boas gilt
als Begründer des \emph{historischen Partikularismus}, der sich gegen
deduktive, umfassende Erklärungsmodelle wie Evolutionismus und
Diffusionismus wandte, die vergleichende Methode und ihre
Analogieschlüsse verwarf und stattdessen eine genaue, empirische
Detailanalyse von Einzelphänomenen betonte. Dabei war Boas
Forschungsansatz im Sinne des \emph{four-field approach}, der
Ethnologie, Archäologie, Linguistik und Physische Anthropologie
zusammenführt, methodisch durchaus breit aufgestellt. Methodisch
vielfältige und empirisch fundierte aber gleichzeitig zeitlich und
räumlich eng begrenzte Fallstudien sollten den Weg zu einer induktiven
Kulturwissenschaft ebnen. Boas begründete damit eine Phase intensiver
Datenaufnahme in der amerikanischen Anthropologie (\emph{Salvage
ethnography}), die seine Kritiker wiederrum als theorielos verurteilten.
1911 erschien sein Werk \emph{The mind of Primitive Man}\footnote{\textcite{Boasmindprimitiveman1911}},
das die wichtigsten Thesen seines \emph{Kulturrelativismus}
zusammenfasst: Es wendet sich gegen biologischen Determinismus, betont
den Einfluss von sozialem Lernen und hebt die Multikausailtät
historischer Entwicklungen hervor. Kultur sei abhängig von einer
Vielzahl natürlicher und zwischenmenschlicher Parameter. Diese
Relativität nähme der uniliniearen Gliederung von Kulturzuständen des
Evolutionismus die Grundlage. Boas war ein politischer Mensch und
argumentierte mit Kulturrelativismus gegen Rassismus und
Faschismus\footnote{\textcite{petermann_geschichte_2004}, 643-655.}.
Schüler von Boas (\emph{Boasianer}) wie Clark Wissler (*1870 - †1947),
Elsie Clews Parsons (*1875 - †1941), Alfred Kroeber (*1876 - †1960),
Alexander Goldenweiser (*1880 - †1940), Robert Lowie (*1883 - †1957),
Paul Radin (*1883 - †1959), Edward Sapir (*1884 - †1939) prägten die
amerikanische Ethnologie nachhaltig und führten über Jahrzehnte einen
erbitterten Diskurs mit Evolutionisten und Neoevolutionisten\footnote{\textcite{petermann_geschichte_2004},
  654-688.}.

\emph{Neoevolutionismus} -- der Begriff wiederrum eine Fremdzuschreibung
-- bezeichnet eine Strömung, die als Reaktion auf berechtigte Kritik am
Evolutionismus in den 30er Jahren des 20. Jahrhunderts und insbesondere
nach dem 2. Weltkrieg an Dynamik gewann. Sie verbindet Ansätze, die sich
zwar sozialdarwinistischem Biodeterminismus verweigern, andererseits
aber dennoch bewusst nach Gesetzmäßigkeiten soziokultureller Prozesse
suchen um der Anthropologie ein höheres Abstraktionsniveau zu
erschließen. Aus dieser Definition heraus lassen sich dem
Neoevolutionismus einige der bedeutensten Ethnologen und Archäologen
zuordnen: Vere Gordon Childe (*1892 - †1957), Karl Wittfogel (*1896 -
†1988), George Murdock (*1897 - †1985), Leslie White (*1900 - †1975) und
Julian Haynes Steward (*1902 - †1972) Auch die Arbeit einer nachfolgende
Generation mit Protagonisten wie Elman Service (*1915 - †1996), Morton
Fried (*1923 - †1986), Roy Rappaport (*1926 - †1997), Marshall Sahlins
(*1930) oder Lewis Binford (*1931 - †2011) ist stark von
neoevolutionistischem Denken geprägt.

Vere Gordon Childe, ursprünglich Philologe aus Australien, etablierte
sich in Europa durch seine großen, synthetischen Werke als
Prähistoriker. Ihm gelang es, die Gliederung der Menschheitsgeschichte
in Entwicklungsphasen -- Childe griff Morgans Unterscheidung von
Wildheit, Barbarei und Zivilisation auf -- durch einen multilinearen
Ansatz neu zu beleben und in kohärenten, archäologischen Narrativen
(z.B. \emph{The Dawn of European Civilization}\footnote{\textcite{childe_dawn_1925}},
\emph{Man Makes himself}\footnote{\textcite{childe_man_1936}} oder
\emph{Social Evolution}\footnote{\textcite{childe_social_1951}}) nutzbar
zu machen. Als überzeugter Marxist etablierte er den Topos
\emph{vorgeschichtlicher Revolutionen}, der Marx \emph{Historischen
Materialismus} weiterentwickelt und konkretisiert. Childes Kritiker
waren zunächst vor allem jene Spezialisten, deren Forschung er in seinen
Büchern zusammen zu fassen und zu vereinfachen auf sich genommen hatte.
Der deutsche Soziologe und Sinologe Karl Wittvogel beschäftigte sich mit
dem Einfluss von Bewässerungssystemen im Entstehungsprozess früher
Hochkulturen. Mit seiner Studie zu \emph{Hydraulischen
Gesellschaften}\footnote{\textcite{wittfogel_oriental_1957}} hat er ein
einflussreiches, evolutionistisches Werk vorgelegt, das Staatenbildung
und die Herausbildung der Hierarchie des \emph{orientalischen
Despotismus} mit Verwaltungsnotwendigkeiten von Bewässerungssystemen
erklärt. Wittvogels Theorie hat bemerkenswerte Rezeption und erfahren
und wurde in eine Vielzahl anderer Kulturzusammenhänge hineinprojiziert.
George Murdock war ein Vorreiter der \emph{Cross-Cultural Analysis} und
Begründer der \emph{Human Relations Area Files}\footnote{\url{http://hraf.yale.edu}
  {[}28.01.2018{]}}. Dieses Archiv ist 1949 aus einer von Murdock
entwickelten Sammlung hervorgegangen, enthält strukturierte
Informationen und Literaturlisten zu Kulturmerkmalen vieler hundert --
meist indigener -- Gesellschaften und wird bis heute gepflegt. Murdocks
\emph{transkultureller Vergleich} basiert auf evolutionistischer
Grundlage und ist stark von quantitativer Auswertung mit
ethnostatistischen Methoden geprägt: Sein Hauptwerk \emph{Social
Structure}\footnote{\textcite{murdock_social_1949}} analysiert und
dokumentiert universelle Regeln und Gesetze sozialer Beziehungen anhand
eines Datensatzes von 250 Ethnien. Im Kontext der Kritik am
Evolutionsimus wurde auch Murdock vorgeworfen, Kulturzüge unsachgemäß
isoliert betrachtet oder einer solchen Betrachtung zugänglich gemacht zu
haben. Der amerikanische Ethnologe Leslie White war einer der
wichtigsten Protagonisten des Neoevolutionismus. Nach seiner Lektüre von
Morgan und anderen Evolutionisten wie Spencer und Tylor suchte er
explizit die Konfrontation mit dem vorherrschenden Partikularismus der
Boasianer und stellte ihr eine umfassende, materialistische
Kulturtheorie gegenüber. Diese würde objektiven Kulturvergleich im Sinne
einer Wissenschaft der \emph{Kulturologie} entlang einer evolutiven
Skala des Prokopfverbrauchs von Energie ermöglichen: \emph{White's
Law}\footnote{\textcite{white_energy_1943},
  \textcite{white_science_1949}}. White betonte die Bedeutung von
Technologie und Wirtschaft für die Herausbildung von Sozialordnung und
Ideologie, erkannte aber auch die einzigartige, symbolschaffende
Kreativität des Menschen an. Kritiker werfen ihm vor, diesen impliziten
Widerspruch niemals aufgelöst zu haben. Dennoch inspirierte Whites
klare, regelbasierte Anthropologie eine Generation von Studierenden die
sich im Kulturrelativismus nicht wiederfinden konnten. Neben White ist
auch Julian Steward eine der tragenden Säulen des Neoevolutionismus.
Steward veröffentlicht 1955 \emph{Theory of Culture Change}\^{}, wo er
\emph{Kulturökologie} als Wissenschaft von definierbaren Ursache-Wirkung
Beziehungen von Natur- und Mensch jenseits des überholten
\emph{Kulturdeterminismus} formuliert. Sein Vorschlag zur Periodisierung
der Ur- und Frühgeschichte folgt einem \emph{multilinearen} Ansatz, der
der \emph{unilinearen} Abfolge von für alle Kulturen immer gleicher
Zustandsformen die Analogentwicklung von \emph{Kulturtypen} -- Typen der
Umweltanpassung -- entgegenstellt. Unter bestimmten natürlichen und
sozialen Bedingugen würden sich bestimmte Verhaltensmuster und Formen
des Zusammenlebens ergeben, nicht aber mit zwingender Notwendigkeit oder
in einer definierten Abfolge. Auch Steward bezog sich methodisch auf
transkulturellen Vergleich, der es ermöglichen sollte, die primären,
subsistenzbezogenen Eigenschaften von techno-ökonomischen
\emph{Kulturkernen} im Gegensatz zum Überbau der sekundären, variablen
Charakterzüge von Kulturen zu definieren. Mehrere Protagonisten der noch
jungen \emph{New Archaeology} wurden von Stewards modernem,
pragmatischer Evolutionismus stark beeinflusst\footnote{\textcite{petermann_geschichte_2004},
  734-761.}.

Eine neue Welle der Auseinandersetzung mit Kulturevolution gewinnt Mitte
der 1970er Jahre an Dynamik\footnote{\textcite{creanza_cultural_2017}}.
Sie lenkt das Interesse weg von Politik und Gesellschaftsstruktur,
sondern abstrahiert auf die basalen Grundzüge menschlichen Denkens.
Dieser Ansatz inkorporiert Ergebnisse und Methoden moderner,
biologischer Verhaltensforschung und erlaubt neue Perspektiven Jenseits
des Evolutionismus und seiner Varianten. Von entscheidender Bedeutung
für die Entstehung dieser Strömungen sind Edward Osborne Wilsons (*1929)
\emph{Sociobiology: The New Synthesis}\footnote{\textcite{WilsonSociobiologynewsynthesis1975}}
und Richard Dawkins (*1941) \emph{The Selfish Gene}\footnote{\textcite{Dawkinsselfishgene1976}},
auf das unten genauer eingegangen werden soll\footnote{\textcite{SmithThreestylesevolutionary2000},
  27.}. Auch Luigi Luca Cavalli-Sforza (*1922), Marcus William Feldmann
(*1942) und andere entwickeln wesentliche Ansätze für den Brückenschlag
zwischen Biologie und Anthropologie\footnote{\textcite{alland_cultural_1972},
  \textcite{cavalli-sforza_models_1973}, \textcite{feldman_models_1975},
  \textcite{feldman_cultural_1976}, \textcite{blum_uncertainty_1978}}.
Um die Jahrtausendwende unterscheidet Eric Aldan Smith schließlich drei
große Strömungen\footnote{\textcite{SmithThreestylesevolutionary2000}.
  Stephen Shennan greift diese Unterscheidung auf
  \autocite[15-18.]{ShennanGenesmemeshuman2002}} in der Untersuchung
menschlichen Verhaltens aus einer Evolutionsperspektive:
\emph{Evolutionary psychology}, \emph{Human behavioural ecology} und
\emph{Dual inheritance theory}.

\emph{Evolutionary psychology} konzentriert sich auf die Entwicklung des
menschlichen Denkens vor dem Hintergrund seiner evolutionären
Geschichte. Selektiver Druck habe zur Ausbildung spezialisierter
Verhaltensmodule geführt, die in bestimmten Situationen bestimmte
Reaktionen auslösen. Von entscheidender Bedeutung für die Entstehung
dieser angepassten Verhaltensmodule sei die \emph{Environment of
Evolutionary Adaptiveness (EEA)}, also die Umgebung, in der sich die
menschliche Entwicklung maßgeblich abgespielt hat. Dabei bezieht sich
die \emph{Evolutionary psychology} auf die Lebensrealität pleistozäner
Jäger- und Sammlergruppen, in der der moderne Mensch den überwältigend
größten Teil selektiv wirksamer Generationszyklen durchlebt hat. Die
Selektionsparameter wären in diesem Zeitraum relativ stabil geblieben.
In der Konsequenz seien Menschen heute beispielsweise ideal an das
nomadische Leben in kleinen Gruppen in großer gegenseitiger Abhängigkeit
adaptiert, Männer würden junge, gesunde und hübsche Sexualpartnerinnen
bevorzugen und süße Speißen wären beliebt, weil Süße bei Früchten ein
Indikator für Reife und Genießbarkeit ist. Alle Aspekte des Verhaltens
seien auf bestimmte Gegebenheiten in der \emph{EEA} optimiert und
entsprechend schlecht für eine andere, etwa neolithische oder
postneolithische Lebensweise geeignet\footnote{\textcite{SmithThreestylesevolutionary2000},
  27-29.}. Der \emph{Evolutionary psychology} wird vorgeworfen, die
unangemessen vereinfachende Annahmen über vorgeschichtliches Verhalten
zu treffen, ohne sich ausreichend mit jenen archäologischen Daten und
Auswertungsergebnissen auseinanderzusetzen, die eine Rekonstruktion der
tatsächlichen Lebensverhältnisse in der Vorgeschichte erlauben würden.
Aus archäologischer Perspektive greift unter anderem
\textcite{Mithenprehistorymindsearch1996} Überlegungen der
\emph{Evolutionary psychology} auf\footnote{\textcite{ShennanGenesmemeshuman2002},
  15.}.

\emph{Human behavioural ecology} überträgt Ansätze aus der
Verhaltensbiologie auf den Menschen\footnote{\textcite{smith_cultural_1992};
  \textcite{winterhalder_analyzing_2000}}. Dabei nimmt sie den
klassisch-darwinistischen Standpunkt ein, menschliches Verhalten könnte
ebenso wie tierisches als permanente Maximierung des
Reproduktionserfolgs durch Selektion verstanden werden\footnote{\textcite{creanza_cultural_2017}}.
Bewusste oder unbewusste Entscheidungen würden hinsichtlich der Frage
getroffen werden, inwiefern das Ergebnis den Erhalt der eigenen
genetischen Information gewährleistet. Im Zentrum steht dabei die
Beziehung zwischen Mensch und natürlicher Umwelt: ``Welche ökologischen
Faktoren (z.B. Ressourcenverfügbarkeit, Populationsdichte, etc.)
schaffen den Rahmen dafür, dass ein bestimmtes Verhalten (z.B.
Altruismus, Vorratshaltung, etc.) zum Erfolg führt?''. Die ökologische
Nische des Menschen in Relation zu seinen Subsistenzstrategien, seinem
Paarungsverhalten und seiner sozialen Struktur sind wesentliche
Forschungsgegenstände der \emph{Human behavioural ecology}\footnote{\textcite{henrich_search_2001};
  \textcite{kaplan_theory_2000}; \textcite{voland_evolutionary_1998};
  \textcite{winterhalder_risk-senstive_1999}}. Die kleinteilige
Aufgliederung der Fragestellungen hinsichtlich einzelner Situationen und
Verhaltensweisen erlaubt es dabei, auch komplexe Fragen quantitativ in
einfachen Modellen abzubilden. Diese Modelle versprechen testbare
Aussagen: ``Wenn Frauen ihre Sexualpartner nach dem Kriterium wählen,
wer den Nachwuchs am besten versorgen kann, dann wäre die Anzahl der
Frauen pro Mann proportional zu seinem Reichtum.''. Die Reduktion auf
direkte, kausale Beziehungen birgt jedoch die Gefahr die vielfältigen
Interdependenzen einzelner Verhaltensweisen zu übersehen. Gerade
Langtzeitstudien spielen dafür eine wichtige Rolle\footnote{\textcite{belovsky_optimal_1988};
  \textcite{broughton_widening_1997}; \textcite{low_population_1993};
  \textcite{stiner_paleolithic_1999}; \textcite{stiner_tortoise_2000};
  \textcite{winterhalder_population_1988}}. \emph{Behavioural ecology}
erklärt die Vielfalt menschlichen Verhaltens aus der großen Diversität
biologischer- und sozialer Nischen, die sehr viele unterschiedliche
Erfolgsstrategien erlaubt. Tatsächlich gäbe es sogar eine Korrelation
zwischen Verhaltensvielfalt und Diversität der sozioökologischen Umwelt.
Sie erlaubt sich eine große Vereinfachung, indem sie die Mechanismen,
die zur Ausbildung einer Verhaltensanpassung führen, nicht hinterfragt:
Die einschränkende Wirkung von Kultur (hier: vererbtes Verhalten) etwa
in Form von Tradition sei untergeordnet, da erfolglose Strategien
unabhängig davon in wenigen Generationen durch biologische Selektion
aussterben würden. Diese bewusste, statistische Vereinfachung von
Übergangsprozessen wird als \emph{phenotypic gambit}
bezeichnet\footnote{\textbf{Zitat!} und \textbf{Prüfen!}}. In dieser
Konsequenz sei auch anzunehmen, dass der Mensch sein Verhalten schnell
und gut an die revolutionären Veränderungen des Holozän oder der
Industrialisierung angepasst habe\footnote{\textcite{SmithThreestylesevolutionary2000},
  29-31.}.

\emph{Dual inheritance theory} postuliert neben der Vererbung von Genen
ein zweites Vererbungssystem von Ideen und Kulturmerkmalen. Auch diese
würden von Generation zu Generation, von Person zu Person und von Tag zu
Tag weitergereicht und stünden unter dem Einfluss von Selektion und
Mutation. Dabei würde sowohl die im genetischen Vererbungssystem
entscheidende, natürliche Selektion wirken als auch eine Selektion durch
bewusste oder unbewusste Entscheidung der Träger von Ideen: Menschen.
Ersterer Selektionsprozess sei Konsequenz der Rückwirkung von Ideen auf
die Fitness ihrer Träger, letzterer ein System von Interdepenzen
verschiedener Ideen, Umweltsituationen und genetischer Determinanten.
Ebenfalls von entscheidender Bedeutung seien die zwischenmenschlichen
Prozesse wie Erziehung, Gefolgschaft oder Freundschaft, die die
Weitergabe von Ideen steuern. Entstehung neuer Ideen aus der Kombination
vorhandener wäre eine Form der Mutation. Da nun also in der
Kulturgeschichte Vererbung, Entstehung von Variabilität und Auswahl nach
Fitnesskriterien als gegeben angenommen werden dürften, und damit große
strukturelle Ähnlichkeit des genetischen und des kulturellen
Vererbungssystems bestünde, sei auch die Übertragung neo-darwinistischer
Methoden auf die Untersuchung von Kulturmerkmalen möglich. Die beiden
Vererbungssysteme könnten unabhängig und in ihrer Interaktion erforscht
werden, wobei Konzepte zur Erklärung des einen potentiell auch zur
Erklärung im anderen geeignet sein könnten. Andererseits gäbe es auch
klare Unterschiede: Beispielsweise erfolgt die Weitergabe genetischer
Information fast ausschließlich vertikal durch sexuelle oder asexuelle
Fortpflanzung, während Ideen beliebig horizontal weitergeben werden,
also unabhängig von Verwandschaft diffundieren können. Individuelle
Menschen sind zwar sowohl Träger vieler Gene als auch vieler
Kulturmerkmale, erstere werden aber nur einmal festgelegt, während
letztere ständigem Wechsel unterliegen. \emph{Dual inheritance theory}
ist sich dieser Unterschiede bewusst, hält sie aber für analystisch
bewätigbar. Da die kulturelle Evolution in anderen zeitlichen,
räumlichen und kausalen Maßstäben agieren würde, könnte diese Theorie
auch das Auftreten von Verhaltensmerkmalen erklären, die aus einer
Reproduktionsperspektive nicht sinnvoll sind. Kulturelle Evolution ist
schneller und flexibler: Anpassung an neue oder für das Überleben von
Menschen ungeeignete Umgebungen geschieht nicht mehr genetisch, sondern
durch Verhaltensanpassung. Genetische Anpassung folgt der kulturellen
langsam, bedeutet aber auch Einschränkungen für die Flexibilität der
kulturellen Evolution\footnote{\textcite{SmithThreestylesevolutionary2000},
  31-33.}.

Smith legt seinem Artikel Tabelle \ref{tab:smiththreestyles} bei, die
die Unterschiede zwischen \emph{Evolutionary psychology}, \emph{Human
behavioural ecology} und \emph{Dual inheritance theory}. übersichtlich
darstellt.

\begin{table*}

\caption{\label{tab:smiththreestyles}Three Styles of Evolutionary Explanation (nach \textcite{SmithThreestylesevolutionary2000})}
\centering
\begin{tabu} to \linewidth {>{\raggedright\arraybackslash}p{15em}>{\raggedright\arraybackslash}p{10em}>{\raggedright\arraybackslash}p{10em}>{\raggedright\arraybackslash}p{10em}}
\toprule
 & Evolutionary psychology & Behavioural ecology & Dual.inheritance theory\\
\midrule
What is being explained: & Psychological mechanisms & Behavioural strategies & Cultural evolution\\
Key constraints: & Cognitive, genetic & Ecological, material & Structural, information\\
Temporal scale of adaptive change: & Long-term (genetic) & Short-term (phenotypic) & Medium-term (cultural)\\
Expected current adaptiveness: & Lowest & Highest & Intermediate\\
Hypothesis generation: & Informal inference & Optimality models & Population-level models\\
\addlinespace
Hypothesis-testing methods: & Survey, lab experiment & Quantitative ethnographic observartion & Mathematical modelling and simulation\\
Favoured topics: & Mating, parenting, sex differences & Subsistence, reproductive strategies & Large-scale cooperation, maladaptation\\
\bottomrule
\end{tabu}
\end{table*}

\emph{Dual inheritance theory} ist in der archäologischen Forschung am
intensivsten reflektiert worden und ist auch die theoretische Grundlage
für das Modell, das für die vorliegende Arbeit entwickelt wurde. Sie
soll also im folgenden genauer beleuchtet werden. Um ihre Ursprünge
nachzuzeichnen, möchte ich zunächst auf Richard Dawkins \emph{Memetik}
eingehen. Memetik wurde mehrfach als populärwissenschaftliche oder sogar
pseudowissenschaftlich kritisiert. Grund dafür ist unter anderem ihr
niederschwelliger Zugang zu Cultural evolution Theorie, der es jedoch
erlaubt grundlegende Konzepte anschaulich zu illustrieren.

\hypertarget{memetik}{%
\section{Memetik}\label{memetik}}

Memetik (engl. \emph{Memetics}) ist eine Variante der oben beschriebenen
\emph{Dual inheritance theory}. Der Begriff \emph{Meme} wurde 1976 vom
britischen Evolutionsbiologen Richard Dawkins in seinem Buch \emph{The
selfish gene} eingeführt\footnote{\textcite{Dawkinsselfishgene1976}. Ich
  werde im folgenden aus einer Neuauflage des Buches zitieren, die 2016
  40 Jahre nach der Erstpublikation veröffentlicht und um Kommentare von
  Dawkins erweitert wurde: \textcite{Dawkinsselfishgene40th2016}.}.
Obgleich populärwissenschaftlich hat es doch in verschiedenen
Fachbereichen beachtliche Rezeption erfahren und darf als Grundstein der
Memetik gelten.

\hypertarget{the-selfish-meme}{%
\subsection{The selfish meme}\label{the-selfish-meme}}

Dawkins führt in \emph{The selfish gene} einen wesentlichen
Perspektivenwechsel durch, indem er Evolution nicht aus der Sicht der
sich entwickelnden Organismen sondern aus der der Gene betrachtet. Gene
würden -- freilich nicht bewusst -- Organismen als komplexe Vehikel für
ihre eigene Reproduktion nutzen: \emph{the gene's eye view}.
\textbf{Ausbauen!}

In Kapitel 11\footnote{\textcite{Dawkinsselfishgene40th2016}, 287-}
bezieht Dawkins schließlich explizit die Spezies Mensch in seine Analyse
mit ein. Ist ist die Menschheit im selben Umfang der Determination durch
seine Gene untertan? Dawkins verneint das: Sein Kulturverhalten würde
den Menschen von allen anderen bekannten Lebewesen abheben. Auch bei
Tieren gibt es Verhaltensmuster, die unabhängig von genetischer
Vererbung von Individuum zu Individuum weitergegeben werden:
beispielsweise bestimmte Melodien des Gesangs von Singvögeln, die
erwachsene Tiere voneinander lernen. Kein anderes bekanntes Lebewesen
erreicht jedoch das Komplexitätsniveau des Menschen, der Sprache, Mode,
Ritual, Kunst, Architektur und Technologie besitzt und sie unter
ständigen Anpassungen tradiert. Die Entwicklungen in diesen Bereichen
über archäologische Zeiträume zeigt eine Tendenz hin zu zunehmend
höherer Komplexität und Vielfalt. Geschwindigkeit und Diversität liegen
weit jenseits dessen, was genetische Evolution zu leisten in der Lage
wäre. Erklärungsversuche dafür von \emph{Evolutionary psychology} und
\emph{Human behavioural ecology} empfindet Dawkins als unzureichend.
Stattdessen abstrahiert er die von ihm postulierte Evolutionstheorie und
führt den Begriff des \emph{Replikators} ein. Wenn irgendeine Form von
Replikator vorhanden sei, dann würde zwangsläufig Evolution stattfinden.
Gene seien Replikatoren -- Ideen, Gedanken, \emph{Meme} aber ebenso.
Glaubt man einer Fußnote in Dawkins später kommentiertem Text, so war
die Aussage, dass das Gen nicht die einzige mögliche Form eines
Replikators ist, bereits die wesentliche in Kapitel 11. Umso
erstaunlicher, dass er den Moment der Schöpfung seines Neologismus Meme
dennoch theatralisch zelebriert:

\begin{quote}
I think that a new kind of replicator has recently emerged on this very
planet. It ist staring us in the face. It is still in its infancy, still
drifting clumsily about in its primeval soup, but already is it
achieving evolutionary change at a rate that leaves the old gene panting
far behind. The new soup is the soup of human culture. We need a name
for the new replicator, a noun that conveys the idea of a unit of
cultural transmission, or a unit of \emph{imitation}. `Mimeme' comes
from a suitable Greek root, but I want a monosyllable that sounds a bit
like `gene'. I hope my classicist friends will forgive me if I
abbreviate mimeme to meme. {[}\ldots{}{]} It should be pronounced to
rhyme with `cream'.

-- \textcite{Dawkinsselfishgene40th2016}, 291.
\end{quote}

Meme seien kleine abgrenzbare Informationseinheiten wie Melodien,
Geflügelte Worte, Kleidungsmoden oder das Wissen um spezifische
technische Prozesse. So wie Gene Lebewesen als Vehikel gebrauchen, so
wären menschliche Gehirne das Medium, in denen sich Gene ausbreiten. Die
Informationsweitergabe ist nicht auf sexuelle oder asexuelle
Fortpflanzung beschränkt, sondern funktioniert über eine Form der
zwischenmenschlichen Kommunikation, die Dawkins unter dem Überbegriff
Imitation zusammenfasst. Er lässt -- zwar mit einiger Zurückhaltung --
durchscheinen, dass er eine physische Existenz von Memen etwa als
Strukturen verschaltener Nervenzellen annimmt. Unabhängig davon sei ihr
Effekt deutlich zu spüren: Entitäten, die unser Denken parasitisch
bewohnen und ihre eigene Ausbreitung bezwecken würden. Dawkins bemüht
für eine erste Illustration das Beispiel des monotheistischen Glaubens
an einen Gott\footnote{Religionskritik ist ein wiederkehrendes Thema in
  Dawkins umfangreichem, populärwissenschaftlichem Werk. Siehe z.B. das
  umstrittene \emph{The God delusion} -- \textcite{dawkins_god_2006}}:

\begin{quote}
Consider the idea of God. {[}..{]} How does it replicate itself? By the
spoken and written word, aided by great music and great art.
{[}\ldots{}{]} What is it about the idea of a god that gives it its
stability and penetrance in the cultural environment? The survival value
of the god meme in the meme pool results from its great psychological
appeal. It provides a superficially plausible answer to deep and
troubling questions about existence. It suggests that injustices in this
world may be rectified in the next. The `everlasting arms' hold out a
cushion against our own inadequacies which, like a doctors placebo, is
none the less effective for being imaginary. These are some of the
reasons why the idea of God is copied so readily by successive
generations of individual brains.

-- \textcite{Dawkinsselfishgene40th2016}, 292.
\end{quote}

Warum ist das menschliche Gehirn empfänglich für Meme? Gibt es einen
klassisch evolutionären Vorteil von dieser Empfänglichkeit? Nach Dawkins
ist die grundsätzliche Kulturfähigkeit des Menschen durchaus ein Effekt
genetischer Mutation und Selektion. Ab einem gewissen Punkt -- in
fließendem Übergang -- sei allerdings der Replikator Meme im Kulturraum
entstanden und hätte die Zügel in die Hand genommen.

\begin{quote}
Whenever conditions arise in which a new kind of replicator \emph{can}
make copies of itself, the new replicators \emph{will} tend to take
over, and start a new kind of evolution of their own. Once this new
evolution begins, it will in no necessary sense be subservient to the
old.

-- \textcite{Dawkinsselfishgene40th2016}, 293.
\end{quote}

Die genetische Evolution habe also den Nährboden bzw. das Medium einer
neuen, viel schnelleren Form der Evolution geschaffen, die andere
Prioritäten für Gesundheit, Langlebigkeit und Fortpflanzungsfähigkeit
ihrer Trägerorganismen anlegt. In vielen Fällen sind diese Prioritäten
ähnlich. Ein Beispiel dafür sind Meme, die positv konnotiert mit Sex
umgehen. Andererseits gibt es auch Meme wie etwa das Zölibat
katholischer Ordensträger, die aus einer Genperspektive nicht sinnvoll
sein können, da sie die Verbreitung der Gene ihrer Träger effektiv
hemmen.

Wenn nun also auch im Medium Kultur die Mechanismen der Evolution
wirken, dann müssten sich die Replikatoren Meme dem selben Druck beugen
wie die Gene in der natürlichen Umwelt. Überleben könnten nur
Replikatorenvarianten mit einer hohen Qualität der Eigenschaften
\emph{longevity}, \emph{fecundity} und
\emph{copyying-fidelity}\footnote{\textcite{Dawkinsselfishgene40th2016},
  47-49.}.

\emph{longevity} -- Langlebigkeit -- sei eine günstige Eigenschaft für
einen Replikatortyp, da er seinen Gesamtbestand im Medium so einerseits
leicht hoch halten kann und ihm außerdem mehr Zeit für Reproduktion zur
Verfügung steht. Einzelne Kopien von Genen sind in ihrer Lebenszeit an
den Organismus gebunden, dessen Aufbau sie kodieren. Instanzen eines
Memes seien dagegen von der menschlichen Gedächtnisleistung abhängig.
Meme könnten aber auch außerhalb von Menschen überdauern, wenn sie etwa
in geschriebener oder digitaler Form abgelegt wurden. Damit könnte das
Meme etwa später wieder einen Menschen infizieren, obgleich kein
direkter Kontakt mit einem Infizierten stattgefunden hat.

\emph{fecundity} -- Fruchtbarkeit -- sei für die Durchsetzungsfähigkeit
eines Replikatortyps noch wichtiger als \emph{longevity}: Um so mehr
Kopien er in kürzerer Zeit von sich selbst anfertigen kann, desto
schneller wird er das Medium dominieren. Die Reproduzierfähigkeit eines
Memes sollte von verschiedenen Eigenschaften abhängen, allem voran
schlicht seiner Beliebtheit in oder außerhalb einer assoziierten
Adressatengruppe.

\emph{copyying-fidelity} -- Kopiertreue -- scheint hier zunächst
deplaziert. Ein gewisser Grad an Mutationsfähigkeit ist unerlässlich für
Anpassung. Tritt allerdings bei einem Replikatortyp eine zu große
Instabilität auf, so argumentiert Dawkins, könnte er seine Identität
nicht aufrechterhalten und würde entweder schnell von Varianten
abgelöst, die aus ihm selbst hervorgegangen sind, oder sich völlig
auflösen. Bei Memen scheint gerade das häufig zu passieren:
Übertragungsfehler oder bewusste Modifikation scheinen die Regel, nicht
die Ausnahme zu sein. Damit muss die Qualität von Memen als Replikatoren
in Frage gestellt werden. Dawkins gibt das zu -- diese Frage nach der
Kopiertreue führt ihn zurück zur Definition von Memen. Welche
Information enthält ein individuelles Meme bzw. -- in einem
Analogieschluss -- das Gen?

Auch das Gen ist keine in mikrobiologischen Begriffen eindeutig
definierte Entität\footnote{\textbf{Zitat!}}. Dawkins bezeichnet damit
einen DNA-Abschnitt mit hinreichender Wirkung und Kopiertreue, um als
selektionsrelevante Einheit zu wirken\footnote{\textbf{Zitat! Kapitel 3}}.
Gene schließen sich auf verschiedenen hierarchischen Ebenen zu Komplexen
zusammen, die als Gruppe gegebenfalls eine Gesamtwirkung entfalten und
wiederum als ganzes Selektionsrelevant wirken kann\footnote{\textbf{Zitat!
  Kapitel 3 \& 5}}. Ein ähnliches Strukturverhalten könnte auch für Meme
angenommen werden. Eine Symphonie setzt sich beispielsweise aus einer
Vielzahl einzelner, für sich wiedererkennbarer Melodieabschnitte und
Figuren zusammen. Eine Religion ist die Gesamtheit vieler verknüpfter
Ideen und Ritualen, die als ganzes tradiert werden, eine Konfession
möglicherweise ein \emph{stable set of mutually-assisting
memes}\footnote{\textcite{Dawkinsselfishgene40th2016}, 299.}.

\begin{quote}
I conjecture that co-adapted meme-complexes evolve in the same kind of
way as co-adapted gene-complexes. Selection favours memes that exploit
their cultural environment to their own advantage. This cultural
environment consists of other memes which are also being selected. The
meme pool therefore comes to have the attributes of an evolutionarily
stable set, which new memes find it hard to invade.

-- \textcite{Dawkinsselfishgene40th2016}, 301.
\end{quote}

Komplexe zusammenhängener Meme wurden später von Dawkins Schülern mit
dem Begriff \emph{Memeplex} belegt (s.u.).

Wie oben ausgeführt, versetzt sich Dawkins in die Perspektive der Gene
hinein und personifiziert sie. Eine empirisch naheliegende und
terminoligisch praktische Metapher um ihre effektive Entwicklung zu
beschreiben. Diese Übertragung möchte er auch für Meme vornehmen. Meme
stünden in starker Konkurrenz zueinander um die Zeit, die Menschen ihnen
widmen und sie gegebenenfalls replizieren: Meme möchten so viele
menschliche Gehirne wie möglich so lange wie möglich dominieren.

\begin{quote}
Time is possibly a more important limiting factor than storage space,
and it is the subject of heavy competition. The human brain, and the
body that it controls, cannot do more than one or a few things at once.
If a meme is to dominate the attention of a human brain, it must do so
at the expense of `rival' memes.

-- \textcite{Dawkinsselfishgene40th2016}, 298.
\end{quote}

Aus dieser Perspektive könnte, so Dawkins, etwa das oben angesprochene
Zölibat-Meme verstanden werden, dass im Memeplex katholischer
Glaubenspraxis Priester freisetzt, keine Zeit an einer Familie zu
verlieren, sondern sich voll auf die Pflege und Verbreitung anderer Meme
der Kirchendoktrin zu konzentrieren. Die Prioritäten von Menschen, Genen
und Memen müssen sich unterscheiden.

\begin{quote}
What we have not previously considered, is that a cultural trait may
have evolved in the way that it has, simply because it is
\emph{advantageous to itself}.

-- \textcite{Dawkinsselfishgene40th2016}, 302.
\end{quote}

Das wirft die philosophische Frage auf, inwiefern Menschen Sklaven ihrer
Gene und Meme sind. Dawkins gibt dazu zu bedenken, dass weder Gene noch
Meme im Gegensatz zum Menschen über Bewusstsein oder Planungsfähigkeit
verfügen. Gene und Meme seien \emph{unconscious, blind,
replicators}\footnote{\textcite{Dawkinsselfishgene40th2016}, 302.}.
Damit könnte sich der Mensch seine Situation bewusst machen, sich
zumindest teilweise den auf ihn wirkenden Entitäten entziehen und neue
Meme schaffen, die seinen Zielen besser dienen: zum Beispiel solche, die
langfristige Kooperation stabilisieren und den immanenten Egoismus von
Genen und Memen ächten.

\hypertarget{dawkins-schuler}{%
\subsection{Dawkins Schüler}\label{dawkins-schuler}}

Memetik selbst hat sich, in Dawkins Terminologie, als außerordentlich
potentes Meme erwiesen. Die Grundidee ist in den Strömungen der
\emph{Dual Inheritance theory} aufgegangen, aber auch die Memetik an
sich hat sich über dieses intitiale Kapitel hinaus weiterentwickelt.
Maßgeblichen Anteil daran hatten unter anderem der Philosoph Daniel
Dennett\footnote{siehe u.a. \textcite{dennett_brainstorms_1978},
  \textcite{dennett_elbow_1984}, \textcite{dennett_consciousness_1991},
  \textcite{dennett_darwins_1995}.}, die Psychologin Susan Blackmoore
und all jene Natur- und Geisteswissenschaftler, die sich im mittlerweile
eingestellten Journal of Memetics\footnote{\url{http://cfpm.org/jom-emit/}
  {[}06.01.2018{]}} zu Wort gemeldet haben. Susan Blackmore bezieht sich
unmittelbar auf Dawkins Ausgangsidee und erweitert sie um einige
Aspekte. Die besondere Qualität ihres Buches \emph{The meme
machine}\footnote{\textcite{blackmore_meme_1999}. Ich werde im folgenden
  aus einer mir vorliegenden, deutschen Ausgabe zitieren:
  \textcite{blackmore_macht_2000}.} liegt in der Synthese vieler
Diskurse und Spannungslinien, die sich rund um die Memetik bis in die
90er Jahre herauskristallisiert hatten. Dennoch lässt sich ihre
Perspektive auf eine einfache Formel reduzieren, die sie in
verschiedenen Kontexten immer wieder angepasst anwendet: Sobald Meme
existieren übernehmen sie die Rolle des dominanten Replikators, der das
Verhalten seiner Träger wesentlich und langfristig beeinflusst.

Sprache

\begin{quote}
Als sich die Imitationsfähigkeit erst einmal entwickelt hatte und Meme
auftauchten, haben diese Meme die Umwelt verändert, in der die Gene
selektiert wurden und zwangen sie so, immmer bessere memverbreitende
Apparate zu schaffen. Mit anderen Worten ist die menschliche
Sprachfähigkeit memgetrieben, und die Funktion der Sprache besteht
darin, Meme zu verbreiten.

-- \textcite{blackmore_macht_2000}, 159.
\end{quote}

Sexuelle Selektion

\begin{quote}
Der Memetik {[}\ldots{}{]} zufolge wird die Partnerwahl nicht nur vom
genetischen, sondern auch vom memetischen Vorteil beeinflusst. Eine
meiner Schlüsselannahmen ist, dass die natürliche Selektion nach
Entstehung der ersten Meme begann, Menschen zu favorisieren, die sich
für eine Paarung mit den besten Imitatoren oder den besten Benutzern und
Verbreitern von Memen entschieden.

-- \textcite{blackmore_macht_2000}, 213.
\end{quote}

Gehirngröße

Altruismus

\begin{quote}
Wenn Leute altruistisch sind, werden sie beliebt, weil sie beliebt sind,
werden sie kopiert, und weil sie kopiert werden, breiten sich ihre Meme
-- \emph{einschließlich der Altruismusmeme selbst} -- weiter aus als die
Meme weniger altruistischer Leute. Das liefert einen Mechanismus für die
Ausbreitung altruistischen Verhaltens.

-- \textcite{blackmore_macht_2000}, 252.
\end{quote}

Zunächst verknüpft sie den Mechanismus der Meme-Übertragung mit
Imitation

Kapitel 3

Kapitel 5

Journal of Memetics

The memes' eye view

\hypertarget{kritik-und-abgesang}{%
\subsection{Kritik und Abgesang?}\label{kritik-und-abgesang}}

\url{http://jom-emit.cfpm.org/2002/vol6/edmonds_b_letter.html}
\url{http://jom-emit.cfpm.org/2005/vol9/edmonds_b.html}

\hypertarget{themen-und-konflikte-der-cultural-evolution-forschung}{%
\section{Themen und Konflikte der Cultural Evolution
Forschung}\label{themen-und-konflikte-der-cultural-evolution-forschung}}

Cultural Evolution ist heute eine wichtige theoretische Strömung der
anthropologischen Forschung. Die oben unterschiedenen Perspektiven
Evolutionary psychology, Human behavioural ecology und Dual inheritance
theory sind Grundlage für abstrakte Modelle, Fallstudien und
theoretische Weiterentwicklung. Besonders hervorgetan haben sich hier in
den vergangenen 30 Jahren neben Cavalli-Sforza und Feldmann auch Robert
Chester Dunnell (*1942 - †2010), Peter James Richerson (*1943), Robert
Boyd (*1948), Stephen Shennan (*1949), Michael John O'Brien (*1950),
Patrice A. Teltser (*1954), Ben Sandford Cullen (*1964 - †1995) und eine
Vielzahl jüngerer Kollegen wie Joseph Henrich, Oren Kolodny, Ken Aoki
oder Alex Mesoudi. Seit 2015 konstituiert sich eine Cultural Evolution
Society als interdisziplinäre Wissenschaftsvereinigung\footnote{\url{https://culturalevolutionsociety.org}
  {[}01.02.2018{]}}.

\textcite{creanza_cultural_2017} geben einen guten Überblick über
aktuelle Fragestellungen der Cultural Evolution Forschung. Ausgehend von
dieser Themensammlung werde ich einige wesentliche Zusammenhänge
nachvollziehen um darauf aufbauend im folgenden Kapitel
Bestattungssitten im Licht der Cultural Evolution Theorie zu
diskutieren. Ein wichtiger Themenkomplex der Cultural Evolution
Forschung, Kultur und Kulturentwicklung in Nicht-menschlichen
Spezies\footnote{\textcite{laland_question_2009}}, soll hier aufgrund
seiner geringen Relevanz in diesem Kontext ignoriert werden. Ebenso die
Diskussion zur Evolution von Sprache in der Linguistik\footnote{\textcite{nowak_evolution_1999}}
und eine Vielzahl von Ansätzen, moderne gesellschaftlichen
Problemstellungen wie Klimawandel\footnote{\textcite{seneviratne_allowable_2016}},
Industrielle Landwirtschaft\footnote{\textcite{garibaldi_farming_2017}}
und Multiresistente Keime\footnote{\textcite{boni_evolution_2005}} aus
einer Cultural Evolution Perspektive zu analysieren. Stattdessen wird
dem Themenfeld Cultural Transmission und seiner Bedeutung für
archäologische Modellbildung in einem eigenen Kapitel viel Raum gegeben.

\hypertarget{unterschiede-und-gemeinsamkeiten-von-kulturentwicklung-und-genetik}{%
\subsection{Unterschiede und Gemeinsamkeiten von Kulturentwicklung und
Genetik}\label{unterschiede-und-gemeinsamkeiten-von-kulturentwicklung-und-genetik}}

Eine der Grundannahmen der Cultural Evolution Theorie ist die
Ähnlichkeit zwischen biologischer Evolution und kultureller Entwicklung.
Das schließt die Übertragung biologischer Konzepte wie Mutation,
Selektion, Transmission und Drift explizit ein\footnote{\textcite{smith_cultural_1992}}.
Das Methodenset der Populationsgenetik kann damit auf Kulturprozesse
übertragen werden. Cavalli-Sforza und Feldmann\footnote{\textcite{cavalli-sforza_cultural_1981}},
Robert Boyd und Peter Richerson\footnote{\textcite{boyd_culture_1985}}
und andere\footnote{\textcite{lumsden_genes_1981};
  \textcite{pulliam_programmed_1980}} legten dafür in den 1980ern
konkrete Ausarbeitungen oft mathematisch formulierter Modelle vor.
Dennoch bestehen klare Unterschiede zwischen biologischer
Populationsgenetik und der Entwicklung und Transmission von Ideen.
Cultural Evolution folgt nicht den Mendelschen Regeln zu Uniformität,
Spaltung und Unabhängigkeit\footnote{\textcite{mesoudi_pursuing_2017}}
und große Teile der Terminologie (z.B. Genotyp vs.~Phänotyp, Homozygotie
vs.~Heterozygotie) sind nicht oder nur unter großen
Bedeutungsverschiebungen anwendbar. Horizontale Transmission spielt in
der biologischen Vererbung eine untergeordnete Rolle und die Übertragung
erfordert große Anpassungen an den vor allem vertikalen, genetischen
Ausgangsmodellen\footnote{\textcite{cavalli-sforza_cultural_1973};
  \textcite{feldman_cultural_1976}}.

Im Gegensatz zur DNS der Genetik, ist die Identität der
Informationsträger kultureller Entwicklung unbekannt \textbf{hier ggf.
Diskussion dazu}. Bestimmte kulturelle Eigenschaften lassen sich binär
oder diskret kategorisieren, andere eher quantitativ bzw. proportional
beschreiben. Zu ersteren gehören beispielsweise das technologische
Wissen um Herstellung und Verwendung eines bestimmten Werkzeugs oder die
Verwendung eines bestimmten Ritzmusters zur Keramikverzierung. Auch die
in der vorliegenden Arbeit vorgenommene Untersuchung von
Bestattungssitten reduziert diese auf die binäre Komponente der Ab- und
Anwesenheit eines bestimmten Aspekts des Rituals. Analysen auf
metrischem Skalenniveau wurden etwa zur Abbildung von
Risikobereitschaft\footnote{\textcite{bisin_economics_2001-1}} in
Gruppen oder einem Kompentenzniveau\footnote{\textcite{baldini_revisiting_2015};
  \textcite{henrich_demography_2004};
  \textcite{kobayashi_innovativeness_2012}} im Umgang mit einem
bestimmten Werkzeug zur Anwendung gebracht.

\hypertarget{menschliches-verhalten-genetische-determination-vs.kulturelles-lernen}{%
\subsection{Menschliches Verhalten: Genetische Determination
vs.~Kulturelles
Lernen}\label{menschliches-verhalten-genetische-determination-vs.kulturelles-lernen}}

Der von \autocite{smith_three_2000} (s.o.) beobachtete Riss durch die
Forschungslandschaft zwischen Evolutionary Psychology, Human behavioral
ecology und Dual inheritance theory wird besonders an der Frage
deutlich, welche Aspekte menschlichen Verhaltens genetisch determiniert
und welche kulturell konstruiert sind. Unter der Annahme, dass die
Transmission von Ideen Menschen eine viel höhere Anpassungsfähigkeit an
widrige Subsistenzumstände ermöglicht, zeigen entsprechend konzipierte
Modelle, dass genetisch transportiertes Verhalten nur in ökologisch sehr
stabilen Umgebungen Relevanz entwickeln kann\footnote{\textcite{aoki_emergence_2005};
  \textcite{aoki_evolution_2014}; \textcite{boyd_cultural_1983}}. Aus
dieser Perspektive ergibt sich das klare Primat kultureller Transmission
für den Menschen, der sich dank seiner Kulturfähigkeit in fast alle auf
der Erde vertretenen Biome hat ausbreiten können.

Auch bei einer Dominanz sozialen Lernens und kultureller Transmission
für die Prägung menschlichen Verhaltens ist der genetische Anteil nicht
zu vernachlässigen -- schon allein aufgrund der häufig zu beobachtenden
Korrelation zwischen einem Verhaltensmuster und biologischer
Verwandtschaft. Diese Übereinstimmung ergibt sich aus vertikalen
Transmissionstrukturen, die biologisch und kulturell oft parallel
verlaufen. Genauso muss die natürliche Umwelt als wesentlicher Faktor
bei der Determination menschlichen Verhaltens in Betracht gezogen
werden. Die genaue Charakterisierung des Einflusses von Genen, Kultur
und Umwelt ist unter den Stichworten Gene-Culture coevolution, Dual
Inheritance theory und Cultural Niche construction intensiv diskutiert
worden\footnote{\textcite{aoki_gene-culture_2017};
  \textcite{boyd_culture_1985}; \textcite{cavalli-sforza_cultural_1981};
  \textcite{chudek_culturegene_2011}; \textcite{feldman_aspects_1979};
  \textcite{mesoudi_towards_2006}; \textcite{richerson_dual_1978}}.

Die Methode der Genomweiten Assoziationsstudie (GWAS, Genome-wide
association study) erlaubt es heute, Menschen und ihr Verhalten mit
zunehmender Präzision auf Korrelation mit der Anwesenheit bestimmter
Genen zu untersuchen. Dadurch wird die Suche nach genetischer Anpassung
etwa an die naturräumliche Rahmensituation erleichtert\footnote{\textcite{berg_population_2014}}.
Die Untersuchung von Verhaltensmerkmalen wie dem IQ oder dem erreichten
Ausbildungsniveau mit diesem Werkzeug\footnote{\textcite{benyamin_childhood_2014};
  \textcite{davies_genome-wide_2011}; \textcite{minkov_genetic_2015};
  \textcite{okbay_genome-wide_2016}} ist jedoch mit Risiken verbunden.
Einerseits eröffnen die Erkenntnisse über solche Zusammenhänge ethische
Implikationen, andererseits ist eine statistische Ergebnissicherheit
nicht gewährleistet: Korrelation von Genen und Verhalten muss nicht
Konsequenz einer kausalen Beziehung sein. Stattdessen könnte sie nur
Nebeneffekt von z.B. räumlicher und sozialer Autokorrelation oder
assortativer Paarung sein\footnote{\textcite{abdellaoui_educational_2015};
  \textcite{domingue_genetic_2014}; \textcite{okbay_genome-wide_2016};
  \textcite{piffer_review_2015}}. Moderne Fallbeispiele, für die
komplexe, sozioökonomische Erklärungen angenommen werden müssen obgleich
auch genetische Korrelation besteht, beschäftigen sich unter anderem mit
Tabakkonsum, Armut, Gesundheit oder Rassismus\footnote{\textcite{maes_genetic_2006};
  \textcite{marden_african_2016};
  \textcite{nugent_geneenvironment_2011};
  \textcite{paradies_racism_2015}}.

\hypertarget{mensch-umwelt-interaktion-und-cultural-niche-construction}{%
\subsection{Mensch-Umwelt Interaktion und Cultural Niche
construction}\label{mensch-umwelt-interaktion-und-cultural-niche-construction}}

Cultural Niche construction hält ein potentes Erklärungsmodell bereit,
um den wechselseitigen Selektionsdruck nachzuvollziehen, den Kultur,
Gene und Umwelt aufeinander ausüben\footnote{\textcite{laland_niche_2000};
  \textcite{odling-smee_niche_2003}; \textcite{laland_cultural_2011};
  \textcite{rendell_runaway_2011}}. Dabei beschreibt Niche construction
in der Biologie Veränderungen der natürlichen Umwelt, die einerseits von
einer Spezies selbst hervorgerufen werden und gleichermaßen die
Selektionsdrücke auf diese Spezies beeinflussen\footnote{\textcite{laland_niche_2006}}.
Für den Menschen ergibt sich daraus ein komplexes Geflecht von
Interdependenzen zwischen Kulturverhalten, genetischer Disposition und
Natur, die die schrittweise Modifikation all dieser Systembestandteile
zur Folge hat\footnote{\textcite{alberti_global_2017};
  \textcite{creanza_models_2012}; \textcite{laland_cultural_2001}}. Viel
beachtete Fallbeispiele dieser Interaktion sind unter anderem
anthropogen induzierte Aussterbeereignisse von Megafauna\footnote{\textcite{barnosky_assessing_2004}},
Feuernutzung für Landschaftseingriffe\footnote{\textcite{bird_fire_2008}},
die Ausbreitung der Links- und Rechtshändigkeit\footnote{\textcite{laland_gene-culture_1995}},
die Entstehung der Laktose-Toleranz\footnote{\textcite{feldman_theory_1989};
  \textcite{ingram_population_2012}} und die rückläufige, demographische
Entwicklung in modernen, westlichen Gesellschaften\footnote{\textcite{borgerhoff_mulder_demographic_1998};
  \textcite{fogarty_role_2013}; \textcite{ihara_cultural_2004}}.

Subsistenzbezogenes Verhalten ist unmittelbar selektionsrelevant, da es
die Sterbe- und Reproduktionswahrscheinlichkeit einer Population
beeinflusst. Der Mensch hat seine Versorgung über den größten Teil
seiner Existenz aus Jagen und Sammeln bestritten. Dabei war er von den
Ressourcen einer natürlichen Umwelt abhängig und hat sie durch
Güterentnahme destabilisiert. Etliche Modelle im Kontext der Human
behavioral ecology dokumentieren, wie diese Wechselwirkung zum
Katalysator von Veränderung im Mensch-Umwelt System wurde\footnote{\textcite{hardy_climatic_2010};
  \textcite{hockett_nutritional_2005}; \textcite{stiner_thirty_2001}}.
Auch die Neolithisierung könnte durch einen solchen Prozess verstanden
werden\footnote{\textcite{rowley-conwy_foraging_2011};
  \textcite{smith_onset_2013}}.

\hypertarget{mikroorganismen-und-pathogene}{%
\subsection{Mikroorganismen und
Pathogene}\label{mikroorganismen-und-pathogene}}

Krankheiten sind ein wesentlicher Selektionsfaktor für den Menschen und
hatten großen Einfluss sowohl auf seine biologische\footnote{\textcite{bustamante_natural_2005};
  \textcite{enard_viruses_2016}; mead\_balancing\_2003;
  \textcite{sabeti_genome-wide_2007}@} als auch auf seine
prähistorisch-kulturelle\footnote{\textcite{martin_health_2002};
  \textcite{oxenham_skeletal_2005}} und historische\footnote{\textcite{alfani_plague_2013};
  \textcite{murray_estimation_2006}} Entwicklung. Malaria hat
beispielsweise wesentliche Veränderungen im menschlichen Erbgut
durchgesetzt\footnote{\textcite{kwiatkowski_how_2005};
  \textcite{tishkoff_haplotype_2001}} -- unter anderem die weitreichende
Verbreitung der Sichelzellenanämie\footnote{\textcite{allison_protection_1954}}.
Die Interaktion des Menschen mit Krankheiten lässt sich nicht auf eine
rein biologische Perspektive reduzieren. Stattdessen sind Krankheiten
und ihre Verbreitung stark durch Kulturverhalten bedingt. Nassfeldanbau
in Westafrika könnte die initiale Verbreitung von Malaria massiv
begünstigt haben\footnote{\textcite{durham_coevolution_1991-1}},
Krankheiten waren ein wesentlicher Bestandteil des Kulturpakets, mit dem
sich die Nordamerikanischen Ureinwohner in Folge von Kolumbus Landung
1492 konfrontiert sahen\footnote{\textcite{nunn_columbian_2010}} und die
Kuru Krankheit, die bis in die 1940er im Hochland von Neuguinea immer
wieder in Epidemien ausbrach, war in ihrer Übertragung abhängig von
kannibalistischen Ritualen\footnote{\textcite{lindenbaum_kuru_2015}}.

Neben Pathogenen ist der Mensch auch Wirt für weniger parasitäre
Mikroorganismen. Die Gesamtheit von Lebensformen, die in und auf dem
menschlichen Körper leben ohne Krankheiten oder Entzündungen
hervorzurufen -- die Normalflora -- hat durchaus Rückwirkung nicht nur
auf den menschlichen Organismus, sondern auch auf dessen Verhalten und
Verhaltensspielraum. Menschen können die Fähigkeit zur
Laktoseverarbeitung beispielsweise nicht nur über eine Mutation des
eigenen Erbguts erlangen, sondern auch indirekt über Bakterien im
Verdauungstrakt. Solche Bakterien haben möglicherweise ebenfalls eine
wichtige Rolle bei der Entstehung der Milchwirtschaft in der
Vorgeschichte gespielt\footnote{\textcite{walter_human_2011}}.

\hypertarget{entstehung-und-wirkung-von-innovationen-cultural-complexity-und-demographie}{%
\subsection{Entstehung und Wirkung von Innovationen: Cultural Complexity
und
Demographie}\label{entstehung-und-wirkung-von-innovationen-cultural-complexity-und-demographie}}

In der biologischen Evolution entstehen neue Varianten durch Mutationen
im Erbgut von Individuuen. Cultural Evolution kennt dagegen eine ganze
Reihe von Prozessen, die zur Entstehung von Innovationen verschiedener
Größenordnungen führen können. Viele Modelle reduzieren diese Prozesse
auf simple Zufallsereignisse oder die Interaktion eines Individuums mit
seiner Umwelt\footnote{\textcite{henrich_evolution_2003};
  \textcite{rendell_why_2010}}. Andere bringen komplexere Mechanismen
ins Spiel, wie die Verknüpfung bestehender Innovationen zu
neuen\footnote{\textcite{enquist_why_2008}} und die Interaktion vieler
Innovationen in einer schnellen, aufeinander aufbauenden Kettenreaktion
von Kombination und Ableitung\footnote{\textcite{fogarty_cultural_2015};
  \textcite{kolodny_evolution_2015};
  \textcite{kolodny_game-changing_2016}}: Eine einzige Idee zieht
möglicherweise viele andere nach sich. In der prähistorischen und
historischen Menschheitsentwicklung gibt es viele Ereignisse, die solche
Effekte nahelegen, etwa die explosionsartige Zunahme an Komplexität im
Steingerätinventar am Übergang von Mittel- zu
Jungpaläolithikum\footnote{\textcite{bar-yosef_nature_1998};
  \textcite{roebroeks_time_2008}} oder die neolithische Revolution im
Vorderen Orient\footnote{\textcite{gopher_when_2001};
  \textcite{veen_agricultural_2010}}.

Die Veränderung der Menge und Art kultureller Eigenschaften einer Gruppe
ist mit dem Begriff der Cultural Complexity Forschung verknüpft. Sie
untersucht die Akkumulation und den Verlust von Innovationen (Cultural
accumulation und Cultural decay) im Abgleich zu systemtheoretischen
Gleichgewichtszuständen. Dabei zeigt sich, dass die
Innovationsverfügbarkeit in einer Population durch die Verschränkung der
verschiedenen Ideen starken Schwankungen unterworfen ist, bis sie einen
stabilen Zustand erreicht\autocite{kolodny_evolution_2015}. Innovationen
können selbst Rückwirkungen auf die Systemdynamik ihrer Wirtpopulationen
nehmen, indem sie zum Beispiel die Subsistenzbedingungen verändern und
Bevölkerungswachstum oder -niedergang katalysieren\footnote{\textcite{kolodny_game-changing_2016}}.
\textcite{crema_revealing_2016} eröffnen mit einer Fallstudie an
neolithischer Keramik die Perspektive dafür, dass die Annahme von
Gleichgewichtszustände in archäologischen Kontexten vor diesem
Hintergrund hinterfragt werden muss.

Populationsgröße und Subsistenzrisiko

\textcite{henrich_demography_2004}, \textcite{collard_what_2011},
\textcite{KobayashiInnovativenesspopulationsize2012},
\textcite{collard_population_2013},
\textcite{BaldiniRevisitingEffectPopulation2015},
\textcite{henrich_understanding_2016}, \textcite{vaesen_population_2016}

150, 151, 152

Unabhängig davon welcher Effekt letztlich die größere Wirkung auf
kulturelle Komplexität entfaltet, gibt es eindeutig einen Zusammenhang
zwischen Kulturverhalten und demographischer Entwicklung einer
Population: Der Übergang von einer Jäger- und Sammlerischen Lebensweise
zu Ackerbau und Viehzucht am Beginn des Holozän geht mit einem starkem
Bevölkerungswachstum einher, das unter dem Stichwort der \emph{Neolithic
demographic transition} als eines der folgenreichsten Auswirkungen der
Neolithisierung diskutiert wird\footnote{\textcite{bocquetappel_paleoanthropological_2002};
  \textcite{gage_what_2009}}. Neben Subsistenzpraktiken beeinflussen
eine Vielzahl von Faktoren wie religiöse Normen, Heiratsgepflogenheiten
oder gewaltsame Konflikte die Altersstruktur und das Wachstum einer
Gesellschaft. Etliche davon reduzieren die Geburtenrate\footnote{\textcite{smith_cultural_1992};
  \textcite{colleran_cultural_2016}; \textcite{richerson_natural_1984}}
und wirken so stabilisierend auf das Mensch-Umwelt-System. Ein Phänomen
dieser Art lässt sich im modernen China und in Teilen Indiens
beobachten: Eine kulturelle Präferenz für männliche Nachkommen, die sich
etwa durch selektive Abtreibung manifestiert, führt lokal zu einem
assymetrischen Überschuss von bis zu 6:5 von Männern gegenüber Frauen.
Diese kulturell induzierte demographische Veränderung hat
erwartungsgemäß schwerwiegende ökonomische Konsequenzen\footnote{\textcite{banister_shortage_2004};
  \textcite{li_cultural_2000}; \textcite{tuljapurkar_high_1995}}.

\hypertarget{altruismus}{%
\subsection{Altruismus}\label{altruismus}}

\textcite{smith_cultural_1992}, 83ff.

\hypertarget{cultural-transmission}{%
\section{Cultural transmission}\label{cultural-transmission}}

Die Ausbreitung von Ideen geschieht im sozialen Raum ihrer Träger. Die
Formen der Kommunikation, die dabei zur Anwendung kommen sind vielfältig
und entwickeln auf unterschiedlichen Skalenniveaus unterschiedliche
Relevanz. Grundsätzlich bewegt sich Information mit ihren Trägern, das
heißt alle Prozesse, die zur Bewegung von Menschen im Raum führen, sind
auch Prozesse, die zur Ausbreitung von Information führen. Zu Cultural
Transmission müssen also alle Modi des Austauschs von der
Massenmigration, über den Frauentausch in Heiratsnetzwerken bis hin zum
einzeln wandernden Händler und Handwerker gezählt werden. Daneben stehen
Prozesse innerhalb kohärenter Gruppen, wie die Kindererziehung (vertical
transmission), Lehre und Ausbildung von einer Generation zur nächsten
(oblique transmission) und der einfache Austausch von Information
zwischen allen Mitgliedern einer Population (horizontal transmission),
wie er durch Sprache, Schrift und Imitation permanent stattfindet.

Einige der weitreichendsten Transformationsereignisse in der Geschichte
der Menschheit, die zu einem tiefgreifenden Wandel der vorhandenen
kulturellen Eigenschaften geführt haben, sind von Populationsbewegungen
zumindest begleitet, wenn nicht sogar initiiert worden\footnote{\textcite{boyd_voting_2009}}.
Der Neanderthaler wurde vor ca. 40.000 Jahren vollständig vom Modernen
Menschen verdrängt\footnote{\textcite{skoglund_origins_2012}}, und mit
ihm ging eine erste -- freilich in ihrer Dynamik umstrittene -- Phase
kultureller Modernität zu Ende, die sich erst durch jüngste
Forschungsergebnisse zu erschließen beginnt\footnote{\textcite{hoffmann_symbolic_2018};
  \textcite{tuniz_did_2012}}. Paläogenetische Ergebnisse legen nahe,
dass die neolithische Revolution in Europa im wesentlichen von
wandernden Siedlern aus dem Vorderen Orient getragen wurde, nicht von
der Übernahme eines Innovationspakets durch lokale Jäger- und
Sammlergruppen\footnote{\textcite{aoki_travelling_1996};
  \textcite{bar-yosef_nature_1998}; \textcite{patterson_modelling_2010};
  \textcite{skoglund_origins_2012}}. Im fortgeschrittenen Neolithikum
bis zum Beginn der Bronzezeit vollzog sich eine weitere genetische und
kulturelle Transformation in Mitteleuropa infolge der Einwanderung
berittener Steppenbewohner aus dem Yamnaya Kulturkomplex\footnote{\textcite{allentoft_population_2015};
  \textcite{goldberg_ancient_2017}}.

Jenseits von Populationsbewegungen ist Cultural Transmission abhängig
vom Austausch zwischen Menschen: \emph{Social Learning}.

\hypertarget{social-learning}{%
\subsection{Social Learning}\label{social-learning}}

Menschen besitzen die ausgeprägteste soziale Lernfähigkeit unter allen
bekannten Spezies. Aus einer anthropozentrischen Perspektive betont das
die menschliche Besonderheit, jenseits davon erweckt es aber durchaus
Zweifel an der Qualität dieses Merkmals:

\begin{quote}
What is so \emph{wrong} with culture that it should be really
conspicuous in only one species?

-- \textcite{smith_cultural_1992}, 70.
\end{quote}

Möchte man diese Frage nicht mit einem Hinweis auf evolutionäre Zufälle
abtun, muss man die Natur des Selektionsdrucks untersuchen, der die
enorme Intensivierung von Imitation begünstigt hat. Prominente Methoden
zur Erforschung dieser Frage sind soziale Experimente mit Menschen unter
konstruierten Bedingungen, mathematische Modelle auf Populationsniveau
und agentenbasierte Computermodelle. Die folgenden Erkenntnisse basieren
vor allem auf minimalistischen mathematischen und computerbasierten
Modellrechnungen:

Soziales Lernen steht neben genetischer Vererbung und individuellem
Lernen. Während individuelles Lernen große Flexibilität mit sich bringt,
dafür aber auf das Individuum begrenzt ist, wirkt genetische Vererbung
nur auf dem Populationsniveau und damit gemessen an der Lebensspanne des
Einzelnen sehr langsam. Soziales Lernen steht zwischen diesen Polen und
erlaubt sowohl kurzfristige und kleinräumige, als auch langfristige,
kummulative und populationsweite Anpassung. Während individuelles Lernen
und Experimentieren viel Zeit und Energie in Anspruch nehmen kann, kann
soziales Lernen Wissen über einen Sachverhalt unmittelbar und risikoarm
transportieren. Gefährliche Fehler beim individuellen Lernen, die durch
die für den Einzelnen geringe Anzahl von Experimentdurchläufen häufig
sind, können durch soziales Lernen vermieden werden\footnote{\textcite{boyd_evolution_1988}}.
Es ist dafür allerdings anfällig für schnelle und schnell
aufeinanderfolgende Veränderungen der natürlichen Umweltbedingungen, da
gegebenefalls ein unangepasstes Verhalten traditionell weitergeführt
wird\footnote{\textcite{rogers_does_1988}}. Vergleicht man eine
Kombination von genetischer Anpassung und individuellen Lernen
einerseits mit einer Kombination von sozialem und individuellem Lernen
andererseits, dann führen erstere nur dann zu besserer Anpassung, wenn
die Umgebung nahezu unverändert bleibt oder sich enorm schnell und
völlig zufällig verändert. In den Fällen zwischen diesen Extrema ist
soziales Lernen überlegen\footnote{\textcite{boyd_culture_1985},
  117-128.}:

\begin{quote}
A cultural system of inheritance combining individual and social
learning ought to provide adaptive advantages in environments with an
intermediate degree of environmental similarity from generation to
generation. This is the regime where the faster tracking due to the
evolutionary force of cumulative, relatively weak, low-cost individual
learning pays off most. Most individuals can depend primarily on
tradition, yet the modest pressure of individual learning is sufficient
to keep culture ``honest''.

-- \textcite{smith_cultural_1992}, 73.
\end{quote}

Diese Hypothesen sind außerhalb der künstlichen Modellumgebungen aus
denen sie abgeleitet wurden schlecht überprüfbar. Fallstudien mit
bedingt sozial lernfähigen Tieren wie Ratten könnten zur Prüfung der
Hauptaussagen geeignet sein. Für die menschliche Entwicklung müssen
entsprechende empirische Belege im archäologischen Befund ausgemacht
werden. Geht man von einer Korrelation von Gehirngröße und sozialer
Lernfähigkeit aus, dann könnten zum Beispiel anthropologische Daten aus
dem klimatisch variablen Pleistozän als starkes Indiz
auftreten\footnote{\textcite{smith_cultural_1992}}.

Soziales Lernen kann zur Konsequenz haben, dass schädliches -- also für
genetische Reproduktion ungeeignetes -- Verhalten unter positiven
Selektionsdruck gerät und sich verbreitet. Genetische Disposition und
individuelles Lernen können diesem Effekt entgegenwirken. Wenn etwa eine
strenge Religion Prüderie und Abkehr vom Weltlichen propagiert, kann
sexuelles Verlangen und eine Liebe zu Kindern der familienverneinenden
Ideologie entgegenwirken. Oft sind die Vor- und Nachteile einer
Verhaltensform für den Einzelnen oder die Gesamtpopulation allerdings
nicht so offensichtlich. Die genetische Anlage des Menschen sieht für
komplexes Kulturverhalten keine adequate Reaktion vor und der Einzelne
ist mit der Evalutation vieler Fragen überfordert.

\begin{quote}
The natural world is complex, hard to understand, and variable from
place to place and time to time. Is witchcraft effective? What causes
malaria? What are the best crops to grow in a particular location? Are
natural events affected by human pleas to their governing spirits?
{[}\ldots{}{]} What sort of person(s) should one marry? What mixture of
devotion to work and family will result in the most happiness or the
highest fitness?

-- \textcite{smith_cultural_1992}, 79.
\end{quote}

Dabei treffen Menschen selbst komplexe Entscheidungen oft auf Grundlage
stark vereinfachter Faustregeln. Die investierte Mühe ergibt sich als
Kompromiss zwischen der erwarteten Belohnung einer richtigen
Entscheidung und den Kosten der Informationssammlung\footnote{\textcite{nisbett_human_1980}}.
Eben weil damit nicht viel Kapazität für nicht drängenden Entscheidungen
übrig bleibt, ist Kultur im wesentlichen ein Vererbungssystem. Ein
großer Teil der Fähigkeiten, Glaubens- und Moralvorstellungen des
Individuums hat es von anderen übernommen, ohne sie zu hinterfragen. Das
macht das Verhalten von Menschen anhand des kulturellen Milieus aus dem
sie stammen vorhersagbar\footnote{\textcite{smith_cultural_1992}}.

Zur näheren Charakterisierung der zwischenmenschlichen
Informationsvererbung grenzen \autocite{cavalli-sforza_cultural_1981}
drei Formen des Sozialen Lernens voneinander ab: \emph{Vertical
Transmission}, \emph{Horizontal Transmission} und \emph{Oblique
Transmission}:

Die vertikale Vererbung kultureller Eigenschaften von Eltern zu Kind
spielt eine entscheidende Rolle.

Vertical Transmission ist stark mit der demographischen Entwicklung
einer Population verknüpft. Geht man von einem klassischen Modell der
Life-history-Theorie aus, das Populationsentwicklung auf Grundlage von
sich reproduzierender Altersklassen beschreibt\footnote{\textcite{leslie_further_1948}}
und erweitert es um kulturelle Merkmale und Transmission, dann ergeben
sich bemerkenswerte Simulationsergebnisse\footnote{\textcite{coratenuto_age_1989};
  \textcite{fogarty_role_2013}}. Sogar Verhaltensmuster, die die
Reproduktionsfähigkeit eines Individuums reduzieren, können bei
ausreichend starker Übertragungsfähigkeit dauerhaft relevant bleiben.
Das gilt besonders dann, wenn eine Idee zwar die Reproduktionsfähigkeit
reduziert, gleichzeitig aber die Überlebenschance des Individuums
erhöht.

\autocite{mulder_intergenerational_2009}

\emph{Horizontal Transmission}

\emph{Oblique Transmission} \autocite{fogarty_evolution_2011}

Horizontaler und Schräger Austausch von Ideen ist günstig um einem
Individuum möglichst viel Auswahl an Strategien zur Verfügung zu
stellen, aus denen es zur Lösung von Problemen wählen kann. Umso stärker
diese Formen der Cultural Transmission in einer Gesellschaft ausgeprägt
sind, desto mehr verschiebt sich der Selektionsdruck zugunsten von
sozialen Führungsrollen wie die von Lehrern, Priestern oder Großeltern.
Elternschaft kann demgegenüber ins Hintertreffen geraten. Ein solches
Verhaltensmuster ist für genetische Selektion ungünstig, da Kinder unter
diesen Umständen nicht die biologische Reproduktion, sondern andere
Lebensmodelle anstreben können\footnote{\textcite{smith_cultural_1992}}.

Social Learning:
\autocites{arbilly_arms_2014}{enquist_evolution_2007}{rendell_cognitive_2011}{rendell_rogers_2010}

\hypertarget{entscheidungsprozesse}{%
\subsection{Entscheidungsprozesse}\label{entscheidungsprozesse}}

Die Intensität und Dauerhaftigkeit der Verbreitung einer Idee in einer
Gesellschaft ist chaotisch und nicht mit Sicherheit vorhersagbar.
Dennoch lassen sich Effekte beschreiben, die wesentlichen Einfluss auf
den Erfolg einer Innovation haben. Menschen treffen die Entscheidung ob
sie eine Idee oder ein Verhaltensmuster übernehmen nicht zufällig.
Stattdessen evaluieren sie oft sowohl wen als auch was sie in der
jeweiligen Situation imitieren. Um so leichter es ist, die Vor- oder
Nachteile verschiedener Verhaltensmuster zu erkennen, desto schneller
kann die Entscheidung für oder gegen einzelne getroffen werden. Die
Konsequenz des Evalutationsverhaltens ist \emph{Biased Transmission}.
Ihr Gewicht nimmt zu, wenn dem Einzelnen durch mehr kulturelle Vielfalt
eine größere Auswahl untschiedlicher Verhaltensmuster zur Verfügung
steht\footnote{\textcite{smith_cultural_1992}}.

Menschen zeigen beispielsweise die Tendenz, das Verhalten erfolgreicher
Menschen oder einzelne, erfolgreiche Strategien zu übernehmen\footnote{\textcite{henrich_evolution_2003}}.
Zwar ist das Modell eines Homo Ökonomikus, der stets die rational beste
Entscheidung in einer gegebenen Situation trifft, zu einfach, dennoch
spielt die Verbesserung der eigenen Situation nach unterschiedlichen
Kriterien eine wichtige Rolle bei Entscheidungsprozessen\footnote{\textcite{mesoudi_cultural_2008};
  \textcite{mesoudi_experimental_2011}}. Die klassische \emph{Diffusion
of Innovation} Forschung identifiziert den individuell wahrgenommenen
Vorteil als wesentliches Kriterien zur Übernahme oder Ablehung einer
Neuerung\footnote{\textcite{rogers_diffusion_1983}}. Aus der Perspektive
der Behavioural ecology kann argumentiert werden, dass das Nervensystem
hinreichend komplexer Lebewesen grundsätzlich Verhaltensweisen
bevorzugt, die zu positiven Stimuli führen. Das sind oft gleichzeitig
jene, die für die Anpassung an eine Umgebung förderlich sind. Biologisch
oder durch vormalige Lernprozesse determinierte Lernregeln führen in
einem Prozess von \emph{Guided Variation} zur Selektion von
Verhaltensmustern\footnote{\textcite{smith_cultural_1992}}. Dieser
postulierte Automatismus besitzt Implikationen für eine mögliche
biologische Selektionswirkung von Innovationen: Imitation kann den
Untergang einer Population in Krisensituationen verhindern oder
zumindest die Anpassung an Umweltveränderungen erheblich beschleunigen
und so den mit biologischer Selektion oft verbundenen
Bevölkerungsrückgang vermeiden.

Ein Dualismus von Konformität (\emph{Conformity Bias}) und Neugierde
(\emph{Novelty bias}) ist entscheidend dafür, ob und wie Innovationen
sich in einer Population verhalten. Menschen neigen besonders in Phasen
von Stabilität dazu\footnote{\textcite{henrich_evolution_1998};
  \textcite{kendal_evolution_2009}}, das Verhalten einer
Bevölkerungsmehrheit zu übernehmen\footnote{\textcite{efferson_conformists_2008};
  \textcite{giraldeau_social_1994}; \textcite{henrich_evolution_1998}}.
Dieser \emph{Frequency Bias} hat zur Konsequenz, dass sich Ideen, die
ohnehin schon weit verbreitet sind, weiter stabilisieren können und
Neuerungen, die in direkter Konkurrenz zu vorhanden Konzepten stehen,
nur langsam an Relevanz gewinnen oder verschwinden: Ein sich selbst
verstärkendes System. Insbesondere Ideen, die nicht direkt
subsistenzrelevant sind, sind in ihrer momentanen Ausbreitungsdynamik
stark davon abhängig, wie groß die Verbreitung der Idee in der
Population bereits ist. Eindrucksvolle Beispiele dafür sind unter
anderem Kleidermode oder Babynamen\footnote{\textcite{acerbi_biases_2014};
  \textcite{acerbi_logic_2012}}. Ist eine Population in teilweise
isolierte Gruppen aufgeteilt, erwirkt ein starker Frequency Bias
Homogenität innerhalb und Heterogenität außerhalb von Gruppen. Die bei
biologischer Evolution umstrittene \emph{Group Selection} kann damit im
Kontext von Cultural Evolution durchaus starke Wirkung
entfalten\footnote{\textcite{smith_cultural_1992}}.

Trotz des Frequency Bias brechen Individuuen jedoch mitunter bewusst aus
dem Verhalten der Mehrheit aus\footnote{\textcite{henrich_evolution_2003}}.
Als Konsequenz des Widerstreits dieser Pole folgt die Verbreitung
kultureller Eigenschaften oft einer logistischen, S-förmigen
Wachstumskurve\footnote{\textcite{henrich_cultural_2001}}. Neue Ideen
werden zunächst von einigen, meist wohlhabenden und gut gebildeten
\emph{Innovators} eingeführt bis die ökonomisch empfindlichere
\emph{Majority} sie übernimmt und nur wenige konservative
\emph{Laggards} zurücklässt, die sich der Neuerung bewusst
verweigern\footnote{\textcite{rogers_diffusion_1983}}.

In archäologischen Zusammenhängen wird häufig über den Einfluss sozialer
Eliten auf das Verhalten einer Gesamtpopulation diskutiert. Tatsächlich
tendieren Menschen dazu, soziale höher gestellte Vorbilder zu wählen und
sie zu kopieren\footnote{\textcite{henrich_evolution_2001}}. Dieses
Kopierverhalten lässt sich experimentell bereits an Kleinkindern
beobachten, die sich an jenen Erwachsenen orientieren, die die
verstärkte Aufmerksamkeit anderer Erwachsenen genießen\footnote{\textcite{chudek_prestige-biased_2012}}.
\emph{Prestige Bias} führt zu \emph{Indirect Bias}: Menschen wählen ihre
Vorbildern oft aufgrund weniger auszeichnender Charakteristika aus. Sie
neigen auch dazu, neben den ursprünglich ausschlaggebenden Eigenschaften
weitere Verhaltensmuster des Vorbilds zu übernehmen. Das hat zur
Konsequenz, dass Konzepte, die für sich genommen keine oder nur geringe
Ausbreitung erfahren würden, mit anderen Konzepten transportiert werden.
Einerseits kann dank dieser Tendenz mehr Information schneller
verbreitet werden, andererseits können sich so auch Ideen durchsetzen,
die ihrem Träger keinen Vorteil oder sogar Nachteile bringen können.
Trotz dieses Nachteils kann es evolutiv sinnvoll sein, einfach das
gesamte Verhalten erfolgreicher Individuuen zu übernehmen -- ohne
kostenaufwändige Reflektion darüber, welche Muster genau den Erfolg
herbeiführen\footnote{\textcite{smith_cultural_1992}}.

\begin{quote}
If wealth partly derives from subsistence or social skills that can be
acquired by imitation, it makes adaptive sense to imitate the wealthy.
The assumption that wealth is correlated with adaptive behavior is
perhabs generally correct; if so it would be sensible to imitate wealthy
people even if it is not always very clear just what components of
wealthy people's behavior are adaptive.

-- \textcite{smith_cultural_1992}, 81.
\end{quote}

Soziale Hierarchien und Prestigesysteme können als Hilfsmittel dienen,
um zu entscheiden, welche Eigenschaften und Verhaltensweisen übernommen
werden sollten\footnote{\textcite{rogers_diffusion_1983}}. Gerade arme
und schlecht gebildete Gruppen orientieren sich oft and
Führungspersonen, die über mehr Risikokapital verfügen, das sie für die
Evalutaion von Innovationen investieren können. Dabei werden bevorzugt
Menschen imitiert, die lokal präsent sind und in ähnlichen Umständen
leben wie der Imitierende.

Ein Phänomen, das für vertikale Transmission besondere Relevanz besitzt,
ist Assortative Paarung (Assortative mating). Partnerwahl beim Menschen
ist kein zufälliger Prozess, sondern zeigt die Tendenz, Individuen mit
hoher Ähnlichkeit körperlicher und kultureller Merkmale zusammen zu
führen. Ein Nachweis dieses Effekts gelang an Merkmalen wie Augenfarbe,
Körpergröße, IQ, Bildungsstand und Tabakkonsum\footnote{\textcite{domingue_genetic_2014};
  \textcite{keller_genetic_2013}; \textcite{laeng_why_2007};
  \textcite{treur_spousal_2015}}. Assortative Paarung führt zu höherer
Korrelation genetischer und kultureller Eigenschaften in einer
Population und kann mehr Vielfalt hervorrufen\footnote{\textcite{feldman_evolution_1977};
  \textcite{rice_multifactorial_1978}}: Seltene Eigenschaften können
sich leichter ausbreiten und behaupten\footnote{\textcite{creanza_complexity_2014};
  \textcite{creanza_models_2012}}. Assortativer Paarung ist dabei auch
ein sich selbst verstärkender Prozess, da aus Beziehungen ähnlicher
Partner statistisch mehr Kinder hervorgehen\footnote{\textcite{thiessen_human_1980}}
und soziale Netzwerke dazu neigen, sich zu reproduzieren\footnote{\textcite{abdellaoui_association_2013};
  \textcite{abdellaoui_educational_2015}}. Das hat auch Rückwirkungen
auf die genetische Zusammensetzung von menschlicher
Populationen\footnote{\textcite{robinson_genetic_2017}}. Sprachgrenzen
können dabei als wesentliche Hürde beim genetischen Austausch
auftreten\footnote{\textcite{barbujani_zones_1990};
  \textcite{de_filippo_y-chromosomal_2011};
  \textcite{karafet_coevolution_2016}}, müssen es aber
keinesfalls\footnote{\textcite{hunley_gene_2005};
  \textcite{hunley_genetic_2008}; \textcite{srithawong_genetic_2015}}.
Homophily, die Präfenrenz mit ähnlichen Menschen zu interagieren,
beschränkt sich nicht nur auf die Partnerwahl, sondern erstreckt sich
auf jede Form zwischenmenschlicher Beziehung: Ideen werden grundsätzlich
schneller zwischen ähnlichen Individuen übertragen\footnote{\textcite{centola_experimental_2011};
  \textcite{centola_spread_2010}}.

\hypertarget{computerbasierte-modellierung}{%
\subsection{(Computerbasierte)
Modellierung}\label{computerbasierte-modellierung}}

Cultural Evolution theory ist angelegt zur Modellbildung in
Anthropologischen Wissenschaften. Edmonds definierte Memetik vor seiner
oben nachvollzogenen Abkehr folgendermaßen:

\begin{quote}
the application of models with an evolutionary or genetic
\emph{structure} to the \emph{domain} of (cultural) information
transmission.

-- \textcite{edmonds_modelling_1998}
\end{quote}

Im selben Artikel \emph{On Modelling in Memetics} aus dem 2. Band des
Journal of Memetics weist er aber bereits auf Risiken der Modellierung
in diesem Kontext hin und formuliert Anforderungen.

\pby[title={Literatur},segment=\therefsegment,heading=subbibintoc]

\hypertarget{bestattungsritus-in-der-europaischen-bronzezeit}{%
\chapter{Bestattungsritus in der Europäischen
Bronzezeit}\label{bestattungsritus-in-der-europaischen-bronzezeit}}

\hypertarget{fallbeispiel-und-betrachtungsperspektive}{%
\section{Fallbeispiel und
Betrachtungsperspektive}\label{fallbeispiel-und-betrachtungsperspektive}}

Ziel dieser Arbeit ist die Formulierung eines computerbasierten Cultural
Evolution Modells, dass \ldots{}. Um die Sinnhaftigkeit dieses Ansatzes
zu erforschen, ist es unerlässlich ein Fallbeispiel heranzuziehen, das
potentiell geeignet ist durch ein solches abgebildet zu werden. Für den
Kontext des Fallbeispiels sollen sich idealerweise Synergieeffekte
ergeben. Das heißt, das Modell sollte geeignet sein, archäologische
Fragestellungen in seinem Kontext zu beantworten oder zumindest aus
einer neuen Perspektive zugänglich zu machen. Scheitert dies, so ist
zwar nicht der gesamte Ansatz zu verwerfen, jedoch darf die Nützlichkeit
der Methode aus archäologischer Perspektive in Zweifel gezogen werden.

Die Wahl des Fallbeispiels hat also wesentliche Konsequenzen für das
Gesamtergebnis der Arbeit. Konkret drückt sich das etwa in der Natur der
untersuchten Meme und Memeplexe aus: Unmittelbar funktional relevante
Meme, die z.B. eine Veränderung der Subsistenzsstrategie hervorrufen,
sind anders zu analysieren als Modememe in Keramikverzierung und
Gewandschmuck. Manche Meme sind äußerst erfolgreich, breiten sich über
ganze Kontinente aus und bleiben über Jahrhunderte verhältnismäßig
stabil, andere dagegen sind nur auf eine eine Siedlung beschränkt und
überdauern nicht einmal ihre Schöpfergeneration. Jedes Fallbeispiel ist
über eine Auswahl archäologischer Daten zugänglich. Diese sind höchst
heterogen strukturiert, mit unterschiedlichen Zielsetzungen -- meist
nicht der einer Cultural Evolution Analyse -- aufgenommen und decken,
ebenso wie die Meme, die sie potentiell abbilden, sehr verschiedene
zeitliche und geographische Spektren und Skalenniveaus ab. Ideal wäre
sicher, selbst Daten zu zu einzelnen Memen und deren Entwicklung zu
sammeln. Das ist aber im Rahmen dieser Arbeit nicht möglich, ohne viel
Zeit zu verlieren, die für theoretische Vorüberlegungen sowie die
Modellimplementierung und -Analyse investiert werden sollte. Die Suche
nach einem Fallbeispiel war aus dieser Konsequenz gleichermaßen die
Suche nach einem Datensatz, bei dem Anküpfungspunkte zur Modellidee zu
erwarten waren.

Ein spezielles Subset der C14-Datenbank Radon-B\footnote{\textcite{jutta_kneisel_radon-b_2013}}
erfüllt diese Bedingung.

Brandbestattung und Körperbestattung sind Meme, die schon lange vor
Beginn der Bronzezeit in Konkurrenz standen. Erstaunlicherweise ist
dieser Konflikt bis heute nicht entschieden -- beide Bestattungsrituale,
freilich immer wieder neu konnotiert und kontextualisiert -- finden in
der Gegenwart in Europa Anwendung. Man könnte den Konflikt aus dieser
Perspektive in seiner gesamten zeitlichen Dimension von der frühesten
Vorgeschichte bis in die Moderne nachzeichnen. Die Cultural Evolution
Perspektive bietet für dieses Phänomen ein kohärentes Erklärungsmodell.
Dennoch konzentriert sich diese Arbeit auf die Bronzezeit. Das geschieht
einerseits aufgrund der zur Verfügung stehenden Daten und weiterhin
aufgrund der unglaublichen Komplexität der Memeplexe, in die sich beide
Meme im Laufe der Geschichte eingegliedert haben. Eine Geschichte von
Brand- und Körperbestattung würde den Rahmen dieser Arbeit bei weitem
sprengen, ist vielleicht überhaupt nicht sinnvoll formulierbar. Ähnlich
verhält es sich mit der Grabüberhügelung: Auch dieser Brauch kann auf
eine lange Geschichte zurückschauen und hat im Laufe der Zeit
mannigfaltige, verwandte weil abgeleitete Meme hervorgebracht.

Dieses Fallbeispiel beobachtet also vier eng verknüpfte Meme
(Körperbestattung, Brandbestattung, Flachgrab, Hügelgrab) über eine
gewisse Phase (2500-500calBC, Bronzezeit) ihrer eigentlich wesentlich
längeren Entwicklung in einem gewissen Raum (Nord-, Ost und Westeuropa),
der nur einen Auschnitt ihrer eigentlichen Verbreitung darstellt.

\hypertarget{voruberlegungen-zur-archaologischen-untersuchung-von-bestattungen}{%
\section{Vorüberlegungen zur archäologischen Untersuchung von
Bestattungen}\label{voruberlegungen-zur-archaologischen-untersuchung-von-bestattungen}}

Betrachtet man Bestattungssitten als in Raum und Zeit verbreitetes
Kulturverhalten dann kann man das doch nicht gänzlich losgelöst von
seiner besonderen Qualität als mit dem Tod assoziiertem Verhalten tun.
Bestattungssitten sind weder funktional -- obgleich auch aus
hygienischer Perspektive und hinsichtlich Materialkosten und Arbeitszeit
für oder gegen bestimmte Rituale argumentiert werden kann -- noch sind
sie Mode, die leichtfertig und ohne Reflexion übernommen wird. Der Tod
von Angehörigen ist meist ein schwerwiegendes Ereignis, das mit einem
besonderen, kulturellen und individuellen Verarbeitungsprozess
einhergeht. Bestattungssitten gehören eben zu diesem
Verarbeitungsverhalten und sind als solche Gegenstand ihrer Erforschung.

Thanatologie ist die Wissenschaft des Todes und seiner Wirkung auf die
Umwelt des Sterbenden und Verstorbenen. Sie ist interdisziplinär
angelegt und beschäftigt sich mit allen biologischen, sozialen,
psychologischen und sonstigen Prozessen im Kontextes des biologischen
und speziell menschlichen Todes\footnote{\textcite{hofmann_rituelle_2008},
  100.}. Thanatoarchäologie beschäftigt sich mit dem Tod in
archäologischen Kontexten, also mit dem Niederschlag, den der Tod von
Menschen im archäologischen Befund hinterlassen hat\footnote{\textcite{hofmann_rituelle_2008},
  123.}. Der wichtigste Befundtyp hierfür ist das Grab, das umgekehrt
eine der wichtigsten Forschungsgegenstände der Archäologie im
allgemeinen darstellt. Seiner Erforschung wird vor dem Hintergrund
chronologischer und sozialer Fragestellungen viel Aufmerksamkeit
gewidmet. Dennoch bleibt ein großer Teil der mit Gräbern assoziierten
Bedeutungsbelegung unbekannt, da Gräber sich nur aus der Wahrnehmung des
Todes ihn ihrer Erzeugerkultur und deren Vorstellungen pränataler- und
postmortaler Zustandsformen in Abgrenzung zum bekannten irdischen Leben
verstehen lassen. Archäologische Quellen geben über diese spirituellen
Abstraktionen keinen oder nur sehr eingeschränkt Aufschluss.

Eben daraus erwächst die große Gefahr, in Ermangelung des Wissens über
das Todesverständnis prähistorischer Gesellschaften moderne, westliche
Vorstellungen auf archäologisch erschlossene Grabzusammenhänge zu
projizieren. Eine eurozentrisches Verhalten, das insbesondere die
Postprozessuale Archäologie in Anlehnung an die Postmoderne
identzifiziert hat. Stattdessen muss eine Auseinandersetzung mit dem
breiten Spektrum an Weltanschauungen und Wahrnehmungen erfolgen, in die
Bestattungssitten eingeordnet werden können. Vollständigkeit kann dabei
nicht erreicht werden, aber zumindest eine erhebliche und wertvolle
Aufweitung der Perspektive.

\hypertarget{der-rituelle-umgang-mit-dem-tod}{%
\subsection{Der rituelle Umgang mit dem
Tod}\label{der-rituelle-umgang-mit-dem-tod}}

Kerstin Hofmanns Dissertationsschrift \emph{Der rituelle Umgang mit dem
Tod -- Untersuchungen zu bronze- und früheisenzeitlichen Bestattungen im
Elbe-Weser-Dreieck}\footnote{\textcite{hofmann_rituelle_2008}} enthält
einige theoretische Vorüberlegungen zur Thanatoarchäologie, die ich hier
verarbeiten und einer Betrachtung von Bestattungssitten im Kontext der
Cultural Evolution Theorie voranstellen möchte. Damit soll einerseits
einer simplistischen und eurozentristischen Deutung von Gräbern
vorgebeugt, andererseits die Besonderheit der Bestattungsmeme betont
werden.

\hypertarget{sterben-als-prozess}{%
\subsubsection{Sterben als Prozess}\label{sterben-als-prozess}}

Die Feststellung, ab wann genau ein Mensch Tod ist, ist mit
erstaunlichen Unsicherheiten und Unschärfen verknüpft. Diese nehmen
ihren Anfang bei den biologischen Prozessen, die es erlauben, den
Eintritt des Todes an verschiedenen Parametern zu messen und
entsprechend unterschiedlich festzulegen. Leben drückt sich im Menschen
in verschiedenen Körperfunktionen wie Atmung, Herzschlag oder
Stoffwechsel aus. Der Ausfall eines Teilsystems bewirkt je nach seiner
Relevanz mehr oder weniger schnell den Zusammenbruch aller anderen
Systeme. Das kann sich über einen langen Zeitraum hinziehen: Auch im
Falle des normalen, sukzessiven Ausfalls aller Teilsysteme stirbt die
letzte Körperzelle viele Stunden nach dem Kreislaufstillstand. Da die
Individualität eines Menschen an die intakte Funktion seines Gehirns
gebunden ist, gilt der Kollaps dieses Teilsystems als eines der
wesentlichen Definitionsmomente für den Eintritt des Todes. Umgekehrt
kennt die Medizin mit dem Hirntod auch den Sonderfall, dass nur das
Gehirn seine Funktion mit irreperablen Schäden eingestellt hat, alle
anderen Körperprozesse allerdings weiter funktionieren. Der Hirntod kann
nur klinisch diagnostiziert werden (Harvard-Kriterium), da in diesem
Fall andere Indikatoren für den Eintritt des Todes fehlen. Letztere
lassen sich grundsätzlich in unsichere und sichere pathophysiologische
Kriterien untergliedern. Zu den unsicheren gehören ein Abkühlen des
Körpers, Reflexlosigkeit, Erschlaffen der Muskeln, Pulslosigkeit,
Atemstillstand, Leichenblässe und ein Vertrocknen an Schleimhäuten und
Wunden. Obgleich diese traditionellen Todesanzeiger weitreichend bekannt
sind und im Laufe der Geschichte wesentlich für die Feststellung des
Todes waren, sind sie einzelnen oder sogar bei gemeinsamem Auftreten
nicht verlässlich. Sie können (zumindest kurz- bis mittelfristig) als
Folge von Erkrankungen oder Umgebungsparametern auftreten. Pulslosigkeit
und Atemstillstand sind, wenn der Zustand anhält, sichere
Todesanzeichen. Dazu gehören auch Totenflecken -- rötliche Verfärbungen
an der Körperunterseite infolge der Unterbrechung des Blutflusses -- und
die Totenstarre -- eine biochemische Körperreaktion, die zur Erstarrung
der Muskulator in einem Zeifenster von 6-9 bis 50-300 Stunden nach dem
Todeszeitpunkt führt. Völlig unzweifelhafte Todesanzeiger sind
schließlich spätere Veränderungen an der Leiche wie Autolyse
(Selbstauflösung/Selbstverdauung), Fäulnis, Mumifizierung,
Fettwachsbildung und Skelettierung\footnote{\textcite{hofmann_rituelle_2008},
  92-94.}.

Der mit naturwissenschaftlichen Kriterien messbare Tod ist in einer
modernen, westlich geprägten Gesellschaft oftmals die entscheidende Form
des Todes. Tatsächlich ergeben sich aber neben dieser
biologisch-technischen auch fundamental abweichende Perspektiven, die
den Tod durch seine Kontextualisierung im kulturell-sozialen Gefüge des
Verstorbenen verstehen. Der Tod ist dabei der Abbruch der sozialen
Beziehungen. Dieses Ereignis muss nicht mit dem biologischen Tod
einhergehen. Tatsächlich kann sowohl ein biologisch Lebender aus einer
Gemeinschaft ausgeschlossen und damit für ``tot'' erklärt werden, als
auch ein biologisch Toter -- etwa im Kontext eines Ahnenkults -- weiter
in zwischenmenschliche Interaktion einbezogen und wie ein Lebender
behandelt werden. Vor diesem Hintergrund hält Hofmann die Einschätzung
ob jemand tot ist oder lebendig für von kultureller Wahrnehmung
abhängig:

\begin{quote}
Niemand kann demnach eine Todesfeststellung kulturfrei vornehmen.

-- \textcite{hofmann_rituelle_2008}, 92.
\end{quote}

Das Urteil, ob biologischer und sozialer Tod gleichzeitig eingetreten
sind, ist darüber hinaus stark mit der Art des Todes verknüpft, die den
Verstorbenen ereilt hat. Ein schneller Unfalltod, ein Mord, ein Tod in
kriegerischem Konflikt oder ein langsames Dahindämmern infolge von Alter
oder Krankheit werden unterschiedlich wahrgenommen und sind kulturell
unterschiedlich konnotiert. Oftmals ist genau das Auschlaggeben dafür,
ob sich der Tod im Einzelfall Angehörigen und Beobachtern als schnelles,
unumkehrbares Überschreiten einer Linie oder als länger andauernder
Transformationsprozess darstellt. Den Rahmen für diese Unterscheidung
bilden Vorstellungen von postmortalem Leben, das das irdische Leben
fortsetzt oder mit ihm interagieren kann. Damit ist der Tod und seine
Erfahrung eng mit grundsätzlichen, weltanschaulichen Fragen verknüpft,
denen jede Kultur mit anderen Paradigmen begegnet\footnote{\textcite{hofmann_rituelle_2008},
  94-95.}.

\hypertarget{kulturubergreifende-wahrnehmung-des-todes}{%
\subsubsection{Kulturübergreifende Wahrnehmung des
Todes}\label{kulturubergreifende-wahrnehmung-des-todes}}

Wie und mit welchen Hoffnungen und Ängsten der Einzelne dem eigenen oder
dem Tod anderer Menschen begegnet, hängt von einer Vielzahl von Faktoren
ab. Prägend dafür ist ein Erfahrungshorizont, der sich aus Kultur- und
Religionszugehörigkeit ergibt, aber auch individuellen Eigenschaften und
Erfahrungen.

\begin{quote}
Die Einstellung zum Tode entstehen aus der dynamischen, sich
verändernden Wechselwirkung zwischen Individuum und Umwelt und sind mit
dem individuellen und kollektiven Bild von Mensch, Natur und
Gesellschaft verknüpft.

-- \textcite{hofmann_rituelle_2008}, 96.
\end{quote}

Nach moderner, naturwissenschaftlicher Erkenntnis muss jeder Mensch
sterben. Diese Wahrnehmung hat ihren Ursprung wahrscheinlich in der
Antike, ist allerdings nicht universell menschlich. In indigenen
Gesellschaften wird der Tod oft als etwas unnatürliches und fremdes
gedeutet, dass durch schädliche äußere Einflüsse -- etwa durch Flüche
oder den Eingriff von Gottheiten -- ausgelöst wird. Viele
Schöpfungssagen beschreiben einen Urzustand, in dem der Tod noch nicht
existierte. Erst ein durch Versehen oder Unwissenheit ausgelöstes
Ereignis habe ihn in die Welt gebracht. Oft wird ein ``guter'', von der
Gemeinschaft begleiteter Tod von einem ``schlechten'', einsamen in der
Fremde abgegrenzt.

Der Glaube an ein postmortales Weiterleben ist in der Mehrzahl bekannter
Kulturen verbreitet. Die Auflösung des Körpers im Anschluss an den Tod
mag ein wichtiger Grund dafür sein, dass die meisten das auch mit einer
Trennung von Körper und Seele in Verbindung bringen. Die Seele vollzieht
einen Transformationsprozess -- eine Reise ins Reich der Toten -- der
sich über einen gewissen Zeitraum hin erstreckt. Oft werden Verstorbene
in diesem Übergangsstadium als gefährlich empfunden, da sie sowohl Macht
als auch das Interesse zur Interaktion mit der Welt der Lebenden
besitzen könnten. Im Gegensatz zum Personenkonzept des westlichen
Individualismus, der Menschen als Einheit aus einem Körper und einer
unteilbaren Seele versteht, unterscheiden andere Gesellschaften
gegebenfalls mehrere Seelenkategorien, die im Todesfall unterschiedlich
reagieren, oder etwa durch unterschiedliche Rituale wie Kremation
freigesetzt werden müssen. Unabhängig davon ist die Seele jedoch in
vielen Weltanschauungen die Entität, die im Jenseits weiterlebt. Das
Totenreich ist in verschiedenen Kulturen mit vielfältigen Assoziationen
belegt -- häufig mit dem Bild des Schlafens, einer spiegelbildlichen
Parralelwelt zum irdischen Dasein oder mit Mechanismen, die Ausgleich
und Sühne schaffen sollen. Kulturen, die ein Reinkarnationskonzept
verinnerlicht haben, verstehen den Tod meist nur als eine kurze Phase
zwischen zwei Daseinsformen.

In ausnahmslos allen Kulturen gibt es ein Totenbrauchtum, das den Umgang
mit Verstorbenen regelt. Die praktizierten Handlungen wie
Leichnamsvorbereitung, Bestattung, Totenmahlzeit, Besuchsfeste oder
Wiederbestattungen sind höchst vielfältig und stark von den oben
beschriebenen, ideologischen Voraussetzungen abhängig. Ebenso sind Zweck
und Bedeutung der Riten unterschiedlich. Sie zeigen jedoch
kulturübergreifend einige Gemeinsamkeiten und richten sich grundsätzlich
sowohl an die Toten als auch die Lebenden. Viele Rituale dienen dazu,
das emotionale Trauma und die Trennung zu verarbeiten. Darüber hinaus
soll der Zusammenhalt der Bestattenden sozialen Gruppe auch über den Tod
des Verstorbenen hinaus aufrecht erhalten werden. Dabei kanalisieren die
rituellen Handlungen den kritischen Übergangsprozess von der dauerhaften
An- zur Abwesenheit des Individuums. Bestattungsbräuche können auch aus
einer Angst vor dem Toten hervorgehen und dazu dienen, ihn zu bannen, zu
besänftigen oder zumindest seinen Einfluss auf die Lebenden zu
verringern. Umgekehrt existiert mit der Totenfürsorge ein Verhalten, den
Verstorbenen mit Grabbeigaben für sein postmortales Dasein auszustatten
oder ihn mit Wegzehr für die Reise ins Totenreich zu versehen.
Bestattungsrituale können auch dazu dienen formalisiert zu Erinnern und
eine bestimmte Form der Erinnerung an den Toten zu konstruieren.
Weltanschauung und Moralvorstellungen können gegebenfalls in der Gruppe
durch Repetition und den besonderen Charakter des Anlasses vertieft
werden. Eben hieran wird deutlich, dass Bestattunsriten nicht losgelöst,
sondern in den religiösen, sozioökonomischen, politischen und sonstigen
Kontexten einer Gesellschaft verankert sind. Umfang und Komplexität
einer Beisetzung hängen oft stark von der Ausprägung der sozialen
Hierarchien und der individuellen Position des Verstorbenen darin ab.
Eine systemtheoretische Perspektive macht deutlich, wie sehr
Totenbrauchtum von anderen Subsystem der Gesellschaft abhängen und diese
widerspiegeln kann\footnote{\textcite{hofmann_rituelle_2008}, 96-99.}.

\hypertarget{tod-in-den-anthropologischen-wissenschaften}{%
\subsubsection{Tod in den anthropologischen
Wissenschaften}\label{tod-in-den-anthropologischen-wissenschaften}}

Hm\ldots{} Das könnte ein wenig zu viel sein\ldots{} ggf.
\textcite{hofmann_rituelle_2008}, 100-122.

Genauso die Forschungsgeschichte der Thanatoarchäologie\ldots{} ggf.
\textcite{hofmann_rituelle_2008}, 132-140.

\hypertarget{die-erforschung-des-todes-in-der-prahistorischen-archaologie}{%
\subsubsection{Die Erforschung des Todes in der prähistorischen
Archäologie}\label{die-erforschung-des-todes-in-der-prahistorischen-archaologie}}

Vergangenes, menschliches Kulturverhalten lässt sich nur über das Medium
archäologischer Quellen erschließen. Diese Materielle Kultur spiegelt
den eigentlichen Forschunsgegenstand allerdings nur mehrfach und stark
gefiltert wieder. Selbst ihre eigene Bedeutung lässt sich nur indirekt
und unvollständig rekonstruieren. Dieser Problematik widmet sich ein
großer Teil der theoretischen, archäologischen Forschung\footnote{\textcite{hofmann_rituelle_2008},
  123-128.}. Wie oben ausgeführt sind Handlungen, die mit dem Tod in
Verbindung stehen, meist besonders bedeutungsgeladen und deswegen schwer
rekonstruierbar.

Die wichtigsten Befundtypen der Thanatoarchäologie sind Gräber --
einzeln oder im Kontext von Gräberfeldern und sonstigen
Kollektivgrabanlagen. Schrift- oder ikonographische Quellen, die
Aufschluss über das Totenritual oder sogar die zugrundeliegende
Vorstellungswelt geben würden, existieren in der prähistorischen
Archäologie nicht oder sind äußerst selten. Im Kontext von Gräberfeldern
können oft neben den eigentlichen Bestattungseinrichtungen auch Gruben,
Steinpflaster, Ustrinen (Verbrennungsplätze, an denen Scheiterhäufen
errichtet wurden) und aufgehende Strukturen wie Zäune, Grabmarkierungen
oder Ritualaufbauten dokumentiert werden. Selbst im Fall von datierbarer
Gleichzeitigkeit müssen jedoch nicht alle Befunde auf einem Gräberfeld
mit dem Totenbrauchtum in Verbindung stehen. Umgekehrt haben nicht alle
Handlungen eines Bestattungsrituals in räumlicher Nähe zum
Bestattungsplatz stattfinden müssen. Auch die Anlage von Gräbern ist
nicht obligatorisch: Viele Bestattungsrituale sehen keine
Grabarchitektur vor und manche schließen den eigentlichen Leichnam aus
der Deponierung aus. Solche Pseudogräber oder Kenotaphe sind schwer von
Hortfunden unterscheidbar und werden meist nur über ihre Position auf
dem Gräberfeld identifiziert. Siedlungsbestattungen sind in wenigen
Kulturzusammenhängen die Regel, treten aber immer wieder auf. Sie
erlauben eine besondere Kontextualisierung der Bestattung über die
räumliche Verknüpfung zu Siedlungsarealen oder Haushalten\footnote{\textcite{hofmann_rituelle_2008},
  128-129.}.

Menschliche Überreste finden sich auch außerhalb von intentionell
angelegten Gräbern -- etwa als Konsequenz unnatürlicher Tode durch
Unglücke, Naturkatastrophen oder Gewalt. Auch diese Quellen sind Teil
der thanatoarchäologischen Forschung, müssen aber anders interpretiert
werden. Aufgrund schlechter Erhaltungssituation durch stärkere
taphonomischer Einflüsse sowie Unsicherheiten über das Kulturverhalten
vormoderner Menschen ist die Entscheidung, ob ein Leichnam in einem
Ritual bewusst niedergelegt oder schlicht zufällig durch
Sedimentbedeckung konserviert wurde besonders in der paläolithischen
Archäologie oftmals schwer\footnote{\textcite{hofmann_rituelle_2008},
  145-147.}.

\hypertarget{quellengattung-grab}{%
\subsubsection{Quellengattung Grab}\label{quellengattung-grab}}

Neben Siedlungen, Horten und Einzelfunden gehören Gräber zu den
Hauptkategorien archäologischer Quellengattungen. Gräber und Depots
heben sich von Siedlungen ab, da sie grundsätzlich eine positive
Artefaktauswahl einschließen, das heißt, die eingebrachten Objekte
wurden intentional in diesem Kontext platziert. Diese Intentionalität
gilt auch für den Grabaufbau. Gräber sind also hochgradig
bedeutungsgeladene Befunde, die als Überrest der rituellen Handlungen
des Totenbrauchtums konserviert werden. Sie bilden religiöse, soziale
und politische Strukturen, Werte und Normen einer Gruppe ab --
allerdings stets schematisiert und gegebenefalls bewusst manipuliert.
Die Vielzahl an Filtermechanismen, die zwischen der Lebensrealität einer
prähistorischen Gesellschaft und dem archäologisch fassbaren Befund
wirken, werden bei der Rekonstruktion von Sozialstrukturen oftmals nicht
ausreichend reflektiert. Das ist umso mehr relevant, wenn aus den
statischen archäologischen Quellen dynamische, chronologische
Entwicklungen und Transformationsprozesse abgelesen werden sollen.

Menschen trafen in der Vorgeschichte immer wieder neu eine Entscheidung
für die Position eines Bestattungsplatzes im natur- und
kulturgeographischen Raum. Der Entscheidungsprozess erschließt sich aus
einer landschaftsarchäologischen Perspektive, die einerseits natürliche
Gegebenheiten wie Topographie, Vegetation oder Wassernähe am
Bestattungsplatz sowohl absolut als auch in Relation zu damit
wahrscheinlich verknüpften Siedlungen betrachtet, als auch die
kulturhistorischen Bezüge zu kontemporärer oder vorangegangener
menschlichen Aktivität in der Umgebung. Das erfordert eine grundsätzlich
mit Unsicherheiten behaftet Rekonstruktion der Landschaft zum Zeitpunkt
der Anlage des Bestattungsplatzes. Funktionale Kriterien wie das Meiden
von hochwassergefährdeten Flächen oder Arealen mit schwacher Bodendecke
mögen zu einer Vorauswahl der Plätze geführt haben. Darüber hinaus sind
dem Feld ideologischer Konnotationen keine Grenzen gesetzt. Das kann zum
Beispiel zur Beachtung astronomischer Relationen oder einer bewusst
erzeugten über- oder unterbetonten Sichtbarkeit der Anlage führen. Ein
Bestattungsplatz ist schließlich selbst landschaftsprägend: Grabanlagen
können Territorien abgrenzen oder Wege markieren. Die Aufgabe eines
Gräberfelds, seine kontinuierliche Nutzung oder die Wiederaufnahme der
Nutzung einer alten Anlage, die gegebenenfalls aus einem
vorangegangenen, archäologischen Kulturzusammenhang stammt, geschieht
oft in einem Prozess, der mit anderen schwerwiegenden Veränderungen in
einer Siedlungsgemeinschaft korreliert.

Jenseits der Frage nach der Position des Bestattungsplatzes stellt sich
eine weitere nach der inneren Gliederung desselben. Wird ein Areal neu
für diesen Zweck erschlossen, ist es zunächst meist ohne Einrichtungen,
die als kulturelle Bedeutungsträger fungieren. Erst die Nutzung für
Bestattungen führt zu einer langsamen Akkumulation von -- aus
archäologischer Perspektive -- Befunden. Architektur wie Grabanlagen
oder Ritualstellen können über längere Nutzungszeiträume erneuert,
umgebaut oder entfernt werden. Gräber können einzelnen in individuellen
Einrichtungen wie Gruben oder Kisten für sich stehen oder durch
Konstruktionen wie Grabhügel, Kammern in Megalithbauten oder Einhegungen
zu Einheiten zusammengefasst werden. Letztere führen zu einer Gliederung
des Bestattungsplatzes in nach verschiedenen, oft unbekannten Kriterien
zusammengehörige Grabkomplexe. Auch die Anordnung von Einzelgräbern auf
Gräberfeldern ist in der Regel nicht zufällig und wird unter dem
Stichwort der Horizontalstratigraphie archäologisch diskutiert: Durch
das sukzessive Sterben von Mitgliedern einer Siedlungsgemeinschaft
stellt sich aus Sicht der Bestattenden für jeden Toten erneut die Frage
der Platzierung in Relation zu den bereits vorhandenen Gräbern. Häufig
bilden sich in der Verteilung der Gräber die chronologische Entwicklung
des Gräberfelds ab, aber auch andere Kategorien wie biologische und
soziale Gruppengliederung, Alters- und Geschlechtsunterschiede sowie
Unterschiede im Rang der Verstorbenen in einer vergangenen
soziopolitischen Hierarchie können sich hier niederschlagen. Ausdruck
dieser Kategorien sind räumliche Verteilungsmuster der Gräber in denen
Merkmalsvariation von einem Zentrum aus oder entlang Achse nachvollzogen
werden können, merkmalsgleiche Gruppen zu voneinander getrennten
Clustern akkumulieren oder Außreißer mit positiv oder negativ
herausragenden Eigenschaften getrennt von der Hauptgruppe platziert
wurden. Eine weitere Beobachtungsgröße ergibt sich daraus, ob Gräber in
andere Gräber eingreifen und diese stören. Das kann bewusst vermieden
werden, zufällig in Einzelfällen auftreten oder ein Gräberfeld als
Charakteristikum auszeichnen. Auch die Beraubung von Gräbern nach der
Beisetzung kann Teil des Bestattungsrituals sein und die innere
Gliederung eines Bestattungsplatzes sowie die Grabarchitektur
beeinflussen.

Die vorliegende Arbeit konzentriert sich mit den Dichotomien Brandgrab
vs.~Körpergrab sowie Hügelgrab vs.~Flachgrab auf Aspekte von
Bestattungsform und Grabbau. Diese Kategorien dürfen als besonders
bedeutungsgeladen verstanden werden: Sie sind kulturell deutlich
unterschiedlich und ihre Merkmale besitzen meist starken, symbolischen
Aussagewert. Auch innerhalb von Kulturzusammenhängen herrscht große
Variabilität -- ein möglicher Indikator für die Intensität sozialer
Reglementierung des Bestattungsrituals. Das erschwert eine umfassende
Klassifizierung, die in der Lage wäre alle Phänomene aufzunehmen.
Wesentliche Gliederungsgrößen sind Ein- und Mehrphasigkeit sowie
Partielle und Vollständige Bestattung, darüber hinaus ergeben sich aus
der Architektur des Grabbaus sowie der Art der Deponierung des Leichnams
Unterscheidungskriterien\footnote{Hofmann sammelt einige der wichtigsten
  Kategorien in einer tabellarischen Aufstellung:
  \textcite{hofmann_rituelle_2008}, 152.}. Grabanlagen besitzen in der
Regel eine innere, unsichtbare Struktur und einen sichtbaren,
oberirdischen Aufbau. Das Innere des Grabes ist oft nur während der
Errichtung und im Moment der Beisetzung offen und zugänglich. Es
addressiert entsprechend neben dem Toten und angenommener Entitäten der
postmortalen Welt vor allem die Bestattenden und eventuelle Zuschauer
der Bestattungszeremonie. Der dauerhaft sichtbare Teil des Grabes hat
einen potentiell größeren Adressatenkreis und damit oft einen anderen
Symbolgehalt.

Grabformen können Strukturen im sozialen Gefüge der Lebenden
widerspiegeln, indem für Verstorbene aus verschiedenen sozialen Gruppen
jeweils unterschiedliche Grabformen genutzt oder indem besondere
Individuen in von der Norm abweichenden Sondergrabformen beigesetzt
werden. In vielen Gesellschaften werden für Führungspersonen und deren
Verwandten aufwendigere und auffälligere Gräber errichtet während fremde
und soziale Außenseiter gegebenenfalls deutlich einfacher bestattet
werden. Kollektivgräber nivillieren demgegenüber soziale Hierarchien:
Die archäologische Forschung bringt sie häufig mit einem betonten
Gemeinschaftsdenken und egalitären Gesellschaftsformen in Verbindung.
Bestattungsform und Grabbau bieten die Möglichkeit, nicht nur Einblicke
in die soziale Organisation sondern auch die spirituelle
Vorstellungswelt einer archäologischen Kultur zu gewinnen. Zwar sind mit
ein und der selben Religion durchaus unterschiedliche Grabriten
vereinbar und eine plakative Trennung der Vorstellungen, die zum
Beispiel hinter Körper- und Brandbestattungen stehen mögen, ist nicht
haltbar. Bestimmte Rituale, wie etwa eine aufwändige Mumifizierung,
geben allerdings begründeten Anstoß zur Vermutung, die Unversehrtheit
des Körpers spiele in der Jenseitsvorstellung der entsprechenden Kultur
eine entscheidende Rolle. In verschiedenen Kulturzusammenhängen erwecken
Gräber, die als Totenhäuser gestaltet sind oder hausförmige Urnen
enthalten, den Eindruck, die Toten würden in ihren Gräbern wie in
Häusern weiterleben. Die systematische Orientierung des Körpers in
Relation zu den Himmelsrichtungen tritt auf prähistorischen
Gräberfeldern häufig auf und könnte mit einer religiösen Begründung gut
erklärt werden. Der Grabbau kann auch durch die Angst vor dem Toten
beziehungsweise dessen Eingriffe in die Welt der Lebenden bestimmt sein.
Das kann sich dadurch ausdrücken, dass der Leichnam bewusst mit schweren
Steinen bedeckt oder gefesselt wird.

Jenseits von Bestattungsform und Grabbau konzentriert sich die
archäologische Erforschung von Gräbern vor allem auf die
Grabausstattung, also jene Artefakte und Überreste, die bei der
Beisetzung intentionell in das Grab eingebracht wurden. Aufgrund
taphonomischer Gegebenheiten erhalten sich bestimmte Materialkategorien
weniger gut oder besser als andere und sind entsprechend im
archäologischen Befund über- oder unterrepräsentiert. Grabbeigaben
können und müssen hinsichtlich ihrer Bedeutung nach verschiedenen
Kriterien untersucht werden. Bestimmte Objekte wurden nur für den
Bestattungskontext hergestellt, andere dem Materialkreislauf der
Lebenden bewusst entzogen. Artefakte können Teil des Grabbaus sein, zur
persönlichen Ausstattung und Tracht des Verstorbenen gehört haben oder
als sonstige Beigaben in den Grabkontext eingebracht worden sein.
Letztere können beispielsweise als Gebrauchsgegenstände für den Toten in
seiner postmortalen Existenz verstanden werden, als durch den Tod
verunreinigt gelten oder zur Selbstdarstellung der Hinterbliebenen im
Bestattungsritual präsentiert werden. Aus archäologischer Perspektive
ist es oft sehr schwierig, die Motive hinter der Deponierung einer
einzelnen Beigabe zu erschließen. Der potentielle Symbolgehalt von Form,
Farbe und Verzierung der Artefakte steigert bringt weitere
Unsicherheiten mit sich. In der archäologischen Literatur werden
beispielsweise immer wieder einzelne Artefakte als Amulette
angesprochen. Meist handelt es sich um Einzelstücke ohne erkennbaren,
funktionalen Nutzen, die nah am Leichnam platziert wurden. Sie könnten
sowohl Funktionen als Glücksbringer für den Toten als auch als
Bannmittel zum Schutz der Lebenden übernommen haben. Nahrungsbeigaben
sind in rezenten Kulturen oft mit der Vorstellung einer Reise ins
Jenseits verknüpft. Der Verstorbene hat auch nach dem Tod noch Bedarf
nach physischer Nahrung. Diese Assoziation ist allerdings nicht
zwingend: Nahrungsbeigaben können auch schlicht eine weitere
Ausdrucksform für die soziale Identität des Toten sein.

Interpretationsansätze für Grabausstattungen betonen meist den
Aussagewert der Beigabensammlung für die Identität des Bestatteten. In
der Regel werden Unterschiede in der Qualität und Quantität von Beigaben
mit dem vertikalen sozialen Status einer Person oder Gruppe in
Verbindung gebracht. Insbesondere Prestigegüter -- auffallende
Einzelobjekte aus heute als wertvoll erachteten Materialien -- werden in
diesem Kontext betont betrachtet. Beigaben können auch die Zugehörigkeit
unter anderem zu einem sozialen Geschlecht, einer Altersgruppe, einem
Berufszweig oder einer Herkunftsregion ausdrücken. Zuordnungen dieser
Art lassen sich mit physisch-anthropologischen oder
naturwissenschaftlichen Daten korrelieren und so gegebenenfalls
verifizieren. Allerdings kann sowohl eine Person mehrere Identitäten in
sich vereinen als auch ein Artefakt mit mehreren Bedeutungsebenen
verknüpft sein. Die Auszeichung von eindeutigen Leit- oder
Faziesartefakten kann zwar statistisch relevant, im Einzelfall aber auch
irreführend sein und zu Zirkelschlüsse führen\footnote{\textcite{hofmann_rituelle_2008},
  145-165.}.

\hypertarget{eine-cultural-evolution-perspektive-auf-bestattungssitten}{%
\subsection{Eine Cultural Evolution Perspektive auf
Bestattungssitten}\label{eine-cultural-evolution-perspektive-auf-bestattungssitten}}

Bergensen 1998, 54ff \& 63 f. Grimes 1998, 131 ff.

\hypertarget{raumliche-und-zeitliche-trends-im-bestattungsritus-der-bronzezeit}{%
\section{Räumliche und zeitliche Trends im Bestattungsritus der
Bronzezeit}\label{raumliche-und-zeitliche-trends-im-bestattungsritus-der-bronzezeit}}

In der europäischen Bronzezeit sind mehrere unterschiedliche
Bestattungstraditionen unterscheidbar, die zeitlich und räumlich
verschiedene Entwicklungen durchlaufen. Dabei können zwei wesentliche
Dimensionen abgegrenzt werden, entlang derer sich fast alle
dokumentierten Grablegungen kategorisieren lassen: 1. Körperbestattungen
im Gegensatz zu Brandbestattungen sowie 2. Flachgräber gegenüber
überhügelten Gräbern. In diesem Spektrum gibt es etliche Varianten
hinsichtlich der Grabanlage- und Vergesellschaftung (z.B. Nachnutzung
neolithischer Megalithanlagen, Gräberfelder, etc.), des Grabbaus (Särge,
Totenhäuser, Bootsgräber etc.) der Beigabenauswahl, der Platzierung des
Leichnams oder des investierten Aufwands für Bestattungszeremonie und
Architektur. Angesichts dieser Variablenvielfalt ist Generalisierung und
die Reduktion des Gesamtzusammenhangs auf die Spannungsfelder Körper-
vs.~Brandbestattung und Flach- vs.~Hügelgrab schwierig. Dennoch soll für
die vorliegende Arbeit diese Perspektive eingenommen werden, da nur zu
diesen primären Variablen Informationen im Radon-B Datensatz (siehe
Kapitel \ref{radonb-dataset}) enthalten sind. Der Datensatz gibt auch
das Forschunsareal und die Abgrenzung künstlicher Regionen vor, die als
Beobachtungseinheiten dienen (siehe Kapitel
\ref{data-prep-and-segmentation}).

Kurz zusammengefasst besagt das klassische Narrativ der Entwicklung
bronzezeitlicher Bestattungssitten folgendes: In der frühen und
mittleren Bronzezeit dominieren Körperbestattungen in verschiedenen
Variationen. Brandgräber kommen in diesem Zeitfenster nur in der
Ungarischen Tiefebene verstärkt vor. Hügelgräber konzentrieren sich auf
Teile des Balkans sowie Ost-, West- und Nordeuropa, während in Zentral-
und Südeuropa Flachgräber -- mehrheitlich Körperbestattungen --
überwiegen. In der mittleren Bronzezeit gewinnt Überhügelung zumindest
in West- und Mitteleuropa an Bedeutung. In der späten Bronzezeit wird
Brandbestattung zum häufigsten Bestattungsbrauch\footnote{\textcite{harding_european_2000},
  75-76. \textbf{Eigentlich Häusler 1977/1994/1996 -\textgreater{}
  ausbauen!}}.

\hypertarget{korperbestattung-und-brandbestattung}{%
\subsection{Körperbestattung und
Brandbestattung}\label{korperbestattung-und-brandbestattung}}

Nach verbreiter Lehrmeinung sind Brandbestattungen in erster Linie ein
Phänomen der Spätbronzezeit. Ein fast universelles Phänomen sind
Flachgräberfelder mit einzeln beigesetzen Urnen. Aus dieser Beobachtung
heraus wird die Spätbronzezeit in Zentral-, Nord- und Westeuropa auch
als \emph{Urnenfelderzeit} bezeichnet. Tatsächlich kommen
Brandbestattungen schon erheblich früher vor.

In Ungarn wird die Brandbestattung schon in der Frühbronzezeit von den
Nagyrév- und Kisapostag Gruppen und in der Mittelbronzezeit von der
Vatya Kultur praktiziert. Auch in Großbritannien treten
Brandbestattungen schon früh und über einen langen Zeitraum parallel zu
Körperbestattungen auf. In der Mittelbronzezeit wird die Verbrennung
dort die dominante Bestattungssitte. Auch in Zentral- Nord und
Südwesteuropa treten Kremationen in geringem Anteil lange vor der
lokalen Spätbronzezeit auf\footnote{\textcite{harding_european_2000},
  111.}.

Grundsätzlich gilt, dass kohärente Bestattungsplätze und Gräberfelder in
einer Belegungsperiode jeweils einheitlich eine Bestattungsform
praktizieren. Das zeigt sich besonders in der Urnenfelderzeit, wo eine
Vielzahl großer und weitreichend untersuchter Gräberfelder in
Zentraleuropa, im Mediterranen Raum, in Frankreich und in Skandinavien
abgesehen von verschwindend wenigen Ausnahmen exklusiv mit Kremationen
belegt sind. Dazu gehören zum Beispiel die Gräberfelder Moravičany
(Mähren) mit 1260 oder Vollmarshausen (Hessen) mit 252 erfassten
Bestattungen. In der Frühen und Mittleren Bronzezeit ist birituelle
Belegung noch erheblich häufiger: Auf dem Tumuluszeitlichen Platz Dolný
Peter (Slowakei) verhalten sich Brand- zu Körperbestattungen in einem
Verhältnis 5:50, in Streda nad Bodrogom (Slovakei) beträgt das
Verhältnis 34:24, wobei weiterhin neun Kenotaphe erfasst wurden. Auf dem
größten und archäologisch wichtigsten Gräberfeld der Mittelbronzezeit
Zentraleuropas in Pitten (Niederösterreich) dominieren Kremationen mit
147:74. Ebenso gibt es aber auch in der Frühbronzezeit Gräberfelder mit
großer Einheitlichkeit wie Gemeinlebarn F (Niederösterreich) wo unter
den 258 erfassten Bestattungen nur eine einzige mit einem
Verbrennungsritual begesetzt wurde und in der Spätbronzezeit
Gräberfelder mit biritueller Belegung wie Przeczyce (Schlesien) mit
einem Verhältnis von 132:727\footnote{\textcite{harding_european_2000},
  112.}.

Von besonderem archäologischen Interessen sind eben jene Kontexte, wo
verschiedene Rituale in größter räumlicher und -- soweit erfassbar --
zeitlicher Nähe zueinander durchgeführt wurden. Systematische
Unterschiede hinsichtlich Beigabenreichtum, Geschlechterschwerpunkt oder
horizontalstratigraphischer Aufteilung von Gräberfeldern bei biritueller
Belegung\ldots{} \textbf{Mal nachsehen!}. Sowohl für das
urnenflederzeitliche Vollmarshausen als auch das frühbronzezeitliche
Gemeinlebarn F deuten sich eine horizontalstratigraphische Trennung nach
Familiengruppen an, deren Nachweis allerdings erst mit genetischen
Mitteln erfolgen könnte, die zum Zeitpunkt der Untersuchung noch nicht
zur Verfügung standen oder im Falle der Brandbestattungen von
Vollmarshausen wahrscheinlich erhaltungsbedingt ausgeschlossen werden
müssen\footnote{\textcite{harding_european_2000}, 114.}.

In Kontakt- und Übergangsbereichen der Bestattungssitten kam es an
verschiedenen Punkten zu überraschenden Überschneidungen der
Ritualausführung. In Periode III der skandinavischen Bronzeit wurde in
Dänemark Leichenbrand in Sarg- und Kistengräbern beigesetzt, die zuvor
für Körperbestattungen verwendet worden waren. In der Champagnen finden
sich Brandbestattungen in Grabgruben, die ausreichend Platz für einen
unverbrannten Körper geboten hätten. Für ein aunjetitzerzeitliches
Gräberfeld in Jeßnitz (Sachsen-Anhalt) rekonstruieren die Ausgräber ein
Ritual, das sekundäre Feuereinwirkung auf schon in Särgen deponierte
Körperbestattungen eingeschlossen hätte. \textbf{Ausbauen!} \footnote{\textcite{harding_european_2000},
  113.}

\hypertarget{flachgrab-und-hugelgrab}{%
\subsection{Flachgrab und Hügelgrab}\label{flachgrab-und-hugelgrab}}

huhu

\hypertarget{regionaler-uberblick-regions-archaeological-overview}{%
\subsection{Regionaler Überblick
\{regions-archaeological-overview\}}\label{regionaler-uberblick-regions-archaeological-overview}}

Schwerpunkt auf Grabform (Brandgrab vs.~Körpergrab) und
Grabkonstruktion. Weitere Aspekte wie Grabbeigaben, Körperhaltung,
Geschlechtsdimorphismen, etc. werden nur dann beachtet, wenn sie direkt
mit diesen Primäraspekten zusammenhängen.

\hypertarget{osterreich-und-tschechische-republik}{%
\subsubsection{Österreich und Tschechische
Republik}\label{osterreich-und-tschechische-republik}}

Die Bronzezeit in Österreich und Tschechien lässt sich in vier
wesentliche Perioden gliedern: Früh-, Mittel-, Spät- und Jungbronzezeit.
In der Frühbronzezeit entwickelte sich nördlich der Donau in Österreich
unter Einfluss der Ungarischen Nagyrév Kultur die Proto-Aunjetitz Kultur
parallel zum Aunjetitz Kreis in Böhmen und Mähren. Im Aunjetitz Areal
folgen die Bestattungssitten einem relativ festen Regelwerk: Üblich
waren Einzel- und Körpergräber. Einzelne Funde von Massengräbern im
Siedlungskontext, die sich auch in der Tumulus und und der
Urnenfelderkultur fortsetzen, dürfen nicht als reguläre
Bestattungskontexte verstanden werden. In Nordbömen, Mähren und den
anschließenden Teilen von Österreich dominieren Flachgräber, während in
Süd- und Westböhmen Hügelgräber häufiger aufreten. Die Flachgräber
gingen den Grabhügeln zeitlich voran und zeichnen sich oft durch eine
gemauerte Grabkiste oder einen Holzsarg aus. In mehreren Fällen konnte
der Nachweiß für eine hölzerne Konstruktion über dem Flachgrab erbracht
werden. Die Bestattungen in den Hügeln sind meist in den Urhumus
eingetieft und durch eine Steinpackung geschützt, die damit
gleichermaßen den Kern der Hügelaufschüttung bildet. Im Aunjetitzer
Kontext dominieren linke Hocker in Süd-Nord Orientierung ohne
Geschlechtsdimorphismus. Obgleich eine Mehrzahl der Gräber
wahrscheinlich zum Ende der Frühbronzeit beraubt wurden, lässt sich ihr
Beigabeninventar rekonstruieren: Neben Gewandelementen und -- in einigen
wenigen Kontexten -- Waffen und Prestigegegenstände aus Bronze, Gold und
Bernstein überwiegen Keramikgefäße. Dabei lässt sich eine diachrone
Entwicklung von größeren zu sehr kleinen, teilweise miniaturisierten
Beigabengefäßen beobachten\footnote{\textcite{lubos_czech_2013}, 789 \&
  794-796}.

Zeitgleich mit der Aunjetitzer Kultur begegnen sich südlich der Donau in
der österreichischen Frühbronzezeit mehrere Lokalgruppen. Die
Leithaprodersdorf Gruppe, die noch in der Frühbronzezeit von der
Wieselburg Kultur abgelöst wurde, findet sich östlich des Wienerwald. Im
Kontext dieser aufeinander folgenden Gruppen sind flache
Körperbestattungen üblich. Die Körperhaltung und Orientierung folgt
einem klaren Geschlechtsdimorphismus und die Qualität und Quantität der
Beigaben ist betont ungleich. Auch hier sind Steinkisten und Baumsärge
ein wichtiger Teil der Grabkonstruktion. Südlich von Wien und westlich
des Wienerwalds lässt sich die Unterwölbing Kultur verorten. Auch hier
sind flache Körperbestattungen die Regel. Die Gräber sind stark
standardisiert, zeigen einen deutlichen Geschlechtsdimorphismus und sind
als mit gesetzten Steinkisten und Baumstammsärgen aufgebaut. Sie sind
meist Teil siedlungsnaher Gräberfelder und vergleichsweise reich mit
Keramik sowie Bronzewaffen und -schmuck ausgestattet. Schwere Halsringe
sind charakteristisch für diesen Kulturzusammenhang. Im westlichen Teil
Niederösterreichs bis nach Tirol findet sich die Straubing Kultur, deren
Verbreitungsschwerpunkt in Bayern liegt. Aus Österreich sind trotz
ausgeprägter Besiedlung keine Bestattungsplätze der Straubing Kultur
bekannt, in Bayern verhält es sich allerdings wie im Unterwölbing Raum.
Auch aus den Alpengebieten sind zu wenige Gräber erforscht, um eine
zuverlässige Aussage über die vorherrschenden Bestattungsbräuche treffen
zu können, jedoch deutet sich für die inneren Alpen ein früher Wandel
hin zu Brandbestattungen an\footnote{\textcite{lubos_czech_2013}, 789 \&
  796-797}.

Am Ende der Frühbronzezeit entstand in Böhmen und Mähren unter starkem
Einfluss aus Südosteuropa die Věteřov Kultur. In Österreich wurde die
Unterwölbing Kultur durch die Böheimkirchen-Gruppe abgelöst. Im Osten
bestand die Wieselburg Kultur parallel zur neue geformten Drassburg
Gruppe weiter. Im Salzkammergut konstituierte sich die Attersee Gruppe.
Im Laufe der Mittelbronzezeit wurden die lokalen Phänomene im Süden
Mährens, fast ganz Böhmen und in Ostösterreich durch die
Mitteldonauländische Tumuluskultur homogenisiert. In dieser Konsequenz
ist die geradezu universelle Bestattungsform in der Mittelbronzezeit
Tschechien und Österreichs das Hügelgrab. Die Hügel sind einfache
Erdhügel auf einem Steinkreisfundament. In den Hügeln wurden -- oft
mehrere -- sowohl Körper- als auch Brandbestatttungen untergebracht,
wobei erstere langsam als dominante Form von letzteren abgelöst werden.
Bei Körperbestattungen sind die Beigaben um den Körper verteilt, wobei
Keramik entweder am Fuß- oder Kopfende der Grube platziert wurde. Im
Kontext der Brandbestattungen wurde die Ausstattung nicht mit verbrannt,
sondern vor der Überhügelung auf dem Leichenbrand deponiert. Ein
Geschlechtsdimorphismus zeigt sich mitunter nicht nur bei der
Beigabenauswahl, sondern auch bei der Grabform: Im Fall der
Grabhügelanlage von Pitten in Niederösterreich überwiegt Brandbestattung
für weibliche Individuen während Männer überwiegend unverbrannt
beigesetzt wurden. Im Westen Österreichs zeichnet sich das Tumulus
Phänomen durch mehr Bezüge zu Süddeutschland aus. Im Salzburger Land
wurden die Bestattungen als Rückenstrecker in Grabhügeln untergebracht.
Brandbestattungen in einfachen, steinbedeckten oder leicht überhügelten
Gruben enthalten die Überreste verbrannter Metall- und Keramikartefakte
in Urnen aus Keramik oder organischem Material. Die wenigen Funde aus
dem inneren Alpengebiet deuten auf beigabenlose Brandbestattungen
hin\footnote{\textcite{lubos_czech_2013}, 790 \& 797-798}.

In der Spät- und Jungbronzezeit wurden Böhmen, Mähren und Österreich
relativ homogen Teil des Urnenfelder Kulturkomplexes. Böhmen ist
überwiegend im Einflussgebiet der Urnenfeldergruppen aus dem Oberen
Donauraum, Siedlungen der Lausitzer Kultur in Norden und Osten Böhmens
lassen sich allerdings besser aus der Perspektive der Nördlichen
Urnenfeldergruppen verstehen und im äußersten Westen besteht mit der
Cheb Urnenfeldergruppe eine kulturelle Verbindung ins Areal des heutigen
Deutschland. Ebenso lassen sich auch Österreich und Mähren in
verschiedene kleinere Sphären gliedern, die als verschiedentlich
beeinflusste Varianten des Urnenfelderphänomens beschrieben werden
können. In all diesen Kontexten folgt die allgemeine Bestattungsform --
mitunter in Nutzungskontinuität der Mittelbronzezeitlichen
Bestattungsanlagen -- der stark vereinheitlichen Urnenfelderpraxis:
Ausgedehnte Felder von flachen Brandgräbern. In einer Mehrzahl der Fälle
ist der Leichenbrand in einer Keramikurne eingelagert, wobei in einer
Grabgrube durchaus mehrere Urnen oder sonstige beigefügte Gefäße
deponiert sein können. Urnen können neben menschlichen Überresten auch
verbrannte Tierknochen und Bronzeartefakte enthalten. Die persönliche
Tracht und Schmuck wie Nadeln oder Armreife wurde in der Regel mit
verbrannt, während Gebrauchsgegenstände wie Messer erst nach der
Verbrennung beigefügt wurden. Trotz der insgesamt großen Homogenität der
Urnenfelderkulturen hinsichtlich ihres Bestattungsbrauches zeigen die
kulturellen Subgruppen im Detail durchaus Abweichungen voneinander
hinsichtlich der Anordnung von Urne und Beigaben im Grab und der
Beigabenauswahl, die auf weiterreichende Unterschiede in Ideologie und
Sozialstruktur schließen lassen. Zudem wurden in verschiedenen Regionen
weiterhin vereinzelt Körpergräber angelegt oder Bestattungen in
Grabhügel eingebracht. In der Frühen und Mittleren Urnenfelderzeit war
die Beisetzung von Urnen und Beigaben in körpergroßen Steinkisten
deutlich standardisiert. In der Späten Urnenfelderzeit verlieren
Steinsetzungen an Bedeutung. Gleichzeitig nimmt die Beigabenmenge ab --
insbesondere Waffen werden nicht mehr beigegeben. In den inneren
Alpenregionen setzen sich Urnenfelder bis weit in die Frühe Eisenzeit
fort\footnote{\textcite{lubos_czech_2013}, 790 \& 798}.

\hypertarget{polen}{%
\subsubsection{Polen}\label{polen}}

Polen kann entlang seiner Nord-Süd Achse naturräumlich in drei Bereiche
gegliedert werden: Ein 400-500km breiter Streifen flachen, seen- und
feuchtgebietreichen Landes an der Ostseeküste, südich davon Hochland und
das Heiligenkreuz Mittelgebirge, an der Südgrenze ein langer Gebirgszug,
der sich von West nach Ost aus Erzgebirge, Sudeten und Karpathen
zusammensetzt. Die größten Flüsse Polens sind Oder und Weichsel. Beide
erstrecken sich über weite Teile Polens, entwässern in die Ostsee und
stellten in der Vorgeschichte wichtige Verkehrswege dar. Aus
archäologisch-kulturhistorischer Perspektive hinsichtlich der Bronzezeit
zwischen 2300/2200-800calBC bietet sich eine andere Dreiteilung in
West-, Nordost und Südost Polen an: Eine westliche Zone (Woiwodschaften
Pommern, Kajuwien-Pommern, Westpommern, Großpolen, Lebus,
Niederschlesien, Lodsch, Oppeln, Schlesien) war nach Osten durch eine
Grenzlinie zwischen Danziger Bucht im Norden und dem Durchbruch zwischen
Sudeten und Karpathen, der Mährischen Pforte, im Süden definiert. Der
Bereich östlich dieser Linie war wiederrum zweigeteilt in einen
nördlichen (Ermland-Masuren, Podlachien, Masowien) und einen südlichen
(Lublin, Heiligkreuz, Kleinpolen, Karpatenvorland) Teil, getrennt am
Ost-West orientierten Übergangsbereichs von Niederrungs- zu Hochland.
Die westliche Zone spielte in der zentraleuropäischen Bronzezeit eine
entscheidende Rolle, da sie mehrfach die Rolle der östlichen oder
nordöstlichen Grenzregion von wichtigen Kulturphänomenen übernahm. Das
betrifft etwa die Aunjetitzer Kultur, die Hügelgräber Kultur und,
später, die Hallstattkultur. Die Hochland-Regionen von Südostpolen
gehörten weitestgehend zum nördlichen Bezugsbereich der Kulturen des
Karpathenbeckens. Nordostpolen wich in seiner Entwicklung deutlich vom
Rest Polens ab. Eine Sonderrolle nahm nicht zuletzt wegen des Reichtums
an Bernstein auch der Küstenstreifen zwischen Oder- und Weichselmündung
ein. Diese Region war traditionell Teil eines Austauschnetzwerks, dass
das Baltikum überspannte und bis in die Nordsee reichte\footnote{\textcite{czebreszuk_bronze_2013},
  767-770.}.

Die früheste Phase der polnischen Bronzezeit dauerte von 2300/2200 bis
2000calBC. In West- und Südostpolen ist diese Proto-Bronzezeit mit der
Glockenbecher Kultur verknüpft. Kuyavien und Pommern gehörten dabei zu
einem Glockenbecher Kreis aus Südskandinavien und Nordostdeutschland,
Niederschlesien war aus Böhmen inspiriert, Kleinpolen im Südosten des
modernen Polens aus Mähren. Aus diesem Glockenbecher Substrat entstand
in Westpolen ab 2300calBC die Aunjetitzer Kultur. Im Südosten Polens
formte sich die Mierzanowice Kultur. Die zunächst enge Verbindung
zwischen beiden Phänomenen löste sich um 2000calBC auf -- Weichsel und
obere Oder wurden zur Kulturgrenze. Während die Aunjetitzer Kultur
hinsichtlich ihrer materiellen Kultur ein klares Profil ausbildete,
hochentwickelte Metallverarbeitung hervorbrachte und an herausragenden
Grabhügeln erkennbare, soziale Differenzierung katalysierte, stagnierte
und zerfaserte die Mierzanowice Kultur in Lokalgruppen. Nordostpolen
blieb lange in einer spätneolithischen und wildbeuterischen Tradition
verhaftet, obgleich Keramikfunde Verbindungen nach Westpolen nahelegen.
Nach 2000calBC begann sich in Kuyawien und Großpolen mit der Trzciniec
Kultur eine neue Größe herauszubilden, die große Teile Nordostpolens und
-- um 1650/1600calBC -- auch Kleinpolens erfasste\footnote{\textcite{czebreszuk_bronze_2013},
  770-772.}.

Die frühe Bronzezeit in Polen war von Körpergräbern in sehr großen
Hügelgräbern auf Geländeerhebungen und Hügelgräberfeldern mit bis zu 60
einzelnen Hügeln dominiert. Flachgräber waren in dieser Zeit erheblich
seltener. Erst am Übergang zur Lausitzer Kultur ab der Mittel- und
Spätbronzezeit setzen diese sich durch. Im Südosten Polens, im Kontext
der Mierzanowice Kultur, wurden die Toten in West-Ost Orientierung und
nach Süden blickend angehockt auf die Seite gelegt. Frauen wurden mit
dem Kopf nach Osten, Männer mit dem Kopf nach Westen bestattet. Im
Südwesten drückte sich das Geschlecht nicht so offensichtlich in der
Bestattungspraxis aus: Die Körper sind Nord-Süd orientiert, der Kopf im
Süden, angehockt, das Gesicht nach Osten gerichtet. Im Nordwesten, und
damit im Aunjetitzer Kulturkreis, scheint das Recht auf Bestattung einer
sozialen Elite vorbehalten zu sein, die in großen Grabhügeln beigesetzt
wurde. Aus diesem Kontext sind entsprechend erheblich weniger
Bestattungen bekannt, es deutet sich aber an, dass die Leichname
üblicherweise Ost-West orientiert, mit dem Kopf nach Westen und mit
Blick nach Süden angeordnet wurden\footnote{\textcite{dabrowski_aeltere_2004},
  73 \& 80-81.; \textcite{czebreszuk_bronze_2013}, 775.}.

Hügelgräber der Frühbronzezeit sind aus fast ganz Polen bekannt, sie
fehlen nur Nordostpolen (Masowien und Podlasien). Ihr Durchmesser
beträgt heute 10 bis 26m, wobei dieser Wert angesichts Jahrhunderte
währender Erosion und landwirtschaftlicher Landnutzung nach unten
korrigiert werden muss: die Mehrzahl der erhaltenen Hügel ist heute
meist nicht mehr als einen Meter hoch. Manche Hügel sind von einem
breiten, mehrschichtigen Steinkranz eingefasst, der darauf hindeutet,
dass sie ursprünglich von einer nunmehr zerstörten Steinschicht bedeckt
waren. Die notwendige Erde wurde aus der unmittelbaren Umfassung der
Aufschüttunge entnommen, wodurch teilweise bis heute sichtbare Gräben
rund um die Hügel eingetieft wurden. In der Hügelaufschüttung finden
sich häufig ein reiches Artefaktinventar sowie Holz- und
Steinkonstruktionen. Drei Hauptbauarten lassen sich unterscheiden:
Einfache Erdhügel mit 1 bis 2 Körper- oder Brandbestattungen, die in den
Urhumus eingegraben oder schlicht darauf gelegt und anschließend
überhügelt wurden, Erdhügel mit einer ausgeprägten Brandschicht, die
auch verbrannte Knochen und Inventar enthält sowie Hügelgräber mit
Steinschutzkonstruktionen. Die Konstruktionen varieren deutlich zwischen
gemauerten Grabkammern mit den Überresten mehrerer Körperbestattungen,
Steinpflastern und Ringen am Boden der Hügel oder ovalen, kreisförmigen
oder rechteckigen Steinabdeckungen, die ein oder mehrere Brand- oder
Körpergräber im Hügelvolumen bedecken. In mehreren Grabkammern deutet
eine chaotische Lage von Knochen und die Anhäufung von Schädeln auf
Mehrfachbeisetzungen und ein komplexes Totenritual hin\footnote{\textcite{dabrowski_aeltere_2004},
  73-77.}.

Flachgräber traten in der Frühbronzezeit ebenfalls in ganz Polen auf. In
Zentral- und Nordostpolen (Masowien, Podlachien, Lodsch) waren sie
jedoch die ausdrücklich vorherrschende Bestattungsform. Sie wurden
überwiegend als Körper-, jedoch auch als Brandgräber ausgeführt. Die
Körperbestattungen wurden teilweise in Särgen abgelegt oder in
Leichentücher eingeschlagen. Auffallend sind Einzel- und
Mehrfachbestattungen, die sowohl Körper- als auch Brandgräber oder sogar
Mischformen mit teilweise angebrannten Skeletten enthalten. In
Brandgräbern wurde der Leichenbrand entweder mit oder ohne Urne,
manchmal in Särgen und sporadisch in anatomischer Lage deponiert. Die
urnenlosen, flachen Brandgräber stimmen manchmal hinsichtlich Ausmaßen
und Orientierung mit den Körpergräbern überein. In großen Brandgräbern
wurden in mehreren Fällen viele -- in einem Fall bis zu 18 -- Individuen
untergebracht. Wie bei Hügelgräbern treten auch bei Flachgräbern
Steinkonstruktionen in Form von Kisten und Pflastern auf, wobei
gelegentlich der Eindruck entsteht, die Steinsetzungen seien bewusst im
Sinne eins Musters oder Symbols ausgelegt worden\footnote{\textcite{dabrowski_aeltere_2004},
  77-80.}.

Der Übergang zur Mittelbronzezeit war in Polen durch neue Einflüsse aus
der Nordischen Bronzezeit ab 1700calBC im Nordwesten und der Hügelgräber
Kultur nach 1600calBC im Südwesten und Südosten geprägt. Trotz der
Unterschiede zwischen diesen Kontexten scheinen sie doch einer
gemeinsamen kulturellen Sphäre zuzugehören. In beiden waren Hügelgräber
-- oft Steinhügel -- und Metallhorte wichtige Kulturphänomene. Ostpolen
war weiterhin von Vertretern der in sich heterogenen Trzciniec Kultur
besiedelt. Manche Bestattungsplätze waren auch über die Transformation
von Mittel zu Spätbronzezeit hinweg kontinuierlich belegt -- in ältere
Grabhügel wurden häufig Nachbestattungen eingebracht. Hügel- und
Flachgräber weisen insgesamt weitreichende, strukturelle Ähnlichkeiten
auf und konnten beide als Einzel- oder Kollektivgräber ausgeführt sein.
In Fortsetzung der Traditionen aus Schnurkeramik und Mierzanowice Kultur
wurde auch hier vor allem in Hügelgräbern bestattet, daneben bestand
allerdings eine Vielfalt unterschiedlicher Phänomene, die die
Heterogenität dieses Kulturraumes widerspiegeln\footnote{\textcite{dabrowski_aeltere_2004},
  73 \& 80-81.; \textcite{czebreszuk_bronze_2013}, 772 \& 775.}.

Die Spätbronzezeit in Polen war von der Lausitzer Kultur und deren
Expansion dominiert. Die Lausitzer Kultur ist die nordöstliche
Ausprägung der Urnenfelder Kultur. Sie lässt sich nach 1400calBC in
Schlesien und Großpolen erstmals archäologisch fassen und brachte
langanhaltende Stabilierung der Siedlungsaktivitäten in großen Teilen
Polens mit sich. Sie dauerte bis 400calBC, ab 800calBC freilich stark
von der Hallstatt Kultur beeinflusst. Kleinpolen geriet ab 1300calBC in
den Einfluss der Lausitzer Kultur, dabei scheint Migration von Siedlern
aus Schlesien eine wichtige Rolle gespielt zu haben. Nordostpolen
beschritt auch hier einen Sonderweg: Die Veränderung durch die Lausitzer
Kultur ist schwieriger fassbar. Die übliche Bestattungsform in der
Spätbronzezeit ist auch in Polen damit die Brandbestattung in Urnen auf
großen Gräberfeldern. Im Einzugsgebiet des San in Südostpolen trat die
Tarnobrzeg Gruppe auf, die unter Einflüssen aus Steppenraum und
Karpathenbecken, eine distinkte kulturelle Sphäre bildete\footnote{\textcite{czebreszuk_bronze_2013},
  772-773 \& 775-776.}.

\hypertarget{suddeutschland}{%
\subsubsection{Süddeutschland}\label{suddeutschland}}

Die kulturhistorische Entwicklung Deutschlands in der Bronzezeit ist
komplex und erlaubt die Unterscheidung etlicher Gruppen, Stile und
Kulturkomplexe. Wesentlich zum Verständnis sind seine geographische
Gliederung und intensive Interdependenzen mit angrenzenden Phänomenen,
die sich als Ergebnis seiner Lage in Zentraleuropa zu allen
Himmelsrichtungen ergeben. Geomorphologisch kann Deutschland von Süd
nach Nord grob in Folgende Regionen untergliedert werden: Die
(Bayrischen) Alpen und zugehörige Vorgebirgszonen, die süddeutsche
Schichtstufenlandschaft, die Mittelgebirge und schließlich das
norddeutsche Flachland mit Küsten und Inseln in Nord- und Ostsee. Die
großen Flusssysteme von Donau, Rhein, Weser, Elbe und Oder stellen
wichtige Kommunikationskanäle dar, die sich im Verlauf der gesamten
Vorgeschichte als Verbindungen und Grenzen in verschiedenen
Austauschsystemen verhalten haben. Süddeutschland stand nach Osten in
unmittelbarem Kontakt zu Regionen in den heutigen Grenzen von Böhmen,
Mähren, Österreich und Ungarn. Nach Süden bestand Kontakt mit den
Alpenregionen der heutigen Schweiz und Norditaliens, nach Osten mit
Frankreich. Die Entwicklungen in Norddeutschland lassen sich am besten
über seine Verbindungen zum Benelux Raum und der Nordischen Bronzezeit
in Dänemark und Südschweden verstehen. Ostdeutschland bildete mit Polen
eine Sphäre intensiver Interaktion. Das gebräuchliche chronologische
System im Süden und bis in zu den Mittelgebirgen ist die für ganz
Zentraleuropa relevante Phasengliederung nach Reinecke (s.o.), während
in Norddeutschland die Periodenunterteilung der Nordischen Bronzezeit
nach Montelius (s.u.) zur Anwendung kommt. Eine übergeordnete
Dreigliederung in früh, mittel und spät ergibt sich aus einer
vereinfachten Betrachtung der Bestattungssitten: Frühbronzezeit meint
einen Zeitraum vom Ende des 3. Jahrtausends bis 1600calBC in dem flache
Hockergräber überwiegen, Mittelbronzezeit das Fenter 1600-1300calBC mit
Körperbestattungen in Grabhügeln und Spätbronzezeit, die
Urnenfelderzeit, den Zeitraum 1300-800calBC\footnote{\textcite{jockenhovel_germany_2013},
  723-725.}.

Für die folgende Zusammenstellung habe ich mich entsprechend meiner
Regionengliederung entschieden, Deutschland zweigeteilt zu betrachten.
Als gedankliche Grenzlinie dient der Main. Süddeutschland meint damit
vor allem das Areal in den heutigen Grenzen von Bayern und
Baden-Württemberg.

In der Frühbronzezeit existierten mehrere lokal begrenzte
Kultureinheiten in Süddeutschland, die als Inseln in einer meist noch
spätneolithischen Umgebung entstanden. Im Süden und Südosten von Bayern
fand sich die Straubing-Gruppe, deren Verbreitungsgebiet sich auch nach
Österreich fortsetzte (s.o.). Westlich grenzte sie an die Ries Gruppe
an. Ausgehend vom Oberrhein fanden sich in Baden-Württemberg, jeweils
nördlich anneinander anschließend, die Singen Gruppe, die
Hochrhein-Oberrhein Gruppe, die Neckar Gruppe und schließlich die
Adlerberg Gruppe am nördlichen Oberrhein und der Untermainebene. Die
übliche Bestattungsform in diesen Kontexten waren Flachgräberfelder mit
angehockten Körperbestattungen. Die Orientierung der Toten ist
geschlechtsabhängig und folgte der Glockenbechertradition. Manche Gräber
sind mit einem Holzsarg oder einer abdeckenden Steinpackung ausgebaut.
Metallbeigaben sind selten und weitestgehend auf Kupferzierrat, Nadeln
und Dolchklingen beschränkt. Knochen und Muschelschmuck wurden dagegen
oft beigegeben\footnote{\textcite{jockenhovel_germany_2013}, 726-727.}.

Die Mittelbronzezeit war in Zentraleuropa eine Phase nachhaltiger
Innovation. Schwerter und Speere kamen auf und verbreiteten sich
schnell. Zweischneidige Rasierklingen, Pinzetten, Messer und Sicheln
erweiterten das Metallwerkzeuginventar. Pferd und Wagen gewannen als
Transportmittel wesentlich an Bedeutung. In der Mittelbronzezeit
entstand und dominierte besonders in Süddeutschland, aber darüber hinaus
in ganz Zentraleuropa, die Tumulus- oder Hügelgräberkultur. Der
Übergangsprozess dahin lief regional unterschiedlich ab, letztendlich
erfasste das Phänomen jedoch einen bemerkenswert großen Raum. Die
rund-ovalen Hügel wurden aus Erde, Sand, Grassoden, Steinen oder einer
Kombination dieser Materialen errichtet. Je nach lokaler Verfügbarkeit
von Baumaterialien unterscheidet sich auch ihre Architektur. Die Hügel
waren oft von einer Steinsetzung, einem Graben oder -- besonders in
Westfalen und den Niederlanden -- Pfostensetzungen eingehegt. Sie kommen
in der Regel nicht einzeln vor, sondern clustern in kleineren bis sehr
großen Gruppen, die gemeinsam ein landschaftsprägendens Gräberfeld mit
oft dutzenden Hügeln bilden. Jeder Hügel gehörte einer kleinen
Familiengruppe, wobei die Anlage ursprünglich meist über einer
Zentralbestattung angelegt wurde. Spätere Bestattungen wurden in den
vorhanden Hügel eingetieft und liegen deswegen meist höher als das
Ursprungsgrab. Zwischen den Hügeln einer Gruppe wurden gelegentlich
Flachgräber angelegt. Die Bestattungen wurden zunächst meist als
Körpergräber ausgeführt, der Anteil von Brandgräbern nahm im Laufe der
Mittelbronzezeit jedoch immer mehr zu. In der Regel wurde der Leichnam
ausgestreckt in Nord-Süd- oder Ost-West-Orientierung deponiert und die
Grabgrube mit Steinsetzungen oder Holzplanken ausgebaut. Die Ausstattung
der Toten ist geschlechtsabhängig und scheint die persönlich Ausstattung
im Leben widerzuspiegeln: Männer wurden mit Waffen wie Schwert, Dolch,
Axt oder Lanzenspitze sowie Schmuck in Form von Nadeln oder Armreifen
versehen, Frauen mit einer reichen Auswahl von Trachtbestandteilen.
Bernsteinperlen aus dem Baltikum erfreuten sich großer Beliebtheit in
Süddeutschland: In Württemberg und Südbayern enthält der archäologische
Befund einzelne Gräber und Horte mit tausenden Perlen. Die
Zusammenstellung des Grabbeigabeninventars ist das wesentliche Merkmal
nach dem Lokalgruppen der Süddeutschen Mittelbronzezeit wie unter
anderem die Alb Gruppe, die Hagenau Gruppe oder die Rhein-Main Gruppe
definiert werden\footnote{\textcite{jockenhovel_germany_2013}, 727-730.}.

Auch in Süddeutschland war die Brandbestattung die wesentliche
Bestattungsform der Spätbronzezeit. Die Deponierung des Leichenbrands in
einer Urne war ab 1100calBC (Ha A2) die universelle Praxis. Die Urnen
wurden zusammen mit anderen Keramikgefäßen -- manchmal Teile eines
zusammengehörigen Services -- in einfache oder mit einer Steinkiste
ausgebaute Grabgruben gelegt. Mit der Einführung der Brandbestattung
ging die Aufgabe des Hügelbaus einher. Vereinzelt wurden jedoch noch
weiter Körpergräber angelegt, die mit einer reichen Beigabenausstattung
die Traditionen der Hügelgräberzeit fortführen. Obgleich die
Spätbronzezeit in Süddeutschland mit größerer kultureller
Standardisierung als die Mittelbronzezeit assoziiert werden kann ist sie
doch bei weitem kein völlig einheitliches oder homogenes Phänomen. Sie
lässt sich ausgehend von Bz D bis Ha B2/3 in mindestens fünf Phasen
gliedern und reicht bis in die Eisenzeit hinein. Zwar sind viele
Metallartefaktkategorien weit verbreitet, Schmuck -- besonders Fibeln --
und Keramik zeigen dagegen starke regionale Variabilität. In
Süddeutschland kann die südliche Bayrische Gruppe im Alpenvorland, die
Fränkisch-Pfälzische Gruppe in Ostbayern und weiter westlich die
Untermainisch-Schwäbische Gruppe unterschieden werden. Auch innerhalb
des Urnenfelderbestattungsrituals gibt es Varianten, die sich vor allem
durch die Beigabenmenge und Auswahl ausdrücken: Die Mehrzahl der Gräber
enthält nur Keramik, manche daneben auch einige wenige Nadeln und
Schmuckgegenstände, reichere dann Messer, Rasierklingen und einfache
Waffen wie ein Bogen mit Pfeilen. Die herausragend reichen Bestattungen
sind mit größeren Waffen wie Schwertern und Speerspitzen, sowie
bronzenem Drinkgeschirr, Wagenteilen und hochwertigem Bronze- und
Goldschmuck ausgestattet. In Südwestdeutschland sind sie häufig mit
einer nord-süd orientierten Kammer aus Holz oder Stein ausgebaut. Diese
Gräber gehören meist zu erwachsenen Männern, denen seitens der
Archäologischen Forschungstradition eine Führungsposition in ihrer
lokalen Gruppe zugesprochen wird. Manche sind als Körperbestattungen
ausgeführt und enthalten außergewöhnliche Beigaben wie Zeremonialwägen,
Rohmetall und Bronzegewichte. Frauengräber dieser Art sind selten,
gelegentlich treten jedoch Gräber auf, in denen die sowohl Überreste
eines Mannes als auch einer Frau beigesetzt sind. Frauengräber sind mit
Schmuck und Trachtbestandteilen ausgestattet und allgemein beigabenärmer
als Männergräber. Geschlecht und Alter korrelieren grundsätzlich mit der
Größe von Grab und Urne. Kindergräber sind üblicherweises mit einer
feminin assoziierten Beigabenauswahl versehen\footnote{\textcite{jockenhovel_germany_2013},
  730-733.}.

\autocites{falkenstein_development_2012}{falkenstein_zum_2017}

\hypertarget{norddeutschland}{%
\subsubsection{Norddeutschland}\label{norddeutschland}}

Norddeutschland dient hier als vereinfachter Begriff in Abgrenzung zu
Süddeutschland und umfasst eigentlich Nord-, Ost- und Mitteldeutschland.
Diese Großregionen sind Teil unterschiedlicher Einflusssphären (s.o.)
und durchlaufen in der Bronzezeit unterschiedliche kulturelle
Entwicklungen. Norddeutschland (im Schwerpunkt Schleswig-Holstein,
Niedersachsen und Mecklenburg-Vorpommern) wird stark aus Skandinavien
(s.u.), Ostdeutschland (im Schwerpunkt Sachsen, Sachsen-Anhalt und
Brandenburg) von Polen (s.o.) und Mitteldeutschland (im Schwerpunkt
Saarland, Rheinland-Pfalz, Nordrhein-Westfalen, Hessen, Thüringen) aus
dem Süden (s.o.) und Westen (s.u.) beeinflusst. Norddeutschland lässt
sich damit als Teil der Nordischen Bronzezeit beschreiben und
Ostdeutschland ist Teil der beiden großen, aufeinander folgenden Kreise
Aunjetitzer sowie Lausitzer Kultur. Mittel- und Westdeutschland gehören
in vielerlei Hinsicht zur süddeutschen Sphäre und durchlaufen
entsprechend eine dazu ähnliche Entwicklung.

\hypertarget{norddeutschland-1}{%
\paragraph{Norddeutschland}\label{norddeutschland-1}}

Aus dem Substrat der spätneolithischen Einzelgrabkultur entstand in
Norddeutschland, Südskandinavien und im Westbaltikum zwischen 2200 und
1600calBC die Nordische Frühbronzezeit. In vielerlei Hinsicht setzte sie
neolithische Traditionen fort -- offensichtlich beispielweise an der
weiten Verbreitung von hochwertigen Feuersteindolchen. Erste
Metallgegenstände gelangten aus Zentral- und Westeuropa nach Norden,
besonders aus dem unmittelbar angrenzenden Aunjetitzer Raum. Eine eigene
Metallverarbeitung emanzipierte sich schnell und eindrucksvoll, wobei im
Norddeutschen Raum Einflüsse aus Süddeutschland und der Schweiz sichtbar
bleiben\footnote{\textcite{jockenhovel_germany_2013}, 735.}.

Zur Mittelbronzezeit, nach 1600calBC, etablierte sich in Norddeutschland
die Sögel-Wohlde Kultur, deren Verbreitungsgebiet sich vom Osten der
Niederlande über Westphalen bis nach Jütland erstreckte. Sie zeichnet
sich durch Körpergräber in Grabhügeln aus. Im Gegensatz zur Situation in
der zeitgleich südlich davon vorherrschenden Hügelgräberkultur wurden
allerdings nur Männer mit diese Bestattungsform bedacht. Die
Beigabenauswahl umfasst Kurzschwerter, Dolche, Randleistenbeile,
Pfeilspitzen, Nadeln und gelegentlich kleine, goldene Spiralringe. Die
Sögel-Wohlde Kultur ging nördlich der Elbe und im heutigen
Schleswig-Holstein in Montelius Perioden II bis III (1450-1250calBC und
1250-1100calBC) in eine vielfältige und dynamische Kulturlandschaft
über. In diesem Kontext wurden große Grabhügel mit Steinkistengräbern
errichtet, die noch heute Landschaftsprägend wirken. Wie in
Zentraleuropa enthalten sie mehrere Gräber, wobei häufig ein Mann und
eine Frau zusammen oder nacheinander in eine Grabkammer eingebracht
wurden. Ein klassische Deutung versteht die Hügel jeweils als
Familiengrabstätte eines einzelnen Hofes in einer weitestgehend
egalitären Gesellschaft. Anhand typischer Waffenkombinationen lassen
sich Lokalgruppen wie die Westholstein Gruppe (Schwert + Speerspitze),
die Segeberg Gruppe (Schwert + Absatzbeil) und die Westmecklenburg
Gruppe (Schwert + Absatzbeil + Dolch) unterscheiden. Unter den
Grabhügeln fallen einzelne als sogenannte Trachthügel durch eine
besonders reiche Ausstattung -- z.B. mit gegossenen Bronzegefäßen -- und
gute Erhaltungssituation auf\footnote{\textcite{jockenhovel_germany_2013},
  735-736.}.

Im Laufe von Montelius Periode III wurde die Körperbestattung in
Norddeutschland langsam zugunsten der Brandbestattung aufgegeben -- ab
Periode IV war letztere die absolute Regel. Zeitlich korrelierte der
Übergang mit Importen von Bronzegefäßen und Kesselwägen aus dem
Böhmischen und Mährischen Raum. Diese Artefakte sind verknüpft mit dem
andersartigen und sicher religiös aufgeladenen
Urnenfelder-Symbolinventar, was eine kausale Verbindung der Phänomene
nahelegt und Hinweise auf den Ursprung des Urnenfelderphänomens gibt.
Die Norddeutschen Urnenfelder können wie ihre Süddeutschen Pendants
mehrere hundert Urnengräber umfassen, wurden oft um ältere Grabanlagen
herum angelegt und bis in die frühe Eisenzeit genutzt. Das
Beigabeninventar ist klein und reduziert auf Kleinwerkzeuge und
Hygieneausstattung wie Rasiermesser, Pinzetten oder Nadeln. Größere
Objekte wie Schwerter treten in einigen Fällen in Miniaturform auf. Oft
wurden Bronzegegenstände in Horten -- Totenschätzen -- außerhalb der
Gräber deponiert. Unter den Urnen fallen Sonderformen wie Haus- oder
Gesichtsurnen auf\footnote{\textcite{jockenhovel_germany_2013}, 736-737.}.

Die Nordische Spätbronzezeit, also nach üblichem Gebrauch Montelius
Perioden IV bis VI, begann mit einer langen Phase der Konsolidierung.
Sie wirkte nun stärker nach außen -- Artefakte der nordischen Bronzezeit
wie Plattenfibeln oder einschneidige, mit Bootsymbolik verzierte
Rasiermesser tauchen in Gräbern in Niedersachsen und Holland bis in die
Niederrhein Region und Pommern auf. In Periode V wurden vereinzelt große
und reich ausgestattete Brandgräber angelegt -- sogenannte
\emph{Königsgräber}. Auch in Periode VI wurde in Urnengräbern bestattet.
Sie leitet über zur eisenzeitlichen Jastorf Kultur\footnote{\textcite{jockenhovel_germany_2013},
  737-738.}.

\hypertarget{ostdeutschland}{%
\paragraph{Ostdeutschland}\label{ostdeutschland}}

In Ost- und Mitteldeutschland, aber auch weit darüber hinaus über
Schlesien, Großpolen, Böhmen, Mähren, der Südwestslowakei und dem
nördliche Teil Niederösterreich hinweg, siedelten in der Frühbronzezeit
in einem Zeitfenster von 2300/2200 bis 1600/1500calBC Vertreter des
Aunjetitzer Kulturkomplexes. Die Wurzeln der Aunjetitzer Kultur -- die
Proto-Aunjetitzer Phase -- liegen im ausgehenden Spätneolithikum und
greifen Elemente von Schnurkeramik und Glockenbecher Kultur auf. Der
übliche Bestattungsmodus in der Aunjetitzer Kultur waren Flachgräber mit
angehockten Körperbestattungen. Sowohl Männer als auch Frauen sind
Nord-Süd orientiert mit Blick nach Osten. Mehrheitlich sind die Gräber
einfache Gruben, gelegentlich wurden sie allerdings auch mit einer
Steinkiste oder einem Baumstammsarg ausgebaut. Manchmal sind in einer
Grabgrube mehrere Tote beigesetzt. Auffällig sind einige Fälle von
Kinderbestattungen in großen Vorratsgefäßen. In der fortgeschrittenen
Aunjetitzer Kultur, etwa ab dem Übergang zum 2. Jahrtausend, wurden
einfache und kleine Kupfergegenstände zur häufigsten Beigabenkategorie.
Die Beigabenmenge und -vielfalt ist insgesamt gering, sieht man von den
wenigen, sehr reich ausgestatteten, großen und weithin sichtbaren
Grabhügeln der Leubingen Gruppe am Dreiländereck Sachsen, Sachsenanhalt
und Thüringen ab, die traditionell als \emph{Fürstengräber} bezeichnet
werden. Sie enthalten nur eine einzige, fast immer männliche Bestattung
aber ein breites und hochqualitatives Beigabeninventar inkluvsive
signifikanter Mengen Goldschmuck. Diese außergewöhnlichen Bestattungen
sind in der Aunjetitzer Kultur fast ohne Vergleich. Sie weisen auf eine
lokal begrenzte aber ausgeprägte soziale Differenzierung hin, die sich
vielleicht durch eine Beherrschung von Handelswegen zwischen Donau und
Baltikum, Salzgewinnung oder Kupfererzabbau im Harzvorland erklären
lassen könnte. Für letzteren existiert allerdings kein archäologischer
Nachweis. Vom Ende der Aunjetitzer Kultur am Beginn der Mittelbronzezeit
sind fast keine Bestattungsbefunde bekannt\footnote{\textcite{jockenhovel_germany_2013},
  725-726.}.

Ostdeutschland gehörte in der Mittelbronzezeit zum Verbreitungsgebiet
der in ganz Ostmitteleuropa von Deutschland über Polen, Böhmen, Mähren
und bis in die Slovakei dominanten Lausitzer Kultur. Sie entstand aus
einem Substrat aus Aunjetitzer Kultur und verschiedenen Lokalgruppen,
die bis ins 2. Jahrtausend an einer spätneolithischen Tradition
festgehalten hatten. Die Lausitzer Kultur selbst ist kein homogenes
Kulturphänomen -- in ihr lassen sich Keramikphasen (Prä-Lausitz Phase I:
Bz B-C, Phase II Bz C-D, Phase III Bz D-Ha A1, Phase IV Ha A2-Ha C1) und
deutlich distinkte, regionale Gruppen unterscheiden. Die Prä-Lausitz
Gruppe in Ostdeutschland, Schlesien und Großpolen war zunächst stark von
der frühen, zentraleuropäischen Urnenfelderkultur geprägt, löste sich
davon aber zu Beginn der Spätbronzezeit. Das große Verbreitungsareal
dieser Westlichen Lausitzer Gruppe reicht bis weit nach
Mitteldeutschland hinein und umfasst die Flusssysteme von Oder, Elbe und
Weichsel. Sie lässt sich in mehrere Subgruppen wie etwa die
Saalemündungsgruppe, die Unstrut Gruppe oder die Elb-Havel Gruppe
gliedern. Die übliche Bestattungsform der Lausitzer Kultur war die
Brandbestattung in Urnen. Die Gräberfelder wurden über viele
Generationen hinweg genutzt und fallen mit oft mehreren tausende
Begräbnissen sehr groß aus. Die Grabbeigaben bestehen fast
ausschließlich aus Gefäßkeramik, davon jedoch große Mengen.
Metallobjekte sind selten und auf kleine Gegenstände wie Nadeln, Schmuck
und Gebrauchsgegenstände wie Messer und Rasierklingen beschränkt. Die
Beigabenarmut erschwert Schlüsse auf soziale Unterschiede innerhalb der
bestattenden Gesellschaft. Besonders in der Elbe-Saale Region kommen
vereinzelt Waffengräber vor\footnote{\textcite{jockenhovel_germany_2013},
  734-735.}.

\hypertarget{mitteldeutschland}{%
\paragraph{Mitteldeutschland}\label{mitteldeutschland}}

Der Raum südlich der Nordischen Bronzezeit, westlich der Aunjetitzer
Kultur und nördlich der Rhein-Main-Linie lässt sich in der
Frühbronzezeit keinem kohärenten Kulturphänomen zuordnen. Zwischen
Niederrhein und Elbe traten verschiedene Varianten der spätneolithischen
Riesenbecher Gruppe auf. Ihre Wickelschnurkeramik zeigt Parallelen zur
Sögel-Wohlde Kultur der Nordischen Bronzezeit, andere Importe weisen auf
Verbindungen nach Süddeutschland, in den Benelux-Raum und nach
Großbritannien hin. Verbreitet war die Körperbestattung in Hügelgräbern
nach Vorbild der Einzelgrabkultur\footnote{\textcite{jockenhovel_germany_2013},
  727.}.

In der Mittelbronzezeit näherten sich Zentral- und Westdeutschland der
süddeutschen Sphäre und damit der Hügelgräberkultur an. Lokalgruppen wie
die Fulda-Werra Gruppe, die Thüringer Gruppe, die Oberpfälzische Gruppe
und -- nördlich der Mittelgebirge -- die Lüneburger Gruppe gehören zum
Kontakt- und Frauentauschnetzwerk der süddeutschen Mittelbronzezeit. Je
weiter nördlich, desto stärker ist jedoch auch die Beziehung zu Periode
II der Nordischen Bronzezeit. Weiter westlich, an Mittel- und
Niederrhein sowie in Westfalen, zerfließen wie schon in der
Frühbronzezeit die scharfen Kulturgrenzen. Starke Einflüsse aus den
Niederlanden offenbaren sich in der Errichtung von
Holzpfostenumfassungen um Grabhügel und der frühen Verbreitung von
Brandgräbern\footnote{\textcite{jockenhovel_germany_2013}, 727-730.}.

Auch in der Spätbronzezeit ist Mittel- und Westdeutschland zwischen
verschiedenen Einflusssphären aufgeteilt. Westlich des Rhein siedelten
Vertreter der Niedermainisch-Schwäbischen Gruppe und östlich des Ober-
und Mittelrheins sowie im Moselgebiet und dem Saarland des
Rhin-Suisse-France orientale (RSFO) Kulturkomplex. Die Areale östlich
davon gehören bereits zur Peripherie der Lausitzer Kultur. Nördlich der
Mittelgebirge, zwischen Niederrhein Saale und Elbe, zeigen Keramik und
Bronzeartefakte weniger distinkte Unterschiede und erlauben so keine
Gruppenuntergliederung. Ab 1000calBC geriet diese Region zunehmend unter
den Einfluss der späten Nordischen Bronzezeit (Periode IV und V). Die
lokale Spätbronzezeit dauerte bis 600calBC, dann gehörte sie auch zum
Verbreitungsgebiet der eisenzeitlichen Jastorf Kultur. Wie in den
umgebenden Regionen ist auch in der mitteldeutschen Spätbronzezeit
Brandbestattung in Urnen üblich. Die Urnen wurden in alte Grabmonumente
wie Hügel oder Langbetten eingebracht oder leicht überhügelten Gruben
deponiert. Im Gegensatz zur Entwicklung in Süddeutschland wurden diese
Tradition bis in die Eisenzeit fortgesetzt\footnote{\textcite{jockenhovel_germany_2013},
  730-733.}.

\hypertarget{nordostfrankreich}{%
\subsubsection{Nordostfrankreich}\label{nordostfrankreich}}

Das bronzezeitliche Frankreich lässt sich in drei geographische und
kulturelle Regionen gliedern: Die Atlantikküste, die starke Impulse von
den Nordseeanrainern, Großbritannien und der Iberischen Halbinsel
erfuhr, Südfrankreich, das besonders von den Entwicklungen im
Westmediterranen Raum beeinflusst wurde und (Nord)ostfrankreich
ausgehend vom Pariser Becken. Die Bronzezeit in Frankreich dauert von
2300calBC bis 800calBC, wobei eine Aufteilung in Frühbronzezeit
(2300-1650calBC), Mittelbronzezeit (1650-1350calBC) und Spätbronzezeit
(1350-800calBC) üblich ist\footnote{\textcite{mordant_bronze_2013}, 571.}.

In Nordostfrankreich (in etwa die modernen, administrativen Regionen
Ile-de-France, Hauts-de-France, Grand-Est und Bourgogne-Franche-Comté)
begegnen sich in der Frühbronzezeit eine westliche Einflusssphäre aus
der heutigen Normandie und Bretagne, eine noch weiter nordöstlich
entlang der Nordsee gelegene Sphäre aus dem heutigen Benelux-Raum und
die Rhone Kultur aus dem Süden. Im Nordwesten Frankreichs wurden riesige
Grabhügel mit 40-50m Durchmesser und bis zu 5-6m Höhe errichtet. Diese
monumentalen Anlagen sind in der Regel nur mit einer einzigen, zentralen
Bestattung in einer großen Grabkammer versehen -- meist Männer und nur
in selten Fällen Frauen oder Kinder. Zusammen mit den außerordentlich
reichen Beigaben (Äxte, Hellebarden, Gold, Silber, Bernstein, Fayence)
sind sie Anzeiger für eine deutliche vertikale Gliederung der
Gesellschaft mit einzelnen, herausragenden Führungspersonen. Auch in
Nordostfrankreich gibt es frühbronzezeitliche Grabhügel, jedoch ist die
Erhaltungssituation erheblich schlechter als im Nordwesten. In diesen
Hügeln wurden sowohl Brand- als auch Körperbestattungen deponiert. Die
Urnen weisen Ähnlichkeiten zu Urnen aus Südengland und Flandern auf.
Obgleich die Umfassungen der Hügel Durchmesser von bis zu 100m erreichen
konnten, enthalten sie nur wenige oder keine Beigaben. Neben Grabhügeln
gab es in Nordostfrankreich zeitgleich auch Brandgräberfelder mit
einfachen Urnenbestattungen. Im südlichen Teil, im Einzugsbereich der
Rhone Kultur, in Burgund und Franche-Comté, kommen kleinere Grabhügel
mit 6-8m Durchmesser und 1m Höhe vor. Daneben wurden Neolithische
Megalithikanlagen sowie Höhlen weiter als Kollektivbestattungsplätze
genutzt. Ebenfalls in Burgund treten Einzelgräberfelder mit
Körperbestattungen in gestreckter und angehockter Lage auf. Insgesamt
ist die Beigabenauswahl in diesen Kontexten limitiert aber deutlich
geschlechtsspezifisch: Dolche, Äxte und Nadeln für Männer, Schmuck nur
für Frauen\footnote{\textcite{mordant_bronze_2013}, 571-572 \& 581.}.

In der Mittelbronzezeit, im Kontext einer massiven Zunahme von Quantität
und Qualität der Metallverarbeitung und Erfindungen wie dem Absatzbeil
und Schwertern, formt sich in Nord- und Westfrankreich eine große,
intensiv vernetzte Kulturregion, die auch Flandern und den Süden
Großbritanniens einschloss. In Bretagne und Normandie war die Errichtung
von herausragenden Individualgrabhügel in Fortsetzung einer Entwicklung,
die schon seit der fortgeschrittenen Frühbronzezeit zur Verkleinerung
der Hügel geführt hatte, weiter rückläufig und wurde schließlich
eingestellt. Darüber hinaus sind die Bestattungspraktiken in diesem Raum
weitestgehend ein Desiderat. Im Nordosten wurden die
Brandbestattungsfelder weitergeführt, die schon in der Frühbronzezeit in
diesem Raum aufgetreten waren. Sie enthalten wenige Beigaben, kein
Metall und höchstens ein Keramikgefäß als Urne. Südlich davon, in
Ostfrankreich, lässt sich eine starke Expansion der östlichen Tumulus
Kultur beobachten, die schließlich bis hinein ins Pariser Becken und ins
Loire Tal wirkte. Die übliche Bestattungform in diesem Kulturkreis war
das Hügelgrab mit mehreren Beisetzungen. Dabei konnten sowohl Körper-
als auch Brandbestattungen eingebracht werden -- letztere gewannen im
Laufe der Mittelbronzezeit an Bedeutung. Die Gräber sind reich mit
Beigaben versehen und zeigen hier eine klare Differenzierung nach
Geschlecht: Männer wurden mit Dolchen, Äxten und Pfeilspitzen, Frauen
mit Nadeln, Perlen und Armreifen ausgestattet. Das Ritual weist Bezüge
zu den kontemporären Praktiken in Süddeutschland auf\footnote{\textcite{mordant_bronze_2013},
  572-574 \& 581-582.}.

Am Übergang zur Spätbronzezeit konsolidierte sich die kulturelle
Ost-West Spaltung Frankreichs. Entlang der Atlantikküste ausgehend von
der Iderischen Halbinsel bis in den Beneluxraum hinein bestand der schon
zuvor etablierte, atlantische Kulturkomplex fort. Auch in Ostfrankreich
lassen sich die Entwicklungen vor dem Hintergrund der bereits in der
Mittelbronzezeit nachvollziehbaren Prozessen verstehen. In einem großen
Areal zwischen und jeweils jenseits von Rhein und Pariser Becken
dominierte der Rhin-Suisse-France orientale (RSFO) Kulturkomplex.
Wiederrum war die Situation in Nordostfrankreich geprägt von zwei
Einflusssphären. Neben traditionelleren Grabformen, wie etwa
Körperbestattungen in Steinkisten, wurde in Ostfrankreich im RSFO Raum
zunehmend Brandbestattung auf Urnenfeldern praktiziert. Auch im
Nordosten war die Brandgrabsitte zunehmend präsent und wurden ab dem 12.
Jahrhundert fast universell. Im westlichen Teil hielt sich die Tradition
sehr einfacher Brandgräber und karger Beigaben, die schon in der
Mittelbronzezeit Verbreitung gefunden hatte. Neben flachen Einzelgräbern
wurden für gesellschaftlich herausragende Individuen -- meist Männer --
auch Grabhügel errichtet. Sie sind mit einer Vielzahl von Beigaben
versehen, manchmal mit einem Schwert oder Wägeausrüstung. Hügelgräber
gewannen in Ostfrankreich zum Ende der Bronzezeit im 9. Jahrhundert
wieder an Bewandtnis\footnote{\textcite{mordant_bronze_2013}, 574-575 \&
  582-583.}.

\hypertarget{sudskandinavien}{%
\subsubsection{Südskandinavien}\label{sudskandinavien}}

Das wichtigste chronologische Instrument der skandinavischen Bronzezeit
ist die -- freilich weiterentwickelte -- Periodengliederung nach Oskar
Montelius. Der Begriff Frühbronzezeit wird für die Perioden I bis III
verwandt, die etwa das Zeitfenster von Reineckes Phasen A2 bis Ha A
abdecken. Spätbronzeit umfasst die Perioden IV bis VI, die von Ha B1 bis
C reichen. Südskandinavien meint Dänemark sowie die schwedischen
Provinzen Scania und Blekinge. Im Vergleich zu nördlicheren Teilen
Skandinaviens bieten sich hier naturgeographisch günstigere Bedingungen
für Ackerbau, was die Region in der Bronzezeit in die Rolle eines
kulturellen Zentrums für Skandinavien versetzt. Im Vergleich zum
restlichen Europa fällt die Nordische Bronzezeit vor allem durch ihre
außergewöhnlich intensive und stilistisch einzigartige
Metallverarbeitung, die Erhaltung vieler tausend Grabhügel und eine
auffällige, weit verbreitete Form der Felskunst auf\footnote{\textcite{thrane_scandinavia_2013},
  746-750.}.

Aus der Nordischen Bronzezeit sind viele Gräber erhalten -- besonders
aus den Perioden II bis IV --, die es erlauben die Entwicklung der
Bestattungssitten gut nachzuzeichnen. Bestattungen in einigen Grabhügeln
Zentraljütlands sind unter Feuchtbodenbedingungen erhalten und damit ein
hervorragendes archäologisches Archiv. Körperbestattungen (meist) in
Grabhügeln waren die Regel, bis in Periode II sporadisch einzelne
Brandbestattungen -- ebenfalls meist in Hügel eingebracht -- auftraten.
Ab Periode III war die Brandbestattung universell, sieht man von der
Situation auf der Insel Gotland ab. Der Übergang zur Bronzezeit in
Südskandinavien vollzog sich in verschiedenen Lebensbereichen langsam.
Neben dem Hausbau, wo weiter die schon im Neolithikum gebräuchlichen
Langhäuser errichtet wurden, und der Flintproduktion, zeigt sich das
auch in der Grabanlage, die an alte Tradition anknüpfte. In Dänemark und
Südschweden wurden die zu bestattenden Leichname in der Frühbronzezeit
in Baumsärgen -- meist ausgehöhlte Eichenstämme -- deponiert. Die Toten
wurden gestreckt auf den Rücken gelegt und mit dem Kopf nach Westen
orientiert. In Periode III änderte sich das hin zu einer Nord-Süd
Orientierung. Oft wurde der Tote innerhalb des Sarges auf eine
Ochsenhaut oder eine Wolldecke gebettet. Über manchen Gräbern wurden
Totenhäuser errichtet und manche Grabhügel bedecken ein Langhaus. In
Periode IV vollzog sich eine Wende hin zu kleineren Grabgruben --
Brandbestattungen brauchen weniger Platz. Große Beigaben wie Schwerter,
die zuvor mit den Toten abgelegt worden waren, entfallen damit
gleichermaßen. Im Laufe der Perioden IV bis VI wurde der Leichenbrand
zunehmend in Urnen deponiert, aber Steinkisten und einfache Erdgruben
kommen dennoch parallel weiter vor. Die Beigabenmenge nimmt im Zuge
dieser Entwicklung weiter ab: Die meisten Brandbestattungen sind nicht
mit Metallartefakten ausgestattet\footnote{\textcite{thrane_scandinavia_2013},
  754-756.}.

Das herausragendste Merkmal der Bestattungskultur der Nordischen
Bronzezeit sind ihre Grabhügel. Aus der Bronzezeit sind schätzungsweise
100.000 Erdhügel erhalten, die meisten in Südskandinavien und aus den
Perioden II bis III. Viele sind durch landwirtschaftliche Aktivität
gefährdet. In der Größe variieren sie zwischen sehr klein (5m
Durchmesser, 0,5m Höhe) und sehr groß (35m Durchmesser, 6m Höhe), wobei
sie im mit im Durchschnitt 25m Durchmesser und 2,5m Höhe insgesamt
beachtlich ausfallen. Neben diesen Erdhügeln gibt es auch etwa 30.000
Steinhügel, vor allem in den Regionen nördlich von Südskandinavien. Die
größten Exemplare in dieser Kategorie messen mehr als 70m im Durchmesser
und sind mehr als 10m hoch. Sie gehören damit zu den größten
vorgeschichtlichen Bauwerken Europas. In der südskandinavischen
Frühbronzezeit wurden Gräber häufig mit einer kleinen Steinpackung
überhügelt und auf diesem Steinhügel anschließend ein Erdhügel aus
Grassoden aufgebaut. In Dänemark gibt es auch einige spätbronzezeitliche
Steinhügel. Während die frühbronzezeitlichen Hügel häufig auf
natürlichen Erhebungen und Geländerücken errichtet wurden, finden sich
die späteren Hügel tendentiell eher in Niederungslagen. Abgesehen von
einigen Sonderformen wie Langbetten, Hügeln mit einer zum Plateau
abgeflachten Spitze und schiffsförmigen Steinsetzungen sind die Hügel
äußerlich größenunabhängig sehr ähnlich. Die Anlage der Gräber und
Steinsetzungen in ihrem Inneren unterscheidet sich jedoch deutlich von
Hügel zu Hügel. Ein Hügel kann in ein oder mehreren Bauphasen errichtet
und für spätere Bestattungen erweitert worden sein. Oft findet sich
jedoch auch in sehr großen Hügeln nur eine einzige Bestattung.
Neolithische Anlagen konnten in der Bronzezeit Weiternutzung oder Ausbau
erfahren haben, während bronzezeitliche Hügel selbst mitunter bis weit
in die Eisenzeit hinein als Grabanlagen genutzt wurden.
Spätbronzezeitliche Urnen wurden häufig in bestehende Hügel eingebracht,
sodass viele Hügel Gräber aus zwei deutlich getrennten Phasen
enthalten\footnote{\textcite{thrane_scandinavia_2013}, 752-754.}.

Trotz der vielen erhaltenen Grabanlagen ist die Erforschung der
bronzezeitlichen Bevölkerung auf Grundlage dieser Grabbefunde ein
Desiderat. Die Erhaltungssituation der Knochen ist insgesamt schlecht.
Zwischen Männern und Frauen sind keine systematischen Unterschiede beim
Bestattungsbrauch erkennbaren, abgesehen von den etwas zahlreicheren
Grabbeigaben in Männergräbern. Jedoch zeigt der Vergleich der
Geschlechtsbestimmung von Brandgräbern aufgrund
physisch-anthropologischer Kriterien gegenüber derjenigen aufgrund von
Beigaben erstaunliche Abweichungen, die die diachrone Relevanz dieses
Ergebnisses in Frage stellen. Die Beigabenmenge und -qualität scheint
immerhin Aufschluss über die soziale Gliederung zu geben, da sich
jenseits von einigen wenigen, absolut herausragenden ``Fürstengräbern''
auch regelmäßige Ausstattungsklassen andeuten: Gräber mit Goldbeigaben
und Waffen lassen sich von einfacheren mit verzierten Messern,
Rasiermessern und Pinzetten unterscheiden. Die Mehrzahl der Gräber
enthält kein oder nur ein Metallobjekt. Der Kontrast zwischen Arm und
Reich ist in der Spätbronzezeit besonders akzentuiert. Kindergräber sind
sehr selten -- erst in der Spätbronzezeit finden sich vereinzelt in
Urnen neben den Überresten eines Erwachsenen auch die eines Kindes.
Nicht alle Bestattungen sind überhügelt oder in Hügel eingebracht, die
Flachgräber verhalten sich aber abgesehen davon ebenso wie die
Hügelgräber. In der Spätbronzeit wurden in Südskandinaviens auch
Flachgräberfelder angelegt, je weiter südlich, desto größer\footnote{\textcite{thrane_scandinavia_2013},
  756-758.}.

\hypertarget{benelux}{%
\subsubsection{Benelux}\label{benelux}}

Im Benelux-Areal, das hier neben den Niederlanden, Belgien und Luxemburg
auch Teile Nordostfrankreichs, Frieslands und des Rheinlands
einschließen soll, begegnen sich in der Bronzezeit drei Einflussspähren:
Die Nordische, die Kontinentale und die Atlantische Bronzezeit. Die
lokale, kulturelle Entwicklung ist in dieser Konsequenz kleinteilig und
abwechselungsreich. Entscheidenden Einfluss darauf hatte die
geographische Gliederung des Areals, wobei zwei wesentliche Dichotomien
zu beachten sind: Der Gegensatz von Fluss- und Küstenniederungen
gegenüber Pleistozänem Hochland und die Trennung in Areale nördlich und
südlich der Mündungsgebiete von Maas, Rhein und Ijssel. Im
Spätneolithikum und der Bronzezeit gehören die Areale nördlich und
östlich dieser Flüsse dem Austauschnetzwerk der Nordischen Bronzezeit
zu. Südlich und Westlich der Flüsse überwiegt der Einfluss aus Nord- und
Westfrankreich sowie Großbritannien. Das Kalksteinplateau in
Südostbelgien einschließlich der Ardennen lässt sich der kontinentalen
Sphäre zurechnen\footnote{\textcite{fokkens_bronze_2013}, 550-551.}.

Für das Spätneolithikum bis 2500calBC lässt sich die Situation
vereinfacht folgendermaßen darstellen: Die späte Vlardingen-Kultur
besiedelte die nördlichen Niederlande und die Stein Gruppe die
Maas-Niederung bis zur belgischen Grenze. In den pleistozänen Höhenlagen
fand sich die Einzelgrabkultur, die die Trichterbecherkultur an dieser
Position abgelöst hatte. Ab 2500calBC war das gesamte Areal bis in die
Ardennen Teil des Glockenbecherkomplexes, wobei sich die vormaligen
kulturellen und geographischen Einheiten auch hier in Subgruppen
auszudrücken scheinen. Diese Regionalität ĺöste sich nicht auf, sondern
wurde am Ende der Frühbronzezeit im Kontext der späten
Glockenbecherkultur in der Verteilung von Leitformen wie
Wickelschnurkeramik und Riesenbecher erneut sichtbar. Ab 1850calBC, in
der Mittelbronzezeit, bildeten sich im Nordosten, Westen und Süden der
Niederlande neue Keramikstile heraus: Im Norden -- nördlich und östlich
von Ijssel und Vechte -- Elp Keramik, südlich der Ijssel Hilversum
Keramik sowie später Drakenstein Keramik mit starken Einflüssen aus
Südengland und Nordwestfrankreich, im Westen Hoogkarspel Keramik. Ab
1200 war der Benelux-Raum Teil des Urnenfelder-Phänomens. Insbesonder
dank der für Urnen verwandten Gefäße lassen sich aber wiederrum
innerhalb desselben deutlich Keramikstile unterscheiden: Im Nordosten
die Ems Gruppe, im Süden der Niederlande Einflüsse aus dem Norddeutschen
Raum und im Süden Belgiens eine starke Orientierung an der
Rhin-Suisse-France Oriental (RSFO) Tradition. Im Benelux Raum dauert die
Bestattung auf Urnenfeldern bis zum Ende der frühen Eisenzeit und damit
länger als in anderen Regionen Europas an\footnote{\textcite{fokkens_bronze_2013},
  552-553.}.

In großen Teilen des Benelux Gebiets wurden ab 2900calBC Grabhügel
errichtet, wobei diese Tradition in einzelnen Regionen, wie etwa
Nordbelgien, erst ab 2600 bzw. 2000calBC zur Regel wurde. Sie hielt dann
bis 1400calBC an. Im Norden und Osten des Benelux Raums waren bis
1200calBC Körperbestattungen, ab der Mittelbronzezeit in gestreckter
Rückenlage, die Regel. Ab 1200 überwiegt die Brandbestattung in
Urnenfeldertradition. Südlich der Maas findet ein anderer Prozess statt:
Brandbestattungen wurden hier bereits im Spätneolithikum im
Glockenbecherkontext praktiziert. Schon in der Mittelbronzezeit, also
mehrere Jahrhunderte vor der Entstehung des Urnenfelderphänomens, wurde
die Brandbestattung die vorherrschende Bestattungssitte. Im Westen der
Niederlande vollzieht sich wiederrum eine andere, lokale Entwicklung:
Nach 1600calBC wurden in Westfriesland keine Grabhügel mehr angelegt.
Auch Urnenfelder kommen hier nicht vor. Da die Region besiedelt war und
der Forschungsstand als gut gelten darf, muss davon ausgegangen werden,
dass hier ein abweichender Bestattungsritus praktiziert wurde, der
archäologisch nicht oder nur schwer zu erfassbar ist\footnote{\textcite{fokkens_bronze_2013},
  557-558.}.

Verglichen mit den Hügeln der Nordischen Bronzezeit fielen die Grabhügel
im Benelux Raum klein aus. Selten übersteigt ihr Durchmesser 15m und sie
sind sämtlich weniger als 1,5m hoch. Um den Hügel herum wurden
Begrenzungs- und Einhegeanlagen in Form von dicht oder locker gepackten
Pfostensetzungen und/oder flachen Gräben errichtet. Diese Anlagen zeigen
eine große Variabilität -- Regionalgruppen deuten sich nicht an. Die
Parameter Beigabenqualität und -quantität, Hügelgröße und Komplexität
der Einhegeanlage scheinen voneinander unabhängig zu sein. Grabhügel
bildeten landschaftliche Bezugspunkte, in deren Nähe überproportional
häufig Siedlungen anlegt, weitere Hügel errichtet oder in die weitere
Gräber eingebracht wurden. Auch Urnenfelder wurden häufig um ältere
Grabhügel herum angelegt. Zwischen den einzelnen Bestattungsereignissen
in einem Hügel konnten lange Zeiträume vergehen, was darauf hindeutet,
dass nur Mitglieder einer sozialen Elite in Grabhügeln beigesetzt
wurden. Die Beigabenarmut vieler Bestattungen, besonders südlich der
Maas, stellt diese Deutung jedoch in Frage. Nördlich des Rhein und im
Nordwesten der Niederlande trat in der Mittelbronzezeit eine
standardisierte Form der Männerbestattung auf. Die Leichname wurden als
Rückenstrecker deponiert und mit einem Rapier, einem Randleistenbeil und
manchmal einer Speerspitze, Arm- oder Haarringen, Pinzetten, einer
Rasierklinge und Pfeilspitzen ausgestattet. Nicht zuletzt aufgrund der
hohen Standardisierung der Beigaben ist die häufig vorgenommene
Assoziation dieser Bestattungen mit dem Begriff ``Fürstengrab''
zweifelhaft\footnote{\textcite{fokkens_bronze_2013}, 558-561.}.

In der Spätbronzezeit ab 1200calBC bis 800calBC wandelte sich die
vormalige Praxis, nur einen kleinen Teil der Verstorbenen archäologisch
fassbar beizusetzen. Stattdessen wurden Urnenfelder mit einer großen
Zahl von Bestattungen errichtet. Die Urnen wurden zunächst in große
Langbetten eingebracht, nach 1000calBC dann in vereinheitliche, kleine
Hügel mit einem flachen Umfassungsgraben und einer Rampe auf der
Südostseite. Obgleich sehr große Urnenfelder existieren ist die Mehrzahl
überschaubar und deutet auf eine Population hin, die sich aus drei oder
vier Familien gespeißt haben könnte\footnote{\textcite{fokkens_bronze_2013},
  561-562.}.

\hypertarget{england}{%
\subsubsection{England}\label{england}}

Die Bronzezeit in Großbritannien und Irland erstreckte sich über einen
Zeitraum von 2500-800/600calBC. Das schließt jedoch auch das lokale
Chalcolithikum ein, das bis 2150calBC reicht. Die Innovation der
Bronzemetallurgie verbreitete sich, erst einmal entdeckt, äußert
schnell. Die Frühbronzezeit dauerte nach britischer Terminologie von
2150-1500calBC, die Mittelbronzezeit von 1500-1150calBC und die
Spätbronzezeit schließlich von 1150calBC bis 800/600calBC. Die Insellage
hebt die Region deutlich von den oben betrachteten Fällen ab und ist
Grund für den Sonderweg, den die Entwicklung in Großbritannien und
Irland im Vergleich zum Festland ging. Gleichermaßen bestanden aber auch
vielfältige und tiefgreifende Verbindungen insbesondere im Kontext der
Atlantischen Bronzezeit nach Nordwestfrankreich und ebenfalls entlang
der in die Nordsee entwässernden Flüsse bis nach Zentraleuropa hinein.
Sowohl Irland als auch Großbritannien sind durch ihre lange,
abwechslungsreiche Küstenlinie geprägt. Großbritannien ist naturräumlich
in landwirtschaftlich gut nutzbare Niederungslagen im Süden und Osten,
in England, gegenüber unwirtlicheren Hochebenen im Westen und besonders
im Norden, also in Wales und Schottland gegliedert. Doch auch in Wales
und Schottland stechen einzelne Regionen durch hohes ackerbauliches
Potential hervor. Irlands Küste ist vielerorts hoch und schroff. Sie
umschließt nieder gelegenes Land, dessen Potential für Landwirtschaft
ebenfalls im Süden und Osten am höchsten ist\footnote{\textcite{roberts_britain_2013},
  531-533.}.

Im Gegensatz zu Frankreich, wo Kupfermetallurgie schon im 4. Jahrtausend
bekannt war, traten die frühesten Kupferartefakte in Großbritannien und
Irland erst ein Jahrtausend später um 2500/2400calBC auf. Sie waren Teil
eines Innovationspakets aus Zentraleuropa, das neben
Glockenbecherkeramik, steinernen Armschutzplatten, gestielten und mit
Widerhaken versehenen Pfeilspitzen auch Kupferdolche und goldene
Körpchenanhänger umfasste. In Großbritannien bildete sich dieses
Ereignis in einer neuen lokalen Glockenbecher Tradition ab, die sich
durch Einzelkörperbestattungen auszeichnete. In Irland führte der
Glockenbechereinfluss nicht zu einer so tiefgreifenden Veränderung der
Bestattungssitte, allerdings wurde auch hier Glockenbecherkeramik Teil
von Handlungen im Kontext des Totenrituals: Sie wurde an älteren
Grabanlagen deponiert und in neue Brandgräber in Grabhügeln eingebracht.
Sowohl in Großbritannien als auch in Irland war die bronzezeitliche
Landschaft geprägt von intentional positionierten und aufeinander
ausgerichteten Monumentalanalagen. Das schließt die neuen Grabhügel für
Glockenbecherbestattungen in Großbritannien und Keilgräber in Irland
ebenso ein wie eine Vielzahl unterschiedlicher, nichtfunktionaler Erd-,
Holz- oder Steinanlagen und neolithische Henges und Megalithanlagen. Wie
in Skandinavien entstand in Großbritannien Felskunst, die ebenfalls der
komplexen Rituallandschaft zugerechnet werden muss. Die Grabhügel und
Monumentalanlagen in Süd- und Nordostengland, Zentralschottland sowie
Ostirland und dem Orkney Archipel sind vergleichsweise größer als in
anderen Regionen, strukturell aber ähnlich\footnote{\textcite{roberts_britain_2013},
  533-535.}.

Im ausgehenden 3. und beginnenden 2. Jahrtausend überwogen in Südengland
Körperbestattungen in Grabhügelgruppen nach dem Modell der Wessex
Kultur. Die Gräber sind teilweise reich und mit exotischen Beigaben
ausgestattet. Außerhalb von Südengland herrscht eine große regionale
Variabilität von Bestattungssitten in Großbritannien und Irland: In
Nordengland hatten Bestattungen in Höhlen einige Bedeutung, in
Ostengland Moorbestattungen. In großen Teilen Westschottlands und
Irlands wurden Flachgräberfelder angelegt. Die Gräber sind einfache
Gruben oder mit einer Steinkiste ausgebaut. In Irland wurden zudem auch
ältere Keilgräber wiederbelegt und herausragend große Grabhügel neu
errichtet. Auch hinsichtlich der Beigaben gab es klare regionale
Vorlieben für einzelne Artefaktkategorien: Gagatcolliers in weiblichen
Bestattungen im Norden Großbritanniens, Bernsteincolliers im Süden,
sowie lokal Bronze- oder Feuersteindolche. In den verschiedenen
Kontexten wurde häufig sowohl Brand- als auch Körperbestattung
praktiziert, wobei die Verbrennung der Verstorbenen ab dem Beginn des 2.
Jahrtausends allgemein zu überwiegen scheint. Grabanlagen wurden häufig
über lange Zeiträume genutzt oder deutlich nach deren ursprünglicher
Anlage wieder neu belegt. Auch Ritualanlagen wie Holz-, Stein- und
Menhirkreise sowie Ring Cairns sind regelmäßig mit Bestattungen
assoziiert\footnote{\textcite{roberts_britain_2013}, 535-536.}.

Mitte des 2. Jahrtausends vollzogen sich sowohl in Großbritannien als
auch in Irland, ganz besonders jedoch im Süden und Osten Englands, eine
Reihe von Veränderungen. Die archäologische Quellenlage verschiebt sich
in diesem Zeitfenster zuungunsten von Grab- und Monumentalanlagen.
Stattdessen nimmt die Informationsdichte hinsichtlich Siedlungen und
Subsitenz zu. Jetzt häufiger befestigte Siedlungen mit kreisförmigem
Aufriss, Rundhäuser und regelmäßige, rechtwinklige Ackersysteme mit
begrenzenden Gräben und Zäunen begannen die Kulturlandschaft zu
dominieren. Wilde Pflanzen und Tiere verloren an Bedeutung in der
Subsistenz, stattdessen ist mit der Einführung von Brunnen sowie neuen
Getreiden und Hülsenfrüchten eine Intensivierung der Landwirtschaft zu
beobachten. Stein, Flint und andere organische und inorganische
Werkstoffe wurden als wichtigste Träger der materiellen Kultur von Gold
und Bronze verdrängt. Letztere sind archäologisch allerdings auch
deswegen viel sichtbarer, da sich eine ausgeprägte Deponierungstradition
etablierte. Diese bildet sich jedoch nicht im stark fragmentierten und
regional heterogenen Bestattungsbefund ab. Ab der Mitte des 2. bis zum
Anfang des 1. Jahrtausends war die Brandbestattung die am weitesten
verbreitete Bestattungsform. In vielen Regionen Englands wurden
siedlungsnahe Flachgräberfelder angelegt. Die Beigabenauswahl war fast
vollständig auf Gefäßkeramik beschränkt. In Irland wurden Urnengräber
angelegt, parallel allerdings auch weiter bis ins erste Jahrtausend
Grabhügel errichtet. Die älteren Tradition wurden langsam durch eine
sehr flexible Praxis ersetzt, nach der der Leichenbrand zusammen mit
unverzierten, groben Keramikgefäßen in Flachgräberfeldern, Grabhügeln,
Gräben und sogar Siedlungen beigesetzt wurde. Neben den eigentlichen
Brandbestattungen kommen auch Pseudogräber verhältnismäßig häufig vor.
Sie wurden unter anderem auf Gräberfeldern deponiert und enthalten große
Mengen verbranntes Getreide. In Schottland wurden die Brandbestattungen
meist in ältere Monumente eingebracht. Im frühen ersten Jahrtausend
wurde vielerorts kein archäologisch fassbares Bestattungsritual mehr
praktiziert. Die Bestattungsbefunde beschränken sich auf kleine Mengen
verbrannter menschlicher Knochen, die in Siedlungen, Gräben oder auf
Äckern sporadisch auftreten\footnote{\textcite{roberts_britain_2013},
  537-542.}.

\hypertarget{fragestellungen}{%
\section{Fragestellungen}\label{fragestellungen}}

\ldots{}

\pby[title={Literatur},segment=\therefsegment,heading=subbibintoc]

\hypertarget{datenauswertung}{%
\chapter{Datenauswertung}\label{datenauswertung}}

\hypertarget{software-und-daten}{%
\section{Software und Daten}\label{software-und-daten}}

Die vorliegende Arbeit wurde in fünf verschiedenen Teilprojekten
entwickelt und hat mindestens vier Hilfsprojekte hervorgebracht oder
inspiriert (siehe Tabelle \ref{tab:projects}). Alle diese Projekte
wurden und werden unabhängig voneinander mit der
Versionskontrollsoftware Git\footnote{\url{https://git-scm.com/}
  {[}31.07.2018{]}} überwacht, die den Arbeitsfortschritt in vielen
hundert einzeln kommentierten Veränderungspaketen -- ``Commits'' --
dokumentiert. Der Entstehungsprozess ist damit weitreichend
nachvollziehbar, sieht man von Vorüberlegungen und Gesprächen ab, die
keine konkreten Ergebnisse gezeitigt haben. Nach Abschluss von Korrektur
und Revision der Arbeit, werden alle Projekte über die Cloud Plattform
Github \footnote{\url{https://github.com/} {[}31.07.2018{]}} zugänglich
gemacht werden. Die im Text verarbeitete und darüber hinaus gesammelte
Literatur ist in drei thematisch getrennte Sammlungen gegliedert (siehe
Tabelle \ref{tab:libraries}) und über das zotero Webportal einsehbar.

\begin{table}

\caption{\label{tab:projects}Projekte und Pakete für diese Arbeit.}
\centering
\fontsize{8}{10}\selectfont
\begin{tabu} to \linewidth {>{\bfseries\raggedleft\arraybackslash}p{0.5em}>{\raggedright\arraybackslash}p{25em}}
\toprule
 & Projekt\\
\midrule
1 & \textbf{neomod\_textdev}\newline \textit{\href{https://www.github.com/nevrome/neomod\_textdev}{github/nevrome/neomod\_textdev}}\newline Textproduktion. Der Text der Masterarbeit wurde in R Markdown\tablefootnote{\url{https://rmarkdown.rstudio.com/ [31.07.2018]}} mit im bookdown Framework\tablefootnote{\textcite{xie_bookdown_2016}; \textcite{xie_bookdown_2018}; \url{https://bookdown.org/ [31.07.2018]}} verfasst. Da ausschließlich das Rendern mittels Pandoc\tablefootnote{\url{https://pandoc.org/ [31.07.2018]}} in LaTeX\tablefootnote{\url{https://www.latex-project.org/ [31.07.2018]}} ins PDF Format vorgesehen war, enthält die Textvorlage auch vereinzelt LaTeX Ausdrücke. Jeder Commit löst dank Continous Integration mit Travis\tablefootnote{\url{https://travis-ci.com/ [31.07.2018]}} ein automatisches Rendern des Texts aus.\\
\addlinespace \hline \addlinespace
2 & \textbf{neomod\_prepresentation}\newline \textit{\href{https://www.github.com/nevrome/neomod\_prepresentation}{github/nevrome/neomod\_prepresentation}}\newline Präsentationen über die Inhalte der Masterarbeit. Vor- während und nach der Arbeit wurden mehrere Präsentation über Planung, Arbeitsfortschritt und Ergebnisse zusammengestellt. Die Präsentationen sind jeweils in R Markdown konstruiert, unterscheiden sich aber je nachdem, ob ein Rendern in HTML oder PDF vorgesehen war.\\
\addlinespace \hline \addlinespace
3 & \textbf{neomod\_analysis}\newline \textit{\href{https://www.github.com/nevrome/neomod\_analysis}{github/nevrome/neomod\_analysis}}\newline Datenanalyse. Sowohl die Realweltdaten als auch die Daten aus der Simulation wurden mit R ausgewertet. Dieses Projekt hat bewusst nicht die Form eines R Pakets, sondern setzt sich aus vielen einzelnen, teilweise redundanten R Skripten zusammen. Hier werden auch Ergebnisdaten und Abbildungen gespeichert. Erstere wurden aufgrund ihres Volumens weitestgehend  nicht mit Versionskontrolle dokumentiert und liegen entsprechend nur in den lokalen Systemen vor, in denen sie erzeugt wurden. Sie müssen bei Bedarf generiert oder -- im  Fall von Quelldaten -- heruntergeladen werden.\\
\addlinespace \hline \addlinespace
4 & \textbf{popgenerator\tablefootnote{\textcite{schmid_popgenerator_2018}}}\newline \textit{\href{https://www.github.com/nevrome/popgenerator}{github/nevrome/popgenerator}}\newline Populationsgenerator. R Paket zur Konstruktion von Populationsgraphen.\\
\addlinespace \hline \addlinespace
5 & \textbf{gluesless\tablefootnote{\textcite{clemens_schmid_gluesless_2018}}}\newline \textit{\href{https://www.github.com/nevrome/gluesless}{github/nevrome/gluesless}}\newline Expansionsmodell. C++ Programm zur Simulation der Ausbreitung von Ideen in einem Populationsgraphen, wie er von popgenerator erzeugt wird.\\
\addlinespace \hline \addlinespace
6 & \textbf{c14bazAAR\tablefootnote{\textcite{schmid_c14bazaar_2018}}}\newline \textit{\href{https://www.github.com/nevrome/c14bazAAR}{github/nevrome/c14bazAAR}}\newline \textsuperscript{14}C-Datenbeschaffung. R Paket zum strukturierten Download von \textsuperscript{14}C-Daten aus verschiedenen Quelldatenbanken -- unter anderem der hier verarbeiteten Radon-B Datenbank.\\
\addlinespace \hline \addlinespace
7 & \textbf{neimann1995}\newline \textit{\href{https://www.github.com/nevrome/neimann1995}{github/nevrome/neimann1995}}\newline Reproduktion eines Artikels. Verständnisübung entlang eines der wesentlichen Artikel in der theoretischen Vorbereitung dieser Arbeit.\\
\addlinespace \hline \addlinespace
8 & \textbf{rdoxygen}\newline \textit{\href{https://www.github.com/nevrome/rdoxygen}{github/nevrome/rdoxygen}}\newline Doxygen Dokumentation. R Paket um Doxygen Dokumentation für Source Code in R Paketen zu erstellen.\\
\addlinespace \hline \addlinespace
9 & \textbf{txtstorage}\newline \textit{\href{https://www.github.com/nevrome/txtstorage}{github/nevrome/txtstorage}}\newline Textdatenspeicher. R Paket zur Verwaltung vo Austauschdateien mit einfachen Analyseergebnissen. Dient vor allem dazu, Zähldaten dynamisch in den Text der Arbeit einzubinden.\\
\bottomrule
\end{tabu}
\end{table}

\begin{table}

\caption{\label{tab:libraries}Literatursammlungen für diese Arbeit.}
\centering
\fontsize{8}{10}\selectfont
\begin{tabu} to \linewidth {>{\bfseries\raggedleft\arraybackslash}p{0.5em}>{\raggedright\arraybackslash}p{25em}}
\toprule
 & Bibliothek\\
\midrule
1 & \textbf{cultural\_evolution}\newline \textit{\href{https://www.zotero.org/groups/2086516/cultural\_evolution}{zotero/cultural\_evolution}}\newline Cultural Evolution. Literatursammlung zu Cultural Evolution und ihren vielen Subthemen wie Memetik, Cultural Transmission oder Sozial Learning. Geht weit über eine rein archäologische Perspektive hinaus, umfasst aber gleichzeitig archäologische Fallstudien ohne großen theoretischen Selbstanspruch.\\
\addlinespace \hline \addlinespace
2 & \textbf{bronze\_age\_burials}\newline \textit{\href{https://www.zotero.org/groups/2199051/bronze\_age\_burials}{zotero/bronze\_age\_burials}}\newline Bestattungssitten in der Bronzezeit. Archäologische Literatur zur Theorie der Thanatoarchäologie und zur kulturhistorischen Entwicklung in der Bronzezeit.\\
\addlinespace \hline \addlinespace
3 & \textbf{software\_packages}\newline \textit{\href{https://www.zotero.org/groups/2211203/software\_packages}{zotero/software\_packages}}\newline Software. Referenzen von wissenschaftlicher Software, die für Datenverarbeitung sowie Text- und Abbildungsvorbereitung zum Einsatz gekommen ist. Vor allem R Pakete und C++ Bibliotheken.\\
\bottomrule
\end{tabu}
\end{table}

Die gesamte Datenanalyse wurde in der Statistikprogrammiersprache
R\footnote{\textcite{RCoreTeamLanguageEnvironmentStatistical2016}}
implementiert. Dabei kam neben Funktionen aus Basispaketen auch eine
große Anzahl von Community-Paketen zum Einsatz, inklusiver mehrerer
selbst entwickelter. Aufgrund der sonst unangemessen großen Menge an
Referenzen, werden im folgenden nur die Pakete genannt, deren Funktionen
tatsächlich unmittelbar aufgerufen wurden und nicht deren oft
umfangreiche Sammlung an Abhängigkeiten. Die Zusammenstellung umfasst
jedoch auch Pakete, die im Laufe der Entwicklung intensiv zum Einsatz
kamen, dann aber aufgrund inhaltlicher oder technischer Veränderungen
ersetzt werden mussten. Über die Entwicklungszeit dieser Arbeit haben
sich viele der verwendeten Pakete ebenfalls weiterentwickelt -- die
angebene Versionsnummer bezieht sich auf diejenige, mit der die in
Version 1.0 dieser Arbeit abgedruckten Ergebnisse erstellt wurden.
Folgende Pakete kamen zum Einsatz zur Text- und Literaturverarbeitung
(bookdown\footnote{\textcite{xie_bookdown_2016};
  \textcite{xie_bookdown_2018}}, citr\footnote{\textcite{aust_citr_2017}},
knitr\footnote{\textcite{xie_dynamic_2015}; \textcite{xie_knitr_2014};
  \textcite{xie_knitr_2018}}, markdown\footnote{\textcite{allaire_markdown_2017}},
rmarkdown\footnote{\textcite{allaire_rmarkdown_2018}}), zur
Datenbeschaffung (c14bazAAR, rnaturalearth\footnote{\textcite{south_rnaturalearth_2017}}),
zur allgemeinen Datenmanipulation (broom\footnote{\textcite{robinson_broom_2018}},
dplyr\footnote{\textcite{wickham_dplyr_2018}}, forcats\footnote{\textcite{wickham_forcats_2018}},
kableExtra\footnote{\textcite{zhu_kableextra_2018}}, pbapply\footnote{\textcite{solymos_pbapply_2018}},
plyr\footnote{\textcite{wickham_split-apply-combine_2011}},
purrr\footnote{\textcite{henry_purrr_2018}}, readr\footnote{\textcite{wickham_readr_2017}},
stringi\footnote{\textcite{gagolewski_r_2018}}, stringr\footnote{\textcite{wickham_stringr_2018}},
tibble\footnote{\textcite{muller_tibble_2018}}, tidyr\footnote{\textcite{wickham_tidyr_2018}},
reshape, reshape2), zur Graphikerstellung (cowplot\footnote{\textcite{wilke_cowplot_2018}},
ggplot2\footnote{\textcite{wickham_ggplot2_2016}}, gridExtra\footnote{\textcite{auguie_gridextra_2017}},
png\footnote{\textcite{urbanek_png_2013}}), für geographische Analysen
(raster\footnote{\textcite{hijmans_raster_2017}}, sf\footnote{\textcite{pebesma_sf_2018}},
sp), für statistische Analysen und Spezialdatenverarbeitung
(Bchron\footnote{\textcite{haslett_simple_2008}}, car\footnote{\textcite{fox_r_2011}},
vegan\footnote{\textcite{oksanen_vegan_2018}}), zur Erstellung von
WebApps und Interaktiven Präsentationen im Shiny Framework
(shiny\footnote{\textcite{chang_shiny_2018}}) und zur allgemeinen Arbeit
und Paketentwicklung in R (devtools\footnote{\textcite{wickham_devtools_2018}},
magrittr\footnote{\textcite{bache_magrittr_2014}}, pryr\footnote{\textcite{wickham_pryr_2018}},
Rcpp\footnote{\textcite{eddelbuettel_extending_2017};
  \textcite{eddelbuettel_rcpp_2011};
  \textcite{eddelbuettel_seamless_2013}}, rlang\footnote{\textcite{henry_rlang_2018}},
roxygen2\footnote{\textcite{wickham_roxygen2_2017}}, testthat\footnote{\textcite{wickham_testthat_2011}}).

Die Expansionssimultion gluesless ist in C++\footnote{\textcite{standard-cpp-foundation_international_2017}}
umgesetzt, um auf dessen höhere Geschwindigkeit und bessere Werkzeuge
für objektorientiertes Programmieren zurückgreifen zu können. Zur
Abbildung des Populationsgraphen kam zunächst die Boost Graph
Library\footnote{\url{https://www.boost.org/doc/libs/1_67_0/libs/graph}
  {[}01.08.2018{]}; \textcite{siek_boost_2002}} (BGL) zum Einsatz, wurde
dann aber aufgrund von Performance-Problemen durch die C++ Bibliothek
des Stanford Network Analysis Project\footnote{\url{https://snap.stanford.edu/}
  {[}01.08.2018{]}; \textcite{leskovec2016snap}} (SNAP) abgelöst.

Die für die Kartengestaltung benötigten Raumdaten, also
Landmasse-Außengrenzen, Administrative Ländergrenzen sowie Flüsse und
Seen, stammen aus dem Natural Earth Projekt\footnote{\url{https://www.naturalearthdata.com}
  {[}02.08.2018{]}}. Verwendet wurden Daten des mittleren
Auflösungssniveaus, das eine Maßstabsperspektive von 1:50.000.000
wiedergeben soll. Die Daten wurden mittels des R Pakets
rnaturalearth\footnote{\textcite{south_rnaturalearth_2017}} direkt in R
heruntergeladen.

\hypertarget{radonb-dataset}{%
\section{Datensatz Radon-B}\label{radonb-dataset}}

Radon-B\footnote{\textcite{kneisel_radon-b_2013}} ist eine öffentlich
verfügbare Datenbank, die einzelne Radiokohlenstoffdatierungen --
\textsuperscript{14}C-Daten -- aus der Bronze- und Frühen Eisenzeit in
Europa sammelt. Sie konzentriert sich auf ein Zeitfenster zwischen 2300
bis 500calBC ab, enthält jedoch auch Daten jenseits dieses Limits. Neben
Radon-B steht mit ihrer Schwesterdatenbank
Radon\autocite{martin_hinz_radon_2012} eine strukturell äquivalente
Sammlung mit einem Schwerpunkt auf neolithischen Daten zur Verfügung.
Jedes Datum ist mit Kerndaten und Metainformationen verknüpft (siehe
Tabelle \ref{tab:radonbparams}). Die Informationen wurden aus einzelnen
Publikationen zusammengetragen und sind teilweise unvollständig,
inkonsistent oder fehlerhaft (siehe auch Kapitel
\ref{source-criticism}).

\begin{table}

\caption{\label{tab:radonbparams}Parameter, die in Radon-B für jedes Datum vorliegen.}
\centering
\fontsize{8}{10}\selectfont
\begin{tabu} to \linewidth {>{\bfseries\raggedleft\arraybackslash}p{0.5em}>{\raggedright\arraybackslash}p{25em}}
\toprule
 & Parameter\\
\midrule
1 & \textbf{Lab Code + Lab Nr.}\newline \textit{z.B. Ua-25144, OxA-1602, HAR-4341}\newline Die allgemeine, individuelle Kennnummer, die sich aus einem Kürzel des Labors, das die Messung durchgeführt hat, und einer fortlaufenden, laborspezifischen Prozessnummer zusammensetzt.\\
\addlinespace \hline \addlinespace
2 & \textbf{BP (Before Present)}\newline Das \textsuperscript{14}C-Alter, das mit der Messung ermittelt wurde in Jahren vor 1950 nach Christus [uncalBP].\\
\addlinespace \hline \addlinespace
3 & \textbf{Std (Standard deviation)}\newline Die messbedingte Standardabweichung des \textsuperscript{14}C-Alters.\\
\addlinespace \hline \addlinespace
4 & \textbf{$\delta$\textsuperscript{13}C}\newline Ein Maß für das Isotopenverhältnis des stabilen Isotops \textsuperscript{13}C / \textsuperscript{12}C zwischen der Probe und einem Standard in Promille [\textperthousand].\\
\addlinespace \hline \addlinespace
5 & \textbf{$\delta$\textsuperscript{13}C Std}\newline Die Standardabweichung des $\delta$\textsuperscript{13}C-Werts.\\
\addlinespace \hline \addlinespace
6 & \textbf{Sample Material}\newline \textit{z.B. charcoal, bone, seed}\newline Oberkategorie des Probenmaterials.\\
\addlinespace \hline \addlinespace
7 & \textbf{Sample Material Comment}\newline \textit{z.B. hazel, oak, barley, boar}\newline Nähere Kategorisierung und Artenzuordnung des Probenmaterials.\\
\addlinespace \hline \addlinespace
8 & \textbf{Feature Type}\newline \textit{z.B. settlement (house), rockshelter,     Grave (cremation)}\newline Befund bzw. Fundplatzkategorie, aus dem das Probenmaterial stammt.\\
\addlinespace \hline \addlinespace
9 & \textbf{Feature}\newline \textit{z.B. House I, from a mass of burnt debris...}\newline Bezeichnung des Befunds in der Grabungsdokumentations des Fundplatzes.\\
\addlinespace \hline \addlinespace
10 & \textbf{Culture}\newline \textit{z.B. Late Bronze Age, Únětice, Nordic Bronze Age}\newline Allgemeine, archäologische Kultur- oder Phasenzuordnung des Probenkontexts.\\
\addlinespace \hline \addlinespace
11 & \textbf{Phase}\newline \textit{z.B. Nagyrév Group, Mierzanowice, Period III}\newline Präzisere Kultur- oder Kontextansprache.\\
\addlinespace \hline \addlinespace
12 & \textbf{Site}\newline \textit{z.B. La Croix-Saint-Ouen, Stedten, Byneset}\newline Bezeichnung des Fundplatzes, aus dem die Probe stammt.\\
\addlinespace \hline \addlinespace
13 & \textbf{Country}\newline \textit{z.B. Germany, France, Poland}\newline Land in dem der Fundplatz liegt.\\
\addlinespace \hline \addlinespace
14 & \textbf{Country Subdivision}\newline \textit{z.B. Baden-Württemberg, Surrey, Greater Poland}\newline Zugehörige administrative Region innerhalb des Landes.\\
\addlinespace \hline \addlinespace
15 & \textbf{Literature}\newline Literaturreferenz auf die Publikation aus der die Informationen über das Datum entnommen wurden.\\
\addlinespace \hline \addlinespace
16 & \textbf{Comment}\newline Freitextkommentarfeld mit Zusatzinformationen zu dem einzelen Datum.\\
\bottomrule
\end{tabu}
\end{table}

\hypertarget{data-prep-and-segmentation}{%
\subsection{Datenvorbereitung und
Gliederung}\label{data-prep-and-segmentation}}

Eine hinsichtlich der Variablenauswahl etwas reduzierte\footnote{Für
  einen Überblick, welche Variablen heruntergeladen und wie umbenannt
  werden:
  \url{https://github.com/ISAAKiel/c14bazAAR/blob/master/data-raw/variable_reference.csv}},
aber hier ausreichende Version von Radon-B wurde mittels des R Pakets
c14bazAAR direkt in R bezogen. Dieser Ausgangsdatensatz enthielt alle zu
diesem Zeitpunkt {[}15.07.2018{]} öffentlichen Einträge: 11.048 Daten
von 2.766 Fundplätzen aus 48 Ländern. Der erste
Datenverarbeitungsschritt war das Entfernen aller Daten ohne
Altersinformation und aller Daten außerhalb der theoretischen Reichweite
der Kalibrationskurve (71-46401calBP) (10956 Daten verblieben). Zur
Kalibration wurde das R Paket Bchron\footnote{\textcite{haslett_simple_2008}}
und die darin enthaltenen Version es IntCal13 Datensatzes\footnote{\textcite{reimer_intcal13_2013}}
verwendet. Bchron berechnet das kalibrierte Alter mittels Numerischer
Integration\footnote{\url{https://github.com/andrewcparnell/Bchron/blob/master/R/BchronCalibrate.R}
  {[}02.08.2018{]}} und liefert für jedes Datum eine normierte
Wahrscheinlichkeitskurve. Alter mit Wahrscheinlichkeiten unterhalb eines
Schwellwerts von \(1\mathrm{e}{-6}\) wurden abgeschnitten und Alter
innerhalb des \(2\sigma\) Wahrscheinlichkeitsbereichs gesondert
markiert. Die so erhaltenen, unterschiedlich wahrscheinlichen,
kalibrierten Alter für jedes einzelne Datum wurden ab hier von calBP in
calBC umgerechnet, um üblichen archäologischen Konventionen und dem
allgemeinen Sprachgebrauch zu entsprechen. Um ein Subset des so
vorbereiteten Gesamtdatensatzes zu erzeugen, das die Anforderungen der
Fragestellung erfüllt, wurde er auf all jene Daten reduziert, die in
ihrem \(2\sigma\) Bereich mindestens ein Alter im Zeitfenster
800-2200calBC (1401 Jahre) vorweisen können (7543 Daten). Radon-B stellt
in der Variable \emph{Feature Type} (\emph{sitetype} in c14bazAAR)
teilweise kategorisierte Informationen zum Befundkontext jedes Datums
zur Verfügung: Fragestellungsrelevant sind die Kategorien
\emph{cemetery}, \emph{Grave}, \emph{Grave (mound)}, \emph{Grave (mound)
inhumation}, \emph{Grave (mound) cremation}, \emph{Grave (flat)},
\emph{Grave (flat) inhumation}, \emph{Grave (flat) cremation},
\emph{Grave (cremation)} und \emph{Grave (inhumation)} (2361 Daten).
Statt der Variablen \emph{Feature Type} wurden mittels Pattern Matching
zwei neue Variablen mit jeweils drei Werten geschaffen:
\emph{burial\_type} mit den Kategorien \emph{inhumation},
\emph{cremation} und \emph{unknown} sowie \emph{burial\_construction}
mit den Kategorien \emph{mound}, \emph{flat} und \emph{unknown}. Die
Fragestellung erfordert es auch, alle Daten ohne Raumbezug, also ohne
Koordinateninformation, zu entfernen (2336 Daten). Nach diesen
Arbeitsschritten lässt sich der Hauptausgangsdatensatz als Tabelle mit
2336 Zeilen und 15 Spalten beschreiben, darunter die hier wesentlichen
mit Angaben zu Labornummer, Koordinaten, Dichteverteilung des
kalibrierten Alters und \emph{burial\_type} sowie
\emph{burial\_construction}. Eine Karte der so vorbereiteten Gräberdaten
zeigt die hohe Heterogenität der Datendichte in verschiedenen Regionen
Europas (siehe Abbildung \ref{fig:general-map}.

\begin{figure}
\includegraphics{../neomod_analysis/figures_plots/general_maps/general_map} \caption[Übersichtskarte der \textsuperscript{14}C Daten an bronzezeitlichen Gräbern in Europa]{Karte der \textsuperscript{14}C-Daten des Radon-B Datensatzes in Europa aus einem Zeitfenster von 2200-800calBC. Daten außerhalb des durch die gewählten Kartengrenzen definierten Areals werden ignoriert. Jedes Datum ist nach seinen Kontextinformationen hinsichtlich der Variablen Burial type und Burial construction in Form und Farbe markiert.}\label{fig:general-map}
\end{figure}

Abbildung \ref{fig:general-map-research-area} zeigt das
Untersuchungsareals dieser Arbeit. Es folgt keinen natürlichen oder
kulturellen Grenzen, sondern wurde rein künstlich in Anbetracht der
räumlichen Verteilung der zusammengestellten \textsuperscript{14}C-Daten
festgelegt. Auf Grundlage visueller Analyse der Punktdichte schien es
angemessen, ein rechteckiges Areal aufzuspannen. Die Projektion, die
dieser Festlegung, allen Kartierungen und auch der Regionengliederung
zugrunde liegt ist bewusst mit EPSG:102013\footnote{\url{https://epsg.io/102013}
  {[}02.08.2018{]}} -- Europe Albers Equal Area Conics gewählt, da diese
auch auf kontinentalem Maßstab und bei Landmassen in betonter Ost-West
Ausdehnung ein hohes Maß an Flächentreue gewährleistet\footnote{\textcite{snyder_map_1987},
  98-99.}. Das ist eine wichtige Eigenschaft für die Definition von
vergleichbaren, räumlichen Untersuchungseinheiten. 1894 der 2336 oben
ausgewählten Daten stammen aus dem Rechteckareal.

\begin{figure}
\includegraphics{../neomod_analysis/figures_plots/general_maps/general_map_research_area} \caption[Karte mit \textsuperscript{14}C Daten und Grenzen des Untersuchungsareals]{Wie Abbildung \ref{fig:general-map}, jedoch mit Fokus auf das rot markierte Untersuchungsareal.}\label{fig:general-map-research-area}
\end{figure}

Der nach oben beschriebenem Vorgehen zusammengestellte Arbeitsdatensatz
umfasst also 1894 Einträge aus der Radon-B Datenbank. Die effektive
Anzahl an \textsuperscript{14}C Daten, die diese Einträge wiedergeben,
ist jedoch geringer: Eine Zählung der Labornummern ergibt 1831
individuelle Werte. Diese Diskrepanz ergibt sich aus Einträgen mit
keiner (\emph{n/a-n/a}) oder unvollständiger (z.B. \emph{MAMS-n/a},
\emph{Gd-n/a}, \emph{Ke-n/a}) Labornummer sowie Daten die mehrfach in
die Datenbank eingegeben wurden (z.B. \emph{OxA-29003},
\emph{GrN-10754}, \emph{BRAMS-1217}). Letzteres betrifft 46 Einträge in
dieser Datenauswahl, die ein und dasselbe \textsuperscript{14}C Datum
zwei- oder mehrfach repräsentieren. Die Anzahl an Gräbern, die durch die
Einträge repräsentiert werden ist noch geringer: Für 498 Einträge gilt,
dass die ihnen zugehörige Kombination aus Fundplatz und Befund von
mindestens einem weiteren Eintrag abgedeckt wird. Der wichtigste Grund
dafür ist, dass für ein Grab häufig mehrere \textsuperscript{14}C Daten
in Auftrag gegeben werden. Die Abweichungen zwischen Einträgen und
\textsuperscript{14}C Daten sowie \textsuperscript{14}C Daten und
Gräbern scheinen also auf den ersten Blick schwerwiegend zu sein.
Nichtsdestoweniger wurde in einem ersten Durchlauf der Berechnungen von
einer Korrektur abgesehen, und tatsächlich waren die Auswirkungen dieses
Versäumnisses auf die relative zeitliche und räumliche Entwicklung der
Hauptuntersuchungsparameter erstaunlich gering. Die Über- und
Unterbetonung der Verhältnisse durch beide Fehler zeigte keine übermäßig
starke Tendenz hinsichtlich der Variablen \emph{burial type} (Absolute
Werteverteilung innerhalb der Dubletten: cremation: 170, inhumation:
104, unknown: 224) und \emph{burial construction} (flat: 82, mound: 86,
unknown: 330) und auch zeitlich und räumlich waren die Abweichungen
scheinbar weitestgehend zufällig verteilt. In dieser Datenkombination
also eher ein statistisches Rauschen. Da allerdings keine Garantie
besteht, dass das auch für andere Datenkombinationen in Zukunft gelten
wird, schien es sinnvoll einen Algorithmus zu entwickeln, um die Fehler
zumindest teilweise auszugleichen.

Erstere Abweichung zwischen der Menge an Einträgen und den tatsächlich
vorhanden \textsuperscript{14}C Daten ist durch unvollständige
Datenpublikation und Fehleingabe bedingt. Sie wird sich mit der
stückweisen Verbesserung des Radon-B Datensatzes in Zukunft hoffentlich
selbst lösen. Da sich die Einträge jenseits der Labornummer häufig
unterscheiden, bleibt im Augenblick nur die Diskussion von Einzelfällen
oder die Inkaufnahme von geringfügigem Datenverlust bei einer
automatisierten Lösung. Um die reproduzierbare Natur dieser Arbeit nicht
in Frage zu stellen, kamen Werkzeuge aus dem c14bazAAR Paket zum
Einsatz, die Einträge mit äquivalenter Labornummer automatisch
zusammenführen. Abweichende Einträge in den Ausgangsdaten werden dabei
als unbekannte Werte behandelt. Von den 1848 oben zusammengestellten
Einträgen blieben 1.848 erhalten.

Die zweite Mengenabweichung zwischen \textsuperscript{14} Daten und
Gräbern ist schwerwiegender, da sie immerhin nahezu ein Drittel der
Einträge betrifft und keine Aussicht besteht, dass sich dieses Problem
mit einer Verbesserung der Datenlage lösen wird: Sie ist Teil der
Semantik des Datensatzes. Da für einzelne Gräber mehrere (bis zu 8)
\textsuperscript{14}C Daten vorliegen, muss für diese jeweils ein
individuelles chronologisches Modell definiert werden, dass alle Daten
vereint. Für einen großen Teil der Gräber könnte zwar angenommen werden,
dass die Datierungen sich tatsächlich nur auf ein einzelnes, zu einem
bestimmten Zeitpunkt in einen geschlossenen Befund eingebrachtes
Individuum beziehen, das geht allerdings nicht aus den in Radon-B
enthaltenen Metainformationen hervor. Stattdessen muss in Betracht
gezogen werden, dass auch Kollektivgräber mit langer Belegungszeit und
vielen einzelnen Bestattungen mit nur einer Befundbezeichnung
charakterisiert wurden. Die Befundangabe für manche Einträge ist sehr
unpräzise (z.B. \emph{Kollektivgrab}, \emph{Einzelgrab}, \emph{from ring
ditch}) und es ist nicht ersichtlich, ob die Daten tatsächlich von einer
einzelnen Bestattung stammen. Die Herausforderung besteht also darin,
auf Grundlage der vorhandenen Daten einerseits das übermäßige Gewicht
zeitlich scharf umgrenzter Gräber mit einzelnen, mehrfach datierten
Bestattungen zu mindern, und andererseits der diachronen Entwicklung in
über lange Zeit genutzten Grabanlagen gerecht zu werden. Um das zu
erreichen wurden die 486 nach der oben durchgeführten Entfernung der
Labornummer-Dubletten verbliebenen Mehrfacheinträge in einem ersten
Schritt weiter auf jene Befundtermini reduziert, die tatsächlich einen
einzelnen Grabbefund meinen könnten. Das sind vor allem jene 252 mit
numerischen Zeichen (z.B. \emph{Bef. 530 Doppelbestattung}, \emph{Grab,
Bef. 35635}, \emph{Objekt 461}), weswegen die Auswahl auf sie beschränkt
wurde. Innerhalb dieser Auswahl wurden nach Fundplatz und Befund
gegliederte Gruppen angelegt und deren kalibrierte Dichteverteilungen
zusammengeführt. Notwendig wäre dafür eigentlich ein individuelles
chronologisches Modell für jede dieser Datengruppen. Stattdessen wurden
die einzelnen Dichteverteilungen addiert und auf die das Gesamtmaximum
bezogen normiert. Die Information, ob ein Alter zum \(2\sigma\) Bereich
eines Datums gehört, wurde immer dann als wahr angenommen, wenn es im
\(2\sigma\) mindestens eines Datums liegt. Aus der Perspektive der
\textsuperscript{14}C Datenverarbeitung ist dieses Vorgehen nicht
korrekt, angesichts der zugrundeliegenden Fragestellung und der
Herausforderungen des Datensatzes jedoch sinnvoll: Jeder Eintrag im
Datensatz soll einen Ort und einen Zeitraum definieren, in dem die mit
ihm assoziierten Angaben für die Primärvariablen \emph{burial\_type} und
\emph{burial\_construction} auftraten. Durch die Reduktion der Daten auf
einen einzelnen Eintrag wird die Überbetonung dieser Information
vermieden. Gleichzeitig wird aber auch der mitunter langen
Belegungsdauer eines durch den einzelnen Eintrag repräsentierten
Grabmonuments Rechnung getragen. Eine Verbesserung der Metainformationen
zu jedem Datum (z.B. relativchronologische Position zu anderen Daten des
selben Grabes) würde eine wesentliche Verbesserung dieses Algorithmus
ermöglichen. Nach der vorgenommenen Reduktion verblieben 1704 jeweils
befundspezifische Einträge.

Innerhalb des Untersuchungsareals wurden künstliche Regionen abgegrenzt,
um die zeitliche und räumliche Entwicklung der Variablen
\emph{burial\_type} und \emph{burial\_construction} in sinnvollen und
der verfügbaren Datenmenge angemessenen Einheiten beobachten zu können
(siehe Abbildung \ref{fig:general-map-research-area-regions}). Der
Prozess der Erstellung dieser Regionen war semiautomatisch und darauf
angelegt kulturelle Makroregionen der Bronzezeit zumindest
näherungsweise abzubilden. Dafür stand mir auch eine unpublizierte,
händisch entworfene Regionengliederung von Jutta Kneisel und Oliver
Nakoinz als Vorlage zur Verfügung. In den Grenzen des
Untersuchungsareals wurde ein Raster von Punkten angelegt, die jeweils
als Zentrum einer der geplant runden Regionen dienen sollten. Dieses
Raster wurde manuell so angepasst, bis es sich den Zentrumspunkten
wesentlicher geographischer, kultureller und/oder
forschungsgeschichtlicher Einheiten annährte. Die Distanz zwischen den
Zentren beträgt in dieser Konfiguration 400km (im Bezugssystem der
EPSG:102013 Projektion). In einem weiteren Schritt wurden kreisförmige
Regionen um die Zentrumspunkte aufgebaut. Der Kreisradius wurde nach
Augenmaß mit 240km so gewählt, dass möglichst alle bekannten
\textsuperscript{14}C Daten (also damit auch Gräber) in mindestens einer
Region verortet sind. Das Überlappen von Regionen wurde dabei in Kauf
genommen. Andere Regionendefinitionen anhand alternativer geometrischer
Formen (Rechtecke, Hexagone), nach der Dichteverteilung von Fundpunkten,
anhand sich zeitlich wandelnder, archäologisch erfasster, kultureller
Einheiten sind denkbar und sollten bei zunehmender Datenverfügbarkeit in
Zukunft evaluiert werden. Das gilt auch hinsichtlich der Größe der
Einheiten, die aufgrund der diachron geringen Datenmenge sehr groß
gewählt werden mussten. Nur aus den acht Regionen, die in Abbildung
\ref{fig:general-map-research-area-regions} definiert werden, sind
ausreichend \textsuperscript{14}C Daten an Gräbern bekannt, um eine
nähere Betrachtung zu rechtfertigen. Der Schwellwert dafür wurde mit 60
Gräbern jedoch sehr niedrig angelegt um das effektive Untersuchungsareal
nicht noch weiter verkleinern zu müssen. Die Benennung der Regionen war
an den modernen, administrativen Einheiten orientiert, die die Kreise im
wesentlichen einschließen. Ihre im folgenden stets eingehaltene,
geographische Reihenfolge von Südost nach Nordwest soll die Lesbarkeit-
und Interpretierbarkeit von Abbildungen erhöhen. Mit angegeben ist die
Menge an Gräbern pro Region: \emph{Austria and Czechia} (70),
\emph{Poland} (134), \emph{Southern Germany} (213), \emph{Northeastern
France} (64), \emph{Northern Germany} (475), \emph{Southern Skandinavia}
(209), \emph{Benelux} (284), \emph{England} (113). Durch die
Regionengliederung verringerte sich das effektive Untersuchungsareal
weiter. Von den 1704 Gräbern im Rechteckareal verblieben 1562. Das ist
der Ausgangsdatensatz auf dem alle folgenden Beobachtungen beruhen.

\begin{figure}
\includegraphics{../neomod_analysis/figures_plots/general_maps/general_map_research_area_regions} \caption[Karte mit \textsuperscript{14}C Daten, dem Untersuchungsareal und den künstlichen Regionen]{Wie Abbildung \ref{fig:general-map-research-area}. Die Regionen sind farblich markiert.}\label{fig:general-map-research-area-regions}
\end{figure}

\hypertarget{deskriptive-analyse}{%
\subsection{Deskriptive Analyse}\label{deskriptive-analyse}}

Aus dem Areal der kreisförmigen, artifiziellen Großregionen, die für
diese Arbeit festgelegt wurden (siehe Abbildung
\ref{fig:general-map-research-area-regions}) liegen in Radon-B
Informationen zu mindestens 1562 Gräbern auf Grundlage von 1701
\textsuperscript{14}C Daten vor (zur Datenauswahl und -vorbereitung
siehe Kapitel \ref{data-prep-and-segmentation}). Geht man davon aus,
dass die Eingaben in Radon-B korrekt sind, dann stammen die
\textsuperscript{14}C Daten von 454 Fundplätzen. Zu den Daten sind 41
verschiedene Periodenbegriffe und 25 archäologische Kulturen
dokumentiert, diese Information ist jedoch aufgrund der Datensituation
sinnvoll auswertbar (siehe Kapitel \ref{source-criticism}). 1160 Daten
wurden an Probenmaterial von Knochen und Zähnen (169 davon verbrannt),
367 von Holz und Holzkohle, ein kleiner Teil (20) von sonstigen
Materialien wie Nüssen, Harz oder Pech gemessen. Für die restlichen 154
Daten ist keine Materialangabe hinterlegt. Hinsichtlich der Variablen
\emph{burial\_type} und \emph{burial\_construction} gelten die in
Tabelle \ref{tab:dprcrosstab} dargestellte Verhältnisse.

\begin{table}[!h]

\caption{\label{tab:dprcrosstab}Kreuztabelle}
\centering
\fontsize{8}{10}\selectfont
\begin{tabu} to \linewidth {>{\raggedright}X>{\raggedleft}X>{\raggedleft}X>{\raggedleft}X}
\toprule
  & flat & mound & unknown\\
\midrule
cremation & 66 & 96 & 241\\
\addlinespace
inhumation & 291 & 95 & 224\\
\addlinespace
unknown & 12 & 117 & 559\\
\bottomrule
\end{tabu}
\end{table}

Von besonderem Interesse für die vorliegende Arbeit ist eine diachrone
Perspektive in der Bestattungssittenentwicklung. Schon eine Kartierung
der Gräber in Zeitschritten von 200 Jahren (siehe Abbildung
@ref\{fig:general-map-research-area-timeslices\}) offenbart generelle
Trends hinsichtlich \emph{burial\_type} und \emph{burial\_construction}
in Früh-, Mittel- und Spätbronzezeit: Flachgrab -- Hügelgrab --
Flachgrab und Körpergrab -- Brandgrab. Diese Beobachtung kann mittels
des erstellten Datensatzes erheblich präzisiert sowie räumlich- und
zeitlich explizit gemacht werden.

\begin{landscape}
\begin{figure}
\includegraphics{../neomod_analysis/figures_plots/general_maps/general_map_research_area_timeslices} \caption[huhu]{huhu}\label{fig:general-map-research-area-timeslices}
\end{figure}
\end{landscape}

Entscheidend ist dabei hier nicht unbedingt, wann exakt wo welche Art
Grab angelegt wurde. Stattdessen soll aus dieser Information eine
quantitative Beschreibung zur Verbreitung und Dominanz von Ideen
extrahiert werden. Zur Erstellung dieses Proxies wurde die Annahme
getroffen, dass eine Idee zu einem bestimmten Zeitpunkt in einer Region
dann als anwesend gewertet werden muss, wenn ein Grab in dieser Region
existiert, dessen \(2\sigma\) Wahrscheinlichkeitsbereich ermittelt aus
einem oder mehreren \textsuperscript{14}C Daten diesen Zeitpunkt
umfasst. Ein Beispiel: Die Idee ``Körperbestattung'' muss 1447calBC
anwesend gewesen sein, da dieses Jahr im \(2\sigma\) Bereich des
\textsuperscript{14}C Datums NZA-32497 liegt, das für die
Körperbestattung I2639/25217 vom Fundplatz Amesbury Down angefertigt
wurde. Liegen mehrere Daten aus einer Region für ein Jahr vor, dann kann
das Auftreten der verschiedenen Ausprägungen von \emph{burial\_type} und
\emph{burial\_construction} gezählt werden. Tut man das für alle Jahre
in allen Regionen mit allen Ausprägungen tut, dann ergeben sich sechs
aufschlussreiche Zeitreihen, die sich jahrweise sinnvoll zur Gesamtzahl
der Beobachtungen addieren (für \emph{burial\_type} siehe Abbildung
\ref{fig:development-amount-regions-burial-type}), für
\emph{burial\_construction} Abbildung
\ref{fig:development-amount-regions-burial-construction}). Aus den
Ausprägungsmengen lässt sich auch das jeweilige Verhältnis der Ideen in
jedem Jahr in jeder Region berechnen (für \emph{burial\_type} Abbildung
\ref{fig:development-proportions-regions-burial-type}), für
\emph{burial\_construction} Abbildung
\ref{fig:development-proportions-regions-burial-construction}). Dabei
wurden die Gräber, für die keine Information zu den Primärvariablen
vorliegt (\emph{unknown}) ignoriert. Die Entwicklung der Verhältnisse
ist die für die Fragestellung dieser Arbeit interessantere Perspektive.
Eine Betrachtung der Stichprobengröße aus der die Proportionen
abgeleitet wurden, ist aber unumgänglich um die Aussagekraft in einer
Region und in einem Zeitfenster beurteilen zu können. Für manche
Regionen und Zeiträume liegen sehr wenige Gräber vor. Die Ergebnisse
aller folgenden Analysen müssen entsprechend mit Vorsicht gelesen
werden. Sie könnten sich bei zunehmender Datenmenge verändern.
Nichtsdestoweniger erlauben die Abbildungen
\ref{fig:development-proportions-regions-burial-type}) und
\ref{fig:development-proportions-regions-burial-construction} schon
jetzt bemerkenswerte, wenn auch teilweise objektiv falsche
Interpretationen (für eine Kontextualisierung siehe
\ref{representativity}):

\begin{figure}
\includegraphics{../neomod_analysis/figures_plots/development/development_amount_regions_burial_type} \caption[huhu]{huhu}\label{fig:development-amount-regions-burial-type}
\end{figure}

\begin{figure}
\includegraphics{../neomod_analysis/figures_plots/development/development_amount_regions_burial_construction} \caption[huhu]{huhu}\label{fig:development-amount-regions-burial-construction}
\end{figure}

\begin{figure}
\includegraphics{../neomod_analysis/figures_plots/development/development_proportions_regions_burial_type} \caption[huhu]{huhu}\label{fig:development-proportions-regions-burial-type}
\end{figure}

\begin{figure}
\includegraphics{../neomod_analysis/figures_plots/development/development_proportions_regions_burial_construction} \caption[huhu]{huhu}\label{fig:development-proportions-regions-burial-construction}
\end{figure}

Definitiv gab es im Laufe der Bronzezeit in Europa einen Trend weg von
der Körperbestattung hin zur Brandbestattung. Um 2200 waren
Brandbestattungen in Polen, Süddeutschland, Nordostfrankreich und
Norddeutschland nahezu unbekannt. Im Nordwesten, in Großbritannien und
im Benelux Raum, und im Südosten, in Österreich und Tschechien, gab es
jedoch frühe Brandbestattungstraditionen. Diese Kontexte könnten als
Ursprungsgebiete des später omnipräsenten Phänomens diskutiert werden.
Während Brandbestattungen sowohl in Österreich und Tschechien als auch
im Benelux Gebiet im Laufe der Frühbronzezeit an Bedeutung verloren und
erst in der Mittelbronzezeit wieder gewannen, stieg ihr Anteil in
England stetig. Körper- und Brandbestattung hielten sich hier lange die
Waage. Ähnlich verhielt es sich in Südskandinavien, wo der Anteil an
Brandbestattungen bis zur Spätbronzezeit allerdings wesentlich geringer
ausfiel. In Polen und Norddeutschland setzte sich Brandbestattung mit
dem Beginn er Spätbronzezeit relativ plötzlich und vollständig durch. In
Nordostfrankreich und Norddeutschland vollzieht sich dieser Wandel schon
in der Mittelbronzezeit. In beiden Kontexten spielten Körperbestattungen
auch danach eine wesentliche Rolle.

Hinsichtlich der Frage nach Grabüberhügelung ist das Bild erheblich
heterogener. In Österreich und Tschechien waren Flachgräber bis in die
Spätbronzezeit die dominante Bestattungsform, Überhügelung gewann aber
ab der Mittelbronzezeit an Relevanz. In Polen hielten sich Flach- und
Hügelgrab bis in die Mittelbronzezeit die Waage, dann setzten sich
Flachgräber durch. Die Datenmenge aus Süddeutschland und
Nordostfrankreich ist außerordentlich gering: Glaubt man der Stichprobe,
dann vollzog sich in Süddeutschland am Beginn der Mittelbronzezeit ein
plötzlicher, radikaler Wechsel von der Bestattung in Flach- zu
Hügelgräbern. In Nordostfrankreich hätte es in der Bronzezeit keine
Flachgräber gegeben. Norddeutschland durchlief eine Entwicklung von der
Dominanz von Flachgräbern in der Frühbronzezeit, einer kurzen Phase
zunehmender Überhügelung in der Mittelbronzezeit gefolgt von erneuter
Dominanz der Flachgrabsitte in der Spätbronzezeit. Eine ähnliche
Entwicklung deutet sich in Südskandinavien an: Flachgräber überwogen
deutlich, wurden über Jahrhunderte aber zunehmend -- fast vollständig --
von Hügeln ersetzt, bis die Beisetzung in Flachgräbern am Ende der
Bronzezeit wieder häufiger wurde. Im Benelux Raum waren Hügelgräber
durchgehend dominant, in der Früh- und Spätbronzezeit traten Flachgräber
jedoch ebenfalls in signifikantem Umfang auf. In England waren
Flachgräber ein kurzes Phänomen in der Frühbronzezeit, das später nicht
wieder auftrat.

\hypertarget{source-criticism}{%
\subsection{Quellenkritik}\label{source-criticism}}

Die Verwendung des Radon-B Datensatzes im Kontext von Methode und
Fragestellungen dieser Arbeit ist aus mehreren Gründen problematisch.
Eine Quellenkritik muss dabei auf verschiedenen Ebenen ansetzen: Bei der
technischen und inhaltlichen Umsetzung der Datenbank, bei der
Repräsentativität der im Datensatz vertretenen Stichprobe und
schließlich bei der Frage, ob diese Art Daten für eine Betrachtung von
Kulturvorgängen im Allgemeinen und im Rahmen der Cultural Evolution
Theorie und Terminologie geeignet ist.

\hypertarget{dateneingabe}{%
\subsubsection{Dateneingabe}\label{dateneingabe}}

Die Radon-B Datenbank hat inhaltliche Unzulänglichkeiten, die sich vor
allem aus inkonsistenter Dateneingabe infolge mangelnder Vorgaben in
Freitextfeldern ergeben. Eine systematische Lösung dieser Probleme wäre
sehr aufwändig, da sie die individuelle, nicht automatisierbare
Reevaluation eines großen Teils der Einträge erfordern würde.

\begin{itemize}
\tightlist
\item
  Viele Einträge sind unvollständig. Die Unvollständigkeiten rühren
  sicher teilweise aus Mängeln in den ausgewerteten Publikationen: Der
  \(\\delta\)\textsuperscript{13}C Wert, Material und Spezies des
  beprobten Überrests oder der kulturhistorische Kontext sind mitunter
  nicht bekannt oder nicht publiziert.
\item
  Mehrere eigentlich kategorisierbare Kontextvariablen sind mit einer
  inkosistenten Kategorienauswahl versehen oder nicht sinnvoll
  hierarchisiert. Während für manche Variablen (Feature Type, Sample
  Material) eine Kategorisierung bewusst festgelegt worden zu sein
  scheint, die dann nur in wenigen Fällen durch freie Einträge erweitert
  wurde, scheint bei anderen (Culture, Phase) keine klare Vorgabe zu
  bestehen, welche semantischen Inhalte in welcher Struktur dort
  eingefügt werden sollen.
\item
  Viele Einträge sind mehrsprachig -- vor allem Englisch und Deutsch --
  wobei in ein und dem selben Datensatz mitunter mehrere Sprachen für
  einzelne Felder auftreten.
\item
  Die Koordinateninformationen sind übermäßig genau, wenn man in
  Betracht zieht, dass sie in der Regel nur den Fundplatz und keine
  Befunde auf demselben verorten. Die scheinbare Genauigkeit reicht
  häufig in den Zentimeter-Bereich.
\end{itemize}

\hypertarget{representativity}{%
\subsubsection{Repräsentativität}\label{representativity}}

Die Abbildungen \ref{fig:general-map},
\ref{fig:development-amount-regions-burial-type} und
\ref{fig:development-amount-regions-burial-construction} zeigen, dass
der Datensatz für einzelne Regionen und Zeiträume verhältnismäßig viele,
für die Mehrzahl jedoch sehr wenige Daten enthält. Diese
Ungleichverteilung der Daten hat viele verschiedene Gründe.

\begin{itemize}
\tightlist
\item
  Die Datenaufnahme in Radon-B ist von Schwerpunkten,
  Forschungsinteressen und Projektfinanzierung der beteiligten
  Wissenschaftler abhängig. Daten wurden zeitlich und räumlich
  bedarfsweise aus der Literatur gesammelt.
\item
  Die Verfügbarkeit von \textsuperscript{14}C Daten in der Literatur ist
  wiederrum stark daran gebunden, ob ein modernes Forschungsprojekt mit
  Konzentration auf eine Region und ein Zeitfenster durchgeführt wurde.
  Neben der zufälligen Verteilung der Interessensgebiete der Forschenden
  spielt hier auch die politische Rahmensituation -- etwa die lang
  andauernde Trennung Europas in West und Ost -- eine entscheidende
  Rolle, die die Arbeit in manchen Räumen erleichtert oder kompliziert.
\item
  Auch im Untersuchungsgebiet dieser Arbeit begegnen sich
  unterschiedliche Forschungstraditionen, die \textsuperscript{14}C
  Daten in der Vergangenheit mehr oder weniger essentiell für die
  Konstruktion einer absolutchronologischen Einschätzung gehalten haben.
  \textsuperscript{14}C Daten sind in verschiedenen metallzeitlichen
  Kontexten der Genauigkeit relativchronologischer Einordnungen -- etwa
  auf Grundlage von Fibeltypologie -- unterlegen und werden deswegen nur
  sporadisch zur Schaffung von absolutchronologischen Einhängepunkten
  genutzt. In Feuchtbodenkontexten und bei Verfügbarkeit der
  erforderlichen Hölzer wird Dendrochronologische Datierung bevorzugt
  angewandt. Die verfügbaren \textsuperscript{14}C Kalibrationskurven
  bilden in manchen Zeiträumen Plateaus aus, was die erreichbare
  Datierungsgenauigkeit der Daten signifikant einschränkt. Ist ein
  solcher Effekt bekannt, dann werden erst gar keine Daten in Auftrag
  gegeben.
\item
  \textsuperscript{14}C Datierung erfordert eine zwar kleine, aber
  hinreichend genau kontextualisierbare Menge organischen Fundmaterials.
  Aufgrund von Altholz- und Altwassereffekt werden kurzlebige
  Probenmaterialien wie beispielsweise Knochen von terrestrischen
  Lebewesen oder verkohlte Getreidekörner bevorzugt. In vielen
  Mineralbodenkontexten ist kein ausreichend gutes, datierbares Material
  vorhanden. Das kann ganze Großregionen betreffen, wenn etwa infolge
  kalkarmer Böden generell schlechte Knochenerhaltung vorherrscht.
\item
  Aus sehr wenigen Regionen und Zeiten der europäischen Bronzezeit sind
  so viele Gräber dokumentiert, dass davon ausgegangen werden kann, eine
  signifikante Stuchprobe der Bestattungskultur der Gesamtpopulation
  erforscht zu haben. Stattdessen ist in mehreren Kontexten
  offensichtlich, dass nur die Bestattungen einzelner sozialen Gruppen,
  eines Geschlechts oder einer ethnischen Gruppe erfasst wurden: Die
  Menge und Natur der bekannten Gräber kann schlicht nicht für alle
  Verstorbenen repräsentativ sein. In diesem Fall ist anzunehmen, dass
  weitere, abweichende Bestattungsrituale praktiziert wurden, die keine
  oder archäologisch nur schwer fassbare Überreste hinterlassen haben.
  Diese Rituale sind in den \textsuperscript{14}C Daten aus Radon-B
  nicht abgebildet.
\end{itemize}

Hinsichtlich der Primärvariablen dieser Arbeit \emph{burial\_type} und
\emph{burial\_construction} ergeben sich weitere, besondere
Effekte\footnote{\textcite{harding_european_2000}, 84-85.}.

\begin{itemize}
\tightlist
\item
  Die Störung und Beraubung von Gräbern war ein in Geschichte und
  Vorgeschichte weit verbreitetes Phänomen. Hügelgräber wurden dabei
  aufgrund ihrer guten Sichtbarkeit üblicherweise stärker angegriffen
  als Flachgräber und könnten infolge dessen in Radon-B
  unterrepräsentiert sein.
\item
  Es ist davon auszugehen, dass ein großer Teil der in der Bronzezeit
  errichteten Hügelgräber durch landwirtschaftliche Aktivität --
  langjähriges Überpflügen -- zerstört wurde. Auch das ist Grund für
  eine Unterrepräsentation im gesamten archäologischen Befund.
\end{itemize}

Die Repräsentativität der relativen Entwicklung der Primärvariablen
(siehe Abbildung \ref{fig:development-proportions-regions-burial-type}
und \ref{fig:development-proportions-regions-burial-construction}) kann
zumindest oberflächlich geprüft werden, indem man die oben vorgestellten
Ergebnisse auf Grundlage des Radon-B Datensatzes mit einer Auswertung
der relativen Aussagen aus der Literatur vergleicht, wie sie in Kapitel
\ref{regions-archaeological-overview} zusammengefasst werden. Für
Abbildung \ref{general-map-regions-countries} wurden die Gräber und die
künstlichen Regionen zur besseren Orientierung auf die modernen
Ländergrenzen projiziert, auf die sich Kapitel
\ref{regions-archaeological-overview} bezieht. Zudem habe ich die
Literaturangaben in eine stark simplifizierende Modellabbildung
\ref{fig:development-proportions-regions-pseudoquant} verarbeitet, die
den Anteil einer Bestattungsform in einer Region und Periode wiedergibt.
Angaben wie ``In der Mittelbronzezeit dominiert in Süddeutschland die
Bestattung in Hügelgräbern'' wurden klassifiziert um pseudoquantitativ
visualierbar zu werden. Von den 5 Klassen wurde die Zuordnung zu 0 --
Merkmal ist nicht vorhanden -- und 4 -- Merkmal ist absolut dominant --
nicht vorgenommen, da selbst bei extrem regelhaften
Bestattungstraditionen in einem archäologischen Kontext stets Ausreißer
auftreten. Außerdem sollte damit auch den berechtigten Unsicherheiten
hinsichtlich der Repräsentativität des archäologischen Befunds Rechnung
getragen werden.

\begin{figure}
\includegraphics{../neomod_analysis/figures_plots/general_maps/general_map_regions_countries} \caption[huhu]{huhu}\label{fig:general-map-regions-countries}
\end{figure}

\begin{figure}
\includegraphics{../neomod_analysis/figures_plots/development/development_pseudoquant} \caption[huhu]{huhu}\label{fig:development-proportions-regions-pseudoquant}
\end{figure}

Österreich und Tschechische Republik

Polen

Süddeutschland

Norddeutschland

Nordostfrankreich

Südskandinavien

Benelux

England

\hypertarget{datenbedeutung}{%
\subsubsection{Datenbedeutung}\label{datenbedeutung}}

\begin{itemize}
\tightlist
\item
  Unterschied: Selbst Hügel anlegen vs.~Grab in vorhandenen Hügel
  einbringen
\end{itemize}

\hypertarget{simulation}{%
\section{Simulation}\label{simulation}}

\hypertarget{simulation-theorie}{%
\subsection{Theoretische Grundlagen und
Funktionalität}\label{simulation-theorie}}

Wie werden warum welche Gegebenheiten der realen Welt im Modell
abgebildet?

Obgleich das die Situation in manchen kulturhistorischen Zusammenhängen
nur unzureichend abbildet, wird hier der Einfachheit davon ausgegangen,
dass vertikale Beziehungen immer in einer 2:1 Relation von Eltern zu
Kind bestehen. Die Eltern müssen älter sein als das Kind, also einen
früheren Geburtszeitpunkt haben, dürfen aber zu diesem Zeitpunkt auch
nicht zu alt -- also tot -- sein. Tatsächlich müsste für die Zuordnung
von Eltern zu Kindern hier eine Vielzahl von Parametern beachtet werden:
Ein Zeitfenster der gemeinsamen Fruchtbarkeit der Eltern, Partnertreue,
Altersäquivalenz der Partner oder eine theoretische Maximalproduktion
von Kindern pro weiblichem Individuum, um nur einige zu nennen. Diesen
Anforderungen kann man stets nur unvollständig gerecht werden, die
Realwelt lässt sich jedoch mit einer Simulation, die die
Populationserzeugung generativ voranschreiten lässt besser abbilden als
mit dem hier gewählten Ansatz. Nach anfänglichen Versuchen musste darauf
jedoch

Ein Filtern nach diesen Kriterien ist aber bei großen Populationszahlen
ein zeit- und rechenaufwändiger Vorgang und wurde nach anfänglichen
Versuchen ebenso verworfen wie weitere Kriterien wie Letztere hätte eine
Unterscheidung nach dem Geschlecht erfordert.

\hypertarget{implementierung-und-algorithmen}{%
\subsection{Implementierung und
Algorithmen}\label{implementierung-und-algorithmen}}

Die Simulationssoftware besteht aus zwei speziell für diese Anwendung
entwickelten Modulen: Der Populationsgenerator popgenerator und die
Expansionssimulation gluesless. Die Module sind aufeinander abgestimmt
und in den verwendeten Versionen funktional auf die Fragestellungen
dieser Arbeit zugeschnitten. Beide ließen sich jedoch relativ leicht für
einen breiteren Anwendungsbereich öffnen, wenn dafür in Zukunft
Notwendigkeit bestehen sollte. Die im folgenden ausgearbeitete
Beschreibungen beziehen sich entsprechend jeweils speziell auf Version
1.0 der Module, die für die Berechnung in dieser Arbeit zum Einsatz
kamen. Im Gegensatz zu den Ausführungen im Kontext von Datenvorbereitung
und -analyse in Kapitel @ref(\#radonb-dataset), wo darauf zugunsten des
Leseflusses bewusst verzichtet wurde, werden hier nun die wesentlichen
Funktionen und Klassen namentlich genannt. Das soll es erleichtern, die
grundsätzlicher Architektur der Software zu verstehen.

\hypertarget{populationsgenerator}{%
\subsubsection{Populationsgenerator}\label{populationsgenerator}}

Der Populationsgenerator ist als R Paket implementiert. Er ist in drei
Submodule gegliedert: \texttt{population\_generator},
\texttt{relations\_generator} und \texttt{ideas\_generator}, die
nacheinander aufgerufen werden können -- die Interfacefunktion
\texttt{prepare\_pops\_rels\_ideas} automatisiert das. Jedes dieser
Module erweitert einen Eingabedatensatz sukzessive um nondeterministisch
generierte Daten hinsichtlich der zeitübergreifenden Gesamtpopulation,
der Beziehungen innerhalb dieser Population und der Verteilung zweier
Ideen darin zu einem hypothetischen Nullzeitpunkt. Der Eingabedatensatz
-- das \texttt{models\_grid} -- muss in Form eines \texttt{data.frames},
also der in R üblichen Datenstruktur für tabellierte Daten, vorliegen.
Jede Zeile in diesem Datensatz enthält zunächst die Parameter -- später
auch die Ausgabedaten -- für eine Population, ihre Relationen und Ideen.
Komplexe Parameter sind dabei in \texttt{list\ columns} gespeichert.
Diese erlauben es, fast beliebige Datenstrukturen in den Zellen eines
\texttt{data.frames} zu schachteln. Tabelle \ref{tab:param-popgenerator}
beschreibt alle Parameter kurz und umreißt ihren theoretischen
Wertebereich. Die tatsächlich für die Simulation relevanten Werte werden
in Kapitel \ref{simulation-parameters} diskutiert. Die Erzeugung von
Populationen und Relationen ist soweit wie möglich vektorisiert, also
ohne Schleifen oder schleifenersetzende Strukturen (\texttt{apply},
\texttt{purrr::map}) programmiert, um die Berechnungsdauer zu
minimieren. Dieser Schritt ist eine Konzession an die Technik und führte
zu einem höheren Abstraktionsgrad der Populationsnetzwerke als
ursprünglich geplant.

\begin{table*}

\caption{\label{tab:param-popgenerator}Parameter des popgenerator Moduls}
\centering
\fontsize{8}{10}\selectfont
\begin{tabu} to \linewidth {>{\raggedright\arraybackslash}p{14em}>{\raggedright\arraybackslash}p{20em}>{\raggedright\arraybackslash}p{20em}}
\toprule
Paramter & Beschreibung & Theoretischer Wertebereich\\
\midrule
\textit{Allgemeine Parameter}\newline \textbf{timeframe}\newline \textit{integer vector} & Eine Liste der Kalenderjahre über die sich die Simulation erstrecken soll. Das kann eine künstliche Sequenz sein, oder sich an dem Realweltphänomen orientieren, das mit der Simulation erforscht werden soll. & Beliebige, unterbrechungsfreie Sequenz von sukzessive aufeinander folgenden Integerwerten:\newline     \texttt{-2200:-800}, \texttt{0:500}, \texttt{seq(-1000, 1000)}\\
\addlinespace \hline \addlinespace
\textit{Populationsparamter}\newline \textbf{unit\_amount}\newline \textit{integer} & Anzahl der Gruppen, in die die Population untergliedert sein soll. & Beliebiger positiver Integerwert.\newline     \texttt{1}, \texttt{8}, \texttt{100}\\
\addlinespace \hline \addlinespace
\textit{Populationsparamter}\newline \textbf{unit\_names}\newline \textit{list of factors} & Eine Liste der Gruppennamen, wahlweise mit einer Definition ihrer inneren Reihenfolge. & Liste mit Namen. Die Anzahl muss \textit{unit\_amount} entsprechen.\newline     \texttt{list(factor("regionA", levels = regionen), factor("regionB", levels = regionen))}\\
\addlinespace \hline \addlinespace
\textit{Populationsparamter}\newline \textbf{unit\_size\_functions}\newline \textit{list of functions} & Funktionen, die die Populationsgröße für jede Gruppe in Abhängigkeit von der Simulationszeit definieren. & Liste mit Funktionen. Die Anzahl muss \textit{unit\_amount} entsprechen.\newline     \texttt{list('regionA' = function(t) \{100\}, 'regionB' = function(t) \{100 + 10 * cos(t * 0.1)\})}\\
\addlinespace \hline \addlinespace
\textit{Populationsparamter}\newline \textbf{age\_distribution\_function}\newline \textit{function} & Funktion, die die durchschnittliche Altersverteilung der Menschen in der Population beschreibt. & Eine Funktion, die einen Wert für jeden Eingabewert aus \textit{age\_range} zurückgibt.\newline     \texttt{function(x) \{1 / (1 + 0.0004 * 0.7\^\ (-7 * log(x)))\}}\\
\addlinespace \hline \addlinespace
\textit{Populationsparamter}\newline \textbf{age\_range}\newline \textit{integer vector} & Altersfenster, auf das die Altersverteilungsfunktion angewandt wird. & Beliebige, unterbrechungsfreie Sequenz von sukzessive aufeinander folgenden Integerwerten im Bereich der menschlichen Lebenserwartung:\newline     \texttt{1:70}, \texttt{1:120}\\
\addlinespace \hline \addlinespace
\textit{Beziehungsparameter}\newline \textbf{amount\_friends}\newline \textit{integer} & Menge an horizontalen Beziehungen, die ein Individuum aufbaut. & Beliebiger positiver Integerwert oder 0.\newline     \texttt{0}, \texttt{5}, \texttt{100}\\
\addlinespace \hline \addlinespace
\textit{Beziehungsparameter}\newline \textbf{unit\_interaction\_matrix}\newline \textit{integer matrix} & Matrix die definiert, welche Gruppe in welchem Umfang mit welcher anderen Gruppe interagiert. Diese Tabelle kann verschiedene -- räumliche, kulturelle, wirtschaftliche -- Distanzen ausdrücken. & Kreuztabelle in Form einer Integermatrix. Die Werte können beliebige positive Integerwert oder 0 sein.\newline     \texttt{matrix(c(0, 1, 1, 0), nrow = 2, ncol = 2)}\\
\addlinespace \hline \addlinespace
\textit{Beziehungsparameter}\newline \textbf{cross\_unit\_proportion\_child\_of}\newline \textit{double} & Anteil der vertikalen Beziehungen, die nicht innerhalb einer Gruppe bestehen, sondern über Gruppengrenzen hinweg reichen. & Double Wert zwischen 0 und 1.\newline     \texttt{0}, \texttt{0.02}, \texttt{0.7}\\
\addlinespace \hline \addlinespace
\textit{Beziehungsparameter}\newline \textbf{cross\_unit\_proportion\_friend}\newline \textit{double} & Anteil der horizontalen Beziehungen, die über Gruppengrenzen hinweg reichen. & Doublewert zwischen 0 und 1.\newline     \texttt{0}, \texttt{0.02}, \texttt{0.7}\\
\addlinespace \hline \addlinespace
\textit{Beziehungsparameter}\newline \textbf{weight\_child\_of}\newline \textit{integer} & Stärke einer vertikalen Beziehung. Die Beziehungsstärke hat Einfluss darauf, ob eine Idee beim Versuch von einem Individuum zum anderen zu springen Erfolg hat. & Beliebiger positiver Integerwert oder 0.\newline     \texttt{0}, \texttt{5}, \texttt{100}\\
\addlinespace \hline \addlinespace
\textit{Beziehungsparameter}\newline \textbf{weight\_friend}\newline \textit{integer} & Stärke einer horizontalen Beziehung. & Beliebiger positiver Integerwert oder 0.\newline     \texttt{0}, \texttt{5}, \texttt{100}\\
\addlinespace \hline \addlinespace
\textit{Ideenparameter}\newline \textbf{names}\newline \textit{character vector} & Namen der Ideen. & Vektor mit Namen.\newline     \texttt{c('ideaA', 'ideaB')}\\
\addlinespace \hline \addlinespace
\textit{Ideenparameter}\newline \textbf{start\_distribution}\newline \textit{data.frame} & Proportionaler Anteil der Ideen in jeder Region zum Startzeitpunkt der Simulation. Eine Tabelle mit einer Zeile für jede Gruppe und einer Spalte für jede Idee. & \textit{data.frame} mit Proportionen pro Gruppe und Idee. In den Zellen ist der Anteil der Idee in dieser Gruppe als Doublewert zwischen 0 und 1 angegeben. Die Zeilensumme muss 1 sein.\newline     \texttt{data.frame(ideaA = c(0.2, 0.5), ideaB = c(0.8, 0.5))}\\
\addlinespace \hline \addlinespace
\textit{Ideenparameter}\newline \textbf{strength}\newline \textit{double vector} & Stärke der Ideen. & Integervektor mit Werten zwischen ...\newline     \texttt{c(1,1)}\\
\bottomrule
\end{tabu}
\end{table*}

\texttt{population\_generator} dient dazu eine große Menge von
Individuen zu generieren, die gemeinsam eine generationenüberschreitende
Population bilden. Menschen sind nur durch ihre Lebenszeit und ihre
Gruppenzugehörigkeit definiert und damit -- wie in Kapitel
\ref{simulation-theorie} beschrieben -- sehr einfach modelliert. Die
zeitliche Auflösung des popgenerator Moduls ist jahrweise und damit an
die Erfordernisse der Fragestellung dieser Arbei angepasst. Für die
Erzeugung von Populationen, also der Verarbeitung der Daten im
\texttt{models\_grid}, werden zunächst die Populationsparameter aus
jeder Zeile in Instanzen der \texttt{S4}-Konfigurationsklasse
\texttt{population\_settings} überführt. Diese und die folgenden
Transformation in solche Konfigurationsobjekte erleichtern die
Datenweitergabe innerhalb des Pakets. Aus jedem einzelnen
\texttt{population\_settings} Objekt wird eine Population geschaffen. Da
jede Population aus einer oder vielen Gruppen besteht, die biologische
Vererbungsgruppen wie Familien oder Clans repräsentieren und jeweils
eine individuelle Größenentwicklung durchlaufen können, wird für jede
Gruppe in der Population ein Konfigurationsobjekt der Klasse
\texttt{unit\_settings} zusammengestellt. Zur Erzeugung einer Gruppe
wird dieses Objekt an die Funktion \texttt{generate\_unit} übergeben,
die die Hauptlast der Menschengenerierung trägt. In ihr wird zunächst
das Integral unter der \texttt{unit\_size\_function} im um einen
Bufferbereich erweiterten Untersuchungszeitfenster berechnet, um zu
ermitteln, wie viele Menschen-Jahr Kombinationen erforderlich sind, um
die vorgegebene Populationsgrößenentwicklung abzubilden. Aus
\texttt{age\_distribution\_function} und \texttt{age\_range} lässt sich
die durchschnittliche Lebenserwartung der Menschen errechnen, die sich
aus den Eingabeparameters ergibt. Beide Informationen zusammen
ermöglichen es, die Anzahl an Menschen zu bestimmen, die insgesamt
erforderlich ist, um die Größenentwicklung näherungsweise aufzubauen. Um
diese Anzahl an Menschen den Erfordernissen der Populationsentwicklung
entsprechend auf dem Zeitstrahl in mehr oder wenigen dichten Gruppen
anzuordnen, wird eine regelmäßiges Sequenz von Geburtsfenstern
abgegrenzt. Die Länge eines Geburstsfensters entspricht der mittleren
Lebenserwartung. Um die oben errechnete Gesamtzahl der Menschen auf die
Fenster aufzuteilen, wird wiederum das Integral der
\texttt{age\_distribution\_function} in jedem Fenster ermittelt und in
Verhältnis zur Gesamtsumme dieser Integrale gesetzt. Damit steht für
jedes Geburtsfenster ein Faktor bereit, um den Anteil der
Gesamtpopulation zu berechnen, der in diesem Zeitfenster existiert. Mit
dieser Information können die entsprechenden Menschen mit der
\texttt{generate\_humans} Funktion zufällig generiert werden. Das
Beispiele in Abbildung \ref{fig:popgen-sizedev-example} belegt die
Funktionalität dieses Ansatzes, illustriert aber auch den Umfang der
Abweichungen von Ergebnispopulationsgröße und Vorgabefunktion.
\textbf{TODO: Diskussion}. Zum Abschluss der Berechnungen in
\texttt{population\_generator} werden die Gruppenpopulationen zur
Gesamtpopulation zusammengeführt. Diese liegt in der Form eines
\texttt{data.frames} vor, wobei jede Zeile ein Indiviuum repräsentiert.
Jedes Individuum erhält eine eindeutige ID und bringt Informationen zu
seiner Lebensdauer, seinem Geburts- und Sterbezeitpunkt sowie seiner
Gruppenzugehörigkeit mit. Das Sortierkriterium im Gesamtdatensatz ist
das Geburtsjahr. Ein solcher Ergebnisdatensatz wird für jedes Modell,
also jede Zeile, im Eingabedatensatz \texttt{models\_grid} erzeugt und
kann entsprechend in der \texttt{list\ column} \emph{populations} dort
hinzugefügt werden.

\begin{figure}
\includegraphics{../neomod_analysis/figures_plots/popgenerator_examples/create_unit_population_size_development_comparison} \caption[huhu]{huhu}\label{fig:popgen-sizedev-example}
\end{figure}

Das Modul \texttt{relations\_generator} erweitert diesen von
\texttt{population\_generator} modifizierten Eingabedatensatz. Es dient
dazu, die vorhandene Gesamtpopulation sinnvoll inneinander zu verknüpfen
um ein diachrones, soziales Netzwerk zu schaffen. Dafür wird zunächst
für jedes Modell ein Objekt der Konfigurationsklasse
\texttt{relations\_settings} instanziiert, das neben der in
\texttt{relations\_generator} generierten Population auch die
Beziehungsparameter enthält, die vorgeben, welche Eigenschaften das
Netzwerk besitzen soll. Dessen eigentliche Erzeugung ist ein
vierteiliger Prozess: Vertikale und Horizontale Beziehungen, die sich
als Kanten im Netzwerk zwischen den Knoten der Individuen ausdrücken,
werden getrennt voneinander aber jeweils innerhalb der oben erzeugten
Gruppen hergestellt. Anschließend wird ein Teil der vorhandenen
Beziehungen so umgelenkt, dass er die Gruppengrenzen überschreitet und
somit die Gesamtpopulation verschränkt. Abschließend werden die
Beziehungen je nach Typ mit einem Kantengewicht versehen. Die Erzeugung
der vertikalen Beziehungen mit \texttt{generate\_vertical\_relations}
funktioniert gruppenweise, verbindet ein jüngeres Individuum mit -- wenn
im entsprechenden Zeitfenster vorhanden -- zwei älteren und orientiert
sich dabei nicht an deren realem Alter sondern an Indexdistanzen (für
eine Erklärung der Hintergründe dieser Lösung siehe Kapitel
\ref{simulation-theorie}). Eltern werden zufällig aus der Perspektive
der Kinder gewählt, indem zunächst ein Indexbereich -- ein Bereich
zwischen zwei individuellen IDs -- festgelegt wird, aus dem die
potentiellen Eltern stammen können. Zwar kann sicher angenommen werden,
dass ein Kind eine niederere ID besitzen muss als seine Eltern, doch
darüber hinaus ist der Umgang mit den Indizes rein approximativ: Die
durchschnittliche Geburtsjahrdistanz eines Indexschrittes hängt von der
Populationsgröße und -entwicklung ab. Um eine Distanz in Jahren in eine
Distanz in Indexschritten umzuwandeln, muss die zeitlich lokale,
mittlere Indexdistanz ermittelt werden. Das angezielte
Altersdistanzfenster dafür wurde zwischen 40 und 15 Jahren vor dem
Geburtsjahr des Kindes festgelegt. Die für jedes Kind individuelle und
effektiv zufällige Auswahl der Eltern erfolgt also aus einem Pool von
Individuuen, deren Index zwischen jenen liegt, die durchschnittlich die
lokale 40 Jahresgrenze über- oder die durchschnittlich 15 Jahresgrenze
unterschreiten. \texttt{generate\_horizontal\_relations} zur Erzeugung
der horizontalen Beziehungen funktioniert nach dem selben technischen
Prinzip wie \texttt{generate\_vertical\_relations}. Hier werden
allerdings abhängig vom Wert von \texttt{amounts\_friends} mitunter
wesentlich mehr Beziehungen hergestellt und das Bezugsfenster ist mit
einer Altersdistanz von 50 Jahren in beide Richtungen ausgehend vom
Geburtszeitpunkt des jeweiligen Individuums deutlich breiter. Sind die
vertikalen und horizontalen Beziehungen gruppenintern etabliert, dann
werden mit \texttt{modify\_relations\_cross\_unit} einige dieser
Verbindungen zugunsten von gruppenübergreifenden Beziehungen aufgelöst.
Die Eingabeparamter \texttt{cross\_unit\_proportion\_child\_of} und
\texttt{cross\_unit\_proportion\_friend} sind entscheidend dafür, in
welchem Umfang das für die beiden Beziehungstypen passiert. Davon
abhängig werden mehr oder weniger Beziehungen für eine Modifikation
zufällig ausgewählt. Diese besteht darin, dass eines der beiden
Individuen durch ein anderes aus einer anderen Gruppe, aber dem selben
Geburtsfenster ersetzt wird. Welche andere Gruppe gewählt wird, wird
über eine Zufallsentscheidung auf Grundlage der
\texttt{unit\_interaction\_matrix} festgelegt. In einem letzten Schritt
innerhalb des \texttt{relations\_generator} Submoduls werden die
Beziehungen nach den Eingabevariablen \texttt{weight\_child\_of} und
\texttt{weight\_friend} mit einem Gewichtswert versehen. Die vier
Teilschritte dienen gemeinsam dazu einen \texttt{data.frame} zu
schaffen, der sinnvolle Beziehungen zwischen Individuen der
Gesamtpopulation dokumentiert. Die \texttt{data.frames} für jedes Modell
werden in der \texttt{list\ column} \emph{relations} an den
Eingabedatensatz \texttt{models\_grid} angehängt.

Das letzte und einfachste Submodul des popgenerator Pakets,
\texttt{idea\_generator}, \ldots{}

popgenerator stellt einige dedizierte Exportfunktionen bereit, die vor
allem die Übergabe des generierten Populationsnetzwerks an die
Expansionssimulation gluesless ermöglichen sollen. Der Austausch erfolgt
über verschiedene, speziell formatierte Textdateien, die von
popgenerator ins Dateisystem abgelegt und von gluesless gelesen werden.

\hypertarget{expansionssimulation}{%
\subsubsection{Expansionssimulation}\label{expansionssimulation}}

Das C++ Programm gluesless simuliert die Expansion von Ideen in einem
Populationsgraphen wie er mittels des popgenerators erzeugt werden kann.
gluesless macht sich die objektorientierte Natur von C++ zu Abbildung
der Simulationswelt in vier Klassen zunutze: \texttt{Networkland},
\texttt{Aether}, \texttt{Timeline}, \texttt{Idea}. Die Klassenmethoden
greifen außerdem auf mehrere globale Hilfsfunktionen zurück. gluesless
kann drei Eingabeparameter verarbeiten: Der Pfade zu einer pajek-Datei
(.paj), die das Populationsnetzwerk beschreibt, der Pfad zu einer
speziell formatierten Textdatei mit der Ideenverteilung zum
Nullzeitpunkt und der Pfad der Ausgabetextdatei. Mit diesen Parametern
kann es einfach auf der Kommandozeile aufgerufen werden.

Wird das Programm mit den korrekten Eingaben gestartet, dann wird die
\texttt{main} Methode ausgeführt und zunächst jeweils eine Instanz der
Klassen \texttt{Networkland}, \texttt{Aether} und \texttt{Timeline}
angelegt. \texttt{Networkland} repräsentiert die Netzwerkwelt in der die
Ideen leben und interagieren. Ihr Hauptbestandteil ist ein Zeiger auf
ein Objekt der Klasse \texttt{TUndirNet}\footnote{\url{https://snap.stanford.edu/snap/doc/snapuser-ref/d8/da8/classTUndirNet.html}
  {[}13.08.2018{]}} aus der SNAP Bibliothek, das dazu dient, den
Populationsgraphen als sehr einfaches, ungerichtetes Netzwerk zu
speichern und sehr schnell zugänglich zu machen. Die Hauptaufgabe der
Klassenmethoden von \texttt{Networkland} ist es, ein bedarfsgerechtes
Interface zu diesem Netzwerkdatentyp bereit zu stellen. Es kann
verschiedene Fragen beantworten: z.B. ``Existiert ein bestimmter Knoten
im Netzwerk?'', ``Welche Nachbarn hat ein Knoten?'', ``Welchen
Gewichtswert hat eine bestimmte Beziehung?''. Außerdem erlaubt es die
Manipulation des Netzwerks, indem Knoten gelöscht werden können. Der
\texttt{Aether} ist die gedankliche Einheit, die die Netzwerkwelt und
alle Ideen umschließt. Er ist die Kapsel, die den aktuellen Zustand von
Welt und Agenten abbildet. Dafür besitzt er einen Zeiger auf die im
Programmablauf angelegte Instanz der \texttt{Networkland} Klasse und
einen Vektor mit Zeigern auf Instanzen der \texttt{Idea} Klasse. Neben
Funktionen, die Informationen zum aktuellen Zustand von Netzwerk und
Ideen zurückgeben, besitzt der \texttt{Aether} auch die \texttt{develop}
Methode. Sie steuert unter welchen Bedingungen und in welcher
Reihenfolge Ideen am Übergang von einem Zeitschritt zum nächsten agieren
dürfen. Die Basiskonfiguration sieht eine zufällige Abfolge vor. Die
\texttt{Timeline} Klasse umschließt wiederrum gedanklich den
\texttt{Aether} und besitzt dafür einen Zeiger auf die Instanz dieser
Klasse. Sie dient dazu, in jedem Zeitschritt diagnostische Werte zum
Zustand des Aethers abzugreifen und in Vektoren aufgelistet vorzuhalten.
Dazu gehört zum Beispiel die verbleibende Größe des Netzwerks. Auch
\texttt{Timeline} verfügt über eine \texttt{develop} Methode, die
einerseits die \texttt{develop} Methode im Aether anstößt und
andererseits die Messung der diagnostischen Werte auslöst. Es ist diese
\texttt{develop} Methode die im Hauptprogrammablauf in einer
while-Schleife so lange immer wieder aufgerufen wird, bis die
Ideenexpansion endet.

Die Hauptlast der Expansionssimulation tragen Methoden in der
\texttt{Idea} Klasse. Das ist in Programmstruktur- und semantik
sinnvoll, da Ideen als aktiv handelnde Agenten modelliert werden sollen:
Jede Form von Aktivität soll von ihnen ausgehen, während etwa die
Netzwerkwelt, die Menschen und ihre Beziehungen abbildet, nur als
passive Landschaft verstanden wird (siehe Kapitel
\ref{simulation-theorie} für eine Erläuterung dieser Perspektive). Ideen
besitzen einen Namen, einen Zeiger auf die Instanz der
\texttt{Networkland} Klasse in der sie leben und zwei Vektoren, die die
IDs der Netzwerkknoten speichern, auf denen sie aktuell sitzen und auf
denen sie zum Zeitpunkt deren Todes saßen. Um letzteres zu verstehen,
muss man den Algorithmus in der \texttt{expand} Methode betrachten, die
die Ausbreitung der Ideen steuert. Zu Programmbeginn besetzen alle Ideen
die ihnen mittels einer Eingabedatei zugewiesenen Startknoten im
Netzwerk. In jedem Zeitschritt der Simulation darf nun jede Idee einmal
handeln -- \texttt{Aether::develop} legt die Reihenfolge dabei fest. Der
erste Schritt der Idee ist es, eine Liste aller Nachbarknoten zu den von
ihr okkupierten zu erstellen. Die Idee versucht all diese Nachbarknoten
einzunehmen, muss dafür allerdings verschiedene Hindernisse überwinden,
die sich auf die Zufallsentscheidung ob sie Erfolg hat auswirken. Da in
diesem Kontext viele Abfragen von Beziehungs- und Knoteninformationen im
Netzwerk durchgeführt werden müssen, findet die Berechnung dieser
Entscheidungen in einem Subnetzwerk statt, dass sich auf den aktuellen
Dominanzbereich einer Idee und den unmittelbaren Nachbarn beschränkt.
Für jeden Nachbarn werden drei Informationen abgefragt: Die Anzahl an
Verbindungen zu Knoten, die die Idee schon hält, das maximale
Kantengewicht unter all diesen Beziehungen und ob der Knoten selbst
bereits von einer anderen Idee besetzt wird. Grundsätzlich wird für die
Entscheidug ob ein Nachbarknoten der Idee zugesprochen wird gegen das
Kantengewicht gewürfelt. Die Wahrscheinlichkeit wird erhöht, wenn
mehrere Verbindungen zu Knoten der Idee bestehen und deutlich
verringert, wenn der Nachbarknoten bereits Teil des Einflussgebiets
einer anderen Idee ist. Nachdem auf dieser Grundlage für jeden
Nachbarknoten entschieden wurde, ob er zu der aktuell handelnden Idee
gehören wird, verlagert die Idee ihre Existenz auf diese Nachbarknoten.
Ihre bisherigen Knoten sterben, das heißt sie werden aus dem Netzwerk
gelöscht. Nur ihre Bezeichnung wird als Eroberung der Idee gespeichert.
Das Löschen der alten Knoten führt automatisch dazu, dass die Ideen
entlang der impliziten Zeitachse des Populationsnetzwerks voran
schreiten und bildet gleichermaßen semantisch den Forschungskontext ab:
Ideen zur Bestattungsformen drücken sich im Tod ihrer Träger aus. Dieser
Algorithmus wird für jede Ideen in jedem Zeitschritt ausgeführt.

Die Expansionssimulation endet, wenn sich die Größe des Netzwerks von
einem Zeitschritt zum nächsten nicht mehr ändert, also wenn die Ideen
alle Knoten in ihrer Reichweite erobert haben. Naturgemäß führt das
dazu, dass nicht alle Knoten im Netzwerk überhaupt von einer Idee
erobert werden. Je nachdem, wie hoch die Hürden für eine erfolgreiche
Knoteneroberung angesetzt sind, verbleiben mehr oder weniger Knoten
(siehe Kapitel \ref{simulation-theorie}. Die \texttt{Timeline} Klasse
stellt schließlich die Exportfunktion \texttt{export\_to\_text\_file}
bereit, die am Ende der \texttt{main} Methode ausgeführt wird. Diese
transformiert und speichert die diagnostischen Daten und den Endzustand
der Ideen hinsichtlich der von ihnen eingenommenen Netzwerkknoten in
eine menschenlesbare Datei, die später zur Auswertung wieder in R
eingelesen werden kann.

\hypertarget{simulation-parameters}{%
\subsection{Wertebereiche der
Simulationsparameter}\label{simulation-parameters}}

\hypertarget{allgemeine-beobachtungen-zum-simulationsverhalten}{%
\subsection{Allgemeine Beobachtungen zum
Simulationsverhalten}\label{allgemeine-beobachtungen-zum-simulationsverhalten}}

\hypertarget{kulturelle-und-raumliche-distanz}{%
\section{Kulturelle und Räumliche
Distanz}\label{kulturelle-und-raumliche-distanz}}

\hypertarget{kausale-interaktionsbeziehungen}{%
\section{Kausale
Interaktionsbeziehungen}\label{kausale-interaktionsbeziehungen}}

\hypertarget{simulation-und-reale-entwicklung}{%
\section{Simulation und reale
Entwicklung}\label{simulation-und-reale-entwicklung}}

\ldots{}

\pby[title={Literatur},segment=\therefsegment,heading=subbibintoc]

\hypertarget{zusammenfassung-und-abschlieende-gedanken}{%
\chapter{Zusammenfassung und Abschließende
Gedanken}\label{zusammenfassung-und-abschlieende-gedanken}}

\hypertarget{abbildungen}{%
\chapter{Abbildungen}\label{abbildungen}}

\printbibliography


\end{document}
