\documentclass[openany,twoside,twocolumn]{book}
\usepackage{lmodern}
\usepackage{amssymb,amsmath}
\usepackage{ifxetex,ifluatex}
\usepackage{fixltx2e} % provides \textsubscript
\ifnum 0\ifxetex 1\fi\ifluatex 1\fi=0 % if pdftex
  \usepackage[T1]{fontenc}
  \usepackage[utf8]{inputenc}
\else % if luatex or xelatex
  \ifxetex
    \usepackage{mathspec}
  \else
    \usepackage{fontspec}
  \fi
  \defaultfontfeatures{Ligatures=TeX,Scale=MatchLowercase}
\fi
% use upquote if available, for straight quotes in verbatim environments
\IfFileExists{upquote.sty}{\usepackage{upquote}}{}
% use microtype if available
\IfFileExists{microtype.sty}{%
\usepackage{microtype}
\UseMicrotypeSet[protrusion]{basicmath} % disable protrusion for tt fonts
}{}
\usepackage[left=2.5cm, right=2cm, top=2cm, bottom=2cm]{geometry}
\usepackage{hyperref}
\hypersetup{unicode=true,
            pdfborder={0 0 0},
            breaklinks=true}
\urlstyle{same}  % don't use monospace font for urls
\usepackage[style=apa]{biblatex}
\ExecuteBibliographyOptions{refsegment=chapter}
\addbibresource{references-cultural-evolution.bib}
\addbibresource{references-bronze-age-burials.bib}
\addbibresource{references-software-and-data-analysis.bib}
\usepackage{longtable,booktabs}
\usepackage{graphicx,grffile}
\makeatletter
\def\maxwidth{\ifdim\Gin@nat@width>\linewidth\linewidth\else\Gin@nat@width\fi}
\def\maxheight{\ifdim\Gin@nat@height>\textheight\textheight\else\Gin@nat@height\fi}
\makeatother
% Scale images if necessary, so that they will not overflow the page
% margins by default, and it is still possible to overwrite the defaults
% using explicit options in \includegraphics[width, height, ...]{}
\setkeys{Gin}{width=\maxwidth,height=\maxheight,keepaspectratio}
\IfFileExists{parskip.sty}{%
\usepackage{parskip}
}{% else
\setlength{\parindent}{0pt}
\setlength{\parskip}{6pt plus 2pt minus 1pt}
}
\setlength{\emergencystretch}{3em}  % prevent overfull lines
\providecommand{\tightlist}{%
  \setlength{\itemsep}{0pt}\setlength{\parskip}{0pt}}
\setcounter{secnumdepth}{5}

%%% Use protect on footnotes to avoid problems with footnotes in titles
\let\rmarkdownfootnote\footnote%
\def\footnote{\protect\rmarkdownfootnote}

%%% Change title format to be more compact
\usepackage{titling}

% Create subtitle command for use in maketitle
\newcommand{\subtitle}[1]{
  \posttitle{
    \begin{center}\large#1\end{center}
    }
}

\setlength{\droptitle}{-2em}

  \title{}
    \pretitle{\vspace{\droptitle}}
  \posttitle{}
    \author{}
    \preauthor{}\postauthor{}
    \date{}
    \predate{}\postdate{}
  
% turn off page header titles
\pagestyle{plain}
% change language to german
\usepackage[ngerman]{babel}
% turn off bibliography at the end of document
\let\pby\printbibliography
\renewcommand{\printbibliography}{}
% footnotes in tables
\usepackage{tablefootnote}
% promille symbol
\usepackage{textcomp}
% move all figures and tables to the end of the document
\usepackage[nomarkers,notables]{endfloat}
\AtBeginFigures{\setcounter{chapter}{0}}
\renewcommand{\figuresection}{Abbildungen}
% fancy headers
\usepackage{fancyhdr}
\pagestyle{fancy}
% section header formatting
\usepackage{titlesec}
\titleformat{\section}[display]{\normalfont\Large\bfseries}{\thesection}{0pt}{}
\titleformat{\subsection}[display]{\normalfont\large\bfseries}{\thesubsection}{0pt}{}
\titleformat{\subsubsection}[display]{\normalfont\normalsize\bfseries}{\thesubsubsection}{0pt}{}
\titleformat{\paragraph}[runin]{\normalfont\normalsize\bfseries}{}{0pt}{}
\titleformat{\subparagraph}[runin]{\normalfont\normalsize\bfseries}{}{0pt}{}
% figure rotation
\usepackage{pdflscape}
\DeclareDelayedFloatFlavor{landscape}{figure}
% ignore note field in citations
\AtEveryBibitem{%
  \clearfield{note}%
}
\usepackage{booktabs}
\usepackage{longtable}
\usepackage{array}
\usepackage{multirow}
\usepackage[table]{xcolor}
\usepackage{wrapfig}
\usepackage{float}
\usepackage{colortbl}
\usepackage{pdflscape}
\usepackage{tabu}
\usepackage{threeparttable}
\usepackage{threeparttablex}
\usepackage[normalem]{ulem}
\usepackage{makecell}

\begin{document}

\begin{titlepage}

  \vspace*{\fill}

    \begin{center}

        \Huge Ein computerbasiertes Cultural Evolution Modell zur Ausbreitungsdynamik europäisch-bronzezeitlicher Bestattungssitten

        \vspace{2cm}

        \huge Masterarbeit

        \large im Fach prähistorische und Historische Archäologie der Philosophischen Fakultät der Christian-Albrechts-Universität zu Kiel

        \vspace{2cm}

        \large vorlegt von \\
        \huge Clemens Schmid

    \end{center}

\vspace{4cm}

\large Erstgutachter: PD Dr. Oliver Nakoinz \\
\large Zweitgutachter: Dr. Martin Hinz

\vspace{1cm}

\large Kiel im September 2018

  \vspace*{\fill}

\end{titlepage}

\renewcommand{\chaptermark}[1]{\markboth{#1}{}}
\renewcommand{\sectionmark}[1]{\markright{\thesection\ #1}}
\fancyhf{}
\fancyhead[LE,RO]{\textbf{\thepage}}
\fancyhead[LO]{\textbf{\nouppercase{\rightmark}}}
\fancyhead[RE]{\textbf{\nouppercase{\leftmark}}}
\fancypagestyle{plain}{%
  \fancyhead{} % get rid of headers
  \renewcommand{\headrulewidth}{0pt} % and the line
}

\setcounter{tocdepth}{3}
\tableofcontents

\parskip 4pt \setlength{\textfloatsep}{10pt plus 1.0pt minus 2.0pt}

\AtEndDocument{
  \begin{titlepage}

  \huge \textbf{Erklärung}

  \vspace{1.5cm}

  \large Hiermit erkläre ich, dass ich die vorliegende Arbeit selbstständig und ohne fremde Hilfe angefertigt und keine anderen als die angegebenen Quellen und Hilfsmittel verwendet habe.

  \vspace{1cm}

  \large Die eingereichte schriftliche Fassung der Arbeit entspricht der auf dem elektronischen Speichermedium.

  \vspace{1cm}

  \large Weiterhin versichere ich, dass diese Arbeit noch nicht als Abschlussarbeit an anderer Stelle vorgelegen hat.

  \vspace{2cm}

  \noindent\rule{8cm}{0.4pt}

\end{titlepage}

}

\hypertarget{intro}{%
\chapter{Einführung}\label{intro}}

Die vorliegende Master-Arbeit entstand 2017-2018 am Institut für Ur- und
Frühgeschichte der Christian-Albrechts-Universität zu Kiel unter
Betreuung von Priv.-Doz. Dr.~Oliver Nakoinz und Dr.~Martin Hinz. Sie ist
quelloffen und voll reproduzierbar.

Ausgangspunkt der Überlegungen für diese Arbeit war die der
\emph{Evolutionary Archaeology} entlehnte Frage, ob kulturhistorische
Transformations- und Ausbreitungsprozesse sinnvoll und gewinnbringen in
einer Modellimplementierung abgebildet werden können, die Ideen,
Traditionen und Innovationen als handlungsfähige Agenten begreift. Um
diesen Ansatz in einer archäologische Abschlussarbeit zu erforschen,
wurde ein Fallbeispiel gewählt, in dem sich gegenseitig ausschließende
Ideen über einen langen Zeitraum in einem großen Raum konkurrieren:
Bestattungstraditionen in der europäischen Bronzezeit. Diese lassen sich
vereinfacht durch zwei Dichotomien beschreiben: Brandbestattung im
Gegensatz zu Körperbestattung sowie das Flachgrab in Abgrenzung zum
Hügelgrab. Für diesen Zusammenhang steht mit Radon-B ein Datensatz von
mehreren tausend \textsuperscript{14}C-datierten Gräbern mit
Metainformationen zur Verfügung, der die Phänomene in einem Zeitfenster
von 2200 bis 800calBC in hoher Auflösung abbildet.

Die Arbeit verfolgt also folgende Hauptziele:

\begin{itemize}
\tightlist
\item
  Die Zusammenfassung der Grundlagen, Geschichte und Fragestellungen der
  \emph{Cultural Evolution Theory} in Hinblick auf Entwicklung und
  Ausbreitung von modernen, komplexen Verhaltensmustern.
\item
  Die Vorstellung von Paradigmen und Herausforderungen der
  Thanatoarchäologie.
\item
  Die Betrachung der besonderen Qualität von Bestattungssitten als mit
  dem Tod verknüpften Ideen aus kulturevolutionärer Perspektive.
\item
  Die Überblicksartige Erfassung der Entwicklung bronezeitlicher
  Bestattungssitten in Nordwest-, Nord-, und Zentraleuropa in den vielen
  hundert relevanten Kulturzusammenhängen.
\item
  Die Auswertung des Radon-B Datensatzes hinsichtlich der zeitlichen und
  räumlichen Verbreitung der primären Bestattungsformen.
\item
  Die Konstruktion und Anwendung eines deduktiven, simulationsgeeigneten
  Modells, das theoretische Vorüberlgeungen aufgreift und zur
  Kontextualisierung und Erforschung der Realweltentwicklung geeignet
  ist.
\end{itemize}

Die Arbeit ist, sieht man von dieser kurzen Einleitung und einer ebenso
kurzen Zusammenfassung am Ende ab, in drei Abschnitte gegliedert:

Ein erster Teil, Kapitel @ref(cultural-evolution, zeichnet die
Geschichte der \emph{Cultural Evolution Theory} nach, führt näher in die
forschungsgeschichtlich wichtige Unterströmung \emph{Memetik} ein und
gibt einen Überblick über aktuelle Entwicklungen und Fragestellungen --
zunächst ganz allgemein, dann mit Schwerpunkt auf den Aspekt
\emph{Cultural Transmission}, der für diese Arbeit von besonderer
Bedeutung ist. Ein solcher Überblick muss allein aufgrund der Vielfalt
und Heterogenität von Themen und Fächern die er berührt stets
unvollständig bleiben. Aus diesem Grund wurde zur bedarfsgerechten
Vertiefung und als Kompendium für die Zukunft viel Literatur angefügt.

Der zweite Teil, Kapitel \ref{bronze-age-burial-rites}, stellt das
Fallbeispiel der Entwicklung bronzezeitlicher Bestattungssitten vor.
Dabei wird zunächst auf eine Reihe von Vorüberlegungen eingegangen, die
den Tod und seine archäologische Erforschung kontextualisieren. Die
damit erfassten Probleme und Paradigmen sind essentiell für eine
Betrachtung von Bestattungssitten aus \emph{Cultural Evolution}
Perspektive. Der größte Teil des Kapitels ist einer Beschreibung der
Entwicklung von Bestattungspraktiken im Untersuchungsareal gewidmet.
Diese nimmt naturgemäß viel Raum ein und muss dennoch als äußerst
reduziert gelten. Ein ausführliches Quellenstudium hätte den Rahmen
dieser Arbeit gesprengt. Die aus wenigen Übersichtstexten kompilierte
Übersicht ist immerhin geeignet, die Repräsentativität des Radon-B
Datensatzes abzuschätzen.

Ein dritter und letzter Teil, Kapitel \ref{data-analysis}, dient der
Präsentation und Auswertung des Datensatzes. Das erarbeitete Modell und
seine Implementierung als computerbasiertes Simulationswerkzeug werden
erklärt, wofür sowohl die inhaltlichen Zusammenhänge als auch die
technische Umsetzung thematisiert werden müssen. Aus dem Vergleich von
realer und simulierter Entwicklung anhand eines Proxies, der sowohl aus
den \textsuperscript{14}C-Daten als auch den Simulationsergebnissen
abgeleitet werden kann, ergeben sich interpretierbare Beobachtungen. Die
Simulation dient im folgenden auch als Vergleichsgegenstand für die
Untersuchung von Unterschieden in kultureller und räumlicher Distanz
sowie der Messung von diachroner Einflussbeziehungen im Netzwerk der
europäischen Großregionen. Kapitel \ref{data-analysis} ist also endlich
der quantitativen Forschung gewidmet, die für diese Arbeit unternommen
wurde.

\hypertarget{cultural-evolution}{%
\chapter{Cultural Evolution}\label{cultural-evolution}}

\hypertarget{definition-und-geschichte}{%
\section{Definition und Geschichte}\label{definition-und-geschichte}}

Die Grundaussage der \emph{Cultural Evolution Theory} ist, dass die
Prozesse der natürlichen Entwicklung von Spezies durch Evolution auch
bei der menschlichen Kulturentwicklung wirken. Mechanismen der Evolution
wie \emph{Selektion} und \emph{Mutation} wären entscheidend dafür,
welche Verhaltensweisen, Ideen und Innovationen sich langfristig
durchsetzen könnten. Entsprechend könnte biologische Terminologie und
Modellbildung zumindest eingeschränkt auch in anthropologischen
Kontexten sinnvoll eingesetzt werden.

Cultural Evolution Theory wird in der archäologischen Fachliteratur vor
allem als \emph{Darwinian Archaeology} oder \emph{Evolutionary
Archaeology} diskutiert. Daneben gab und gibt es in der
Forschungsgeschichte eine ganze Reihe weiterer Begriffe und Schulen, die
mit dem Evolutionsbegriff verknüpft sind. Das ist kein rein
archäologisches Forschungsgebiet: Unter anderem Verhaltensbiologie,
Neurologie, Genetik, Soziologie und alle Anthropologischen Fächer sind
betroffen und haben sich an dieser Diskussion beteiligt. Die Übertragung
biologisch-evolutiver Wirkmechanismen zur Erklärung menschlichen
Verhaltens war bereits Gegenstand akademischer Debatte, lange bevor
Charles Darwins (*1809 - †1882) Evolutionstheorie mit den Standardwerken
\emph{On the Origin of Species} \footnote{\textcite{Darwinoriginspeciesmeans1859}}
und \emph{The Descent of Man}\footnote{\textcite{Darwindescentmanselection1871}}
in Fachwelt und Öffentlichkeit verarbeitet wurden\footnote{\textcite{petermann_geschichte_2004},
  501-502.}: Parallel zu den Entwicklungen in den Naturwissenschaften --
allerdings mit allgemein geringen Wechselwirkungen -- wurde
Evolutionstheorie im wissenschaftlichen Diskurs der Sozialwissenschaften
reflektiert. Eine erste wesentliche Spannungslinie, die hier betrachtet
werden muss, reicht von \emph{Evolutionismus} über
\emph{Neoevolutionismus} hin zu \emph{Kulturrelativismus} und
\emph{Multilinearer Evolution}. Sie hat in der archäologischen
Theoriediskussion große Wirkung entfaltet und ist untrennbar mit der
Geschichte des Faches verknüpft.

\hypertarget{biology}{%
\subsection{Evolutionsbiologie: Von Darwinismus zu Erweiterter
Synthese}\label{biology}}

Die biologische Forschung ist nicht bei Charles Darwin stehen geblieben
sondern hat sich über die Korrekturen im \emph{Neo-Darwinismus} um 1890,
über die \emph{Synthetische Theorie der biologischen Evolution} um 1940
und die \emph{Erweiterte Synthetischen Theorie} ab dem Ende der 1990er
bis in die Gegenwart weiterentwickelt. Ende des 19. Jahrhunderts wurden
wesentliche Aspekte biologischen Evolutionstheorie noch kontrovers
diskutiert\footnote{\textcite{bowler_evolution_1989}, 188-202.}.
Insbesondere der Streit zwischen darwinistischer Evolution durch
Selektion und \emph{lamarckistischer Evolution} durch Vererbung
erworbener Eigenschaften war nicht entschieden. Jean-Baptiste de Lamarck
(*1744 - †1829) war zwar weitestgehend überholt, aber sein
Adaptionsgedanke lebte in \emph{Neo-Lamarckismus}\footnote{\textcite{bowler_evolution_1989},
  236-247.} und \emph{Orthogenese}\footnote{\textcite{bowler_evolution_1989},
  247-250.} fort, die als Alternativen für den vor allem von August
Weismann (*1834 - †1914) und Alfred Russel Wallace (*1823 - †1913)
propagierten \emph{Neo-Darwinismus}\footnote{\textcite{bowler_evolution_1989},
  251-260.} diskutiert wurden. Weismann vertrat einen dogmatischen
\emph{Selektionismus} und führte mit der \emph{Keimplasmatheorie} eine
Erklärung für Vererbung ein, die wichtige Aspekte der Genetik vorwegnahm
und lamarckistische Adaption ausschloss. Die frühe \emph{Genetik} ging
jedoch nicht aus darwinistischem Selektionismus hervor. Stattdessen
wurde die Wiederentdeckung der bereits von Gregor Mendel (*1822 - †1884)
1866 publizierten \emph{Mendelschen Vererbungsregeln} um 1900 vor allem
im Kontext der \emph{Saltationstheorie} diskutiert, die nicht Selektion,
sondern tiefgreifende, spontane Mutationen als Motor der Evolution
favorisierte\footnote{\textcite{bowler_evolution_1989}, 260-261.}. Ein
bekannter, streitbarer Vertreter dieser Schule war William Bateson
(*1861 - †1926). Er prägte den Begriff \emph{Genetik} und trug
maßgeblich zur Popularisierung der Mendelschen Regeln bei. Ihm entgegen
stand die ebenfalls noch junge Wissenschaft der \emph{Biometrie}, die
statistische Methoden zur Untersuchung von Populationen einführte und
die Bedeutung von Selektion hervorhob. Darwins Cousin Francis Galton
(*1822 - †1911) gilt als Vorreiter dieser Strömung, vertrat aber eine
fehlerhafte, inkohärente Vererbungslehre. Erst Nachfolgern wie Walter
Frank Raphael Weldon (*1860 - †1906) und Karl Pearson (*1857 - †1936)
gelang der Nachweis, dass Selektion zur nachhaltiger Veränderung in
Populationen führen kann\footnote{\textcite{bowler_evolution_1989},
  256-260.}. Die Debatte um den genauen Mechanismus der Evolution war
entscheidend für die Biologie im späten 19. und frühen 20. Jahrhundert
-- die Interdependenzen von Mutation, Adaption und Selektion waren noch
nicht verstanden.

In den ersten Jahrzehnten des 20. Jahrhunderts wurde Hugo de Vries
(*1848 - †1935) \emph{Mutationstheorie} die in Fachkreisen am weitesten
verbreitete Evolutionstheorie. Nach de Vries funktioniert Mutation wie
in der \emph{Saltationstheorie} als schnelles Hervorbringen neuer,
vollständiger Varianten, die dann durch Selektion sortiert werden. Viele
seiner Anhänger verwarfen die Notwendigkeit für Selektion jedoch -- so
z.B. Thomas Hunt Morgan (*1866 - †1945) oder Wilhelm Johannsen (*1857 -
†1927), die moralisch und inhaltlich gegen eine tragende Rolle von
Selektion argumentierten und gleichzeitig wesentliche Beiträge zur
Definition der Vererbungseinheiten im Kontext der noch jungen
\emph{Genetik} leisteten. Morgans Forschung an Fruchtfliegen führte zu
einem signifikant besseren Verständnis von Vererbung, das den langen
Konflikt zwischen Mendelianern und Biometrikern effektiv löste. Ab 1920
setzten sich in Großbritannien und den USA \emph{Präformationslehre} und
die Mechanismen \emph{Natürliche Selektion} und kleinteilige, zufällige
Mutation als die wesentlichen, theoretischen Grundlagen der Evolution
der Arten durch, nachdem alle anderen zuvor diskutierten Theorien
weitestgehend ausgeschlossen worden waren. In einzelnen Fachbereichen
und in Kontinentaleuropa wurden alternative Ansätze -- insbesondere
\emph{lamarckistische Evolution} -- allerdings noch wesentlich länger
diskutiert und gelehrt\footnote{\textcite{bowler_evolution_1989},
  268-273.}. Der nun folgende Prozess der Konsolidierung und
Vereinheitlichung der Evolutionstheorie in allen Subdisziplinen der
Biologie dauerte bis in die 40er Jahre an und wird als \emph{Synthese}
bezeichnet. Sie entwickelte sich aus einem langjähriger akademischen
Diskurs in vielen wesentlichen Publikationen. Letztlich ging aus ihr die
moderne Evolutionsbiologie hervor und viele Bereiche wie Paläontologie,
Populationsbiologie und die botanische und zoologische Feldforschung
erhielten deutliche Anstöße -- auch zur Quantifizierung und
Systematisierung von Forschung. Die \emph{Synthese} wurde von
Protagonisten wie Julian S. Huxley (*1887-†1975), Sewall G. Wright
(*1889 - †1988), Ronald A. Fisher (*1890 - †1962), John B. S. Haldane
(*1892-†1964), Theodosius G. Dobzhansky (*1900-†1975), Bernhard Rensch
(*1900-†1990), Edmund B. Ford (*1901 - †1988), George G. Simpson
(*1902-†1984) und neben vielen anderen vor allem Ernst Mayr (*1904 -
†2005) getragen. Trotz ihrer augenscheinlich anregenden Wirkung verblieb
berechtigte Kritik an der \emph{Synthetischen Evolutionstheorie}: Der
rigide durchgesetzte Schwerpunkt auf Selektionismus auf Darwinismus
führte etwa zunächst zu einer globalen Ablehnung später rehabilitierter
Phänomene wie zum Beispiel \emph{Genetischer Drift}\footnote{\textcite{bowler_evolution_1989},
  325-327 \& 333-339.}.

Ab den späten 1990ern und besonders nach der Jahrtausendwende wurde
immer häufiger der Wunsch nach einer Erneuerung des Paradigmas des
\emph{Modernen Synthese} artikuliert. Die Methoden und
Erkenntnismöglichkeiten der biologischen Subdisziplinen hatten sich
massiv weiterentwickelt, und es schien sinnvoll, die alten Maximen zu
ersetzen oder zumindest zu erweitern. Die Diskussion um diese
\emph{Erweiterte Synthese} hält bis in Gegenwart an. Wesentliche
Konzepte, die die alte \emph{Synthese} noch nicht kennen konnte, sind
zum Beispiel \emph{Evolvierbarkeit}\footnote{\textcite{wagner_robustness_2013}},
\emph{phänotypische Plastizität}\footnote{\textcite{pigliucci_phenotypic_2001}}
oder der neue aufgegriffene Fachbereich der Evolutionären
Entwicklungsbiologie (\emph{EvoDevo})\footnote{\textcite{muller_evodevo_2007}}.
Ihre Integration und die Reflexion über die Mechanismen der
Artenentwicklung wird auch in Zukunft Gegenstand der biologischen
Fachdiskussion bleiben\footnote{\textcite{pigliucci_elements_2010}}.

\hypertarget{evolutionismus-und-sozialdarwinismus}{%
\subsection{Evolutionismus und
Sozialdarwinismus}\label{evolutionismus-und-sozialdarwinismus}}

Klassischer \emph{Evolutionismus} ist ein Überbegriff für die erste
Übertragung biologischer Evolutionsforschung auf die Kulturgeschichte.
Er betont den Aspekt des schrittweisen, kulturellen Aufstiegs und der
Zunahme organisatorischer Komplexität. Zivilisation hätte sich über
mehrere Fortschrittsstufen von einem primitiven Urzustand zur modernen
Industriegesellschaft weiterentwickelt. Die Beschreibung einer Kultur
kann vor diesem Hintergrund in sehr einfachen Begriffen und mit wenigen
Parametern erfolgen\footnote{\textcite{noauthor_evolutionismus_1986}}.
Bei der ersten Formulierung Evolutionistischer Theorie hat Darwin nur
eine untergeordnete Rolle gespielt. Protagonisten wie Herbert Spencer
(*1820 - †1903) und John Lubbok (*1834 - †1913) orientierten sich
stärker an Charles Lyell (*1797 - †1875), der in der ersten Hälfte des
19. Jahrhunderts mit den geologischen Schlüsselprinzipien
\emph{Aktualismus} (rezente, natürliche Phänomene haben so auch in der
Vergangenheit stattgefunden) und \emph{Gradualismus} (geologischer
Wandel ist langsam und stetig) wesentliche Grundlagen für die
Evolutionsforschung gelegt hatte. Die Prinzipien gaben der
stratigraphischen Vergesellschaftung menschlicher Skelettüberreste mit
pleistozänen Tierknochen eine neue Bedeutung, die eine auf breiter Front
\emph{Vergleichende Methode} rechtfertigte. Damit wurden
vorgeschichtliche Gesellschaften dem Vergleich mit `primitiven',
rezenten Gesellschaften zugänglich. Evolutionismus konzentrierte sich
nicht auf Mechanismen der Evolution wie Mutation und Selektion, sondern
griff ein dem Kapitalismus entlehntes Konzept von Wettbewerb und
Weiterentwicklung der Kulturen auf, das durch Vergleich mit rezenten
Gesellschaften und deren Organisationsgrad versteh- und kategorisierbar
geworden war. Die Evolutionisten bildeten keine kohärente Schule.
Stattdessen wurde eine Gruppe von Individuen -- maßgeblich Lewis Henry
Morgan (*1818 - †1881), Herbert Spencer, John Ferguson McLennan (*1827 -
†1881), Edward Burnett Tylor (*1832 - †1917) und John Wesley Powell
(*1834 - †1902) -- abschätzig von Gegnern mit diesem Begriff belegt. Dem
Evolutionismus wurde vorgeworfen, die Aussagekraft materieller Kultur
über die soziale Organisation vorgeschichtlicher Gesellschaften
positivistisch überbewertet zu haben. \emph{Konjekturalgeschichte} und
\emph{Vergleichende Methode} hätten zu einer Perspektive unlinearer
Entwicklung geführt, die durch Stufengliederung der
Menschheitsgeschichte kulturelle Vielfalt unangemessen reduziert und
durch die Konzentration auf progressive Entwicklungsabläufe zu falschen
ethnologischen Beobachtungen geführt habe\footnote{\textcite{petermann_geschichte_2004},
  464-474, 734.}. Zuletzt wäre die vorgenommene Abgrenzung von
Entwicklungsstadien mit einer Teleologisierung auf die moderne,
westliche Gesellschaft verbunden und damit Grundlage einer
Rechtfertigung von Rassismus, Eurozentrismus und Imperialismus. Damit
wurde der Begriff \emph{Sozialdarwinismus} assoziiert\footnote{\textcite{ShennanGenesmemeshuman2002},
  11.}.

\emph{Sozialdarwinismus} ist ebenso wie Evolutionismus keine kohärente
wissenschaftstheoretische Schule, sondern eine polemische Zuschreibung
wissenschaftlicher, ideologischer und politischer Gegner. Die heftige
Kontroverse, die rund um Evolutionstheorie in der zweiten Hälfte des 19.
Jahrhunderts entstand, wurde von Propagandisten wie Thomas Henry Huxley
(*1825 - †1895) (\emph{Darwin's Bulldog}) oder, im deutschsprachigen
Raum, Ernst Haeckel (*1834 - †1919) getragen. Die Erkenntnisse hatten
Konsequenzen für fundamentale weltanschauliche Fragen -- entsprechend
wurde die Diskussion von der Presse aufgegriffenen und einer breiten
Öffentlichkeit präsentiert. Das hatte starke, oft unangemessene
Vereinfachung der Themenstellung zufolge. Die Reduktion von
Evolutionstheorie auf griffige Phrasen wie \emph{Survival of the
Fittest} und \emph{Natural Selection} wirkte sich schließlich auch auf
den Diskurs in den Sozialwissenschaften aus. Spencer entwickelt in
seinem Hauptwerk \emph{The Principles of Sociology}\footnote{\textcite{SpencerHerbertSpencerPrinciples1898}}
das Narrativ eines evolutionären Kampf ums Dasein, der nur in den
jüngsten Phasen der Menschheitsgeschichte von Altruismus begleitet
wird\footnote{\textcite{petermann_geschichte_2004}, 501-510.}. Diese
sozialphilosophische Theorie fällt im Klima der fortgeschrittenen
Industrialisierung und deren Konkurrenzgesellschaft auf fruchtbaren
Boden. Noch heute wirkt der Gedanke eines Überlebenskampfs im
marktwirtschaftlichen Geschehen nach und hat sich etwa über christliche
Prädestinationslehre zu jenem traditionell amerikanischen Topos
stabilisiert, der sich politisch gegen staatliche Eingriffe ins
Wirtschaftssystem und für individuelle, zwischenmenschliche Solidarität
ausspricht. Spencer beeinflusste eine ganze Reihe amerikanischer
Ethnologen und Soziologen\footnote{\textcite{smith_cultural_1992}, 62.},
darunter William Graham Sumner (*1840 - †1910), Lester Frank Ward (*1841
- †1913) und Franklin Henry Giddings (*1855 - †1931). Sie teilten
Spencers Verständnis biosozialer Evolution und deren
empirisch-positivistischer Erforschbarkeit, jeder repräsentiert
gleichermaßen aber gegensätzliche Ansichten darüber, wie stark die
evolutiven Prozesse menschliche Gesellschaften determinieren. Europas
Sozialdarwinisten waren keine Spencerianer, dafür aber umso stärker
Theorien radikal-biologischen und rassistischen Existenzkampfs
verpflichtet. Zu nennen sind unter anderem Gustav Ratzenhofer (*1842 -
†1904), Jakov Novicov (*1849 - †1912), Michelangelo Vaccaro (*1854 -
†1937) und besonders der jüdisch-polnische Jurist und Soziologe Ludwig
Gumplowicz (*1838 - †1909), der mit seinem wissenschaftlichen Rassismus
in einer Rede im September 1933 von Adolf Hitler fast wörtlich zitiert
wurde\footnote{\textcite{petermann_geschichte_2004}, 511-524.}:

\begin{quote}
Nie und nirgends sind Staaten anders entstanden als durch Unterwerfung
fremder Stämme seitens eines oder mehrerer verbündeter oder geeinigter
Stämme.

-- \autocite{GumplowiczGrundrissSoziologie1885}
\end{quote}

Ein wichtiger Antrieb für Sozialdarwinistische Theorie war die
biometrische Forschung von Galton, der intellektuelle Fähigkeit als eine
maßgeblich biologisch vererbbare Eigenschaft beschrieb. Ethnische
Herkunft hielt er in einer Form von Rassenlehre für das entscheidende
Kriterium für die Intelligenz eines Individuums. Er sprach sich in
dieser Konsequent für bewusste Zuchtwahl beim Menschen aus und prägte
den Begriff \emph{Eugenik}\footnote{\textcite{bowler_evolution_1989},
  256-257.}.

\hypertarget{cultural-relativism-neoevolutionism}{%
\subsection{Kulturrelativismus und
Neoevolutionismus}\label{cultural-relativism-neoevolutionism}}

Kritiker des Evolutionismus in der ersten Hälfte des zwanzigsten
Jahrhunderts waren Vertreter der britischen \emph{Social Anthropology},
deutscher \emph{Kulturgeschichte} und vor allem der von Franz Boas
(*1858 - †1942) etablierten, amerikanischen \emph{Kulturanthropologie}.
Die Gemeinsamkeit dieser Schulen und Strömungen liegt an ihrem
traditionellen Fokus auf den jeweiligen naturräumlichen, historischen
und soziopolitischen Kontext einzelner kultureller Ausprägung. Boas war
Jude, absolvierte ein naturwissenschaftliches Studium in Deutschland und
emigrierte nach seiner Zuwendung zur Ethnologie in die USA. Er gilt als
Begründer des \emph{historischen Partikularismus}, der sich gegen
deduktive, umfassende Erklärungsmodelle wie Evolutionismus und
Diffusionismus wandte. Letzterer hatte sich parallel zu ersterem vor
allem in Europa aus den Arbeiten von Friedrich Ratzel (*1844 - †1904),
Leo Frobenius (*1873 - †1938) -- Gründer der Kulturkreislehre -- sowie
Hermann Baumann (*1902 - †1972), Gustaf Kossinna (*1858 - †1931) und
verschiedenen Autoren der \emph{Wiener Schule} der Völkerkunde
herausgebildet. Boas verwarf diese Weltmodelle, die ihnen zugrunde
liegende \emph{Vergleichende Methode} und ihre Analogieschlüsse und
betonte stattdessen eine genaue, empirische Detailanalyse von
Einzelphänomenen. Dabei war Boas Forschungsansatz im Sinne des
\emph{four-field approach}, der Ethnologie, Archäologie, Linguistik und
Physische Anthropologie zusammenführt, breit aufgestellt. Methodisch
vielfältige und empirisch fundierte aber gleichzeitig zeitlich und
räumlich eng begrenzte Fallstudien sollten den Weg zu einer induktiven
Kulturwissenschaft ebnen. Boas begründete damit eine Phase intensiver
Datenaufnahme in der amerikanischen Anthropologie (\emph{Salvage
Ethnography}), die seine Kritiker wiederum als theorielos verurteilten.
1911 erschien sein Werk \emph{The mind of Primitive Man}\footnote{\textcite{Boasmindprimitiveman1911}},
das die wichtigsten Thesen seines \emph{Kulturrelativismus}
zusammenfasst: Es wendet sich gegen biologischen Determinismus, betont
den Einfluss von \emph{Social Learning} und hebt die Multikausalität
historischer Entwicklungen hervor. Kultur sei abhängig von einer
Vielzahl natürlicher und zwischenmenschlicher Parameter. Diese
Relativität nahm der uniliniearen Gliederung von Kulturzuständen des
Evolutionismus die Grundlage. Andererseits enthielt Boas modernes
Verständnis der Interaktion zwischen Gruppen bereits Grundaussagen der
\emph{Cultural Transmission Theory}\footnote{\textcite{obrien_epistemological_2002}}
(siehe Kapitel \ref{cultural-transmission}). Boas war ein politischer
Mensch und argumentierte mit Kulturrelativismus gegen Rassismus und
Faschismus\footnote{\textcite{petermann_geschichte_2004}, 643-655.}.
Schüler von Boas (\emph{Boasianer}) wie Clark Wissler (*1870 - †1947),
Elsie Clews Parsons (*1875 - †1941), Alfred Kroeber (*1876 - †1960),
Alexander Goldenweiser (*1880 - †1940), Robert Lowie (*1883 - †1957),
Paul Radin (*1883 - †1959), Edward Sapir (*1884 - †1939) prägten die
amerikanische Ethnologie nachhaltig und führten über Jahrzehnte einen
erbitterten Diskurs mit Evolutionisten und Neoevolutionisten\footnote{\textcite{petermann_geschichte_2004},
  654-688.}.

\emph{Neoevolutionismus} -- der Begriff wiederum eine Fremdzuschreibung
-- bezeichnet eine Strömung, die als Reaktion auf berechtigte Kritik am
Evolutionismus in den 30er Jahren des 20. Jahrhunderts und insbesondere
nach dem 2. Weltkrieg an Dynamik gewann. Sie verbindet Ansätze, die sich
zwar sozialdarwinistischem Biodeterminismus verweigern, andererseits
aber dennoch bewusst nach Gesetzmäßigkeiten soziokultureller Prozesse
suchen um der Anthropologie ein höheres Abstraktionsniveau zu
erschließen. Aus dieser Definition heraus lassen sich dem
Neoevolutionismus einige der bedeutendsten Ethnologen und Archäologen
zuordnen: Vere Gordon Childe (*1892 - †1957), Karl Wittfogel (*1896 -
†1988), George Murdock (*1897 - †1985), Leslie White (*1900 - †1975) und
Julian Haynes Steward (*1902 - †1972) Auch die Arbeit einer nachfolgende
Generation mit Protagonisten wie Elman Service (*1915 - †1996), Morton
Fried (*1923 - †1986), Roy Rappaport (*1926 - †1997), Marshall Sahlins
(*1930) oder Lewis Binford (*1931 - †2011) ist stark von
neoevolutionistischem Denken geprägt.

Vere Gordon Childe, ursprünglich Philologe aus Australien, etablierte
sich in Europa durch seine großen, synthetischen Werke als
Prähistoriker. Ihm gelang es, die Gliederung der Menschheitsgeschichte
in Entwicklungsphasen -- Childe griff Morgans Unterscheidung von
Wildheit, Barbarei und Zivilisation auf -- durch einen multilinearen
Ansatz neu zu beleben und in kohärenten, archäologischen Narrativen
(z.B. \emph{The Dawn of European Civilization}\footnote{\textcite{childe_dawn_1925}},
\emph{Man Makes himself}\footnote{\textcite{childe_man_1936}} oder
\emph{Social Evolution}\footnote{\textcite{childe_social_1951}}) nutzbar
zu machen. Als überzeugter Marxist etablierte er den Topos
\emph{vorgeschichtlicher Revolutionen}, der Marx \emph{Historischen
Materialismus} weiterentwickelt und konkretisiert. Childes Kritiker
waren zunächst vor allem jene Spezialisten, deren Forschung er in seinen
Büchern zusammen zu fassen und zu vereinfachen auf sich genommen hatte.
Der deutsche Soziologe und Sinologe Karl Wittvogel beschäftigte sich mit
dem Einfluss von Bewässerungssystemen im Entstehungsprozess früher
Hochkulturen. Mit seiner Studie zu \emph{Hydraulischen
Gesellschaften}\footnote{\textcite{wittfogel_oriental_1957}} hat er ein
einflussreiches, evolutionistisches Werk vorgelegt, das Staatenbildung
und die Herausbildung der Hierarchie des \emph{orientalischen
Despotismus} mit Verwaltungsnotwendigkeiten von Bewässerungssystemen
erklärt. Wittvogels Theorie hat bemerkenswerte Rezeption und erfahren
und wurde in eine Vielzahl anderer Kulturzusammenhänge hineinprojiziert.
George Murdock war ein Vorreiter der \emph{Cross-Cultural Analysis} und
Begründer der \emph{Human Relations Area Files}\footnote{\url{http://hraf.yale.edu}
  {[}28.01.2018{]}}. Dieses Archiv ist 1949 aus einer von Murdock
entwickelten Sammlung hervorgegangen, enthält strukturierte
Informationen und Literaturlisten zu Kulturmerkmalen vieler hundert --
meist indigener -- Gesellschaften und wird bis heute gepflegt. Murdocks
\emph{transkultureller Vergleich} basiert auf evolutionistischer
Grundlage und ist stark von quantitativer Auswertung mit
ethnostatistischen Methoden geprägt: Sein Hauptwerk \emph{Social
Structure}\footnote{\textcite{murdock_social_1949}} analysiert und
dokumentiert universelle Regeln und Gesetze sozialer Beziehungen anhand
eines Datensatzes von 250 Ethnien. Im Kontext der Kritik am
Evolutionismus wurde auch Murdock vorgeworfen, Kulturzüge unsachgemäß
isoliert betrachtet oder einer solchen Betrachtung zugänglich gemacht zu
haben. Der amerikanische Ethnologe Leslie White war einer der
wichtigsten Protagonisten des Neoevolutionismus. Nach seiner Lektüre von
Morgan und anderen Evolutionisten wie Spencer und Tylor suchte er
explizit die Konfrontation mit dem vorherrschenden Partikularismus der
Boasianer und stellte ihr eine umfassende, materialistische
Kulturtheorie gegenüber. Diese würde objektiven Kulturvergleich im Sinne
einer Wissenschaft der \emph{Kulturologie} entlang einer evolutiven
Skala des Pro-Kopf-Verbrauchs von Energie ermöglichen: \emph{White's
Law}\footnote{\textcite{white_energy_1943},
  \textcite{white_science_1949}}. White betonte die Bedeutung von
Technologie und Wirtschaft für die Herausbildung von Sozialordnung und
Ideologie, erkannte aber auch die einzigartige, symbolschaffende
Kreativität des Menschen an. Kritiker werfen ihm vor, diesen impliziten
Widerspruch niemals aufgelöst zu haben. Dennoch inspirierte Whites
klare, regelbasierte Anthropologie eine Generation von Studierenden die
sich im Kulturrelativismus nicht wiederfinden konnten. Neben White ist
auch Julian Steward eine der tragenden Säulen des Neoevolutionismus.
Steward veröffentlicht 1955 \emph{Theory of Culture Change}\^{}, wo er
\emph{Kulturökologie} als Wissenschaft von definierbaren Ursache-Wirkung
Beziehungen von Natur- und Mensch jenseits des überholten
\emph{Kulturdeterminismus} formuliert. Sein Vorschlag zur Periodisierung
der Ur- und Frühgeschichte folgt einem \emph{multilinearen} Ansatz, der
der \emph{unilinearen} Abfolge von für alle Kulturen immer gleicher
Zustandsformen die Analogentwicklung von \emph{Kulturtypen} -- Typen der
Umweltanpassung -- entgegenstellt. Unter bestimmten natürlichen und
sozialen Bedingungen würden sich bestimmte Verhaltensmuster und Formen
des Zusammenlebens ergeben, nicht aber mit zwingender Notwendigkeit oder
in einer definierten Abfolge. Auch Steward bezog sich methodisch auf
transkulturellen Vergleich, der es ermöglichen sollte, die primären,
subsistenzbezogenen Eigenschaften von techno-ökonomischen
\emph{Kulturkernen} im Gegensatz zum Überbau der sekundären, variablen
Charakterzüge von Kulturen zu definieren. Mehrere Protagonisten der noch
jungen \emph{New Archaeology} wurden von Stewards modernem,
pragmatischem Evolutionismus stark beeinflusst\footnote{\textcite{petermann_geschichte_2004},
  734-761.}.

\hypertarget{evolutionism-modern-theories}{%
\subsection{Moderne Theorien zur
Kulturevolution}\label{evolutionism-modern-theories}}

Eine neue Welle der Auseinandersetzung mit Kulturevolution gewinnt Mitte
der 1970er Jahre an Dynamik\footnote{\textcite{creanza_cultural_2017}}.
Sie lenkt das Interesse weg von Politik und Gesellschaftsstruktur,
sondern abstrahiert auf die basalen Grundzüge menschlichen Denkens.
Dieser Ansatz inkorporiert Ergebnisse und Methoden moderner,
biologischer Verhaltensforschung und erlaubt neue Perspektiven Jenseits
des Evolutionismus und seiner Varianten. Von entscheidender Bedeutung
für die Entstehung dieser Strömungen sind Edward Osborne Wilsons (*1929)
\emph{Sociobiology: The New Synthesis}\footnote{\textcite{WilsonSociobiologynewsynthesis1975}}
und Richard Dawkins (*1941) \emph{The Selfish Gene}\footnote{\textcite{Dawkinsselfishgene1976}},
auf das unten genauer eingegangen werden soll\footnote{\textcite{SmithThreestylesevolutionary2000},
  27.}. Auch Luigi Luca Cavalli-Sforza (*1922), Marcus William Feldmann
(*1942) und andere entwickeln wesentliche Ansätze für den Brückenschlag
zwischen Biologie und Anthropologie\footnote{\textcite{alland_cultural_1972},
  \textcite{cavalli-sforza_models_1973}, \textcite{feldman_models_1975},
  \textcite{feldman_cultural_1976}, \textcite{blum_uncertainty_1978}}.
Um die Jahrtausendwende unterscheidet Eric Aldan Smith schließlich drei
große Strömungen\footnote{\textcite{SmithThreestylesevolutionary2000}.
  Stephen Shennan greift diese Unterscheidung auf
  \autocite[15-18.]{ShennanGenesmemeshuman2002}} in der Untersuchung
menschlichen Verhaltens aus einer Evolutionsperspektive:
\emph{Evolutionary Psychology}, \emph{Human Behavioural Ecology} und
\emph{Dual Inheritance Theory}.

\emph{Evolutionary Psychology} konzentriert sich auf die Entwicklung des
menschlichen Denkens vor dem Hintergrund seiner evolutionären
Geschichte. Selektiver Druck habe zur Ausbildung spezialisierter
Verhaltensmodule geführt, die in bestimmten Situationen bestimmte
Reaktionen auslösen. Von entscheidender Bedeutung für die Entstehung
dieser angepassten Verhaltensmodule sei die \emph{Environment of
Evolutionary Adaptiveness (EEA)}, also die Umgebung, in der sich die
menschliche Entwicklung maßgeblich abgespielt hat. Dabei bezieht sich
die Evolutionary Psychology auf die Lebensrealität pleistozäner Jäger-
und Sammlergruppen, in der der moderne Mensch den überwältigend größten
Teil selektiv wirksamer Generationszyklen durchlebt hat. Die
Selektionsparameter wären in diesem Zeitraum relativ stabil geblieben.
In der Konsequenz seien Menschen heute beispielsweise ideal an das
nomadische Leben in kleinen Gruppen in großer gegenseitiger Abhängigkeit
adaptiert, Männer würden junge, gesunde und hübsche Sexualpartnerinnen
bevorzugen und süße Speisen wären beliebt, weil Süße bei Früchten ein
Indikator für Reife und Genießbarkeit ist. Alle Aspekte des Verhaltens
seien auf bestimmte Gegebenheiten in der \emph{EEA} optimiert und
entsprechend schlecht für eine andere, etwa neolithische oder
postneolithische Lebensweise geeignet\footnote{\textcite{SmithThreestylesevolutionary2000},
  27-29.}. Der Evolutionary Psychology wird vorgeworfen, die
unangemessen vereinfachende Annahmen über vorgeschichtliches Verhalten
zu treffen, ohne sich ausreichend mit jenen archäologischen Daten und
Auswertungsergebnissen auseinanderzusetzen, die eine Rekonstruktion der
tatsächlichen Lebensverhältnisse in der Vorgeschichte erlauben würden.
Aus archäologischer Perspektive griff allen voran Steven Mithen
Überlegungen der Evolutionary Psychology auf\footnote{\textcite{Mithenprehistorymindsearch1996};
  \textcite{mithen_cognitive_1997}}.

\emph{Human Behavioural Ecology} überträgt Ansätze aus der
Verhaltensbiologie auf den Menschen\footnote{\textcite{smith_cultural_1992};
  \textcite{winterhalder_analyzing_2000}}. Dabei nimmt sie den
klassisch-darwinistischen Standpunkt ein, menschliches Verhalten könnte
ebenso wie tierisches als permanente Maximierung des
Reproduktionserfolgs durch Selektion verstanden werden\footnote{\textcite{creanza_cultural_2017}}.
Bewusste oder unbewusste Entscheidungen würden hinsichtlich der Frage
getroffen werden, inwiefern das Ergebnis den Erhalt der eigenen
genetischen Information gewährleistet. Im Zentrum steht dabei die
Beziehung zwischen Mensch und natürlicher Umwelt: ``Welche ökologischen
Faktoren (z.B. Ressourcenverfügbarkeit, Populationsdichte, etc.)
schaffen den Rahmen dafür, dass ein bestimmtes Verhalten (z.B.
Altruismus, Vorratshaltung, etc.) zum Erfolg führt?''. Die ökologische
Nische des Menschen in Relation zu seinen Subsistenzstrategien, seinem
Paarungsverhalten und seiner sozialen Struktur sind wesentliche
Forschungsgegenstände der Human Behavioural Ecology\footnote{\textcite{henrich_search_2001};
  \textcite{kaplan_Theory_2000}; \textcite{voland_evolutionary_1998};
  \textcite{winterhalder_risk-senstive_1999}}. Die kleinteilige
Aufgliederung der Fragestellungen hinsichtlich einzelner Situationen und
Verhaltensweisen erlaubt es dabei, auch komplexe Fragen quantitativ in
einfachen Modellen abzubilden. Diese Modelle versprechen testbare
Aussagen: ``Wenn Frauen ihre Sexualpartner nach dem Kriterium wählen,
wer den Nachwuchs am besten versorgen kann, dann wäre die Anzahl der
Frauen pro Mann proportional zu seinem Reichtum.''. Die Reduktion auf
direkte, kausale Beziehungen birgt jedoch die Gefahr die vielfältigen
Interdependenzen einzelner Verhaltensweisen zu übersehen. Gerade
Langzeitstudien spielen dafür eine wichtige Rolle\footnote{\textcite{belovsky_optimal_1988};
  \textcite{broughton_widening_1997}; \textcite{low_population_1993};
  \textcite{stiner_paleolithic_1999}; \textcite{stiner_tortoise_2000};
  \textcite{winterhalder_population_1988}}. Behavioural Ecology erklärt
die Vielfalt menschlichen Verhaltens aus der großen Diversität
biologischer- und sozialer Nischen, die sehr viele unterschiedliche
Erfolgsstrategien erlaubt. Tatsächlich gäbe es sogar eine Korrelation
zwischen Verhaltensvielfalt und Diversität der sozioökologischen Umwelt.
Sie erlaubt sich eine große Vereinfachung, indem sie die Mechanismen,
die zur Ausbildung einer Verhaltensanpassung führen, nicht hinterfragt:
Die einschränkende Wirkung von Kultur (hier: vererbtes Verhalten) etwa
in Form von Tradition sei untergeordnet, da erfolglose Strategien
unabhängig davon in wenigen Generationen durch biologische Selektion
aussterben würden. Diese bewusste, statistische Vereinfachung von
Übergangsprozessen wird als \emph{Phenotypic Gambit} bezeichnet (und
kritisiert\footnote{\textcite{rubin_phenotypic_2016}}). In ihrer
Konsequenz sei auch anzunehmen, dass der Mensch sein Verhalten schnell
und gut an die revolutionären Veränderungen des Holozän oder der
Industrialisierung angepasst habe\footnote{\textcite{SmithThreestylesevolutionary2000},
  29-31.}.

\emph{Dual Inheritance Theory} postuliert neben der Vererbung von Genen
ein zweites Vererbungssystem von Ideen und Kulturmerkmalen. Auch diese
würden von Generation zu Generation, von Person zu Person und von Tag zu
Tag weitergereicht und stünden unter dem Einfluss von Selektion und
Mutation. Dabei würde sowohl die im genetischen Vererbungssystem
entscheidende, natürliche Selektion wirken als auch eine Selektion durch
bewusste oder unbewusste Entscheidung der Träger von Ideen: Menschen.
Ersterer Selektionsprozess sei Konsequenz der Rückwirkung von Ideen auf
die Fitness ihrer Träger, letzterer ein System von Interdependenzen
verschiedener Ideen, Umweltsituationen und genetischer Determinanten.
Ebenfalls von entscheidender Bedeutung seien die zwischenmenschlichen
Prozesse wie Erziehung, Gefolgschaft oder Freundschaft, die die
Weitergabe von Ideen steuern. Entstehung neuer Ideen aus der Kombination
vorhandener wäre eine Form der Mutation. Da nun also in der
Kulturgeschichte Vererbung, Entstehung von Variabilität und Auswahl nach
Fitnesskriterien als gegeben angenommen werden dürften, und damit große
strukturelle Ähnlichkeit des genetischen und des kulturellen
Vererbungssystems bestünde, sei auch die Übertragung neo-darwinistischer
Methoden auf die Untersuchung von Kulturmerkmalen möglich. Die beiden
Vererbungssysteme könnten unabhängig und in ihrer Interaktion erforscht
werden, wobei Konzepte zur Erklärung des einen potentiell auch zur
Erklärung im anderen geeignet sein könnten. Andererseits gäbe es auch
klare Unterschiede: Beispielsweise erfolgt die Weitergabe genetischer
Information fast ausschließlich vertikal durch sexuelle oder asexuelle
Fortpflanzung, während Ideen beliebig horizontal weitergeben werden,
also unabhängig von Verwandschaft diffundieren können. Individuelle
Menschen sind zwar sowohl Träger vieler Gene als auch vieler
Kulturmerkmale, erstere werden aber nur einmal festgelegt, während
letztere ständigem Wechsel unterliegen. Dual Inheritance Theory ist sich
dieser Unterschiede bewusst, hält sie aber für analytisch bewältigbar.
Da die kulturelle Evolution in anderen zeitlichen, räumlichen und
kausalen Maßstäben agieren würde, könnte diese Theorie auch das
Auftreten von Verhaltensmerkmalen erklären, die aus einer
Reproduktionsperspektive nicht sinnvoll sind. Kulturelle Evolution ist
schneller und flexibler: Anpassung an neue oder für das Überleben von
Menschen ungeeignete Umgebungen geschieht nicht mehr genetisch, sondern
durch Verhaltensanpassung. Genetische Anpassung folgt der kulturellen
langsam, bedeutet aber auch Einschränkungen für die Flexibilität der
kulturellen Evolution\footnote{\textcite{SmithThreestylesevolutionary2000},
  31-33.}.

Smith legt seinem Artikel Tabelle \ref{tab:smiththreestyles} bei, die
die Unterschiede zwischen Evolutionary Psychology, Human Behavioural
Ecology und Dual Inheritance Theory zusammenfasst.

\begin{table*}

\caption{\label{tab:smiththreestyles}Three Styles of Evolutionary Explanation (nach \textcite{SmithThreestylesevolutionary2000})}
\centering
\begin{tabu} to \linewidth {>{\raggedright\arraybackslash}p{15em}>{\raggedright\arraybackslash}p{10em}>{\raggedright\arraybackslash}p{10em}>{\raggedright\arraybackslash}p{10em}}
\toprule
 & Evolutionary Psychology & Behavioural Ecology & Dual.Inheritance Theory\\
\midrule
What is being explained: & Psychological mechanisms & Behavioural strategies & Cultural Evolution\\
Key constraints: & Cognitive, genetic & Ecological, material & Structural, information\\
Temporal scale of adaptive change: & Long-term (genetic) & Short-term (phenotypic) & Medium-term (cultural)\\
Expected current adaptiveness: & Lowest & Highest & Intermediate\\
Hypothesis generation: & Informal inference & Optimality models & Population-level models\\
\addlinespace
Hypothesis-testing methods: & Survey, lab experiment & Quantitative ethnographic observartion & Mathematical modelling and simulation\\
Favoured topics: & Mating, parenting, sex differences & Subsistence, reproductive strategies & Large-scale cooperation, maladaptation\\
\bottomrule
\end{tabu}
\end{table*}

Dual Inheritance Theory ist ein wesentlicher Teil der theoretischen
Grundlage für die Expansionssimulation, die für die vorliegende Arbeit
entwickelt wurde (siehe Kapitel \ref{simulation-theorie}). Um sie besser
zu verstehen, lohnt es sich, einen wichtigen Teil ihrer
Entstehungsgeschichte nachzuzeichnen: Richard Dawkins \emph{Memetik}.

\hypertarget{memetics}{%
\section{Memetik}\label{memetics}}

Memetik (\emph{Memetics}) ist eine Variante der oben beschriebenen Dual
Inheritance Theory. Der Begriff \emph{Meme} wurde 1976 vom britischen
Evolutionsbiologen Richard Dawkins in \emph{The selfish gene}\footnote{\textcite{Dawkinsselfishgene1976}.
  Ich werde im folgenden aus einer Neuauflage des Buches zitieren, die
  2016 40 Jahre nach der Erstpublikation veröffentlicht und um
  Kommentare von Dawkins erweitert wurde:
  \textcite{Dawkinsselfishgene40th2016}.} eingeführt. Obgleich
populärwissenschaftlich hat es doch in verschiedenen Fachbereichen
beachtliche Rezeption erfahren und darf als wichtiger Grundstein dieser
intellektuellen Strömung gelten. Memetik ist eine außergewöhnlich
(öffentlichkeits)wirksame Nischenwissenschaft, die von Kritikern als
irrelevant, unpraktikabel, Ideologie oder Pseudowissenschaft abgelehnt
(siehe Kapitel \ref{memetics-critique}) wurde. Das liegt nicht zuletzt
an ihrem niederschwelligen Zugang zu Cultural Evolution Theory. Ihr
größter Verdienst ist es, Grundgedanken zur Kulturevolutionsforschung zu
Ende zu denken und radikal vereinfacht auszuformulieren. Das hat den
Diskurs in mehrere Fächer getragen und zu einer neuen Reife geführt:
Memetik hat sich im akademischen Diskurs selbst abgeschafft. In dieser
Arbeit steht Memetik auch exemplarisch für andere Strömungen der
Cultural Evolution Theory, die sich an der Formulierung eines einer
einheitlichen und umfassenden Entwicklungsmodells versucht haben -- z.B.
\emph{Cultural Virus Theory}\footnote{\textcite{cullen_contagious_2000}}.

\hypertarget{memetics-dawkins}{%
\subsection{\texorpdfstring{Meme in Dawkins \emph{The selfish
gene}}{Meme in Dawkins The selfish gene}}\label{memetics-dawkins}}

Dawkins führt in \emph{The selfish gene} einen wesentlichen
Perspektivwechsel durch, indem er Evolution nicht aus der Sicht der sich
entwickelnden Organismen sondern aus der sich durch die Organismen
ausbreitenden Gene betrachtet. Gene würden -- freilich nicht bewusst --
Lebewesen als komplexe Vehikel für ihre eigene Reproduktion nutzen und
so die Entwicklung derselben mittel- und langfristig auf
Populationsniveau steuern: \emph{the gene's eye view}. In Kapitel 11,
\emph{Memes: the new replicators}\footnote{\textcite{Dawkinsselfishgene40th2016},
  287-303}, bezieht Dawkins explizit die Spezies Mensch in seine Analyse
mit ein und prüft, ob die Menschheit im selben Umfang dieser
Determination durch den statistischen Willen ihres Erbguts untertan ist?
Dawkins verneint das: Sein Kulturverhalten würde den Menschen von allen
anderen bekannten Lebewesen abheben.

Auch bei Tieren gibt es Verhaltensmuster, die unabhängig von genetischer
Vererbung von Individuum zu Individuum weitergegeben werden:
beispielsweise bestimmte Melodien des Gesangs von Singvögeln, die
erwachsene Tiere voneinander lernen. Kein anderes bekanntes Lebewesen
erreicht jedoch das Komplexitätsniveau des Menschen, der Sprache, Mode,
Ritual, Kunst, Architektur und Technologie besitzt und sie unter
ständigen Anpassungen tradiert. Die Entwicklungen in diesen Bereichen
über archäologische Zeiträume zeigt eine Tendenz hin zu zunehmend
höherer Komplexität und Vielfalt. Geschwindigkeit und Diversität liegen
weit jenseits dessen, was genetische Evolution zu leisten in der Lage
wäre. Erklärungsversuche dafür von Evolutionary Psychology und Human
Behavioural Ecology empfindet Dawkins als unzureichend. Stattdessen
abstrahiert er die von ihm postulierte Evolutionstheorie und führt den
Begriff des \emph{Replikators} ein. Wenn irgendeine Form von Replikator
vorhanden sei, dann würde zwangsläufig Evolution stattfinden. Gene seien
Replikatoren -- Ideen, Gedanken, Meme aber ebenso. Glaubt man einer
Fußnote in Dawkins später kommentiertem Text, so war die Aussage, dass
das Gen nicht die einzige mögliche Form eines Replikators ist, bereits
die wesentliche in Kapitel 11. Umso erstaunlicher, dass er den Moment
der Schöpfung seines Neologismus Meme dennoch theatralisch zelebriert:

\begin{quote}
I think that a new kind of replicator has recently emerged on this very
planet. It ist staring us in the face. It is still in its infancy, still
drifting clumsily about in its primeval soup, but already is it
achieving evolutionary change at a rate that leaves the old gene panting
far behind. The new soup is the soup of human culture. We need a name
for the new replicator, a noun that conveys the idea of a unit of
cultural transmission, or a unit of \emph{imitation}. `Mimeme' comes
from a suitable Greek root, but I want a monosyllable that sounds a bit
like `gene'. I hope my classicist friends will forgive me if I
abbreviate mimeme to meme. {[}\ldots{}{]} It should be pronounced to
rhyme with `cream'.

-- \textcite{Dawkinsselfishgene40th2016}, 291.
\end{quote}

Meme seien kleine abgrenzbare Informationseinheiten wie Melodien,
Geflügelte Worte, Kleidungsmoden oder das Wissen um spezifische
technische Prozesse. So wie Gene Lebewesen als Vehikel gebrauchen, so
wären menschliche Gehirne das Medium, in denen sich Gene ausbreiten. Die
Informationsweitergabe ist nicht auf sexuelle oder asexuelle
Fortpflanzung beschränkt, sondern funktioniert über eine Form der
zwischenmenschlichen Kommunikation, die Dawkins unter dem Überbegriff
Imitation zusammenfasst. Er geht davon aus, dass Meme physische als
Strukturen verschalteter Nervenzellen existieren. Unabhängig davon sei
ihr Effekt deutlich zu spüren: Entitäten, die unser Denken parasitisch
bewohnen und ihre eigene Ausbreitung bezwecken. Dawkins bemüht für eine
erste Illustration das Beispiel des monotheistischen Glaubens an einen
Gott\footnote{Religionskritik ist ein wiederkehrendes Thema in Dawkins
  umfangreichem, populärwissenschaftlichem Werk. Siehe z.B.
  \textcite{dawkins_god_2006}}:

\begin{quote}
Consider the idea of God. {[}..{]} How does it replicate itself? By the
spoken and written word, aided by great music and great art.
{[}\ldots{}{]} What is it about the idea of a god that gives it its
stability and penetrance in the cultural environment? The survival value
of the god meme in the meme pool results from its great psychological
appeal. It provides a superficially plausible answer to deep and
troubling questions about existence. It suggests that injustices in this
world may be rectified in the next. The `everlasting arms' hold out a
cushion against our own inadequacies which, like a doctors placebo, is
none the less effective for being imaginary. These are some of the
reasons why the idea of God is copied so readily by successive
generations of individual brains.

-- \textcite{Dawkinsselfishgene40th2016}, 292.
\end{quote}

Warum ist das menschliche Gehirn empfänglich für Meme? Gibt es einen
klassisch evolutionären Vorteil von dieser Empfänglichkeit? Nach Dawkins
ist die grundsätzliche Kulturfähigkeit des Menschen durchaus ein Effekt
genetischer Mutation und Selektion. Ab einem gewissen Punkt -- in
fließendem Übergang -- sei allerdings der Replikator Meme im Kulturraum
entstanden und hätte die Zügel in die Hand genommen.

\begin{quote}
Whenever conditions arise in which a new kind of replicator \emph{can}
make copies of itself, the new replicators \emph{will} tend to take
over, and start a new kind of evolution of their own. Once this new
evolution begins, it will in no necessary sense be subservient to the
old.

-- \textcite{Dawkinsselfishgene40th2016}, 293.
\end{quote}

Die genetische Evolution habe also den Nährboden bzw. das Medium einer
neuen, viel schnelleren Form der Evolution geschaffen, die andere
Prioritäten für Gesundheit, Langlebigkeit und Fortpflanzungsfähigkeit
ihrer Trägerorganismen anlegt. In vielen Fällen sind diese Prioritäten
ähnlich. Ein Beispiel dafür sind Meme, die positiv konnotiert mit Sex
umgehen. Andererseits gibt es auch Meme wie etwa das Zölibat
katholischer Ordensträger, die aus einer Genperspektive schwerer zu
erklären sind, da sie die Verbreitung der Gene ihrer Träger effektiv
hemmen.

Wenn nun also auch im Medium Kultur die Mechanismen der Evolution
wirken, dann müssten sich die Replikatoren Meme dem selben Druck beugen
wie die Gene in der natürlichen Umwelt. Überleben könnten nur
Replikatorenvarianten mit einer hohen Qualität der Eigenschaften
\emph{Longevity}, \emph{Fecundity} und \emph{Copying-Fidelity}\footnote{\textcite{Dawkinsselfishgene40th2016},
  47-49.}.

\emph{Longevity} -- Langlebigkeit -- sei eine günstige Eigenschaft für
einen Replikatortyp, da er seinen Gesamtbestand im Medium so einerseits
leicht hoch halten kann und ihm außerdem mehr Zeit für Reproduktion zur
Verfügung steht. Einzelne Kopien von Genen sind in ihrer Lebenszeit an
den Organismus gebunden, dessen Aufbau sie kodieren. Instanzen eines
Memes seien dagegen von der menschlichen Gedächtnisleistung abhängig.
Meme könnten aber auch außerhalb von Menschen überdauern, wenn sie etwa
in geschriebener oder digitaler Form abgelegt wurden. Damit könnte das
Meme etwa später wieder einen Menschen infizieren, obgleich kein
direkter Kontakt mit einem Infizierten stattgefunden hat.

\emph{Fecundity} -- Fruchtbarkeit -- sei für die Durchsetzungsfähigkeit
eines Replikatortyps noch wichtiger als longevity: Um so mehr Kopien er
in kürzerer Zeit von sich selbst anfertigen kann, desto schneller wird
er das Medium dominieren. Die Reproduzierfähigkeit eines Memes sollte
von verschiedenen Eigenschaften abhängen, allem voran schlicht seiner
Beliebtheit in oder außerhalb einer assoziierten Adressatengruppe.

\emph{Copying-Fidelity} -- Kopiertreue -- scheint hier zunächst
deplatziert. Ein gewisser Grad an Mutationsfähigkeit ist unerlässlich
für Anpassung. Tritt allerdings bei einem Replikatortyp eine zu große
Instabilität auf, so argumentiert Dawkins, könnte er seine Identität
nicht aufrechterhalten und würde entweder schnell von Varianten
abgelöst, die aus ihm selbst hervorgegangen sind, oder sich völlig
auflösen. Bei Memen scheint gerade das häufig zu passieren:
Übertragungsfehler oder bewusste Modifikation scheinen die Regel, nicht
die Ausnahme zu sein. Damit muss die Qualität von Memen als Replikatoren
in Frage gestellt werden. Dawkins gibt das zu -- diese Frage nach der
Kopiertreue führt ihn zurück zur Definition von Memen. Welche
Information enthält ein individuelles Meme bzw. -- in einem
Analogieschluss -- das Gen?

Das Gen hat hinsichtlich seines mikrobiologischen Aufbaus eine
langwierige Definitionsgeschichte hinter sich\footnote{\textcite{gerstein_what_2007};
  siehe auch Kapitel \ref{biology}}. Dawkins definiert es als einen
dedizierten DNA-Abschnitt mit hinreichender Wirkung und Kopiertreue, um
als selektionsrelevante Einheit zu wirken. Gene schließen sich auf
verschiedenen hierarchischen Ebenen zu Komplexen zusammen, die als
Gruppe gegebenenfalls eine Gesamtwirkung entfalten und wiederum als
ganzes selektionsrelevant wirken kann\footnote{\textcite{Dawkinsselfishgene40th2016},
  36-37.}. Ein ähnliches Strukturverhalten könnte auch für Meme
angenommen werden. Eine Symphonie setzt sich beispielsweise aus einer
Vielzahl einzelner, für sich wiedererkennbarer Melodieabschnitte und
Figuren zusammen. Eine Religion ist die Gesamtheit vieler verknüpfter
Ideen und Ritualen, die als ganzes tradiert werden, eine Konfession
möglicherweise ein \emph{stable set of mutually-assisting
memes}\footnote{\textcite{Dawkinsselfishgene40th2016}, 299. Komplexe
  zusammenhängender Meme wurden später von Dawkins Schülern mit dem
  Begriff \emph{Memeplex} belegt (siehe Kapitel \ref{memetics-history}.}.

\begin{quote}
I conjecture that co-adapted meme-complexes evolve in the same kind of
way as co-adapted gene-complexes. Selection favours memes that exploit
their cultural environment to their own advantage. This cultural
environment consists of other memes which are also being selected. The
meme pool therefore comes to have the attributes of an evolutionarily
stable set, which new memes find it hard to invade.

-- \textcite{Dawkinsselfishgene40th2016}, 301.
\end{quote}

Wie oben ausgeführt, versetzt sich Dawkins in die Perspektive der Gene
hinein und personifiziert sie. Eine empirisch nahe liegende und
terminologisch praktische Metapher um ihre effektive Entwicklung zu
beschreiben. Diese Übertragung möchte er auch für Meme vornehmen. Meme
stünden in starker Konkurrenz zueinander um die Zeit, die Menschen ihnen
widmen und sie gegebenenfalls replizieren: Meme möchten so viele
menschliche Gehirne wie möglich so lange wie möglich dominieren.

\begin{quote}
Time is possibly a more important limiting factor than storage space,
and it is the subject of heavy competition. The human brain, and the
body that it controls, cannot do more than one or a few things at once.
If a meme is to dominate the attention of a human brain, it must do so
at the expense of `rival' memes.

-- \textcite{Dawkinsselfishgene40th2016}, 298.
\end{quote}

Aus dieser Perspektive könnte, so Dawkins, etwa das oben angesprochene
Zölibat-Meme verstanden werden, dass im Memeplex katholischer
Glaubenspraxis Priester freisetzt, keine Zeit an einer Familie zu
verlieren, sondern sich voll auf die Pflege und Verbreitung anderer Meme
der Kirchendoktrin zu konzentrieren. Die Prioritäten von Menschen, Genen
und Memen müssen sich unterscheiden.

\begin{quote}
What we have not previously considered, is that a cultural trait may
have evolved in the way that it has, simply because it is
\emph{advantageous to itself}.

-- \textcite{Dawkinsselfishgene40th2016}, 302.
\end{quote}

Das wirft die philosophische Frage auf, inwiefern Menschen Sklaven ihrer
Gene und Meme sind. Dawkins gibt dazu zu bedenken, dass weder Gene noch
Meme im Gegensatz zum Menschen über Bewusstsein oder Planungsfähigkeit
verfügen. Gene und Meme seien \emph{unconscious, blind,
replicators}\footnote{\textcite{Dawkinsselfishgene40th2016}, 302.}.
Damit könnte sich der Mensch seine Situation bewusst machen, sich
zumindest teilweise den auf ihn wirkenden Entitäten entziehen und neue
Meme schaffen, die seinen Zielen besser dienen: zum Beispiel solche, die
langfristige Kooperation stabilisieren und den immanenten Egoismus von
Genen und Memen ächten.

\hypertarget{memetics-history}{%
\subsection{Kurze Geschichte der Memetik}\label{memetics-history}}

Dawkins war nicht der erste Autor, der den Meme-Begriff für Einheiten
der Kulturübertragung genutzt hat, obgleich die subjektive Orginalität
seiner Wortschöpfung der Wahrheit entsprechen mag\footnote{\textcite{laurent_note_1999}}.
Dawkins Quellen und Inspiration zu hinterfragen würde den Rahmen dieser
Arbeit sprengen, es seien aber immerhin Ted Cloak\footnote{\textcite{cloak_cultural_1966};
  \textcite{cloak_is_1975}} sowie Cavalli-Sforza und Feldman\footnote{\textcite{cavalli-sforza_models_1973}}
erwähnt, die Teile des Replikatormodells vorweg genommen hatten und von
Dawkins darin auch zitiert werden. Ausgehend von \emph{The selfish gene}
hat sich Memetik -- in Dawkins Terminologie -- als außerordentlich
potentes Meme erwiesen. Eine intensive Auseinandersetzung damit fand
allerdings erst mit einigem zeitlichem Abstand in den 1990ern statt. Die
Cultural Evolution Debatte, in deren größeren Kontext Memetik
eingeordnet werden muss, erfuhr indes auch in den 80ern wesentliche
Beiträge: 1981 wurden mit Charles Lumsdens (*1949) und Edward Wilsons
\emph{Genes, Mind and Culture} \autocite[ -- Lumsden und Wilson führen
mit \emph{Culturgen} eine dem Meme ähnliche Beobachtungseinheit
ein.]{lumsden_genes_1981} sowie Cavalli-Sforzas und Feldmans
\emph{Cultural Transmission and Evolution}\footnote{\textcite{cavalli-sforza_cultural_1981}}
zwei wichtige und gleichermaßen kontroverse Monographien veröffentlicht,
die Modelle zur Inkorporation evolutionärer Ansätze für die
anthropologische Forschung ausformulierten. Die Geistes- und
Geschichtswissenschaften standen soziobiologischen Ansätzen in den 70ern
und 80ern grundsätzlich kritisch gegenüber\footnote{u.a.
  \textcite{sahlins_use_1976}} -- entsprechend zurückhaltend war der
Umgang mit diesen Publikationen. Peter James Richerson (*1943) und
Robert Boyd (*1948) nahmen mit \emph{Culture and the Evolutionary
Process}\footnote{\textcite{boyd_culture_1985}} wesentlichen Einfluss
auf diese Diskussion indem sie mit einem expliziten Schwerpunkt auf Dual
Inheritance Theory Teile der festgefahrenen Soziobiologie-Konflikts
vermieden. Gleichzeitig übernahmen sie damit ein der Memetik ähnliches
Gerüst, ohne jedoch den Meme-Begriff zu referenzieren. Die 90er
schließlich waren das Jahrzehnt der Memetik. Maßgeblichen Anteil daran
hatten neben Dawkins\footnote{\textcite{dawkins_viruses_1993}} unter
anderem die Psychologin Susan Blackmore (*1951) mit \emph{The Meme
Machine} \footnote{\textcite{blackmore_meme_1999} -- Ich werde im
  folgenden aus einer mir vorliegenden, deutschen Ausgabe zitieren:
  \textcite{blackmore_macht_2000}}, der Philosoph Daniel Dennett (*1942)
mit \emph{Consciousness Explained}\footnote{\textcite{dennett_consciousness_1991}}
und \emph{Darwin's Dangerous Idea}\footnote{\textcite{dennett_darwins_1995};
  siehe auch \textcite{dennett_brainstorms_1978} und
  \textcite{dennett_elbow_1984}} sowie eine große Zahl von Natur- und
Geisteswissenschaftlern, die sich unter anderem im 1997 eigens
gegründeten Journal of Memetics\footnote{\url{http://cfpm.org/jom-emit/}
  {[}06.01.2018{]}} zu Wort gemeldet haben. Darunter genannt werden
sollen Aaron Lynch\footnote{u.a. \textcite{lynch_thought_1996}} (*1880 -
†1940), Francis Heylighen\footnote{u.a.
  \textcite{heylighen_evolution_1996} und
  \textcite{heylighen_selfish_1992}} (*1960), und Derek
Gatherer\footnote{u.a. \textcite{gatherer_identifying_2002} und
  \textcite{gatherer_spread_2002}}. In den frühen 2000ern wurden die
kritischen Stimmen innerhalb und außerhalb der Community of Practice
immer lauter\footnote{z.B. in \textcite{aunger_darwinizing_2000}} und
leiteten den Abgesang der Memetik ein: Memetik wurde zunehmend weniger
referenziert und dem einem Stigma der Pseudowissenschaftlichkeit
assoziiert. Das Journal of Memetics wurde 2005 in Ermangelung von
Beiträgen eingestellt\footnote{\textcite{vada_what_2015}}. Die
sichtbarste Referenz zur Memetik in der archäologischen
Literaturlandschaft ist Stephen Shennans (*1949) \emph{Genes, memes, and
human history}\footnote{\textcite{shennan_genes_2002}} geblieben, das
sich aber weniger der Memetik als vielmehr der Evolutionary Archaeology
im Allgemeinen widmet.

Die Qualität Susan Blackmores \emph{The meme machine} liegt in der
Synthese vieler Diskurse und Spannungslinien, die sich rund um die
Memetik bis in die 90er Jahre herauskristallisiert hatten. Im ersten
Teil ihres Buches formuliert Dawkins Grundgedanken sehr bildhaft aus:
Meme sind das Ergebnis von Imitation, \emph{Universeller Darwinismus}
greift wo immer Replikatoren auftreten, Kulturentwicklung zieht
biologische Evolution an einer \emph{langen Leine} hinter sich her, Meme
haben Ähnlichkeit zu Krankheiten und Computerviren, Meme haben --
zumindest statistisch -- Willen und Agency (\emph{the memes' eye view}).
Bevor sie diese Konzepte im Letzten Teil des Buches für
gesellschaftlichen Kommentar instrumentalisiert (\emph{Meme des New
Age}, \emph{Ins Internet}, \emph{Religionen als Memplexe}), wendet sie
sich im mittleren wesentlich Fragen der Menschheitsentwicklung aus
memetischer Perspektive zu. Dabei lässt sich ihr Erklärunsmodell auf
eine einfache Formel reduzieren: Sobald Meme existieren übernehmen sie
die Rolle des dominanten Replikators, der das Verhalten seiner Träger
wesentlich und langfristig beeinflusst. Dieses Interpretationsmuster
bringt sie so beispielsweise für die Entstehung von Sprache, als
Mechanismus sexueller Selektion oder als Begründung für
zwischenmenschlichem Altruismus zur Anwendung:

\begin{quote}
Als sich die Imitationsfähigkeit erst einmal entwickelt hatte und Meme
auftauchten, haben diese Meme die Umwelt verändert, in der die Gene
selektiert wurden und zwangen sie so, immmer bessere memverbreitende
Apparate zu schaffen. Mit anderen Worten ist die menschliche
Sprachfähigkeit memgetrieben, und die Funktion der Sprache besteht
darin, Meme zu verbreiten.

-- \textcite{blackmore_macht_2000}, 159.
\end{quote}

\begin{quote}
Der Memetik {[}\ldots{}{]} zufolge wird die Partnerwahl nicht nur vom
genetischen, sondern auch vom memetischen Vorteil beeinflusst. Eine
meiner Schlüsselannahmen ist, dass die natürliche Selektion nach
Entstehung der ersten Meme begann, Menschen zu favorisieren, die sich
für eine Paarung mit den besten Imitatoren oder den besten Benutzern und
Verbreitern von Memen entschieden.

-- \textcite{blackmore_macht_2000}, 213.
\end{quote}

\begin{quote}
Wenn Leute altruistisch sind, werden sie beliebt, weil sie beliebt sind,
werden sie kopiert, und weil sie kopiert werden, breiten sich ihre Meme
-- \emph{einschließlich der Altruismusmeme selbst} -- weiter aus als die
Meme weniger altruistischer Leute. Das liefert einen Mechanismus für die
Ausbreitung altruistischen Verhaltens.

-- \textcite{blackmore_macht_2000}, 252.
\end{quote}

Daniel Dennett verarbeitet Dawkins Memetik ausführlich in seinen
philosophischen Beiträgen zu Religion, Moral und der Natur des
menschlichen Denkens. Sein Engagement in der Diskussion um evolutionäre
Perspektiven auf die Kulturentwicklung ist ein Beleg dafür, dass Memetik
-- wie schon weiter oben angedeutet -- weniger als wissenschaftliche
Theorie denn als philosophische Strömung verstanden werden kann. In
\emph{Darwin's Dangerous Idea}\footnote{\textcite{dennett_darwins_1995}}
betrachtet er den Kontrast zwischen der von Darwin initiierten
Evolutionstheorie und Entstehungsmodellen, die übernatürliche
Mechanismen -- \emph{Skyhooks} -- inkorporieren. Evolutionstheorie
müsste trotz ihrer immanenten Überlegenheit als modernes,
wissenschaftliches Erklärungsmodell auch eine fundierte Begründung für
die scheinbare und effektive Zweckhaftigkeit und Formähnlichkeit
zufälliger, biologischer Entwicklung und menschlichen Kulturhandelns
formulieren. Dafür führt Dennett das Konzept des \emph{Design Space}
ein, der der natürlichen Evolution Grenzen aufzwingt und sie lenkt.
Evolution versteht Dennett als streng \emph{algorithmischen} Prozess von
Anpassung durch Selektion. Memetik dient ihm als philosophisches
Werkzeug, um diesen Mechanismus aus dem Natur- in den Kulturkontext zu
übertragen. Das erlaubt es, sogar über den Menschen hinaus --
hinsichtlich künstlicher Intelligenz -- ein und dasselbe
Erklärungsmodell für unterschiedliche Domänen zur Anwendung zu bringen.

\begin{quote}
Then a few billion years passed, while multicellular life forms explored
various nooks and crannies of Design Space until, one fine day, another
invasion began, in a single species of multicellular organism, a sort of
primate, which had developed a variety of structures and capacities
{[}\ldots{}{]} that just happend to be particularly well suited for
these invaders. It is not surprising that the invaders where well
adapted for finding homes in their new hosts, since they were themselves
created by their hosts {[}\ldots{}{]}. In a twinkling -- less than a
hundred thousand years -- these new invaders transformed the apes who
were their unwitting hosts into something altogether new: \emph{witting}
hosts, who, thanks to their huge stock of newfangled invaders, could
imagine the heretofore unimaginable, leaping through Design Space as
nothing had ever done before. Following Dawkins (1976), I call the
invaders \emph{memes} {[}\ldots{}{]}.

-- \textcite{dennett_darwins_1995}, 341.
\end{quote}

\begin{quote}
There is no denying that there is cultural evolution, in the
Darwin-neutral sense that cultures change over time, accumulating and
losing features, while also maintaining features from earlier ages. The
history of the idea of say, crucifixion, or of a dome on squinches, or
powered flight, is undeniably a history of transmission through various
nongenetic media of a family of variations on a central theme. But
whether such evolution is weakly or strongly analogous to, or parallel
to, genetic evolution, the process that Darwinian Theory explains so
well, is an open question. {[}\ldots{}{]} At one extreme, we may
imagine, it could turn out that cultural evolution recapitulates
\emph{all} the features of genetic evolution: not only are there gene
analogues (memes), but there are strict analogues of phenotypes,
genotypes, sexual reproduction, {[}\ldots{}{]}. At the other extreme,
cultural evolution could be discovered to operate according to entirely
different principles {[}\ldots{}{]}, so that there was no help at all to
be found amid the concepts of biology. {[}\ldots{}{]} In between the
extremes lie the likely and valuable prospects: that there is a large
(or largish) and important (or merely mildly interesting) transfer of
concepts from biology to the human sciences.

-- \textcite{dennett_darwins_1995}, 345-346.
\end{quote}

Das Journal of Memetics -- Evolutionary Models of Information
Transmission (JoM-EMIT) wurde als Online Zeitschrift mit Peer-Review
Prozess eingerichtet. Während seiner kurzen Existenz wurden in ihm nur
45 Artikel\footnote{\url{http://cfpm.org/jom-emit/all.html}
  {[}23.08.2018{]}} zu kulturevolutionärer Grundlagenforschung,
Wissenschaftstheorie, Philosophie sowie Modellierung und Empirie rund um
und mit der Terminologie der Memetik veröffentlicht. Die Mehrzahl der --
häufig sehr kurzen -- Beiträge beschäftigt sich mit Begriffsdefinition,
der experimentellen Anwendung der Memetik auf Fragestellungen in
diversen Fachbereichen sowie Vorschläge zur computerbasierten Simulation
von Memeexpansionsprozessen. Die Autoren versuchten Memetik als
anwendungsorientierte Wissenschaft zu definieren:

\begin{quote}
{[}\ldots{}{]} the application of models with an evolutionary or genetic
\emph{structure} to the \emph{domain} of (cultural) information
transmission.

-- \textcite{edmonds_modelling_1998}
\end{quote}

Obgleich das Ziel der Zeitschrift insofern erreicht wurde, als dass sie
als wichtige Kommunikationsplattform für die Diskussion rund um Memetik
wahrgenommen wurde, ist sie gleichermaßen die Dokumentation ihres
Scheiterns als eigenständige Wissenschaft: Mehrere Autoren stellen die
Aussichten der Memetik zuletzt explizit in Frage.

\hypertarget{memetics-critique}{%
\subsection{Kritik}\label{memetics-critique}}

Die Kritik an der Memetik ist vielfältig und setzt sowohl an ihren
Grundsätzen als auch Details an. Auf eine umfassende Darstellung muss
hier verzichtet werden\footnote{Viele Aspekte werden etwa von Tim Tyler
  aufgelistet (und vermeintlich entkräftet), dessen Webpräsenz
  \url{http://memetics.timtyler.org/criticisms/} {[}23.08.2018{]} und
  Monographie \emph{Memetics: memes and the science of cultural
  evolution} (\textcite{tyler_memetics_2011}) ein Beleg dafür sind,
  welchen Anklang Memetik an den Grenzen der Wissenschaftlichkeit und in
  pseudo- und alternativwissenschaftlichen Sphären gefunden hat}. Nur
drei Angriffspunkte sollen herausgegriffen werden, die von verschiedenen
Kritikern wiederholt wurden und zusammen entscheidend sind, den
Niedergang der Memetik zu erklären.

\begin{enumerate}
\def\labelenumi{\arabic{enumi}.}
\tightlist
\item
  \emph{Die Übertragung biologischer Terminologie und Erkentnisse auf
  Kulturprozesse ist grundsätzlich unzulässig oder zumindest in der
  Memetik zu stark vereinfacht.} Diese Fundamentalkritik geht weit über
  die Memetik hinaus -- trifft also die Grundidee der Cultural Evolution
  Theory -- hat sich aber besonders an der niederschwelligen und
  gleichzeitig in Absoluten argumentierenden Memetik entladen. Sie
  ähnelt der frühen Kritik am Evolutionismus durch die Boasianer (siehe
  Kapitel \ref{cultural-relativism-neoevolutionism}). Einer der
  sichtbarsten Vertreter dieser ausgesprochen antisoziobiologischen
  Perspektive war Stephen Jay Gould (*1941 - †2002). Memetik entwickelte
  sich also als kontroverse Nische in einem ohnehin heiß umkämpften
  Umfeld.
\end{enumerate}

\begin{quote}
I am convinced that comparisons between biological evolution and human
cultural or technological change have done vastly more harm than good --
and examples abound of this most common of intellectual traps
{[}\ldots{}{]}. Biological Evolution is powered by natural selection,
cultural evolution by a different set of principles that I understand
but dimly.

-- \textcite{gould_pandas_1991}, 63.
\end{quote}

\begin{enumerate}
\def\labelenumi{\arabic{enumi}.}
\setcounter{enumi}{1}
\tightlist
\item
  \emph{Memetik bietet keine ausreichenden Ansatzpunkte für
  systematische Falsifizierung oder quantitative Modellierung.}
  Besonders aus dem Kreis jener Autoren, die tatsächlich den Versuch
  unternommen haben Memetik für ihre Forschung zu applizieren, gingen
  mehrere hervor, die die Anwendbarkeit von memetischer
  Problemformulierung und Modellierung schließlich stark in Frage
  stellen mussten. Darunter Bruce Edmonds, Francisco Gil-White und
  Robert Aunger. Memetik gelang es nicht, prüf- und reproduzierbare,
  wissenschaftliche Ergebnisse zu erzielen. Dieses Scheitern ist
  selbsterklärt.
\end{enumerate}

\begin{quote}
The central core, the meme-gene analogy, has not been a wellspring of
models and studies which have provided ``explanatory leverage'' upon
observed phenomena. Rather, it has been a short-lived fad whose effect
has been to obscure more than it has been to enlighten. I am afraid that
memetics, as an identifiable discipline, will not be widely missed.

-- \textcite{edmonds_revealed_2005}
\end{quote}

\begin{enumerate}
\def\labelenumi{\arabic{enumi}.}
\setcounter{enumi}{2}
\tightlist
\item
  \emph{Die Terminologie der Memetik bringt keinen substantiellen
  Zugewinn im Kontext der Cultural Evolution Forschung.} Diese
  Erkenntnis brachte der Wissenschaft Memetik den Todesstoß. Cultural
  Evolution Theory entwickelte sich permanent weiter, die Bezugnahme auf
  den Meme Begriff war allerdings minimal -- er scheint schlicht nicht
  erforderlich zu sein und keinen nennenswerten Vorteil für die
  Beschreibung und Erforschung von Kulturprozessen zu bieten.
\end{enumerate}

\begin{quote}
Indeed, memetics -- at least for now -- doesn't seem to add anything to
the standard view of gene-culture co-evolution that was developed well
before Dawkins put down his ideas in The Selfish Gene. Ideas clearly do
evolve, and there is in fact a somewhat undeniable analogy between memes
and the evolution of genes. But we don't need to push that analogy too
far, and we certainly don't need a whole new vocabulary to make sense of
it.

-- \textcite{pigliucci_trouble_2007}
\end{quote}

Die trotzdem enorme Resonanz der Memetik in der Öffentlichkeit ist
jedoch geeignet, die Wissenschaftskommunikation der anthropologischen
Wissenschaften zu hinterfragen\footnote{\textcite{bloch_where_2005}}.

\hypertarget{themen-und-konflikte-der-cultural-evolution-forschung}{%
\section{Themen und Konflikte der Cultural Evolution
Forschung}\label{themen-und-konflikte-der-cultural-evolution-forschung}}

Cultural Evolution ist heute eine wichtige theoretische Strömung der
anthropologischen Forschung. Die oben unterschiedenen Perspektiven
Evolutionary Psychology, Human Behavioural Ecology und Dual Inheritance
Theory sind Grundlage für abstrakte Modelle, Fallstudien und
theoretische Weiterentwicklung. Besonders hervorgetan haben sich hier in
den vergangenen 30 Jahren neben Cavalli-Sforza, Feldmann, Richerson,
Boyd und Shennan auch Robert Chester Dunnell (*1942 - †2010), Michael
John O'Brien (*1950), Patrice A. Teltser (*1954), Ben Sandford Cullen
(*1964 - †1995) und eine Vielzahl jüngerer Kollegen wie R. Lee Lyman,
Joseph Henrich, Oren Kolodny, Ken Aoki, Alex Mesoudi oder Enrico Crema.
Seit 2015 konstituiert sich eine Cultural Evolution Society als
interdisziplinäre Wissenschaftsvereinigung\footnote{\url{https://culturalevolutionsociety.org}
  {[}01.02.2018{]}}.

\textcite{creanza_cultural_2017} geben einen guten Überblick über
aktuelle Fragestellungen der Cultural Evolution Forschung. Ausgehend von
dieser Themensammlung sollen einige wesentliche Zusammenhänge
nachvollzogen werden um die breite Aufstellung des Felds zu
veranschaulichen. Weitere wichtige Themenkomplexe, wie Kultur und
Kulturentwicklung in Nicht-menschlichen Spezies\footnote{Kulturverhalten
  und \emph{Social Learning} wurde in vielen Spezies beobachtet,
  darunter Vögel, Delphine, Wale, Primaten, Elefanten und Fische --
  \textcite{eerkens_cultural_2007}; \textcite{laland_question_2009}},
Evolution von Sprache in der Linguistik\footnote{\textcite{nowak_evolution_1999}}
oder die Entstehung von altruistischem Verhalten im Menschen\footnote{\textcite{boyd_origin_2005}
  -- Abschnitt drei ist mit sieben Beiträgen zwischenmenschlicher
  Kooperation und \emph{Reziprozität} gewidmet}, sollen hier aufgrund
ihrer geringen Relevanz im Kontext dieser Arbeit beiseite gelassen
werden. Ebenso die Vielzahl von Ansätzen, moderne gesellschaftlichen
Problemstellungen wie Klimawandel\footnote{\textcite{seneviratne_allowable_2016}},
Industrielle Landwirtschaft\footnote{\textcite{garibaldi_farming_2017}}
und Multiresistente Keime\footnote{\textcite{boni_evolution_2005}} aus
einer Cultural Evolution Perspektive zu analysieren. Stattdessen wird
dem Themenfeld \emph{Cultural Transmission} und seiner Bedeutung für
archäologische Modellbildung im einem anschließenden, eigenen Kapitel
viel Raum gegeben.

\hypertarget{menschliches-verhalten-genetische-determination-vs.kulturelles-lernen}{%
\subsection{Menschliches Verhalten: Genetische Determination
vs.~Kulturelles
Lernen}\label{menschliches-verhalten-genetische-determination-vs.kulturelles-lernen}}

Eine der Grundannahmen der Cultural Evolution Theorie ist die
Ähnlichkeit zwischen biologischer Evolution und kultureller Entwicklung.
Das schließt die Übertragung biologischer Konzepte wie Mutation,
Selektion, \emph{Flow} und \emph{Drift} explizit ein\footnote{\textcite{smith_cultural_1992}}.
Das Methodenset der Populationsgenetik kann damit auf Kulturprozesse
übertragen werden. Cavalli-Sforza und Feldmann\footnote{\textcite{cavalli-sforza_cultural_1981}},
Robert Boyd und Peter Richerson\footnote{\textcite{richerson_dual_1978};
  \textcite{boyd_culture_1985}} und andere\footnote{\textcite{campbell_variation_1965};
  \textcite{pulliam_programmed_1980}; \textcite{lumsden_genes_1981}}
legten dafür in den 1980ern konkrete Ausarbeitungen oft mathematisch
formulierter Modelle vor. Dennoch bestehen klare Unterschiede zwischen
biologischer Populationsgenetik und der Entwicklung und Transmission von
Ideen. Cultural Evolution folgt nicht den Mendelschen Regeln zu
\emph{Uniformität}, \emph{Spaltung} und \emph{Unabhängigkeit}\footnote{\textcite{mesoudi_pursuing_2017}}
und große Teile der Terminologie (z.B. \emph{Genotyp vs.~Phänotyp},
\emph{Homozygotie vs.~Heterozygotie}) sind nicht oder nur unter großen
Bedeutungsverschiebungen anwendbar. \emph{Horizontale} Transmission von
einem lebenden Organismus zum anderen spielt in der biologischen
Vererbung -- zumindest bei multizellularen Organismen\footnote{Im
  Gegensatz zu den Möglichkeiten sehr einfacher Lebenwesen:
  \textcite{woese_new_2004}; \textcite{woese_interpreting_2000}} -- eine
untergeordnete Rolle und die Übertragung erfordert große Anpassungen an
den vor allem \emph{vertikalen}, genetischen Ausgangsmodellen\footnote{\textcite{cavalli-sforza_cultural_1973};
  \textcite{feldman_cultural_1976}}.

Im Gegensatz zur DNS der Genetik, ist die Identität der
Informationsträger kultureller Entwicklung unbekannt. Im Kontext der
Memetik wurde die Frage nach der physischen Existenz von Memen intensiv
diskutiert\footnote{\textcite{delius_nature_1991};
  \textcite{wilkinson_memes_1999}; \textcite{blackmore_macht_2000}
  105-108.} -- und diese damit der Cultural Evolution Community neu
präsent. Zwar sind die neurologischen Strukturen zur Speicherung
einzelner Assoziationen unbekannt, gemessen werden kann Cultural
Transmission dennoch. \textcite{pocklington_cultural_1997} formulieren
auf Grundlage von Dawkins:

\begin{quote}
The appropriate units of selection will be \emph{the largest units of
socially transmitted information that reliably and repeatedly withstand
transmission}.

-- \textcite{pocklington_cultural_1997}, 81.
\end{quote}

Zwar können Ideen wie Gene Gruppen bilden, ihre Übertragung ist
allerdings viel volatiler -- nicht an die vertikale Transmission bei der
biologischen Reproduktion gebunden. Die Modifikation von Ideen ist nicht
nur auf zufällige Mutation beschränkt, sondern kann durchaus durch
bewusste Innovation, Kombination oder Manipulation ausgelöst
werden\footnote{\textcite{eerkens_cultural_2007}}. Hinsichtlich ihrer
Wirkung besteht eine unbestreitbare Parallel zwischen Genen und
Kulturverhalten: Sie erzeugen Phänotypen mit distinkter aber
vergleichbarer Merkmalsausprägung\footnote{\textcite{lyman_culture_2001};
  \textcite{lyman_rise_1997}}. Bestimmte kulturelle Eigenschaften lassen
sich binär oder diskret kategorisieren, andere eher quantitativ bzw.
proportional beschreiben. Zu ersteren gehören beispielsweise das
technologische Wissen um Herstellung und Verwendung eines bestimmten
Werkzeugs oder die Verwendung eines bestimmten Ritzmusters zur
Keramikverzierung. Damit sind also die herkömmlichen Einheiten der
archäologischen Typen- und Formengliederung direkt in das Cultural
Transmission Framework einpassbar\footnote{\textcite{lipo_science_2001};
  \textcite{lyman_cultural_2003}}. Auch die in der vorliegenden Arbeit
vorgenommene Untersuchung von Bestattungssitten reduziert diese auf die
binäre Komponente der Ab- und Anwesenheit eines bestimmten Aspekts des
Rituals. Analysen auf metrischem Skalenniveau wurden etwa zur Abbildung
von Risikobereitschaft\footnote{\textcite{bisin_economics_2001-1}} in
Gruppen oder dem Kompetenzniveau\footnote{\textcite{baldini_revisiting_2015};
  \textcite{henrich_demography_2004};
  \textcite{kobayashi_innovativeness_2012}} im Umgang mit einem
bestimmten Werkzeug zur Anwendung gebracht.

Der von \autocite{smith_three_2000} (siehe Kapitel
\ref{evolutionism-modern-theories}) beobachtete Riss durch die
Forschungslandschaft zwischen Evolutionary Psychology, Human behavioral
Ecology und Dual Inheritance Theory wird besonders an der Frage
deutlich, welche Aspekte menschlichen Verhaltens genetisch determiniert
und welche kulturell konstruiert sind. Unter der Annahme, dass die
Transmission von Ideen Menschen eine viel höhere Anpassungsfähigkeit an
widrige Subsistenzumstände ermöglicht, zeigen entsprechend konzipierte
Modelle, dass genetisch transportiertes Verhalten nur in ökologisch sehr
stabilen Umgebungen Relevanz entwickeln kann\footnote{\textcite{aoki_emergence_2005};
  \textcite{aoki_evolution_2014}; \textcite{boyd_cultural_1983}}. Aus
dieser Perspektive ergibt sich das klare Primat kultureller Transmission
für den Menschen, der sich dank seiner Kulturfähigkeit in fast alle auf
der Erde vertretenen Biome hat ausbreiten können.

Auch bei einer Dominanz sozialen Lernens und kultureller Transmission
für die Prägung menschlichen Verhaltens ist der genetische Anteil nicht
zu vernachlässigen -- schon allein aufgrund der häufig zu beobachtenden
Korrelation zwischen einem Verhaltensmuster und biologischer
Verwandtschaft. Diese Übereinstimmung ergibt sich aus vertikalen
Transmissionsstrukturen, die biologisch und kulturell oft parallel
verlaufen. Genauso muss die natürliche Umwelt als wesentlicher Faktor
bei der Determination menschlichen Verhaltens in Betracht gezogen
werden. Die genaue Charakterisierung des Einflusses von Genen, Kultur
und Umwelt ist unter den Stichworten \emph{Gene-Culture coevolution},
Dual Inheritance Theory und \emph{Cultural Niche Construction} intensiv
diskutiert worden\footnote{\textcite{aoki_gene-culture_2017};
  \textcite{boyd_culture_1985}; \textcite{cavalli-sforza_cultural_1981};
  \textcite{chudek_culturegene_2011}; \textcite{feldman_aspects_1979};
  \textcite{mesoudi_towards_2006}; \textcite{richerson_dual_1978}}.

Die Methode der \emph{Genomweiten Assoziationsstudie} (\emph{GWAS},
\emph{Genome-wide association study}) erlaubt es heute, Menschen und ihr
Verhalten mit zunehmender Präzision auf Korrelation mit der Anwesenheit
bestimmter Genen zu untersuchen. Dadurch wird die Suche nach genetischer
Anpassung etwa an die naturräumliche Rahmensituation
erleichtert\footnote{\textcite{berg_population_2014}}. Die Untersuchung
von Merkmalen wie dem IQ oder dem erreichten Ausbildungsniveau\footnote{\textcite{benyamin_childhood_2014};
  \textcite{davies_genome-wide_2011}; \textcite{minkov_genetic_2015};
  \textcite{okbay_genome-wide_2016}} ist jedoch mit ethischen und
wissenschaftlichen Risiken verbunden. Einerseits eröffnen die
Erkenntnisse über solche Zusammenhänge moralische Implikationen,
andererseits ist eine statistische Ergebnissicherheit nicht
gewährleistet: Korrelation von Genen und Verhalten muss nicht Konsequenz
einer kausalen Beziehung sein. Stattdessen könnte sie nur Nebeneffekt
von z.B. räumlicher und sozialer \emph{Autokorrelation} oder
\emph{assortativer Paarung} sein\footnote{\textcite{abdellaoui_educational_2015};
  \textcite{domingue_genetic_2014}; \textcite{okbay_genome-wide_2016};
  \textcite{piffer_review_2015}}. Moderne Fallbeispiele, für die
komplexe, sozioökonomische Erklärungen angenommen werden müssen obgleich
auch genetische Korrelation besteht, beschäftigen sich unter anderem mit
Tabakkonsum, Armut, Gesundheit oder Rassismus\footnote{\textcite{maes_genetic_2006};
  \textcite{marden_african_2016};
  \textcite{nugent_geneenvironment_2011};
  \textcite{paradies_racism_2015}}.

\hypertarget{mensch-umwelt-interaktion-cultural-niche-construction-und-pathogene}{%
\subsection{Mensch-Umwelt Interaktion: Cultural Niche Construction und
Pathogene}\label{mensch-umwelt-interaktion-cultural-niche-construction-und-pathogene}}

Cultural Niche Construction hält ein potentes Erklärungsmodell bereit,
um den wechselseitigen Selektionsdruck nachzuvollziehen, den Kultur,
Gene und Umwelt aufeinander ausüben\footnote{\textcite{laland_niche_2000};
  \textcite{odling-smee_niche_2003}; \textcite{laland_cultural_2011};
  \textcite{rendell_runaway_2011}}. Dabei beschreibt \emph{Niche
Construction} in der Biologie Veränderungen der natürlichen Umwelt, die
einerseits von einer Spezies selbst hervorgerufen werden und
gleichermaßen die Selektionsdrücke auf diese Spezies
beeinflussen\footnote{\textcite{laland_niche_2006}}. Für den Menschen
ergibt sich daraus ein komplexes Geflecht von Interdependenzen zwischen
Kulturverhalten, genetischer Disposition und Natur, die die schrittweise
Modifikation all dieser Systembestandteile zur Folge hat\footnote{\textcite{alberti_global_2017};
  \textcite{arbilly_arms_2014}; \textcite{creanza_models_2012};
  \textcite{laland_cultural_2001}}. Ähnlichkeiten der natürlichen Umwelt
ansonsten völlig unabhängiger Populationen können Auslöser für
\emph{Konvergenz} sein, also Übereinstimmungen im Kulturverhalten, die
nicht durch \emph{Cultural Transmission} erklärt werden können\footnote{\textcite{eerkens_cultural_2007}}.

Subsistenzbezogenes Verhalten ist unmittelbar selektionsrelevant, da es
die Sterbe- und Reproduktionswahrscheinlichkeit einer Population
beeinflusst. Der Mensch hat seine Versorgung über den größten Teil
seiner Existenz aus Jagen und Sammeln bestritten. Dabei war er von den
Ressourcen einer natürlichen Umwelt abhängig und hat sie durch
Güterentnahme destabilisiert. Etliche Modelle im Kontext der Human
behavioral Ecology dokumentieren, wie diese Wechselwirkung zum
Katalysator von Veränderung im Mensch-Umwelt System wurde\footnote{\textcite{hardy_climatic_2010};
  \textcite{hockett_nutritional_2005}; \textcite{stiner_thirty_2001}}.
Viel beachtete Fallbeispiele dieser Interaktion sind unter anderem
anthropogen induzierte Aussterbeereignisse von Megafauna\footnote{\textcite{barnosky_assessing_2004}},
Feuernutzung für Landschaftseingriffe\footnote{\textcite{bird_fire_2008}},
die Ausbreitung der Links- und Rechtshändigkeit\footnote{\textcite{laland_gene-culture_1995}},
die Entstehung der Laktose-Toleranz\footnote{\textcite{feldman_Theory_1989};
  \textcite{ingram_population_2012}} und die rückläufige, demographische
Entwicklung in modernen, westlichen Gesellschaften\footnote{\textcite{borgerhoff_mulder_demographic_1998};
  \textcite{fogarty_role_2013}; \textcite{ihara_cultural_2004}}. Auch
die Neolithisierung könnte durch einen solchen Prozess verstanden
werden\footnote{\textcite{rowley-conwy_foraging_2011};
  \textcite{smith_onset_2013}}.

Krankheiten sind ein wesentlicher Selektionsfaktor für den Menschen und
hatten großen Einfluss sowohl auf seine biologische\footnote{\textcite{bustamante_natural_2005};
  \textcite{enard_viruses_2016}; \textcite{mead_balancing_2003};
  \textcite{sabeti_genome-wide_2007}@} als auch auf seine
prähistorisch-kulturelle\footnote{\textcite{martin_health_2002};
  \textcite{oxenham_skeletal_2005}} und historische\footnote{\textcite{alfani_plague_2013};
  \textcite{murray_estimation_2006}} Entwicklung. Malaria hat
beispielsweise wesentliche Veränderungen im menschlichen Erbgut
durchgesetzt\footnote{\textcite{kwiatkowski_how_2005};
  \textcite{tishkoff_haplotype_2001}} -- unter anderem die weitreichende
Verbreitung der Sichelzellenanämie\footnote{\textcite{allison_protection_1954}}.
Die Interaktion des Menschen mit Krankheiten lässt sich nicht auf eine
rein biologische Perspektive reduzieren. Stattdessen sind Krankheiten
und ihre Verbreitung stark durch Kulturverhalten bedingt. Nassfeldanbau
in Westafrika könnte die initiale Verbreitung von Malaria massiv
begünstigt haben\footnote{\textcite{durham_coevolution_1991-1}},
Krankheiten waren ein wesentlicher Bestandteil des Kulturpakets, mit dem
sich die Nordamerikanischen Ureinwohner in Folge von Kolumbus Landung
1492 konfrontiert sahen\footnote{\textcite{nunn_columbian_2010}} und die
Kuru Krankheit, die bis in die 1940er im Hochland von Neuguinea immer
wieder in Epidemien ausbrach, war in ihrer Übertragung abhängig von
kannibalistischen Ritualen\footnote{\textcite{lindenbaum_kuru_2015}}.

Neben Pathogenen ist der Mensch auch Wirt für weniger parasitäre
Mikroorganismen. Die Gesamtheit von Lebensformen, die in und auf dem
menschlichen Körper leben ohne Krankheiten oder Entzündungen
hervorzurufen -- die Normalflora -- hat durchaus Rückwirkung nicht nur
auf den menschlichen Organismus, sondern auch auf dessen Verhalten und
Verhaltensspielraum. Menschen können die Fähigkeit zur
Laktoseverarbeitung beispielsweise nicht nur über eine Mutation des
eigenen Erbguts erlangen, sondern auch indirekt über Bakterien im
Verdauungstrakt. Solche Bakterien haben möglicherweise ebenfalls eine
wichtige Rolle bei der Entstehung der Milchwirtschaft in der
Vorgeschichte gespielt\footnote{\textcite{walter_human_2011}}.

\hypertarget{entstehung-und-wirkung-von-innovationen-cultural-complexity}{%
\subsection{Entstehung und Wirkung von Innovationen: Cultural
Complexity}\label{entstehung-und-wirkung-von-innovationen-cultural-complexity}}

In der biologischen Evolution entstehen neue Varianten durch Mutationen
im Erbgut von Individuen. Cultural Evolution kennt dagegen eine ganze
Reihe von Prozessen, die zur Entstehung von Innovationen verschiedener
Größenordnungen führen können. Viele Modelle reduzieren diese Prozesse
auf simple Zufallsereignisse oder die Interaktion eines Individuums mit
seiner Umwelt\footnote{\textcite{henrich_evolution_2003};
  \textcite{rendell_why_2010}}. Andere bringen komplexere Mechanismen
ins Spiel, wie die Verknüpfung bestehender Innovationen zu
neuen\footnote{\textcite{enquist_why_2008}} und die Interaktion vieler
Innovationen in einer schnellen, aufeinander aufbauenden Kettenreaktion
von Kombination und Ableitung\footnote{\textcite{fogarty_cultural_2015};
  \textcite{kolodny_evolution_2015};
  \textcite{kolodny_game-changing_2016}}: Eine einzige Idee zieht
möglicherweise viele andere nach sich. Der akkumulative Ablauf von
Kulturentwicklung gehört zum Kern der Cultural Evolution
Theory\footnote{\textcite{basalla_evolution_1988};
  \textcite{boyd_culture_1985}; \textcite{boyd_evolutionary_1988};
  \textcite{cavalli-sforza_cultural_1981};
  \textcite{durham_adaptive_1976}; \textcite{feldman_gene-culture_1996};
  \textcite{henrich_evolution_2003}; \textcite{lumsden_genes_1981}}. In
der prähistorischen und historischen Menschheitsentwicklung gibt es
viele Ereignisse, die diesen Effekt nahelegen, etwa die explosionsartige
Zunahme an Komplexität im Steingerätinventar am Übergang von Mittel- zu
Jungpaläolithikum\footnote{\textcite{bar-yosef_nature_1998};
  \textcite{roebroeks_time_2008}} oder die neolithische Revolution im
Vorderen Orient\footnote{\textcite{gopher_when_2001};
  \textcite{veen_agricultural_2010}}.

Die Veränderung der Menge und Art kultureller Eigenschaften einer
Population ist mit dem Begriff der \emph{Cultural Complexity} Forschung
verknüpft. Sie untersucht die Akkumulation und den Verlust von
Innovationen (\emph{Cultural accumulation} und \emph{Cultural decay})
sowie \emph{Gleichgewichtszustände} (\emph{equilibria}) die in diesem
Wechselspiel erreicht werden können. Dabei zeigt sich, dass die
Innovationsverfügbarkeit in einer Population durch die Verschränkung der
verschiedenen Ideen starken Schwankungen unterworfen ist, bis sie einen
stabilen Zustand erreicht\autocite{kolodny_evolution_2015}. Innovationen
können selbst Rückwirkungen auf die Systemdynamik ihrer Wirtpopulationen
nehmen, indem sie zum Beispiel die Subsistenzbedingungen verändern und
Bevölkerungswachstum oder -niedergang katalysieren\footnote{\textcite{kolodny_game-changing_2016}}.
\textcite{crema_revealing_2016} eröffnen mit einer Fallstudie an
neolithischer Keramik jedoch auch die Perspektive dafür, dass die
Annahme von Gleichgewichtszustände in archäologischen Kontexten
grundsätzlich fragwürdig ist.

Das Ansammeln von Wissen und Kompetenz in einer Gruppe ist ein
kumulativer Prozess, wobei ein vorhandener Innovationsfundus maßgeblich
durch Rekombination und Ausbau vorhandener Ideen und erweitert wird --
\emph{Cumulative Cultural Evolution}. Weit über die archäologische
Forschung hinaus relevant ist die Frage, welche Variablen die Intensität
dieses Prozesses in welchem Umfang beeinflussen. Ein wichtiger Beitrag
darin von Joseph Henrich\footnote{\textcite{henrich_demography_2004}}
hat 2004 eine sehr dynamische und kontroverse Debatte darüber los
getreten. Sein Modell identifiziert die \emph{effektive
Populationsgröße} als entscheidenden Parameter um die Innovationsrate in
einer Population nachherzusagen: Eine Zunahme der Personen im sozialen
Netzwerk steigert die Menge neuer Erfindungen, während ein
Bevölkerungsrückgang zu Wissensverlust führt. Henrichs simple Simulation
wurde schrittweise erweitert\footnote{\textcite{KobayashiInnovativenesspopulationsize2012};
  \textcite{henrich_understanding_2016}}, relativiert\footnote{\textcite{collard_population_2013},
  \textcite{BaldiniRevisitingEffectPopulation2015}} und
kritisiert\footnote{\textcite{vaesen_population_2016}}. Andere
Parameter, die neben der Populationsgröße als wichtige Einflussfaktoren
vorgeschlagen wurden, sind die Mobilität der Mitglieder einer Gruppe
oder das Subsistenzrisiko durch naturräumliche Einflüsse\footnote{\textcite{collard_what_2011};
  \textcite{collard_risk_2013}; \textcite{buchanan_drivers_2016};
  \textcite{fitzhugh_risk_2001};
  \textcite{winterhalder_risk-senstive_1999}}.

Unabhängig davon welcher Effekt letztlich die größere Wirkung auf
kulturelle Komplexität entfaltet, gibt es eindeutig einen Zusammenhang
zwischen Kulturverhalten und demographischer Entwicklung einer
Population: Der Übergang von einer Jäger- und Sammlerischen Lebensweise
zu Ackerbau und Viehzucht am Beginn des Holozän geht mit einem starkem
Bevölkerungswachstum einher, das unter dem Stichwort der \emph{Neolithic
Demographic Transition} als eines der folgenreichsten Auswirkungen der
Neolithisierung diskutiert wird\footnote{\textcite{bocquetappel_paleoanthropological_2002};
  \textcite{gage_what_2009}}. Neben Subsistenzpraktiken beeinflussen
eine Vielzahl von Faktoren wie religiöse Normen, Heiratsgepflogenheiten
oder gewaltsame Konflikte die Altersstruktur und das Wachstum einer
Gesellschaft. Etliche davon reduzieren die Geburtenrate\footnote{\textcite{smith_cultural_1992};
  \textcite{colleran_cultural_2016}; \textcite{richerson_natural_1984}}
und wirken so stabilisierend auf das Mensch-Umwelt-System. Ein Phänomen
dieser Art lässt sich im modernen China und in Teilen Indiens
beobachten: Eine kulturelle Präferenz für männliche Nachkommen, die sich
etwa durch selektive Abtreibung manifestiert, führt lokal zu einem
asymmetrischen Überschuss von bis zu 6:5 von Männern gegenüber Frauen.
Diese kulturell induzierte demographische Veränderung hat
erwartungsgemäß schwerwiegende ökonomische Konsequenzen\footnote{\textcite{banister_shortage_2004};
  \textcite{li_cultural_2000}; \textcite{tuljapurkar_high_1995}}.

\hypertarget{cultural-transmission}{%
\section{Cultural transmission}\label{cultural-transmission}}

Die Ursprünge der Cultural Transmission Theory sind im klassischen
Diffusionismus (siehe Kapitel \ref{cultural-relativism-neoevolutionism})
und seinen etwas reiferen Ausprägungen nach der Fundamentalkritik der
Boasianer zu suchen -- etwa bei Alfred Louis Kroeber (*1876 - †1960)
oder Wilhelm Koppers (*1886 - †1961). Die Ausbreitung und Entwicklung
von Ideen lässt sich getrennt voneinander erforschen, die Integration
von Cultural Transmission in Cultural Evolution und Dual Inheritance
Theory ermöglicht jedoch eine sinnvolle Verbindung der
Perspektiven\footnote{\textcite{eerkens_cultural_2007};
  \textcite{cavalli-sforza_cultural_1981}, 53-54.}: Ideen breiten sich
nicht zufällig aus, verändern sich nach erforschbaren Regeln und
entfalten weitreichende Wirkung im sozialen Raum ihrer Träger.
Kulturelle Evolution ist in hohem Maße Konsequenz von (fehlerhafter)
Replikation in den sozialen Netzwerken, die Cultural Transmission
erforscht. Zwischenmenschliche Kommunikation, entlang derer die
Ausbreitung von Ideen abläuft, ist vielfältig und entwickeln auf
unterschiedlichen Skalenniveaus unterschiedliche Relevanz. Grundsätzlich
bewegt sich Information mit ihren Trägern, das heißt alle Prozesse, die
zur Bewegung von Menschen im Raum führen, sind auch Prozesse, die zur
Ausbreitung von Information führen. Zu Cultural Transmission müssen also
alle Modi der Migration von der Völkerwanderung, über den Frauentausch
in Heiratsnetzwerken bis hin zum alleine wandernden Händler und
Handwerker gezählt werden. Daneben stehen Prozesse innerhalb kohärenter
Gruppen, wie die Kindererziehung, Lehre und Ausbildung von einer
Generation zur nächsten und der einfache Austausch von Information
zwischen allen Mitgliedern einer Population wie er durch Sprache,
Schrift und Imitation permanent stattfindet.

Einige der weitreichendsten Transformationsereignisse in der Geschichte
der Menschheit, die zu einem tiefgreifenden Wandel der vorhandenen
kulturellen Eigenschaften geführt haben, sind von Populationsbewegungen
zumindest begleitet, wenn nicht sogar initiiert worden\footnote{\textcite{boyd_voting_2009}}.
Der Neanderthaler wurde vor ca. 40.000 Jahren vollständig vom Modernen
Menschen verdrängt\footnote{\textcite{skoglund_origins_2012}}, und mit
ihm ging eine erste -- freilich in ihrer Dynamik umstrittene -- Phase
kultureller Modernität zu Ende, die sich erst durch jüngste
Forschungsergebnisse zu erschließen beginnt\footnote{\textcite{hoffmann_symbolic_2018};
  \textcite{tuniz_did_2012}}. Paläogenetische Ergebnisse legen nahe,
dass die neolithische Revolution in Europa im wesentlichen von
wandernden Siedlern aus dem Vorderen Orient getragen wurde, nicht von
der Übernahme eines Innovationspakets durch lokale Jäger- und
Sammlergruppen\footnote{\textcite{aoki_travelling_1996};
  \textcite{bar-yosef_nature_1998}; \textcite{patterson_modelling_2010};
  \textcite{skoglund_origins_2012}}. Im fortgeschrittenen Neolithikum
bis zum Beginn der Bronzezeit vollzog sich eine weitere genetische und
kulturelle Transformation in Mitteleuropa infolge der Einwanderung
berittener Steppenbewohner aus dem Yamnaya Kulturkomplex\footnote{\textcite{allentoft_population_2015};
  \textcite{goldberg_ancient_2017}}.

Populationsbewegungen dürfen nicht unterschätzt werden, sind aber
gleichermaßen nicht für jede Form kulturellen Wandels verantwortlich.
Eine ganzheitliche Perspektive muss in erster Linie versuchen die
Prozesse innerhalb menschlicher Gesellschaften nachzuvollziehen --
einige Strukturen und Phänomene werden im folgenden unter den
Stichworten \emph{Social Learning} und \emph{Biased Transmission}
vorgestellt. Für die vorliegende Arbeit ist es von besonderem Interesse
nachzuvollziehen, wie Cultural Transmission als Element der Cultural
Evolution Theory in der archäologischen Forschung reflektiert und
praktisch zur Anwendung gebracht wurde. Dazu sollen auf Grundlage der
umfassenden Überblicksartikel von \textcite{eerkens_cultural_2007} und
\textcite{garvey_current_2018-1} einige wesentliche Leitlinien und
Beiträge besonders der vergangenen 30 Jahre nachgezeichnet werden.

\hypertarget{social-learning}{%
\subsection{Trajektorien der Wissensvermittlung: Social
Learning}\label{social-learning}}

Menschen besitzen die ausgeprägteste soziale Lernfähigkeit unter allen
bekannten Spezies. Aus anthropozentrischer Perspektive betont das die
menschliche Besonderheit, jenseits davon erweckt es aber durchaus
Zweifel an der Qualität dieses Merkmals:

\begin{quote}
What is so \emph{wrong} with culture that it should be really
conspicuous in only one species?

-- \textcite{smith_cultural_1992}, 70.
\end{quote}

Möchte man Fragen nach Entstehung und Ablauf von Cultural Transmission
nicht mit einem Hinweis auf evolutionäre Zufälle abtun, muss man
einerseits die Natur des Selektionsdrucks untersuchen, der die enorme
Intensivierung von Imitation ursprünglich begünstigt hat, und
andererseits Prozesse der Wissens- und Ideenvermittlung beobachten,
kategorisieren und quantifizieren. \emph{Social Learning} ist der
Überbegriff für alle Mechanismen der Übertragung von Ideen und Verhalten
von einem Organismus auf den nächsten\footnote{\textcite{eerkens_cultural_2007};
  \textcite{rendell_cognitive_2011}}. Prominente Methoden zur
Erforschung von Social Learning sind soziale Experimente mit Menschen
unter konstruierten Bedingungen, mathematische Modelle auf
Populationsniveau und -- gegebenenfalls agentenbasierte --
Computermodelle.

Soziales Lernen steht neben genetischer Vererbung und individuellem
Lernen. Während individuelles Lernen große Flexibilität mit sich bringt,
dafür aber auf das Individuum begrenzt ist, wirkt genetische Vererbung
nur auf dem Populationsniveau und damit gemessen an der Lebensspanne des
Einzelnen sehr langsam. Soziales Lernen steht zwischen diesen Polen und
erlaubt sowohl kurzfristige und kleinräumige, als auch langfristige,
kumulative und populationsweite Anpassung. Während individuelles Lernen
und Experimentieren viel Zeit und Energie in Anspruch nehmen kann, kann
soziales Lernen Wissen über einen Sachverhalt unmittelbar und risikoarm
transportieren\footnote{\textcite{rendell_rogers_2010}}. Gefährliche
Fehler beim individuellen Lernen, die durch die für den Einzelnen
geringe Anzahl von Experimentdurchläufen häufig sind, können durch
soziales Lernen vermieden werden\footnote{\textcite{boyd_evolution_1988}}.
Es ist dafür allerdings anfällig für schnelle und schnell
aufeinanderfolgende Veränderungen der natürlichen Umweltbedingungen, da
gegebenenfalls ein unangepasstes Verhalten traditionell weitergeführt
wird\footnote{\textcite{rogers_does_1988}}. Vergleicht man eine
Kombination von genetischer Anpassung und individuellen Lernen
einerseits mit einer Kombination von sozialem und individuellem Lernen
andererseits, dann führen erstere nur dann zu besserer Anpassung, wenn
die Umgebung nahezu unverändert bleibt oder sich enorm schnell und
völlig zufällig verändert. In den Fällen zwischen diesen Extrema ist
soziales Lernen überlegen\footnote{\textcite{boyd_culture_1985},
  117-128.}:

\begin{quote}
A cultural system of Inheritance combining individual and Social
Learning ought to provide adaptive advantages in environments with an
intermediate degree of environmental similarity from generation to
generation. This is the regime where the faster tracking due to the
evolutionary force of cumulative, relatively weak, low-cost individual
learning pays off most. Most individuals can depend primarily on
tradition, yet the modest pressure of individual learning is sufficient
to keep culture ``honest''.

-- \textcite{smith_cultural_1992}, 73.
\end{quote}

Diese Hypothesen sind außerhalb der künstlichen Modellumgebungen aus
denen sie abgeleitet wurden schlecht überprüfbar. Fallstudien mit
bedingt sozial lernfähigen Tieren wie Ratten könnten zur Prüfung der
Hauptaussagen geeignet sein. Für die menschliche Entwicklung müssen
entsprechende empirische Belege im archäologischen Befund ausgemacht
werden. Geht man von einer Korrelation von Gehirngröße und sozialer
Lernfähigkeit aus, dann könnten zum Beispiel anthropologische Daten aus
dem klimatisch variablen Pleistozän als starkes Indiz
auftreten\footnote{\textcite{smith_cultural_1992}}.

Zur Charakterisierung der zwischenmenschlichen Informationsübertragung
grenzen \textcite{cavalli-sforza_cultural_1981} in Anlehnung an Begriffe
aus der Epidemiologie drei Formen des Sozialen Lernens voneinander ab:
\emph{Vertical Transmission}, \emph{Horizontal Transmission} und
\emph{Oblique Transmission}\footnote{\textcite{cavalli-sforza_cultural_1981},
  53-59}.

\emph{Vertical Transmission} meint die Übertragung von Wissen, Ideen,
Verhalten und kultureller Eigenschaften von Eltern zu Kind. Diese
Übertragungsform spielte wahrscheinlich in der Menschheitsgeschichte die
mit Abstand größte Rolle, bedenkt man, dass die wildbeuterische
Lebensweise in kleinen Gruppen für einen überragend langen Zeitraum die
einzige relevante Form des menschlichen Zusammenlebens darstellte.
Geschlechtsspezifische Arbeitsverteilung in einer Gesellschaft kann dazu
führen, dass die Informationsübertragung \emph{uniparental} abläuft, in
vielen Fällen spielt jedoch zwangsläufig ein Einfluss von beiden (oder
mehreren) Elternteilen eine Rolle darin, wie ein Verhaltensmuster
tradiert wird. Obgleich die Elternrolle meist mit biologischer
Elternschaft einhergeht, ist vertikale Ideenvererbung nicht an sie
gebunden: Andere denkbare Beziehungen sind Stief- oder
Adoptivelternschaft, wenn es zu einer dauerhaften Verlagerung der
Erziehungsrolle kommt. Vertikale Beziehungen sind nicht nur aufgrund von
biologischer Äquivalenz und Erziehung relevant: Eltern vererben oft auch
ihren sozialen Rang, ihr Vermögen, Privilegien und
Abhängigkeiten\footnote{\textcite{mulder_intergenerational_2009}}.
Dieser Umstand erhöht die Bedeutung dieser Übertragungslinie in
menschlichen Populationen noch weiter. Geht man von einem klassischen
Modell der \emph{Life History Theory} aus, das Populationsentwicklung
auf Grundlage von sich reproduzierender Altersklassen
beschreibt\footnote{\textcite{leslie_further_1948}} und erweitert es um
kulturelle Merkmale und Transmission, dann ergeben sich bemerkenswerte
Simulationsergebnisse\footnote{\textcite{coratenuto_age_1989};
  \textcite{fogarty_role_2013}}: Sogar Verhaltensmuster, die die
Reproduktionsfähigkeit eines Individuums reduzieren, können bei
ausreichend starker, vertikaler Übertragungsfähigkeit dauerhaft relevant
bleiben. Das gilt besonders dann, wenn eine Idee zwar die
Reproduktionsfähigkeit reduziert, gleichzeitig aber die Überlebenschance
des Individuums erhöht.

Cavalli-Sforza und Feldman trennen zwischen \emph{Horizontal
Transmission} und \emph{Oblique Transmission}. Ersteres bezieht sich auf
Übertragung innerhalb einer Generation, während letzteres jene
Generationsgrenzen überschreitenden Beziehungen bezeichnet, die unter
die Elternschaft fallen. Mit diesen beiden Begriffen lassen sich viele
unterschiedliche Formen der zwischenmenschlichen Beziehung bezeichnen.
Potentielle Gegenüberkategorien sind Familien- oder nicht biologisch
verwandte Gruppenmitglieder der Elterngeneration (Tanten und Onkel
gegenüber Neffen und Nichten, Freunde der Kernfamilie, Nachbarn),
Großeltern (und Enkel) und Mitglieder der Großelterngeneration(en),
Geschwister, Cousins und Cousinen, Mitglieder der selben Altersgruppe
bzw. Generation (Freunde, Nachbarn, Kollegen, romantische Partner),
Lehrer, Politische Führer. Jede dieser Kategorien bringt eigene
Besonderheiten mit sich: Enge Familienmitglieder können mitunter das
selbe Niveau der Einflussnahme wie die biologischen Eltern erreichen,
Beziehungen zwischen Angehörigen der selben Altersgruppe fallen je nach
Rang- und Persönlichkeitskonfiguration höchst unterschiedlich aus,
Lehrer und soziopolitische Führer erreichen mit ihren Ideen ein größeres
Publikum\footnote{\textcite{fogarty_evolution_2011}} oder können sogar
die Akzeptanz ihrer Botschaft durch Druck oder Manipulation erzwingen.
Das Beziehungsgeflecht einer menschlichen Population wird darüber hinaus
weiter kompliziert durch Gruppengliederung: Menschen formen räumlich
oder sozial zusammengehörige Einheiten, wobei sich Gruppengrenzen
vielfach überschneiden können. Horizontaler und Schräger Austausch von
Ideen ist günstig um einem Individuum möglichst viel Auswahl an
Strategien zur Verfügung zu stellen, aus denen es zur Lösung von
Problemen wählen kann. Umso stärker diese nicht-veertikalen Formen der
Cultural Transmission in einer Gesellschaft ausgeprägt sind, desto
größer ist die Diversität, die schon innerhalb eines Haushalts
angetroffen werden kann\footnote{\textcite{shennan_genes_2002}} und
desto mehr verschiebt sich der Selektionsdruck zugunsten von sozialen
Führungsrollen wie die von Lehrern, Priestern oder Großeltern\footnote{\textcite{macdonald_subsistence_1998}}.
Elternschaft kann demgegenüber ins Hintertreffen geraten. Ein solches
Verhaltensmuster ist für genetische Selektion gegebenfalls ungünstig, da
Kinder unter diesen Umständen nicht die biologische Reproduktion,
sondern andere Lebensmodelle anstreben können\footnote{\textcite{smith_cultural_1992}}.

Cultural Transmission mittels Social Learning läuft stets über viele
Kanäle gleichzeitig ab, dass heißt die Reduktion einer Fragestellung auf
Vertikale oder Horizontale Übertragung bedeutet meist eine zu starke
Vereinfachung. Das erschwert die Anwendung methodischer Werkzeuge der
Evolutionsbiologie wie beispielsweise die \emph{Phylogenetischen Bäume}
der \emph{Kladistik} auf kulturhistorische Zusammenhänge. Eine
Konsequenz dieser Beobachtung ist, dass nur ein moderner,
archäologischer Kulturbegriff\footnote{\textcite{furholt_nordlichen_2009},
  21-26.}, der die vielfältige Verknüpfung von Individuen innerhalb und
über ethnische und soziale Grenzen hinweg respektiert, zur
Kategorisierung menschlicher Gesellschaften geeignet sein kann\footnote{\textcite{lipo_science_2001};
  \textcite{lipo_population_1997}; \textcite{palmer_tools_2005};
  \textcite{palmer_cultural_1995-1}; \textcite{palmer_categories_1997};
  \textcite{mcelreath_shared_2003}}.

Eine weitere problematische Vereinfachung, die im Rahmen der Cultural
Transmission Forschung in aller Regel vorgenommen wird, um
Realweltphänomene in Modellen abbilden zu können, betrifft die Modi der
Wissensspeicherung. Vereinfacht werden Menschen als Aufnahmesysteme
beschrieben, die eine Vielzahl von distinkten Ideen tragen
können\footnote{\textcite{mithen_cognitive_1997}}. Tatsächlich ist die
Informationsspeicherung im menschlichen Gehirn wesentlich komplizierter
und funktioniert mittels intensiver, assoziativer Verschaltung von
Ideen. Selbst in der Evolutionsbiologie wurde eine vollständig isolierte
Betrachtung von Genen als \emph{Bean Bag Genetics} verworfen\footnote{\textcite{de_winter_beanbag_1997};
  \textcite{mayr_where_1959}}. Wie, wie schnell und mit welchen
Konsequenzen Menschen neue oder alte Ideen aufnehmen, verarbeiten, zur
Anwendung bringen und weitergeben hängt in hohem Maße von ihrer
kulturellen Gesamtkonfiguration ab, die als allgemeine Weltsicht die
Summe ihrer Erziehung und Erfahrungen spiegelt\footnote{\textcite{gabora_ideas_2004};
  \textcite{sperber_explaining_1996}}. Das hat zur Konsequenz, dass
Menschen, die zusammen leben und intensiven Austausch pflegen, ähnliche
und sich selbst verstärkende Weltsichten aufbauen, Ideen und
Innovationen auf ähnliche Art und Weise verarbeiten und kumulativ in
ihre bisherige Vorstellungswelt integrieren\footnote{\textcite{eerkens_cultural_2007};
  \textcite{basalla_evolution_1988}}.

Ebenso wie die Speicherung von Ideen im Gehirn ist auch die Übertragung
von kultureller Information von einem Menschen zum anderen komplex und
kann auf etliche verschiedene Weisen ablaufen. Die Komplexität der
einzelnen Idee, das Medium über das sie transportiert wird, die Art und
Anzahl der Wiederholungen, denen ein Individuum durchschnittlich
ausgesetzt ist und schließlich ihre innere Struktur nehmen Einfluss auf
ihre Verbreitung in einer Population. Komplexität drückt sich --
technisch besehen -- in der Länge einer Informationseinheit aus. Die
Fehlerwahrscheinlichkeit bei der Übertragung langer Datenketten nimmt
statistisch zu\footnote{\textcite{eerkens_cultural_2007}}, dass heißt
die Übertragung komplexer Ideen ist stärker mutationsanfällig, besonders
wenn kein objektives Korrektiv durch die funktionale Anwendung einer
Kulturinformation besteht. Das Medium der Informationsübertragung
(verbale Erklärung, praxisnahe, visuelle Veranschaulichung, Schrift) und
die damit verknüpfte sensorische Bandbreite spielen eine wichtige Rolle
darin, in welchem Umfang gelerntes von Menschen wiedergegeben werden
kann\footnote{\textcite{eerkens_cultural_2005};
  \textcite{eerkens_practice_2000}; \textcite{eerkens_techniques_2001}}.
Information, die in vielen Wiederholungen präsentiert wurde, kann
tendenziell besser gemerkt werden\footnote{\textcite{cover_elements_2012};
  \textcite{shannon_mathematical_1949}}. Die Übertragung von Information
zwischen Menschen hat nicht nur Mutation zur Folge, sondern auch
Restrukturierung und Hierarchisierung: Menschen abstrahieren
Informationen und können Daten mit sozialen Bezügen und Daten aus
vertrauten, kulturellen Umständen besser verinnerlichen\footnote{\textcite{mesoudi_hierarchical_2004};
  \textcite{mesoudi_Bias_2006M}; \textcite{washburn_remembering_2001}}.
Das ist sicher ein Grund dafür, warum soziale und religiöse Systeme sich
selbst reproduzieren und so über lange Zeit Bestand haben
können\footnote{\textcite{kuijt_people_2000};
  \textcite{kuijt_place_2001}}.

\hypertarget{Biased-transmission}{%
\subsection{Entscheidungsprozesse der Ideenadoption: Biased
Transmission}\label{Biased-transmission}}

Die Intensität und Dauerhaftigkeit der Verbreitung einer Idee in einer
Gesellschaft ist chaotisch und nicht mit Sicherheit vorhersagbar.
Dennoch lassen sich Effekte beschreiben, die wesentlichen Einfluss auf
den Erfolg und die Übertragungskorrektheit von Innovation haben.
Menschen treffen die Entscheidung ob sie eine Idee oder ein
Verhaltensmuster übernehmen nicht zufällig. Stattdessen evaluieren sie
oft sowohl wen als auch was sie in der jeweiligen Situation imitieren.
Um so leichter es ist, die Vor- oder Nachteile verschiedener
Verhaltensmuster zu erkennen, desto schneller kann die Entscheidung für
oder gegen einzelne getroffen werden. Die Konsequenz des
Evalutationsverhaltens ist \emph{Biased Transmission}\footnote{\textcite{henrich_cultural_2001}}.
Ihr Gewicht nimmt zu, wenn dem Einzelnen durch mehr kulturelle Vielfalt
eine größere Auswahl unterschiedlicher Verhaltensmuster zur Verfügung
steht\footnote{\textcite{smith_cultural_1992}}.

\begin{quote}
The essential character of Biased transmission is that information may
come from different sources within a population in spite of being
transmitted in a similar direction and involving the same number of
people.

-- \textcite{eerkens_cultural_2007}, 251.
\end{quote}

Gleichzeitig treffen Menschen selbst komplexe Entscheidungen jedoch oft
auf Grundlage stark vereinfachter Faustregeln. Die investierte Mühe
ergibt sich als Kompromiss zwischen der erwarteten Belohnung einer
richtigen Entscheidung und den Kosten der Informationssammlung\footnote{\textcite{nisbett_human_1980}}.
Eben weil damit nicht viel Kapazität für nicht drängenden Entscheidungen
übrig bleibt, ist Kultur im wesentlichen ein Vererbungssystem. Ein
großer Teil der Glaubens- und Moralvorstellungen eines Individuums hat
es von anderen übernommen, ohne sie zu hinterfragen. Aus diesem Grund
sind Modellimplementierungen, die den Prozess der Informationsweitergabe
als zufälliges Kopieren beschreiben, durchaus berechtigt -- und
zahlreich\footnote{\textcite{bentley_academic_2006};
  \textcite{bentley_cultural_2003}; \textcite{bentley_random_2004-1};
  \textcite{hahn_drift_2003}; \textcite{herzog_random_2004};
  \textcite{lipo_science_2001}; \textcite{lipo_population_1997};
  \textcite{neiman_stylistic_1995}; \textcite{shennan_ceramic_2001}}.

Soziales Lernen kann zur Konsequenz haben, dass schädliches -- also für
genetische Reproduktion ungeeignetes -- Verhalten unter positiven
Selektionsdruck gerät und sich verbreitet\footnote{\textcite{eerkens_cultural_2007};
  \textcite{enquist_evolution_2007}}. Genetische Disposition und
individuelles Lernen können diesem Effekt entgegenwirken. Wenn etwa eine
strenge Religion Prüderie und Abkehr vom Weltlichen propagiert, kann
sexuelles Verlangen und Kinderliebe der familienverneinenden Ideologie
entgegenwirken. Oft sind die Vor- und Nachteile einer Verhaltensform für
den Einzelnen oder die Gesamtpopulation allerdings nicht so
offensichtlich. Die genetische Anlage des Menschen sieht für komplexes
Kulturverhalten keine adäquate Reaktion vor und der Einzelne ist mit der
Evaluation vieler Fragen überfordert.

\begin{quote}
The natural world is complex, hard to understand, and variable from
place to place and time to time. Is witchcraft effective? What causes
malaria? What are the best crops to grow in a particular location? Are
natural events affected by human pleas to their governing spirits?
{[}\ldots{}{]} What sort of person(s) should one marry? What mixture of
devotion to work and family will result in the most happiness or the
highest fitness?

-- \textcite{smith_cultural_1992}, 79.
\end{quote}

Menschen zeigen die Tendenz, das Verhalten erfolgreicher Menschen oder
einzelne, erfolgreiche Strategien zu übernehmen\footnote{\textcite{henrich_evolution_2003}}.
Zwar ist das Modell eines Homo Ökonomikus, der stets die rational beste
Entscheidung in einer gegebenen Situation trifft, zu einfach, dennoch
spielt die Verbesserung der eigenen Situation nach unterschiedlichen
Kriterien eine wichtige Rolle bei Entscheidungsprozessen\footnote{\textcite{mesoudi_cultural_2008};
  \textcite{mesoudi_experimental_2011}}. Die klassische \emph{Diffusion
of Innovation} Forschung identifiziert den individuell wahrgenommenen
Vorteil als wesentliches Kriterium zur Übernahme oder Ablehnung einer
Neuerung\footnote{\textcite{rogers_diffusion_1983}}. Aus der Perspektive
der Behavioural Ecology kann argumentiert werden, dass das Nervensystem
hinreichend komplexer Lebewesen grundsätzlich Verhaltensweisen
bevorzugt, die zu positiven Stimuli führen. Das sind oft gleichzeitig
jene, die für die Anpassung an eine Umgebung förderlich sind. Biologisch
oder durch vormalige Lernprozesse determinierte Lernregeln führen in
einem Prozess von \emph{Guided Variation} zur Selektion von
Verhaltensmustern\footnote{\textcite{smith_cultural_1992}}. Dieser
postulierte Automatismus besitzt Implikationen für eine mögliche
biologische Selektionswirkung von Innovationen: Imitation kann den
Untergang einer Population in Krisensituationen verhindern oder
zumindest die Anpassung an Umweltveränderungen erheblich beschleunigen
und so den mit biologischer Selektion oft verbundenen
Bevölkerungsrückgang vermeiden.

Ein Dualismus von Konformität (\emph{Conformity Bias}) und Neugierde
(\emph{Novelty Bias}) ist entscheidend dafür, ob und wie Innovationen
sich in einer Population verhalten. Menschen neigen besonders in Phasen
von Stabilität dazu\footnote{\textcite{henrich_evolution_1998};
  \textcite{kendal_evolution_2009}}, das Verhalten einer
Bevölkerungsmehrheit zu übernehmen\footnote{\textcite{bikhchandani_learning_1998};
  \textcite{efferson_conformists_2008};
  \textcite{giraldeau_social_1994}; \textcite{henrich_evolution_1998};
  \textcite{heinrich_why_2001}; \textcite{smith_conformity_1994}}.
Dieser \emph{Frequency Bias} hat zur Konsequenz, dass sich Ideen, die
ohnehin schon weit verbreitet sind, weiter stabilisieren können und
Neuerungen, die in direkter Konkurrenz zu vorhanden Konzepten stehen,
nur langsam an Relevanz gewinnen oder verschwinden: Ein sich selbst
verstärkendes System. Insbesondere Ideen, die nicht direkt
subsistenzrelevant sind, sind in ihrer momentanen Ausbreitungsdynamik
stark davon abhängig, wie groß die Verbreitung der Idee in der
Population bereits ist. Eindrucksvolle Beispiele dafür sind unter
anderem Kleidermode oder Babynamen\footnote{\textcite{acerbi_Biases_2014};
  \textcite{acerbi_logic_2012}}. Ist eine Population in teilweise
isolierte Gruppen aufgeteilt, erwirkt ein starker Frequency Bias
Homogenität innerhalb und Heterogenität außerhalb von Gruppen. Die bei
biologischer Evolution umstrittene \emph{Group Selection} kann damit im
Kontext von Cultural Evolution durchaus Wirkung entfalten\footnote{\textcite{smith_cultural_1992}}.

Trotz des Frequency Bias brechen Individuen jedoch manchmal bewusst aus
dem Verhalten der Mehrheit aus\footnote{\textcite{henrich_evolution_2003}}.
Als Konsequenz des Widerstreits dieser Pole folgt die Verbreitung
kultureller Eigenschaften oft einer logistischen, S-förmigen
Wachstumskurve\footnote{\textcite{henrich_cultural_2001}}. Neue Ideen
werden zunächst von einigen, meist wohlhabenden und gut gebildeten
\emph{Innovators} eingeführt bis die ökonomisch empfindlichere
\emph{Majority} sie übernimmt und nur wenige konservative
\emph{Laggards} zurücklässt, die sich der Neuerung bewusst
verweigern\footnote{\textcite{rogers_diffusion_1983}}.

In archäologischen Zusammenhängen wird häufig über den Einfluss sozialer
Eliten auf das Verhalten einer Gesamtpopulation diskutiert:
\emph{Prestige Bias}. Tatsächlich tendieren Menschen dazu, soziale höher
gestellte Vorbilder zu wählen und sie zu kopieren\footnote{\textcite{barkow_prestige_1975};
  \textcite{henrich_evolution_2001}; \textcite{schlag_why_1998}}. Gerade
arme und schlecht gebildete Gruppen orientieren sich oft an
Führungspersonen, die über mehr Risikokapital verfügen, das sie für die
Evaluation von Innovationen investieren können. Dieses Kopierverhalten
lässt sich experimentell bereits an Kleinkindern beobachten, die sich an
jenen Erwachsenen orientieren, die die verstärkte Aufmerksamkeit anderer
Erwachsenen genießen\footnote{\textcite{chudek_prestige-Biased_2012}}.
\emph{Prestige Bias} führt auch zu \emph{Indirect Bias}: Menschen wählen
ihre Vorbildern oft aufgrund weniger auszeichnender Charakteristika aus.
Sie neigen auch dazu, neben den ursprünglich ausschlaggebenden
Eigenschaften weitere Verhaltensmuster des Vorbilds zu übernehmen. Das
hat zur Konsequenz, dass Konzepte, die für sich genommen keine oder nur
geringe Ausbreitung erfahren würden, mit anderen Ideen transportiert
werden\footnote{\textcite{obrien_style_2003}}. Einerseits kann dank
dieser Tendenz mehr Information schneller verbreitet werden,
andererseits können sich so auch Ideen durchsetzen, die ihrem Träger
keinen Vorteil oder sogar Nachteile bringen können. Trotz dieses Risikos
kann es evolutiv sinnvoll sein, einfach das gesamte Verhalten
erfolgreicher Individuen zu übernehmen -- ohne kostenaufwändige
Reflektion darüber, welche Muster genau den Erfolg
herbeiführen\footnote{\textcite{smith_cultural_1992}}. Auch in der
Genetik wurde das Phänomen evolutiv überflüssig tradierter DNS-Sequenzen
beobachtet: \emph{Junk DNA}\footnote{\textcite{doolittle_selfish_1980};
  \textcite{gibbs_unseen_2003}; \textcite{orgel_selfish_1980}}.

\begin{quote}
If wealth partly derives from subsistence or social skills that can be
acquired by imitation, it makes adaptive sense to imitate the wealthy.
The assumption that wealth is correlated with adaptive behavior is
perhabs generally correct; if so it would be sensible to imitate wealthy
people even if it is not always very clear just what components of
wealthy people's behavior are adaptive.

-- \textcite{smith_cultural_1992}, 81.
\end{quote}

Soziale Hierarchien und Prestigesysteme können als Hilfsmittel dienen,
um zu entscheiden, welche Eigenschaften und Verhaltensweisen übernommen
werden sollten\footnote{\textcite{rogers_diffusion_1983}}. Information
von Autoritätspersonen oder -einrichtungen, sowie Information, die unter
dem Siegel der Geheimhaltung übermittelt wird, wird mehr Bedeutung
beigemessen und statistisch fehlerärmer weitergegeben\footnote{\textcite{rowlands_role_1993}}.
Der situative Kontext in dem eine Information vermittelt wird hat
generell großen Einfluss auf die Korrektheit der Übertragung und darauf,
ob die Empfänger sie als eigenes Wissen übernehmen\footnote{\textcite{barth_cosmologies_1990};
  \textcite{barth_guru_1990}; \textcite{labov_principles_1994};
  \textcite{whitehouse_memorable_1992}}.

Grundsätzlich werden bevorzugt Menschen imitiert, die lokal präsent sind
und in ähnlichen Umständen leben wie der Imitierende. \emph{Homophily},
die Präferenz mit gleichgesinnten Menschen zu interagieren, erstreckt
sich auf jede Form zwischenmenschlicher Beziehung: Ideen werden
grundsätzlich schneller zwischen Individuen mit ähnlichem Weltbild
übertragen\footnote{\textcite{centola_experimental_2011};
  \textcite{centola_spread_2010}, \textcite{schlag_why_1998}}. Ein
Phänomen, das in diesem Zusammenhang für vertikale Transmission
besondere Relevanz besitzt, ist \emph{Assortative Paarung}
(\emph{Assortative Mating}). Partnerwahl beim Menschen ist kein
zufälliger Prozess, sondern folgt statistisch einem erforschbaren
Regelwerk. Beispielsweise neigen Individuen bei der Partnersuche zu
Gegenübern mit hoher Ähnlichkeit körperlicher und kultureller Merkmale.
Ein Nachweis dieses Effekts gelang in modernen Kontexten bei
Charakteristika wie Augenfarbe, Körpergröße, IQ, Bildungsstand und
Tabakkonsum\footnote{\textcite{domingue_genetic_2014};
  \textcite{keller_genetic_2013}; \textcite{laeng_why_2007};
  \textcite{treur_spousal_2015}}. Assortative Paarung führt zu höherer
Korrelation genetischer und kultureller Eigenschaften in einer
Population und kann dennoch mehr Vielfalt hervorrufen\footnote{\textcite{feldman_evolution_1977};
  \textcite{rice_multifactorial_1978}}: Seltene Eigenschaften können
sich leichter ausbreiten und behaupten\footnote{\textcite{creanza_complexity_2014};
  \textcite{creanza_models_2012}}. Assortativer Paarung ist dabei auch
ein sich selbst verstärkender Prozess, da aus Beziehungen ähnlicher
Partner statistisch mehr Kinder hervorgehen\footnote{\textcite{thiessen_human_1980}}
und soziale Netzwerke dazu neigen, sich zu reproduzieren\footnote{\textcite{abdellaoui_association_2013};
  \textcite{abdellaoui_educational_2015}}. Das hat auch Rückwirkungen
auf die genetische Zusammensetzung von menschlicher
Populationen\footnote{\textcite{robinson_genetic_2017}}. Sprachgrenzen
können dabei als wesentliche Hürde beim genetischen Austausch
auftreten\footnote{\textcite{barbujani_zones_1990};
  \textcite{de_filippo_y-chromosomal_2011};
  \textcite{karafet_coevolution_2016}}, müssen es aber
keinesfalls\footnote{\textcite{hunley_gene_2005};
  \textcite{hunley_genetic_2008}; \textcite{srithawong_genetic_2015}}.

\hypertarget{stylistic-variability}{%
\subsection{Cultural Transmission in der archäologischen Forschung:
Stylistic Variability}\label{stylistic-variability}}

Cultural Transmission greift eine Kernidee der Wissenschaft Archäologie
auf: Ähnlichkeit zwischen Artefakten eines kulturhistorischen
Zusammenhangs in Raum- und Zeit können dadurch erklärt werden, dass sie
als Teil einer von Generation zu Generation übermittelten
Fertigungstradition im diachronen, sozialen Gefüge einer
Gesamtpopulation verstanden werden müssen\footnote{\textcite{lyman_culture_2001};
  \textcite{lyman_measuring_2000}; \textcite{lyman_rise_1997};
  \textcite{obrien_epistemological_2002}}. Die Perspektive, die Cultural
Transmission Theory auf diesen Sachverhalt eröffnet, hat bemerkenswerte
Forschung ausgelöst, aber auch berechtigte Kritik hervorgerufen.
Letztere greift bei Parametern an, die bisher von Cultural Transmission
Modellen vernachlässigt wurden\footnote{\textcite{dobres_creativity_2000}},
verwirft die Vorstellung der isolierten Betrachtung von Ideen und
Kulturentwicklung rundweg\footnote{\textcite{mithen_cognitive_1997}}
oder stellt die praktische Nutzbarkeit des Paradigmas in Frage\footnote{\textcite{dunnell_archaeology_1992};
  \textcite{schiffer_memes_2003}}. Eine Forschungsrichtung, die diesen
Vorwürfen eine vielversprechende Debatte entgegenstellt und für die
vorliegende Arbeit besondere Relevanz besitzt, kann mit dem Schlagwort
\emph{Stylistic Variability} überschrieben werden.

1978 formulierte Robert Dunnell in einem Beitrag\footnote{\textcite{dunnell1978style}}
eine schon zuvor diskutierte\footnote{\textcite{eerkens_cultural_2007}}
und im folgenden intensiv aufgegriffene Unterscheidung: In
archäologischen Kontexten und über lange Zeiträume müssten Ideen und
Verhaltensmuster in zwei Kategorien unterschieden werden --
\emph{Funktion} (\emph{functions}) und \emph{Form} (\emph{style}).
Funktionale Ideen, wie das Wissen um die Nutzung einer bestimmten
Getreideform, die Kompetenz bestimmte Werkzeuge oder Waffen zu fertigen
oder soziale Anerkennung von Kinderreichtum sind unmittelbar
selektionsrelevant. Hier greift die biologische Evolution, die anhand
der Überlebensrate in Krisensituationen oder schlicht der
demographischen Entwicklung dahin wirkt, bestimmte Ideen über ihre
Träger zu begünstigen. Diesen Ideen steht eine Vielzahl anderer
entgegen, \emph{Neutral Traits}, die nicht unmittelbar
selektionsrelevant sein sollten, so wie Keramikverzierung, Trachtmode
oder Schmuckform. Das Überleben oder die Fortpflanzungsfähigkeit einer
Population sind normalerweise nicht von der Natur der geometrischen
Muster auf ihren Vorratsgefäßen abhängig. Dennoch durchlaufen diese
Merkmale eine teilweise atemberaubend schnelle Entwicklung und erlauben
Archäologen Einblicke in Relativchronologie und Sozialstruktur. Sie
würden, so Dunnell, im Kontext der Evolutionary Archaeology trotz ihrer
genetisch untergeordneten Schlagkraft besondere Relevanz verdienen --
und ein dediziertes Methodenset: Dunnell postuliert, dass funktionale
Ideen unter Beachtung ihrer Adaptionsqualität modelliert werden könnten,
während stilistische Ideen im wesentlichen zufälligen, stochastischen
Prozessen (\emph{Neutral Transmission}) folgen würden.

Dunnells Unterscheidung passt sich gut in den weiteren Kontext der Dual
Inheritance Theory (siehe Kapitel \ref{evolutionism-modern-theories})
ein und wurde vielfältig aufgegriffen und erweitert\footnote{\textcite{lipo_population_1997};
  \textcite{lipo_science_2001}; \textcite{lyman_measuring_2000};
  \textcite{neiman_conspicuous_1997-1};
  \textcite{neiman_stylistic_1995}; \textcite{rindos_darwinian_1985};
  \textcite{rindos_undirected_1989}; \textcite{shennan_ceramic_2001};
  \textcite{teltser_culture_1995}}. Tatsächlich sind auch die meisten
\emph{Neutrale Varianten} harten Selektionskriterien nicht gänzlich
erhaben: Ihre Erstellung verursacht in unterschiedlichem Umfang Kosten
in Hinblick auf Arbeitszeit und Material\footnote{\textcite{meltzer_study_1981}}.
Da die Unterscheidung stilistischer und funktionaler Merkmale nicht
immer offensichtlich ist\footnote{\textcite{bettinger_style_1996}}, kann
es erfordrlich sein, die funktionale Nützlichkeit verschiedener
Varianten mit Experimenten und anderen Methoden der
Ingenieurswissenschaften zu vergleichen\footnote{\textcite{kornbacher_building_2001};
  \textcite{obrien_evolutionary_1994}; \textcite{obrien_variation_1990};
  \textcite{pfeffer_engineering_2001-1};
  \textcite{wilhelmsen_building_2001}}: Je ähnlicher die Nützlichkeit
aller verfügbaren Varianten in einem Kontext, desto größer die
Wahrscheinlichkeit, dass sie sich relativ zueinander selektiv neutral
verhalten.

Fraser Neimans Artikel \emph{Stylistic Variation in Evolutionary
Perspective: Inferences from Decorative Diversity and Interassemblage
Distance in Illinois Woodland Ceramic Assemblages}\footnote{\textcite{neiman_stylistic_1995}}
ergründet die von Dunnell vorgeschlagenen stochastischen Prozesse und
erarbeitet ein einfaches Framework für die Simulation der Ausbreitung
nicht funktionaler Ideen. Dabei werden die wesentlichen Phänomene
\emph{Drift} und \emph{Flow} aus der Populationsgenetik in die
Archäologie eingeführt und zur Interpretation von Kulturdistanzen
genutzt. Neimans Beitrag wurde in der Fachwelt sehr intensiv
rezipiert\footnote{u.a. \textcite{bentley_cultural_2003};
  \textcite{eerkens_cultural_2005}; \textcite{kohler_vessels_2004-1};
  \textcite{shennan_ceramic_2001} -- Microsoft Academic listet im
  Augenblick 393 Referenzen auf Neimans Artikel --
  \url{https://academic.microsoft.com/\#/detail/2316012851}
  {[}27.8.2018{]}} -- besonders zu nennen sind Carl Lipos Arbeiten zu
Keramik im Mississippi-Unterlauf, die Cultural Transmission Simulation
sinnvoll mit klassischer archäologischer Seriation verknüpfen\footnote{\textcite{lipo_community_2001};
  \textcite{lipo_neutralitystyle_2001}; \textcite{lipo_population_1997};
  \textcite{lipo_science_2001}} -- und war auch für die Erstellung der
Simulation dieser Arbeit eine wesentliche Inspiration.

\newpage
\pby[title={Literatur},segment=\therefsegment,heading=subbibintoc]

\hypertarget{bronze-age-burial-rites}{%
\chapter{Bestattungsritus in der Europäischen
Bronzezeit}\label{bronze-age-burial-rites}}

\hypertarget{fallbeispiel-und-betrachtungsperspektive}{%
\section{Fallbeispiel und
Betrachtungsperspektive}\label{fallbeispiel-und-betrachtungsperspektive}}

Ausgangspunkt dieser Arbeit war die Formulierung eines computerbasierten
Cultural Evolution Modells. Um sich mit diesem Ansatz inhaltlich und
technisch auseinander zu setzen war es unerlässlich ein Fallbeispiel
heranzuziehen, das potentiell geeignet ist überhaupt durch ein solches
abgebildet zu werden. Für den Kontext des Fallbeispiels sollen sich
idealerweise Synergieeffekte ergeben. Das heißt, das Modell sollte
geeignet sein, archäologische Fragestellungen in seinem Kontext zu
beantworten oder zumindest aus einer neuen Perspektive zugänglich zu
machen.

Die Wahl des Fallbeispiels hat wesentliche Konsequenzen für die
Modellimplementierung. Unmittelbar funktional relevante Innovationen,
die z.B. eine Veränderung der Subsistenzstrategie hervorrufen, sind
anders zu analysieren als Mode in Keramikverzierung und Gewandschmuck
(siehe Kapitel \ref{stylistic-variability}). Manche Ideen sind äußerst
erfolgreich, breiten sich über ganze Kontinente aus und bleiben über
Jahrhunderte verhältnismäßig stabil, andere dagegen sind nur auf eine
eine Siedlung beschränkt und überdauern nicht einmal ihre
Schöpfergeneration. Jedes Fallbeispiel ist über eine Auswahl
archäologischer Daten zugänglich. Diese sind höchst heterogen
strukturiert, mit unterschiedlichen Zielsetzungen -- meist nicht der
einer Cultural Evolution Analyse -- aufgenommen und decken, ebenso wie
das repräsentierte Kulturverhalten, sehr verschiedene zeitliche und
geographische Spektren und Skalenniveaus ab. Ideal wäre sicher, selbst
Daten zu zu einzelnen Ideen und deren Entwicklung zu sammeln. Das war
aber im Rahmen dieser Arbeit nicht möglich, ohne viel Zeit zu verlieren,
die für theoretische Vorüberlegungen sowie die Modellimplementierung und
-analyse investiert werden sollte. Die Suche nach einem Fallbeispiel war
also gleichermaßen die Suche nach einem Datensatz, bei dem
Anküpfungspunkte zur Modellidee zu erwarten waren.

Ein spezielles Subset der C14-Datenbank Radon-B\footnote{\textcite{jutta_kneisel_radon-b_2013}}
erfüllt diese Bedingung (siehe Kapitel \ref{radonb-dataset}). Es enthält
Informationen zur zeitlichen und räumlichen Verteilung bronzezeitlicher
Bestattungssitten: Die Fallstudie dieser Arbeit konzentriert sich auf
vier eng verknüpfte Ideen -- Körperbestattung, Brandbestattung,
Flachgrab, Hügelgrab -- die im Laufe der europäischen Bronzezeit (hier:
2200-800calBC) in Nord-, Ost und Westeuropa eine komplexe
Verbreitungsgeschichte durchleben.

Brandbestattung und Körperbestattung sind Traditionen, die schon lange
vor Beginn der Bronzezeit in Konkurrenz standen. Erstaunlicherweise ist
dieser Konflikt bis heute nicht entschieden -- beide Bestattungsrituale,
freilich immer wieder neu konnotiert und kontextualisiert -- finden in
der Gegenwart in Europa Anwendung. Man könnte den Konflikt aus dieser
Perspektive in seiner gesamten zeitlichen Dimension von der frühesten
Vorgeschichte bis in die Moderne nachzeichnen. Er ist auch nicht auf
Europa beschränkt: Diese Bestattungsformen sind derart universell, dass
sie in einer Vielzahl von Kulturen auf der ganzen Welt in lokalen
Ausprägungen praktiziert wurden. Dennoch konzentriert sich diese Arbeit
auf die europäische Bronzezeit. Das geschieht einerseits aufgrund der
zur Verfügung stehenden Daten und weiterhin aufgrund der unglaublichen
Komplexität der Kulturphänomene, in die sich beide Praktiken im Laufe
der Geschichte eingegliedert haben. Eine Geschichte von Brand- und
Körperbestattung würde den Rahmen dieser Arbeit bei weitem sprengen, ist
vielleicht überhaupt nicht sinnvoll formulierbar. Ähnlich verhält es
sich mit der Grabüberhügelung: Auch dieser Brauch kann auf eine lange
Geschichte zurückschauen und hat im Laufe der Zeit mannigfaltige,
verwandte weil abgeleitete Rituale hervorgebracht.

\hypertarget{thanatoarchaologie}{%
\section{Thanatoarchäologie}\label{thanatoarchaologie}}

\emph{Thanatologie} ist die Wissenschaft des Todes und seiner Wirkung
auf die Umwelt des Sterbenden und Verstorbenen. Sie ist interdisziplinär
angelegt und beschäftigt sich mit allen biologischen, sozialen,
psychologischen und sonstigen Prozessen im Kontextes des biologischen
und speziell menschlichen Todes\footnote{\textcite{hofmann_rituelle_2008},
  100.}. \emph{Thanatoarchäologie} beschäftigt sich mit dem Tod in
archäologischen Kontexten, also mit dem Niederschlag, den der Tod von
Menschen im archäologischen Befund hinterlassen hat\footnote{\textcite{hofmann_rituelle_2008},
  123.} -- eine Subdisziplin mit langer Geschichte\footnote{\textcite{hofmann_rituelle_2008},
  132-140.}. Der wichtigste Befundtyp der Thanatoarchäologie ist das
Grab, das umgekehrt einer der wichtigsten Forschungsgegenstände der
Archäologie im allgemeinen ist. Seiner Erforschung wird vor dem
Hintergrund chronologischer und sozialer Fragestellungen viel
Aufmerksamkeit gewidmet. Dennoch bleibt ein großer Teil der mit Gräbern
assoziierten Bedeutungsbelegung unbekannt, da Gräber sich nur aus der
Wahrnehmung des Todes ihn ihrer Erzeugerkultur und deren Vorstellungen
pränataler- und postmortaler Zustandsformen in Abgrenzung zum bekannten
irdischen Leben verstehen lassen. Archäologische Quellen geben über
diese spirituellen Abstraktionen keinen oder nur sehr eingeschränkt
Aufschluss.

Eben daraus erwächst die große Gefahr, in Ermangelung des Wissens über
das Todesverständnis prähistorischer Gesellschaften moderne, westliche
Vorstellungen auf archäologisch erschlossene Grabzusammenhänge zu
projizieren. Ein eurozentrisches Verhalten, das die
\emph{Postprozessuale Archäologie} in Anlehnung an die
\emph{Postmoderne} als solches identifiziert hat. Stattdessen muss eine
Auseinandersetzung mit dem breiten Spektrum an Weltanschauungen und
Wahrnehmungen erfolgen, in die Bestattungssitten eingeordnet werden
können. Vollständigkeit kann dabei nicht erreicht werden, aber zumindest
eine erhebliche und wertvolle Aufweitung der Perspektive.

Kerstin Hofmanns Dissertationsschrift \emph{Der rituelle Umgang mit dem
Tod -- Untersuchungen zu bronze- und früheisenzeitlichen Bestattungen im
Elbe-Weser-Dreieck}\footnote{\textcite{hofmann_rituelle_2008}} enthält
umfangreiche, theoretische Vorüberlegungen zur Thanatoarchäologie, die
hier verarbeiten und einer kurzen Einordnung von Bestattungssitten im
Kontext der Cultural Evolution Theory vorangestellt werden sollen. Damit
soll einerseits einer zu simplistischen Deutung von Gräbern vorgebeugt,
andererseits die Besonderheiten von Bestattungssitten als tradiertes
Kulturverhalten betont werden.

\hypertarget{sterben-als-prozess}{%
\subsection{Sterben als Prozess}\label{sterben-als-prozess}}

Die Feststellung, ab wann genau ein Mensch Tod ist, ist mit
erstaunlichen Unsicherheiten und Unschärfen verknüpft. Diese nehmen
ihren Anfang bei den biologischen Prozessen, die es erlauben, den
Eintritt des Todes an verschiedenen Parametern zu messen und
entsprechend unterschiedlich festzulegen. Leben drückt sich im Menschen
in verschiedenen Körperfunktionen wie Atmung, Herzschlag oder
Stoffwechsel aus. Der Ausfall eines Teilsystems bewirkt je nach seiner
Relevanz mehr oder weniger schnell den Zusammenbruch aller anderen
Systeme. Das kann sich über einen langen Zeitraum hinziehen: Auch im
Falle des normalen, sukzessiven Ausfalls aller Teilsysteme stirbt die
letzte Körperzelle viele Stunden nach dem Kreislaufstillstand. Da die
Individualität eines Menschen an die intakte Funktion seines Gehirns
gebunden ist, gilt der Kollaps dieses Teilsystems als eines der
wesentlichen Definitionsmomente für den Eintritt des Todes. Umgekehrt
kennt die Medizin mit dem Hirntod auch den Sonderfall, dass nur das
Gehirn seine Funktion mit irreparablen Schäden eingestellt hat, alle
anderen Körperprozesse allerdings weiter funktionieren. Der Hirntod kann
nur klinisch diagnostiziert werden (Harvard-Kriterium), da in diesem
Fall andere Indikatoren für den Eintritt des Todes fehlen. Letztere
lassen sich grundsätzlich in unsichere und sichere pathophysiologische
Kriterien untergliedern. Zu den unsicheren gehören ein Abkühlen des
Körpers, Reflexlosigkeit, Erschlaffen der Muskeln, Pulslosigkeit,
Atemstillstand, Leichenblässe und ein Vertrocknen an Schleimhäuten und
Wunden. Obgleich diese traditionellen Todesanzeiger weitreichend bekannt
sind und im Laufe der Geschichte wesentlich für die Feststellung des
Todes waren, sind sie einzelnen oder sogar bei gemeinsamem Auftreten
nicht verlässlich. Sie können (zumindest kurz- bis mittelfristig) als
Folge von Erkrankungen oder Umgebungsparametern auftreten. Pulslosigkeit
und Atemstillstand sind, wenn der Zustand anhält, sichere
Todesanzeichen. Dazu gehören auch Totenflecken -- rötliche Verfärbungen
an der Körperunterseite infolge der Unterbrechung des Blutflusses -- und
die Totenstarre -- eine biochemische Körperreaktion, die zur Erstarrung
der Muskulator in einem Zeifenster von 6-9 bis 50-300 Stunden nach dem
Todeszeitpunkt führt. Völlig unzweifelhafte Todesanzeiger sind
schließlich spätere Veränderungen an der Leiche wie Autolyse
(Selbstauflösung/Selbstverdauung), Fäulnis, Mumifizierung,
Fettwachsbildung und Skelettierung\footnote{\textcite{hofmann_rituelle_2008},
  92-94.}.

Der mit naturwissenschaftlichen Kriterien messbare Tod ist in einer
modernen, westlich geprägten Gesellschaft oftmals die entscheidende Form
des Todes. Tatsächlich ergeben sich aber neben dieser
biologisch-technischen auch fundamental abweichende Perspektiven, die
den Tod durch seine Kontextualisierung im kulturell-sozialen Gefüge des
Verstorbenen verstehen. Der Tod ist dabei der Abbruch der sozialen
Beziehungen. Dieses Ereignis muss nicht mit dem biologischen Tod
einhergehen. Tatsächlich kann sowohl ein biologisch Lebender aus einer
Gemeinschaft ausgeschlossen und damit für ``tot'' erklärt werden, als
auch ein biologisch Toter -- etwa im Kontext eines Ahnenkults -- weiter
in zwischenmenschliche Interaktion einbezogen und wie ein Lebender
behandelt werden. Vor diesem Hintergrund hält Hofmann die Einschätzung
ob jemand tot ist oder lebendig für von kultureller Wahrnehmung
abhängig:

\begin{quote}
Niemand kann demnach eine Todesfeststellung kulturfrei vornehmen.

-- \textcite{hofmann_rituelle_2008}, 92.
\end{quote}

Das Urteil, ob biologischer und sozialer Tod gleichzeitig eingetreten
sind, ist darüber hinaus stark mit der Art des Todes verknüpft, die den
Verstorbenen ereilt hat. Ein schneller Unfalltod, ein Mord, ein Tod in
kriegerischem Konflikt oder ein langsames Dahindämmern infolge von Alter
oder Krankheit werden unterschiedlich wahrgenommen und sind kulturell
unterschiedlich konnotiert. Oftmals ist genau das Ausschlaggebend dafür,
ob sich der Tod im Einzelfall Angehörigen und Beobachtern als schnelles,
unumkehrbares Überschreiten einer Linie oder als länger andauernder
Transformationsprozess darstellt. Den Rahmen für diese Unterscheidung
bilden Vorstellungen von postmortalem Leben, das das irdische Leben
fortsetzt oder mit ihm interagieren kann. Damit ist der Tod und seine
Erfahrung eng mit grundsätzlichen, weltanschaulichen Fragen verknüpft,
denen jede Kultur mit anderen Paradigmen begegnet\footnote{\textcite{hofmann_rituelle_2008},
  94-95.}.

\hypertarget{kulturubergreifende-wahrnehmung-des-todes}{%
\subsection{Kulturübergreifende Wahrnehmung des
Todes}\label{kulturubergreifende-wahrnehmung-des-todes}}

Wie und mit welchen Hoffnungen und Ängsten der Einzelne dem eigenen oder
dem Tod anderer Menschen begegnet, hängt von einer Vielzahl von Faktoren
ab. Prägend dafür ist ein Erfahrungshorizont, der sich aus Kultur- und
Religionszugehörigkeit ergibt, aber auch individuellen Eigenschaften und
Erfahrungen.

\begin{quote}
Die Einstellung zum Tode entstehen aus der dynamischen, sich
verändernden Wechselwirkung zwischen Individuum und Umwelt und sind mit
dem individuellen und kollektiven Bild von Mensch, Natur und
Gesellschaft verknüpft.

-- \textcite{hofmann_rituelle_2008}, 96.
\end{quote}

Nach moderner, naturwissenschaftlicher Erkenntnis muss jeder Mensch
sterben. Diese Wahrnehmung hat ihren Ursprung wahrscheinlich in der
Antike, ist allerdings nicht universell menschlich. In indigenen
Gesellschaften wird der Tod oft als etwas unnatürliches und fremdes
gedeutet, dass durch schädliche äußere Einflüsse -- etwa durch Flüche
oder den Eingriff von Gottheiten -- ausgelöst wird. Viele
Schöpfungssagen beschreiben einen Urzustand, in dem der Tod noch nicht
existierte. Erst ein durch Versehen oder Unwissenheit ausgelöstes
Ereignis habe ihn in die Welt gebracht. Oft wird ein ``guter'', von der
Gemeinschaft begleiteter Tod von einem ``schlechten'', einsamen in der
Fremde abgegrenzt.

Der Glaube an ein postmortales Weiterleben ist in der Mehrzahl bekannter
Kulturen verbreitet. Die Auflösung des Körpers im Anschluss an den Tod
mag ein wichtiger Grund dafür sein, dass die meisten das auch mit einer
Trennung von Körper und Seele in Verbindung bringen. Die Seele vollzieht
einen Transformationsprozess -- eine Reise ins Reich der Toten -- der
sich über einen gewissen Zeitraum hin erstreckt. Oft werden Verstorbene
in diesem Übergangsstadium als gefährlich empfunden, da sie sowohl Macht
als auch das Interesse zur Interaktion mit der Welt der Lebenden
besitzen könnten. Im Gegensatz zum Personenkonzept des westlichen
Individualismus, der Menschen als Einheit aus einem Körper und einer
unteilbaren Seele versteht, unterscheiden andere Gesellschaften
gegebenenfalls mehrere Seelenkategorien, die im Todesfall
unterschiedlich reagieren, oder etwa durch unterschiedliche Rituale wie
Kremation freigesetzt werden müssen. Unabhängig davon ist die Seele
jedoch in vielen Weltanschauungen die Entität, die im Jenseits
weiterlebt. Das Totenreich ist in verschiedenen Kulturen mit
vielfältigen Assoziationen belegt -- häufig mit dem Bild des Schlafens,
einer spiegelbildlichen Parallelwelt zum irdischen Dasein oder mit
Mechanismen, die Ausgleich und Sühne schaffen sollen. Kulturen, die ein
Reinkarnationskonzept verinnerlicht haben, verstehen den Tod meist nur
als eine kurze Phase zwischen zwei Daseinsformen.

In ausnahmslos allen Kulturen gibt es ein Totenbrauchtum, das den Umgang
mit Verstorbenen regelt. Die praktizierten Handlungen wie
Leichnamsvorbereitung, Bestattung, Totenmahlzeit, Besuchsfeste oder
Wiederbestattungen sind höchst vielfältig und stark von den oben
beschriebenen, ideologischen Voraussetzungen abhängig. Ebenso sind Zweck
und Bedeutung der Riten unterschiedlich. Sie zeigen jedoch
kulturübergreifend einige Gemeinsamkeiten und richten sich grundsätzlich
sowohl an die Toten als auch die Lebenden. Viele Rituale dienen dazu,
das emotionale Trauma und die Trennung zu verarbeiten. Darüber hinaus
soll der Zusammenhalt der Bestattenden sozialen Gruppe auch über den Tod
des Verstorbenen hinaus aufrecht erhalten werden. Dabei kanalisieren die
rituellen Handlungen den kritischen Übergangsprozess von der dauerhaften
An- zur Abwesenheit des Individuums. Bestattungsbräuche können auch aus
einer Angst vor dem Toten hervorgehen und dazu dienen, ihn zu bannen, zu
besänftigen oder zumindest seinen Einfluss auf die Lebenden zu
verringern. Umgekehrt existiert mit der Totenfürsorge ein Verhalten, den
Verstorbenen mit Grabbeigaben für sein postmortales Dasein auszustatten
oder ihn mit Wegzehr für die Reise ins Totenreich zu versehen.
Bestattungsrituale können auch dazu dienen formalisiert zu Erinnern und
eine bestimmte Form der Erinnerung an den Toten zu konstruieren.
Weltanschauung und Moralvorstellungen können gegebenenfalls in der
Gruppe durch Repetition und den besonderen Charakter des Anlasses
vertieft werden. Eben hieran wird deutlich, dass Bestattungssitten nicht
losgelöst, sondern in den religiösen, sozioökonomischen, politischen und
sonstigen Kontexten einer Gesellschaft verankert sind. Umfang und
Komplexität einer Beisetzung hängen oft stark von der Ausprägung der
sozialen Hierarchien und der individuellen Position des Verstorbenen
darin ab. Eine systemtheoretische Perspektive macht deutlich, wie sehr
Totenbrauchtum von anderen Subsystem der Gesellschaft abhängen und diese
widerspiegeln kann\footnote{\textcite{hofmann_rituelle_2008}, 96-99.}.

\hypertarget{tod-in-den-anthropologischen-wissenschaften}{%
\subsection{Tod in den anthropologischen
Wissenschaften}\label{tod-in-den-anthropologischen-wissenschaften}}

Hm\ldots{} Das könnte ein wenig zu viel sein\ldots{} ggf.
\textcite{hofmann_rituelle_2008}, 100-122.

Genauso die Forschungsgeschichte der Thanatoarchäologie\ldots{} ggf.

\hypertarget{die-erforschung-des-todes-in-der-prahistorischen-archaologie}{%
\subsection{Die Erforschung des Todes in der prähistorischen
Archäologie}\label{die-erforschung-des-todes-in-der-prahistorischen-archaologie}}

Vergangenes, menschliches Kulturverhalten lässt sich nur über das Medium
archäologischer Quellen erschließen. Diese Materielle Kultur spiegelt
den eigentlichen Forschungsgegenstand allerdings nur mehrfach und stark
gefiltert wieder. Selbst ihre eigene Bedeutung lässt sich nur indirekt
und unvollständig rekonstruieren. Dieser Problematik widmet sich ein
großer Teil der theoretischen, archäologischen Forschung\footnote{\textcite{hofmann_rituelle_2008},
  123-128.}. Wie oben ausgeführt sind Handlungen, die mit dem Tod in
Verbindung stehen, meist besonders bedeutungsgeladen und deswegen schwer
rekonstruierbar.

Die wichtigsten Befundtypen der Thanatoarchäologie sind Gräber --
einzeln oder im Kontext von Gräberfeldern und sonstigen
Kollektivgrabanlagen. Schrift- oder ikonographische Quellen, die
Aufschluss über das Totenritual oder sogar die zugrunde liegende
Vorstellungswelt geben würden, existieren in der prähistorischen
Archäologie nicht oder sind äußerst selten. Im Kontext von Gräberfeldern
können oft neben den eigentlichen Bestattungseinrichtungen auch Gruben,
Steinpflaster, Ustrinen (Verbrennungsplätze, an denen Scheiterhaufen
errichtet wurden) und aufgehende Strukturen wie Zäune, Grabmarkierungen
oder Ritualaufbauten dokumentiert werden. Selbst im Fall von datierbarer
Gleichzeitigkeit müssen jedoch nicht alle Befunde auf einem Gräberfeld
mit dem Totenbrauchtum in Verbindung stehen. Umgekehrt haben nicht alle
Handlungen eines Bestattungsrituals in räumlicher Nähe zum
Bestattungsplatz stattfinden müssen. Auch die Anlage von Gräbern ist
nicht obligatorisch: Viele Bestattungsrituale sehen keine
Grabarchitektur vor und manche schließen den eigentlichen Leichnam aus
der Deponierung aus. Solche Pseudogräber oder Kenotaphe sind schwer von
Hortfunden unterscheidbar und werden meist nur über ihre Position auf
dem Gräberfeld identifiziert. Siedlungsbestattungen sind in wenigen
Kulturzusammenhängen die Regel, treten aber immer wieder auf. Sie
erlauben eine besondere Kontextualisierung der Bestattung über die
räumliche Verknüpfung zu Siedlungsarealen oder Haushalten\footnote{\textcite{hofmann_rituelle_2008},
  128-129.}.

Menschliche Überreste finden sich auch außerhalb von intentionell
angelegten Gräbern -- etwa als Konsequenz unnatürlicher Tode durch
Unglücke, Naturkatastrophen oder Gewalt. Auch diese Quellen sind Teil
der thanatoarchäologischen Forschung, müssen aber anders interpretiert
werden. Aufgrund schlechter Erhaltungssituation durch stärkere
taphonomischer Einflüsse sowie Unsicherheiten über das Kulturverhalten
vormoderner Menschen ist die Entscheidung, ob ein Leichnam in einem
Ritual bewusst niedergelegt oder schlicht zufällig durch
Sedimentbedeckung konserviert wurde besonders in der paläolithischen
Archäologie oftmals schwer\footnote{\textcite{hofmann_rituelle_2008},
  145-147.}.

\hypertarget{quellengattung-grab}{%
\subsection{Quellengattung Grab}\label{quellengattung-grab}}

Neben Siedlungen, Horten und Einzelfunden gehören Gräber zu den
Hauptkategorien archäologischer Quellengattungen. Gräber und Depots
heben sich von Siedlungen ab, da sie grundsätzlich eine positive
Artefaktauswahl einschließen, das heißt, die eingebrachten Objekte
wurden intentional in diesem Kontext platziert. Diese Intentionalität
gilt auch für den Grabaufbau. Gräber sind also hochgradig
bedeutungsgeladene Befunde, die als Überrest der rituellen Handlungen
des Totenbrauchtums konserviert werden. Sie bilden religiöse, soziale
und politische Strukturen, Werte und Normen einer Gruppe ab --
allerdings stets schematisiert und gegebenenfalls bewusst manipuliert.
Die Vielzahl an Filtermechanismen, die zwischen der Lebensrealität einer
prähistorischen Gesellschaft und dem archäologisch fassbaren Befund
wirken, werden bei der Rekonstruktion von Sozialstrukturen oftmals nicht
ausreichend reflektiert. Das ist umso mehr relevant, wenn aus den
statischen archäologischen Quellen dynamische, chronologische
Entwicklungen und Transformationsprozesse abgelesen werden sollen.

Menschen trafen in der Vorgeschichte immer wieder neu eine Entscheidung
für die Position eines Bestattungsplatzes im natur- und
kulturgeographischen Raum. Der Entscheidungsprozess erschließt sich aus
einer landschaftsarchäologischen Perspektive, die einerseits natürliche
Gegebenheiten wie Topographie, Vegetation oder Wassernähe am
Bestattungsplatz sowohl absolut als auch in Relation zu damit
wahrscheinlich verknüpften Siedlungen betrachtet, als auch die
kulturhistorischen Bezüge zu kontemporärer oder vorangegangener
menschlichen Aktivität in der Umgebung. Das erfordert eine grundsätzlich
mit Unsicherheiten behaftet Rekonstruktion der Landschaft zum Zeitpunkt
der Anlage des Bestattungsplatzes. Funktionale Kriterien wie das Meiden
von hochwassergefährdeten Flächen oder Arealen mit schwacher Bodendecke
mögen zu einer Vorauswahl der Plätze geführt haben. Darüber hinaus sind
dem Feld ideologischer Konnotationen keine Grenzen gesetzt. Das kann zum
Beispiel zur Beachtung astronomischer Relationen oder einer bewusst
erzeugten über- oder unterbetonten Sichtbarkeit der Anlage führen. Ein
Bestattungsplatz ist schließlich selbst landschaftsprägend: Grabanlagen
können Territorien abgrenzen oder Wege markieren. Die Aufgabe eines
Gräberfelds, seine kontinuierliche Nutzung oder die Wiederaufnahme der
Nutzung einer alten Anlage, die gegebenenfalls aus einem
vorangegangenen, archäologischen Kulturzusammenhang stammt, geschieht
oft in einem Prozess, der mit anderen schwerwiegenden Veränderungen in
einer Siedlungsgemeinschaft korreliert.

Jenseits der Frage nach der Position des Bestattungsplatzes stellt sich
eine weitere nach der inneren Gliederung desselben. Wird ein Areal neu
für diesen Zweck erschlossen, ist es zunächst meist ohne Einrichtungen,
die als kulturelle Bedeutungsträger fungieren. Erst die Nutzung für
Bestattungen führt zu einer langsamen Akkumulation von -- aus
archäologischer Perspektive -- Befunden. Architektur wie Grabanlagen
oder Ritualstellen können über längere Nutzungszeiträume erneuert,
umgebaut oder entfernt werden. Gräber können einzelnen in individuellen
Einrichtungen wie Gruben oder Kisten für sich stehen oder durch
Konstruktionen wie Grabhügel, Kammern in Megalithbauten oder Einhegungen
zu Einheiten zusammengefasst werden. Letztere führen zu einer Gliederung
des Bestattungsplatzes in nach verschiedenen, oft unbekannten Kriterien
zusammengehörige Grabkomplexe. Auch die Anordnung von Einzelgräbern auf
Gräberfeldern ist in der Regel nicht zufällig und wird unter dem
Stichwort der Horizontalstratigraphie archäologisch diskutiert: Durch
das sukzessive Sterben von Mitgliedern einer Siedlungsgemeinschaft
stellt sich aus Sicht der Bestattenden für jeden Toten erneut die Frage
der Platzierung in Relation zu den bereits vorhandenen Gräbern. Häufig
bilden sich in der Verteilung der Gräber die chronologische Entwicklung
des Gräberfelds ab, aber auch andere Kategorien wie biologische und
soziale Gruppengliederung, Alters- und Geschlechtsunterschiede sowie
Unterschiede im Rang der Verstorbenen in einer vergangenen
soziopolitischen Hierarchie können sich hier niederschlagen. Ausdruck
dieser Kategorien sind räumliche Verteilungsmuster der Gräber in denen
Merkmalsvariation von einem Zentrum aus oder entlang Achse nachvollzogen
werden können, merkmalsgleiche Gruppen zu voneinander getrennten
Clustern akkumulieren oder Außreißer mit positiv oder negativ
herausragenden Eigenschaften getrennt von der Hauptgruppe platziert
wurden. Eine weitere Beobachtungsgröße ergibt sich daraus, ob Gräber in
andere Gräber eingreifen und diese stören. Das kann bewusst vermieden
werden, zufällig in Einzelfällen auftreten oder ein Gräberfeld als
Charakteristikum auszeichnen. Auch die Beraubung von Gräbern nach der
Beisetzung kann Teil des Bestattungsrituals sein und die innere
Gliederung eines Bestattungsplatzes sowie die Grabarchitektur
beeinflussen.

Die vorliegende Arbeit konzentriert sich mit den Dichotomien Brandgrab
vs.~Körpergrab sowie Hügelgrab vs.~Flachgrab auf Aspekte von
Bestattungsform und Grabbau. Diese Kategorien dürfen als besonders
bedeutungsgeladen verstanden werden: Sie sind kulturell deutlich
unterschiedlich und ihre Merkmale besitzen meist starken, symbolischen
Aussagewert. Auch innerhalb von Kulturzusammenhängen herrscht große
Variabilität -- ein möglicher Indikator für die Intensität sozialer
Reglementierung des Bestattungsrituals. Das erschwert eine umfassende
Klassifizierung, die in der Lage wäre alle Phänomene aufzunehmen.
Wesentliche Gliederungsgrößen sind Ein- und Mehrphasigkeit sowie
Partielle und Vollständige Bestattung, darüber hinaus ergeben sich aus
der Architektur des Grabbaus sowie der Art der Deponierung des Leichnams
Unterscheidungskriterien\footnote{Hofmann sammelt einige der wichtigsten
  Kategorien in einer tabellarischen Aufstellung:
  \textcite{hofmann_rituelle_2008}, 152.}. Grabanlagen besitzen in der
Regel eine innere, unsichtbare Struktur und einen sichtbaren,
oberirdischen Aufbau. Das Innere des Grabes ist oft nur während der
Errichtung und im Moment der Beisetzung offen und zugänglich. Es
adressiert entsprechend neben dem Toten und angenommener Entitäten der
postmortalen Welt vor allem die Bestattenden und eventuelle Zuschauer
der Bestattungszeremonie. Der dauerhaft sichtbare Teil des Grabes hat
einen potentiell größeren Adressatenkreis und damit oft einen anderen
Symbolgehalt.

Grabformen können Strukturen im sozialen Gefüge der Lebenden
widerspiegeln, indem für Verstorbene aus verschiedenen sozialen Gruppen
jeweils unterschiedliche Grabformen genutzt oder indem besondere
Individuen in von der Norm abweichenden Sondergrabformen beigesetzt
werden. In vielen Gesellschaften werden für Führungspersonen und deren
Verwandten aufwendigere und auffälligere Gräber errichtet während fremde
und soziale Außenseiter gegebenenfalls deutlich einfacher bestattet
werden. Kollektivgräber nivellieren demgegenüber soziale Hierarchien:
Die archäologische Forschung bringt sie häufig mit einem betonten
Gemeinschaftsdenken und egalitären Gesellschaftsformen in Verbindung.
Bestattungsform und Grabbau bieten die Möglichkeit, nicht nur Einblicke
in die soziale Organisation sondern auch die spirituelle
Vorstellungswelt einer archäologischen Kultur zu gewinnen. Zwar sind mit
ein und der selben Religion durchaus unterschiedliche Grabriten
vereinbar und eine plakative Trennung der Vorstellungen, die zum
Beispiel hinter Körper- und Brandbestattungen stehen mögen, ist nicht
haltbar. Bestimmte Rituale, wie etwa eine aufwändige Mumifizierung,
geben allerdings begründeten Anstoß zur Vermutung, die Unversehrtheit
des Körpers spiele in der Jenseitsvorstellung der entsprechenden Kultur
eine entscheidende Rolle. In verschiedenen Kulturzusammenhängen erwecken
Gräber, die als Totenhäuser gestaltet sind oder hausförmige Urnen
enthalten, den Eindruck, die Toten würden in ihren Gräbern wie in
Häusern weiterleben. Die systematische Orientierung des Körpers in
Relation zu den Himmelsrichtungen tritt auf prähistorischen
Gräberfeldern häufig auf und könnte mit einer religiösen Begründung gut
erklärt werden. Der Grabbau kann auch durch die Angst vor dem Toten
beziehungsweise dessen Eingriffe in die Welt der Lebenden bestimmt sein.
Das kann sich dadurch ausdrücken, dass der Leichnam bewusst mit schweren
Steinen bedeckt oder gefesselt wird.

Jenseits von Bestattungsform und Grabbau konzentriert sich die
archäologische Erforschung von Gräbern vor allem auf die
Grabausstattung, also jene Artefakte und Überreste, die bei der
Beisetzung intentionell in das Grab eingebracht wurden. Aufgrund
taphonomischer Gegebenheiten erhalten sich bestimmte Materialkategorien
weniger gut oder besser als andere und sind entsprechend im
archäologischen Befund über- oder unterrepräsentiert. Grabbeigaben
können und müssen hinsichtlich ihrer Bedeutung nach verschiedenen
Kriterien untersucht werden. Bestimmte Objekte wurden nur für den
Bestattungskontext hergestellt, andere dem Materialkreislauf der
Lebenden bewusst entzogen. Artefakte können Teil des Grabbaus sein, zur
persönlichen Ausstattung und Tracht des Verstorbenen gehört haben oder
als sonstige Beigaben in den Grabkontext eingebracht worden sein.
Letztere können beispielsweise als Gebrauchsgegenstände für den Toten in
seiner postmortalen Existenz verstanden werden, als durch den Tod
verunreinigt gelten oder zur Selbstdarstellung der Hinterbliebenen im
Bestattungsritual präsentiert werden. Aus archäologischer Perspektive
ist es oft sehr schwierig, die Motive hinter der Deponierung einer
einzelnen Beigabe zu erschließen. Der potentielle Symbolgehalt von Form,
Farbe und Verzierung der Artefakte steigert bringt weitere
Unsicherheiten mit sich. In der archäologischen Literatur werden
beispielsweise immer wieder einzelne Artefakte als Amulette
angesprochen. Meist handelt es sich um Einzelstücke ohne erkennbaren,
funktionalen Nutzen, die nah am Leichnam platziert wurden. Sie könnten
sowohl Funktionen als Glücksbringer für den Toten als auch als
Bannmittel zum Schutz der Lebenden übernommen haben. Nahrungsbeigaben
sind in rezenten Kulturen oft mit der Vorstellung einer Reise ins
Jenseits verknüpft. Der Verstorbene hat auch nach dem Tod noch Bedarf
nach physischer Nahrung. Diese Assoziation ist allerdings nicht
zwingend: Nahrungsbeigaben können auch schlicht eine weitere
Ausdrucksform für die soziale Identität des Toten sein.

Interpretationsansätze für Grabausstattungen betonen meist den
Aussagewert der Beigabensammlung für die Identität des Bestatteten. In
der Regel werden Unterschiede in der Qualität und Quantität von Beigaben
mit dem vertikalen sozialen Status einer Person oder Gruppe in
Verbindung gebracht. Insbesondere Prestigegüter -- auffallende
Einzelobjekte aus heute als wertvoll erachteten Materialien -- werden in
diesem Kontext betont betrachtet. Beigaben können auch die Zugehörigkeit
unter anderem zu einem sozialen Geschlecht, einer Altersgruppe, einem
Berufszweig oder einer Herkunftsregion ausdrücken. Zuordnungen dieser
Art lassen sich mit physisch-anthropologischen oder
naturwissenschaftlichen Daten korrelieren und so gegebenenfalls
verifizieren. Allerdings kann sowohl eine Person mehrere Identitäten in
sich vereinen als auch ein Artefakt mit mehreren Bedeutungsebenen
verknüpft sein. Die Auszeichung von eindeutigen Leit- oder
Faziesartefakten kann zwar statistisch relevant, im Einzelfall aber auch
irreführend sein und zu Zirkelschlüsse führen\footnote{\textcite{hofmann_rituelle_2008},
  145-165.}.

\hypertarget{eine-cultural-evolution-perspektive-auf-bestattungssitten}{%
\section{Eine Cultural Evolution Perspektive auf
Bestattungssitten}\label{eine-cultural-evolution-perspektive-auf-bestattungssitten}}

Betrachtet man Bestattungssitten als in Raum und Zeit verbreitetes
Kulturverhalten dann kann man das doch nicht gänzlich losgelöst von
seiner besonderen Qualität als mit dem Tod assoziiertem Verhalten tun.
Bestattungssitten sind weder funktional -- obgleich auch aus
hygienischer Perspektive und hinsichtlich Materialkosten und Arbeitszeit
für oder gegen bestimmte Rituale argumentiert werden kann -- noch sind
sie Mode, die leichtfertig und ohne Reflexion übernommen wird. Der Tod
von Angehörigen ist meist ein schwerwiegendes Ereignis, das mit einem
besonderen, kulturellen und individuellen Verarbeitungsprozess
einhergeht. Bestattungssitten gehören eben zu diesem
Verarbeitungsverhalten und sind als solche Gegenstand ihrer Erforschung.

Bergensen 1998, 54ff \& 63 f. Grimes 1998, 131 ff.

\hypertarget{regions-archaeological-overview}{%
\section{Räumliche und zeitliche Trends im Bestattungsritus der
Bronzezeit}\label{regions-archaeological-overview}}

In der europäischen Bronzezeit sind mehrere unterschiedliche
Bestattungstraditionen unterscheidbar, die zeitlich und räumlich
verschiedene Entwicklungen durchlaufen. Dabei können zwei wesentliche
Dimensionen abgegrenzt werden, entlang derer sich fast alle
dokumentierten Grablegungen kategorisieren lassen: 1. Körperbestattungen
im Gegensatz zu Brandbestattungen sowie 2. Flachgräber gegenüber
überhügelten Gräbern. In diesem Spektrum gibt es etliche Varianten
hinsichtlich der Grabanlage- und Vergesellschaftung (z.B. Nachnutzung
neolithischer Megalithanlagen, Gräberfelder, etc.), des Grabbaus (Särge,
Totenhäuser, Bootsgräber etc.) der Beigabenauswahl, der Platzierung des
Leichnams oder des investierten Aufwands für Bestattungszeremonie und
Architektur. Angesichts dieser Variablenvielfalt ist Generalisierung und
die Reduktion des Gesamtzusammenhangs auf die Spannungsfelder Körper-
vs.~Brandbestattung und Flach- vs.~Hügelgrab schwierig. Dennoch soll für
die vorliegende Arbeit diese Perspektive eingenommen werden, da nur zu
diesen primären Variablen Informationen im Radon-B Datensatz (siehe
Kapitel \ref{radonb-dataset}) enthalten sind. Der Datensatz gibt auch
das Forschunsareal und die Abgrenzung künstlicher Regionen vor, die als
Beobachtungseinheiten dienen (siehe Kapitel
\ref{data-prep-and-segmentation}).

Kurz zusammengefasst besagt das klassische Narrativ der Entwicklung
bronzezeitlicher Bestattungssitten folgendes: In der frühen und
mittleren Bronzezeit dominieren Körperbestattungen in verschiedenen
Variationen. Brandgräber kommen in diesem Zeitfenster nur in der
Ungarischen Tiefebene verstärkt vor. Hügelgräber konzentrieren sich auf
Teile des Balkans sowie Ost-, West- und Nordeuropa, während in Zentral-
und Südeuropa Flachgräber -- mehrheitlich Körperbestattungen --
überwiegen. In der mittleren Bronzezeit gewinnt Überhügelung zumindest
in West- und Mitteleuropa an Bedeutung. In der späten Bronzezeit wird
Brandbestattung zum häufigsten Bestattungsbrauch\footnote{\textcite{harding_european_2000},
  75-76. \textbf{Eigentlich Häusler 1977/1994/1996 -\textgreater{}
  ausbauen!}}.

Todo: Schwerpunkt auf Grabform (Brandgrab vs.~Körpergrab) und
Grabkonstruktion. Weitere Aspekte wie Grabbeigaben, Körperhaltung,
Geschlechtsdimorphismen, etc. werden nur dann beachtet, wenn sie direkt
mit diesen Primäraspekten zusammenhängen.

Chronologieüberblick Abbildung \ref{fig:general-chronology}\footnote{Die
  Abbildung basiert auf der Zusammenstellung in
  \textcite{roberts_old_2013}. Dafür wurde auf folgende
  Quellenpublikationen zurückgegriffen:
  \textcite{arnoldussen_bronze_2008}; \textcite{bourgeois_lage_2005};
  \textcite{burgess_bronze_1974}; \textcite{burgess_age_1980};
  \textcite{brindley_dating_2007}; \textcite{eogan_accomplished_1994};
  \textcite{gerloff_reineckes_2007}; \textcite{gerloff_atlantic_2010};
  \textcite{de_laet_belgique_1982};
  \textcite{lanting_14c-chronologie_2001};
  \textcite{louwe_kooijmanns_prehistory_2005};
  \textcite{needham_chronology_1996};
  \textcite{needham_independent_1997}; \textcite{needham_first_2010}}

\begin{landscape}
\begin{figure}
\includegraphics{../neomod_analysis/figures_plots/chronology/bronze_age_europe_chronology} \caption[Chronologiesysteme der Bronzezeit in Zentral-, Nord- und Nordwesteuropa]{Chronologiesysteme der Bronzezeit in Zentral-, Nord- und Nordwesteuropa. Bz = Bronzezeit, Br. = Age du bronze, Per. = Periode, BA = Bronze Age, MA = Metalwork Assemblage. Zusammengestellt aus \textcite{roberts_old_2013}.}\label{fig:general-chronology}
\end{figure}
\end{landscape}

\hypertarget{slowakei-und-ungarn}{%
\subsection{Slowakei und Ungarn}\label{slowakei-und-ungarn}}

Die Slowakei ist vor allem durch die Gebirgsrücken von Ost- und
Westkarpaten geprägt. Im Südosten und Südwesten davon öffnet sich das
Land nach Ungarn zum Karpatenbecken mit Kleiner und Großer Ungarischer
Tiefebene. Die ausgedehnte, flache Landschaft wird und wurde von den
Flüssen Donau und Theiß sowie deren Nebenflüssen dominiert, die das
Becken in Nord-Süd-Richtung durchziehen. Vor ihrer Begradigung im 19.
Jahrhundert mäandrierten beide stark und ihr Verlauf änderte sich häufig
und in kurzen Intervallen. Neben ihrer geomorphologisch prägenden
Wirkung stellten sie wichtige Kommunikationswege und kulturelle Grenzen
dar: Im Karpatenbecken begneten West- und Mitteleuropa den
Steppengebieten, dem Ostbalkan und dem östlichen Mittelmeergebiet. Für
die bronzezeitliche kulturgeographie ist es sinnvoll, von West nach Ost
in drei Areale zu untergliedern: Der Donauraum (Transdanubien), das
Zwischenstromland zwischen Donau und Theiß und die Theißregion. Aufgrund
der großen Heterogenität und inneren Komplexität der Entwicklungen in
diesen Regionen, wird der Schwerpunkt der folgenden Ausführungen auf dem
Westen Ungarns und der Slowakei liegen, also jenen Gebieten, die auch im
Untersuchungsareal dieser Arbeit liegen. Die slowakische,
bronzezeitliche Chronologie folgt einem modifizierten Schema nach
Reinecke, in Ungarn dagegen ist eine an der Tellsiedlung Toszeg
erarbeitete Chronologie verbreitet. Eine Aufteilung in Früh-, Mittel-,
Spät und Endbronzezeit ist jedoch in beiden Kontexten gebräuchlich und
kann Grenzenüberschreitend zur Anwendung gebracht werden\footnote{\textcite{markova_slovakia_2013},
  813-814.}.

Ungarn und die Slowakei waren zu Beginn der Frühbronzezeit, die nach
Definition von 2500/2300 bis 1550/1450calBC dauerte, kulturell vielfältg
und territorial fragmentiert. Im Laufe dieser Phase fand eine langsame
Konsolidierung von Gruppen statt, die mit Bevölkerungswachstum,
zunehmender Metallverarbeitung und steigender Siedlungsdichte und
-kontinuität einherging. Etliche Siedlungen haben sich als bis heute in
der Landschaft sichtbare Tellhügel erhalten. Westlich der Donau entstand
am Beginn der Frühbronzezeit aus spätneolithischen Substrat der
Makó-Kosihy-Čaka Komplex. In diesem Kontext waren isolierte
Brandbestattungen mit oder ohne Urne üblich, später auch gelegentlich
Körperbestattungen. Die Region wurde von verschiedenen Phänomenen in
angrenzenden Räumen beeinflusst, besonders von der Somogyvár-Vinkovci
Kultur aus dem Nord- und Nordwestbalkan. In Süd- und
Nordwesttransdanubien waren Körperbestattungen aus dieser Tradition
heraus wesentlich häufiger als Urnenbestattungen. Grabhügel und
Flachgräber traten nebeneinander auf. Im Süden waren reiche Grabbeigaben
üblich, sie werden seltener, sucht man weiter im Norden. Von Westen
kommend über Südmähren und die westlichen Ausläufer der Karpatevorland
in der Slowakei bis ins Areal des heutigen Budapest trat eine lokale
Variante -- die Csepel Gruppe -- der Glockenbecherkultur auf. Im
Gegensatz zu zu anderen Glockenbecherkontexten waren Körperbestattungen
hier zugunsten von Brandbestattungen eher selten. In der Südwestslowakei
siedelten außerdem Vertreter einer Variante der frühen Schnurkeramik --
die Chłopice-Veselé Kultur. Das ist Ausdruck einer weiteren Verbindung
nach Mähren und bis nach Kleinpolen. Flache Körpergräber waren die Regel
in diesem Kontext. Mit dem Fortschreiten der Frühbronzezeit ab Bz A1
entstand in der Südwestslowakei aus Chłopice-Veselé Kultur,
Glockenbecherkultur und Makó-Kosihy-Čaka Kultur die Nitra Kultur. Die
übliche Bestattungsform in diesem Zusammenhang waren nun
Ost-West-orientierte Körpergräber in Hockerlage. Grabaufbauten wie
Totenhäuser geben Indizien über den soziale Rang der Bestatteten. Schon
in der Materiellen Kultur des Nitra Kontext finden sich starke Einflüsse
aus dem Aunjetitzer Kulturraum und ab dem Ende von Bz A1 ist es
angemessen, die Kulturerscheinungen in der Südwestslowakei als lokale
Ausprägung der Aunjetitzer Kultur zu beschreiben, obgleich besonders
hinsichtlich der Bestattungspraktiken Nitratraditionen fortbestanden.
Sozialen Unterschiede in Beigabenmenge und Grabkonstruktion
manifestierten sich hier nicht so stark wie im Kernraum der Aunjetitzer
Kultur. Auch drang die Aunjetitzer Kultur nicht weiter nach Süden nach
Transdanubien vor. Am Übergang von Bz A2 zu Bz B1 entwickelte sie sich
stattdessen lokal weiter zur Maďarovce Kultur, der östlichen Ausprägung
des österreichischen und mährischen Maďarovce-Věteřov-Böheimkirchen
Komplexes. Im Maďarovce Kreis wurden zunächst vor allem flache
Körpergräber angelegt, später mehr Hügelbestattungen und Brandgräber. Am
Übergang zur Mittelbronzezeit in BZ B1 löste sich das Kulturphänomen
langsam auf. Der westliche Zugang zum Karpatenbecken war zeitweise durch
die frühbronzezeitlichen Entwicklungen im Nordosten Österreichs geprägt,
wo sich mit der Leithaprodersdorf Kultur und der Wieselburg gut
abgrenzbare Einheiten ausbreiteten. Weiter südlich in Ungarn wurde die
Makó-Kosihy-Čaka Kultur noch in Bz A von der Kisapostag Kultur abgelöst.
Hier wurden Brandbestattungen in Urnen oder mit einer Verstreuung des
Leichenbrands praktiziert. Körperbestattungen treten zwar auch auf, sind
aber selten. Auf die Kisapostag Kultur folgte die Kultur Inkrustierter
Keramik (Transdanubian Encrusted Pottery oder North Pannonian culture)
mit Brandbestattungen in einfachen Gruben und mit Urnen.
Körperbestattungen traten sporadisch auf. Westlich davon, zwischen Donau
und Theiß, entstand am Beginn der Bronzezeit in bz A0 un Bz A1 die
Nagyrév Kultur. Die übliche Form der Bestattung waren auch in diesem
Kontext Brandbestattungen, verstreut oder in Urnen. Körperbestattungen
treten nur vereinzelt in der Anfangsphase auf. Vor dem Hintergrund von
Nagyrév und Kisapostag Kultur entstand hier im weiteren Verlauf der
Frühbronzezeit die Vatya Kultur -- mit einem sehr umfangreichen und
fortgeschrittenen Metallinventar. Der Bestattungsritus dieser Gruppe war
stark normiert und bestand aus strukturiert auf Gräberfeldern
angeordneten Brandbestattungen mit stark differenzierender
Beigabenausstattung\footnote{\textcite{markova_slovakia_2013}, 814-821.}.

Die Mittelbronzezeit in der Slowakei und in Ungarn ist eine
Vergleichsweise kurze Phase von 1500/1450 bis 1200/1150calBC (Bz B1 bis
Bz C). Zur Frage, ob der Beginn des Übergangs von Früh- zu
Mittelbronzezeit aus Westen oder Osten angestoßen wurde, herrscht in der
Fachdiskussion keine einheitliche Meinung. Als Resultat setzte sich
jedoch im gesamten Raum die Grabhügelkultur durch. Ihre lokale
Ausprägung -- im Westen Ungarns und der Slowakei die
Mitteldonauländische Grabhügelkultur -- war sehr ähnlich zu jener in
anderen Teilen Zentraleuropas. Grabhügel fanden in diesem Zeitraum weite
Verbreitung, die bisherigen, lokalen Präferenzen für Körper- oder
Brandbestattung wurden allerdings beibehalten und mit der Veränderung im
Grabaufbau kombiniert. Die Bestattungen zeigen allgemein einen
Geschlechtsdimorphismus und konnten auf Erdbodennivau platziert und dann
überhügelt, oder in eine Grabgrube unter dem Hügel eingebracht werden.
Am Übergang zur Spätbronzezeit, in Bz D, wurde die Grabhügelkultur durch
die Urnenfelderkultur -- konkret die Mitteldonauländische
Urnenfelderkultur -- abgelöst. Östlich der Donau zeichnete sich die
Grabhügelkultur durch größere räumliche und zeitliche Heterogenität aus.
Im Norden, in der Slowakei, lässt sich die Entwicklung in der
Mittelbronzezeit als Abfolge ausgehend von der Dolný Peter Phase, ein
Übergangshorizont zwischen Maďarovce und Grabhügelkultur, und dann der
Frühen, Klassischen und Späten Grabhügelkultur beschreiben. Südlich
davon gelten die Tápé, die Egyek und die Hajdúbagos Gruppe als Vertreter
der Grabhügelkultur im Karpathenbecken. Von Region zu Region
unterschiedlich wurden Körper- oder Brandbestattung praktiziert. Auch
die Errichtung von Grabhügeln war nicht obligatorisch: Mehrfach treten
Flachgräberfelder auf Westlich der Theiß spielte der Einfluss der
Grabhügelkultur eine untergeordnete Rolle\footnote{\textcite{markova_slovakia_2013},
  825-827.}.

Die kulturhistorische Entwicklung der Spätbronzezeit in Slowakei und
Ungarn lässt sich in drei Großregionen gliedern, die von den zur
Beschreibung von Früh- und Mittelbronzezeit gewählten abweichen: Die
Donauregion im Westen bleibt eine relevante Beobachtungsgröße, nun
lassen sich jedoch die Areale östlich der Donau zusammenfassen.
Stattdessen muss den Gebirgstälern im Norden der Slowakei mehr
Aufmerksamkeit geschenkt werden -- sie waren in der Spätbronzezeit
dichter besiedelt und durchliefen eine eigenständige Entwicklung. Im
gesamten nördlichen Karpatengebiet dominierte die Urnenfelderkultur, die
dem Areal mehr Zentralisierung, Weiterentwicklungen in der
Bronzemetallurgie und eine Intensivierung des Fernhandels brachte --
also Entwicklungen der Mittelbronzezeit fortsetzte. Die übliche
Bestattungssitte der Urnenfelderzeit war die namensgebende Beisetzung
von Leichenbrand in Urnenen auf oft ausgedehnten Gräberfeldern. Die
einzelnen Gräber konnten als Flachgräber ausgeführt, leicht überhügelt,
in bestehende Grabmonumente und Hügel eingebracht oder selbst monumental
Überhügelt werden. Generell wurde diese Bestatuungsform sowohl für
Männer als auch Frauen zur Anwendung gebracht. Die Verbrennung schloss
das Einbringen von Grabbeigaben nicht aus. Tatsächlich sind aus Slowakei
und Ungarn aus der Spätbronzezeit herausragende Hügelbestattungen von
Männern und Frauen bekannt. Im Zeitraum Bz D bis Ha A1 traten in der
Südwestslowakei zwei Varianten der Mitteldonauländischen
Urnenfelderkultur auf: Die Velatice Kultur, die aus der
Mitteldonauländischen Grabhügelkultur westlich der Donau hervorging, und
die Čaka Kultur, die die Lokalformen der Grabhügelkultur des
Karpatenbeckens östlich der Donau ablöste. Die Waag bildet die Grenze
zwischen diesen Phänomenen. In Transdanubien können ebenfalls mehrere
Subgruppen unterschieden werden: Eine nordwestliche Gruppe mit Paralleln
in den Ostalpen, eine südliche und eine nordöstliche Gruppe. Die
Velatice Kultur endet in Ha A2, lebt in Einflüssen in der Lausitzkultur
weiter nördlich jedoch noch einige Zeit weiter. Jene hatte sich am
Beginn der Spätbronzezeit bis in die Höhenlagen im Nordwesten der
Slowakei ausgebreitet und bildete damit die südliche Grenze des
Nördlichen Urnenfelderkreises. Ihr Subsistenzmodell war an die trockenen
Gebirgsregionen angepasst und sie überdauerte hier in einzelnen Enklaven
bis in die fortgeschrittenen Eisenzeit. Die typische Bestattungsform in
diesem Kontext waren flache Urnengräber. Nichtsdestoweniger wurden
vereinzelt ach Hügelgräber angelegt, manchmal mit mehreren Bestattungen.
Die Hügel durchliefen dabei eine architektonische Entwicklung von
Steineinfassungen zu vollständiger Steinbedeckung. Östlich der Donau, im
Nordosten Ungarns und damit außerhalb des Untersuchungsareals dieser
Arbeit, traten im Kontext der Piliny Kultur ab Bz B1 die frühesten
Urnenfelder Zentraleuropas auf\footnote{\textcite{markova_slovakia_2013},
  827-833.}.

\hypertarget{osterreich-und-tschechische-republik}{%
\subsection{Österreich und Tschechische
Republik}\label{osterreich-und-tschechische-republik}}

Die Bronzezeit in Österreich und Tschechien lässt sich in vier
wesentliche Perioden gliedern: Früh-, Mittel-, Spät- und Jungbronzezeit.
In der Frühbronzezeit entwickelte sich nördlich der Donau in Österreich
unter Einfluss der Ungarischen Nagyrév Kultur die Proto-Aunjetitz Kultur
parallel zum Aunjetitz Kreis in Böhmen und Mähren. Im Aunjetitz Areal
folgen die Bestattungssitten einem relativ festen Regelwerk: Üblich
waren Einzel- und Körpergräber. Einzelne Funde von Massengräbern im
Siedlungskontext, die sich auch in der Tumulus und und der
Urnenfelderkultur fortsetzen, dürfen nicht als reguläre
Bestattungskontexte verstanden werden. In Nordbömen, Mähren und den
anschließenden Teilen von Österreich dominieren Flachgräber, während in
Süd- und Westböhmen Hügelgräber häufiger aufreten. Die Flachgräber
gingen den Grabhügeln zeitlich voran und zeichnen sich oft durch eine
gemauerte Grabkiste oder einen Holzsarg aus. In mehreren Fällen konnte
der Nachweiß für eine hölzerne Konstruktion über dem Flachgrab erbracht
werden. Die Bestattungen in den Hügeln sind meist in den Urhumus
eingetieft und durch eine Steinpackung geschützt, die damit
gleichermaßen den Kern der Hügelaufschüttung bildet. Im Aunjetitzer
Kontext dominieren linke Hocker in Süd-Nord Orientierung ohne
Geschlechtsdimorphismus. Obgleich eine Mehrzahl der Gräber
wahrscheinlich zum Ende der Frühbronzeit beraubt wurden, lässt sich ihr
Beigabeninventar rekonstruieren: Neben Gewandelementen und -- in einigen
wenigen Kontexten -- Waffen und Prestigegegenstände aus Bronze, Gold und
Bernstein überwiegen Keramikgefäße. Dabei lässt sich eine diachrone
Entwicklung von größeren zu sehr kleinen, teilweise miniaturisierten
Beigabengefäßen beobachten\footnote{\textcite{lubos_czech_2013}, 789 \&
  794-796}.

Zeitgleich mit der Aunjetitzer Kultur begegnen sich südlich der Donau in
der österreichischen Frühbronzezeit mehrere Lokalgruppen. Die
Leithaprodersdorf Gruppe, die noch in der Frühbronzezeit von der
Wieselburg Kultur abgelöst wurde, findet sich östlich des Wienerwald. Im
Kontext dieser aufeinander folgenden Gruppen sind flache
Körperbestattungen üblich. Die Körperhaltung und Orientierung folgt
einem klaren Geschlechtsdimorphismus und die Qualität und Quantität der
Beigaben ist betont ungleich. Auch hier sind Steinkisten und Baumsärge
ein wichtiger Teil der Grabkonstruktion. Südlich von Wien und westlich
des Wienerwalds lässt sich die Unterwölbing Kultur verorten. Auch hier
sind flache Körperbestattungen die Regel. Die Gräber sind stark
standardisiert, zeigen einen deutlichen Geschlechtsdimorphismus und sind
als mit gesetzten Steinkisten und Baumstammsärgen aufgebaut. Sie sind
meist Teil siedlungsnaher Gräberfelder und vergleichsweise reich mit
Keramik sowie Bronzewaffen und -schmuck ausgestattet. Schwere Halsringe
sind charakteristisch für diesen Kulturzusammenhang. Im westlichen Teil
Niederösterreichs bis nach Tirol findet sich die Straubing Kultur, deren
Verbreitungsschwerpunkt in Bayern liegt. Aus Österreich sind trotz
ausgeprägter Besiedlung keine Bestattungsplätze der Straubing Kultur
bekannt, in Bayern verhält es sich allerdings wie im Unterwölbing Raum.
Auch aus den Alpengebieten sind zu wenige Gräber erforscht, um eine
zuverlässige Aussage über die vorherrschenden Bestattungsbräuche treffen
zu können, jedoch deutet sich für die inneren Alpen ein früher Wandel
hin zu Brandbestattungen an\footnote{\textcite{lubos_czech_2013}, 789 \&
  796-797}.

Am Ende der Frühbronzezeit entstand in Böhmen und Mähren unter starkem
Einfluss aus Südosteuropa die Věteřov Kultur. In Österreich wurde die
Unterwölbing Kultur durch die Böheimkirchen-Gruppe abgelöst. Im Osten
bestand die Wieselburg Kultur parallel zur neue geformten Drassburg
Gruppe weiter. Im Salzkammergut konstituierte sich die Attersee Gruppe.
Im Laufe der Mittelbronzezeit wurden die lokalen Phänomene im Süden
Mährens, fast ganz Böhmen und in Ostösterreich durch die
Mitteldonauländische Tumuluskultur homogenisiert. In dieser Konsequenz
ist die geradezu universelle Bestattungsform in der Mittelbronzezeit
Tschechien und Österreichs das Hügelgrab. Die Hügel sind einfache
Erdhügel auf einem Steinkreisfundament. In den Hügeln wurden -- oft
mehrere -- sowohl Körper- als auch Brandbestatttungen untergebracht,
wobei erstere langsam als dominante Form von letzteren abgelöst werden.
Bei Körperbestattungen sind die Beigaben um den Körper verteilt, wobei
Keramik entweder am Fuß- oder Kopfende der Grube platziert wurde. Im
Kontext der Brandbestattungen wurde die Ausstattung nicht mit verbrannt,
sondern vor der Überhügelung auf dem Leichenbrand deponiert. Ein
Geschlechtsdimorphismus zeigt sich mitunter nicht nur bei der
Beigabenauswahl, sondern auch bei der Grabform: Im Fall der
Grabhügelanlage von Pitten in Niederösterreich überwiegt Brandbestattung
für weibliche Individuen während Männer überwiegend unverbrannt
beigesetzt wurden. Im Westen Österreichs zeichnet sich das Tumulus
Phänomen durch mehr Bezüge zu Süddeutschland aus. Im Salzburger Land
wurden die Bestattungen als Rückenstrecker in Grabhügeln untergebracht.
Brandbestattungen in einfachen, steinbedeckten oder leicht überhügelten
Gruben enthalten die Überreste verbrannter Metall- und Keramikartefakte
in Urnen aus Keramik oder organischem Material. Die wenigen Funde aus
dem inneren Alpengebiet deuten auf beigabenlose Brandbestattungen
hin\footnote{\textcite{lubos_czech_2013}, 790 \& 797-798}.

In der Spät- und Jungbronzezeit wurden Böhmen, Mähren und Österreich
relativ homogen Teil des Urnenfelder Kulturkomplexes. Böhmen ist
überwiegend im Einflussgebiet der Urnenfeldergruppen aus dem Oberen
Donauraum, Siedlungen der Lausitzer Kultur in Norden und Osten Böhmens
lassen sich allerdings besser aus der Perspektive der Nördlichen
Urnenfeldergruppen verstehen und im äußersten Westen besteht mit der
Cheb Urnenfeldergruppe eine kulturelle Verbindung ins Areal des heutigen
Deutschland. Ebenso lassen sich auch Österreich und Mähren in
verschiedene kleinere Sphären gliedern, die als verschiedentlich
beeinflusste Varianten des Urnenfelderphänomens beschrieben werden
können. In all diesen Kontexten folgt die allgemeine Bestattungsform --
mitunter in Nutzungskontinuität der Mittelbronzezeitlichen
Bestattungsanlagen -- der stark vereinheitlichen Urnenfelderpraxis:
Ausgedehnte Felder von flachen Brandgräbern. In einer Mehrzahl der Fälle
ist der Leichenbrand in einer Keramikurne eingelagert, wobei in einer
Grabgrube durchaus mehrere Urnen oder sonstige beigefügte Gefäße
deponiert sein können. Urnen können neben menschlichen Überresten auch
verbrannte Tierknochen und Bronzeartefakte enthalten. Die persönliche
Tracht und Schmuck wie Nadeln oder Armreife wurde in der Regel mit
verbrannt, während Gebrauchsgegenstände wie Messer erst nach der
Verbrennung beigefügt wurden. Trotz der insgesamt großen Homogenität der
Urnenfelderkulturen hinsichtlich ihres Bestattungsbrauches zeigen die
kulturellen Subgruppen im Detail durchaus Abweichungen voneinander
hinsichtlich der Anordnung von Urne und Beigaben im Grab und der
Beigabenauswahl, die auf weiterreichende Unterschiede in Ideologie und
Sozialstruktur schließen lassen. Zudem wurden in verschiedenen Regionen
weiterhin vereinzelt Körpergräber angelegt oder Bestattungen in
Grabhügel eingebracht. In der Frühen und Mittleren Urnenfelderzeit war
die Beisetzung von Urnen und Beigaben in körpergroßen Steinkisten
deutlich standardisiert. In der Späten Urnenfelderzeit verlieren
Steinsetzungen an Bedeutung. Gleichzeitig nimmt die Beigabenmenge ab --
insbesondere Waffen werden nicht mehr beigegeben. In den inneren
Alpenregionen setzen sich Urnenfelder bis weit in die Frühe Eisenzeit
fort\footnote{\textcite{lubos_czech_2013}, 790 \& 798}.

\hypertarget{polen}{%
\subsection{Polen}\label{polen}}

Polen kann entlang seiner Nord-Süd Achse naturräumlich in drei Bereiche
gegliedert werden: Ein 400-500km breiter Streifen flachen, seen- und
feuchtgebietreichen Landes an der Ostseeküste, südich davon Hochland und
das Heiligenkreuz Mittelgebirge, an der Südgrenze ein langer Gebirgszug,
der sich von West nach Ost aus Erzgebirge, Sudeten und Karpathen
zusammensetzt. Die größten Flüsse Polens sind Oder und Weichsel. Beide
erstrecken sich über weite Teile Polens, entwässern in die Ostsee und
stellten in der Vorgeschichte wichtige Verkehrswege dar. Aus
archäologisch-kulturhistorischer Perspektive hinsichtlich der Bronzezeit
zwischen 2300/2200-800calBC bietet sich eine andere Dreiteilung in
West-, Nordost und Südost Polen an: Eine westliche Zone (Woiwodschaften
Pommern, Kajuwien-Pommern, Westpommern, Großpolen, Lebus,
Niederschlesien, Lodsch, Oppeln, Schlesien) war nach Osten durch eine
Grenzlinie zwischen Danziger Bucht im Norden und dem Durchbruch zwischen
Sudeten und Karpathen, der Mährischen Pforte, im Süden definiert. Der
Bereich östlich dieser Linie war wiederrum zweigeteilt in einen
nördlichen (Ermland-Masuren, Podlachien, Masowien) und einen südlichen
(Lublin, Heiligkreuz, Kleinpolen, Karpatenvorland) Teil, getrennt am
Ost-West orientierten Übergangsbereichs von Niederrungs- zu Hochland.
Die westliche Zone spielte in der zentraleuropäischen Bronzezeit eine
entscheidende Rolle, da sie mehrfach die Rolle der östlichen oder
nordöstlichen Grenzregion von wichtigen Kulturphänomenen übernahm. Das
betrifft etwa die Aunjetitzer Kultur, die Hügelgräber Kultur und,
später, die Hallstattkultur. Die Hochland-Regionen von Südostpolen
gehörten weitestgehend zum nördlichen Bezugsbereich der Kulturen des
Karpathenbeckens. Nordostpolen wich in seiner Entwicklung deutlich vom
Rest Polens ab. Eine Sonderrolle nahm nicht zuletzt wegen des Reichtums
an Bernstein auch der Küstenstreifen zwischen Oder- und Weichselmündung
ein. Diese Region war traditionell Teil eines Austauschnetzwerks, dass
das Baltikum überspannte und bis in die Nordsee reichte\footnote{\textcite{czebreszuk_bronze_2013},
  767-770.}.

Die früheste Phase der polnischen Bronzezeit dauerte von 2300/2200 bis
2000calBC. In West- und Südostpolen ist diese Proto-Bronzezeit mit der
Glockenbecher Kultur verknüpft. Kuyavien und Pommern gehörten dabei zu
einem Glockenbecher Kreis aus Südskandinavien und Nordostdeutschland,
Niederschlesien war aus Böhmen inspiriert, Kleinpolen im Südosten des
modernen Polens aus Mähren. Aus diesem Glockenbecher Substrat entstand
in Westpolen ab 2300calBC die Aunjetitzer Kultur. Im Südosten Polens
formte sich die Mierzanowice Kultur. Die zunächst enge Verbindung
zwischen beiden Phänomenen löste sich um 2000calBC auf -- Weichsel und
obere Oder wurden zur Kulturgrenze. Während die Aunjetitzer Kultur
hinsichtlich ihrer materiellen Kultur ein klares Profil ausbildete,
hochentwickelte Metallverarbeitung hervorbrachte und an herausragenden
Grabhügeln erkennbare, soziale Differenzierung katalysierte, stagnierte
und zerfaserte die Mierzanowice Kultur in Lokalgruppen. Nordostpolen
blieb lange in einer spätneolithischen und wildbeuterischen Tradition
verhaftet, obgleich Keramikfunde Verbindungen nach Westpolen nahelegen.
Nach 2000calBC begann sich in Kuyawien und Großpolen mit der Trzciniec
Kultur eine neue Größe herauszubilden, die große Teile Nordostpolens und
-- um 1650/1600calBC -- auch Kleinpolens erfasste\footnote{\textcite{czebreszuk_bronze_2013},
  770-772.}.

Die frühe Bronzezeit in Polen war von Körpergräbern in sehr großen
Hügelgräbern auf Geländeerhebungen und Hügelgräberfeldern mit bis zu 60
einzelnen Hügeln dominiert. Flachgräber waren in dieser Zeit erheblich
seltener. Erst am Übergang zur Lausitzer Kultur ab der Mittel- und
Spätbronzezeit setzen diese sich durch. Im Südosten Polens, im Kontext
der Mierzanowice Kultur, wurden die Toten in West-Ost Orientierung und
nach Süden blickend angehockt auf die Seite gelegt. Frauen wurden mit
dem Kopf nach Osten, Männer mit dem Kopf nach Westen bestattet. Im
Südwesten drückte sich das Geschlecht nicht so offensichtlich in der
Bestattungspraxis aus: Die Körper sind Nord-Süd orientiert, der Kopf im
Süden, angehockt, das Gesicht nach Osten gerichtet. Im Nordwesten, und
damit im Aunjetitzer Kulturkreis, scheint das Recht auf Bestattung einer
sozialen Elite vorbehalten zu sein, die in großen Grabhügeln beigesetzt
wurde. Aus diesem Kontext sind entsprechend erheblich weniger
Bestattungen bekannt, es deutet sich aber an, dass die Leichname
üblicherweise Ost-West orientiert, mit dem Kopf nach Westen und mit
Blick nach Süden angeordnet wurden\footnote{\textcite{dabrowski_aeltere_2004},
  73 \& 80-81.; \textcite{czebreszuk_bronze_2013}, 775.}.

Hügelgräber der Frühbronzezeit sind aus fast ganz Polen bekannt, sie
fehlen nur Nordostpolen (Masowien und Podlasien). Ihr Durchmesser
beträgt heute 10 bis 26m, wobei dieser Wert angesichts Jahrhunderte
währender Erosion und landwirtschaftlicher Landnutzung nach unten
korrigiert werden muss: die Mehrzahl der erhaltenen Hügel ist heute
meist nicht mehr als einen Meter hoch. Manche Hügel sind von einem
breiten, mehrschichtigen Steinkranz eingefasst, der darauf hindeutet,
dass sie ursprünglich von einer nunmehr zerstörten Steinschicht bedeckt
waren. Die notwendige Erde wurde aus der unmittelbaren Umfassung der
Aufschüttunge entnommen, wodurch teilweise bis heute sichtbare Gräben
rund um die Hügel eingetieft wurden. In der Hügelaufschüttung finden
sich häufig ein reiches Artefaktinventar sowie Holz- und
Steinkonstruktionen. Drei Hauptbauarten lassen sich unterscheiden:
Einfache Erdhügel mit 1 bis 2 Körper- oder Brandbestattungen, die in den
Urhumus eingegraben oder schlicht darauf gelegt und anschließend
überhügelt wurden, Erdhügel mit einer ausgeprägten Brandschicht, die
auch verbrannte Knochen und Inventar enthält sowie Hügelgräber mit
Steinschutzkonstruktionen. Die Konstruktionen varieren deutlich zwischen
gemauerten Grabkammern mit den Überresten mehrerer Körperbestattungen,
Steinpflastern und Ringen am Boden der Hügel oder ovalen, kreisförmigen
oder rechteckigen Steinabdeckungen, die ein oder mehrere Brand- oder
Körpergräber im Hügelvolumen bedecken. In mehreren Grabkammern deutet
eine chaotische Lage von Knochen und die Anhäufung von Schädeln auf
Mehrfachbeisetzungen und ein komplexes Totenritual hin\footnote{\textcite{dabrowski_aeltere_2004},
  73-77.}.

Flachgräber traten in der Frühbronzezeit ebenfalls in ganz Polen auf. In
Zentral- und Nordostpolen (Masowien, Podlachien, Lodsch) waren sie
jedoch die ausdrücklich vorherrschende Bestattungsform. Sie wurden
überwiegend als Körper-, jedoch auch als Brandgräber ausgeführt. Die
Körperbestattungen wurden teilweise in Särgen abgelegt oder in
Leichentücher eingeschlagen. Auffallend sind Einzel- und
Mehrfachbestattungen, die sowohl Körper- als auch Brandgräber oder sogar
Mischformen mit teilweise angebrannten Skeletten enthalten. In
Brandgräbern wurde der Leichenbrand entweder mit oder ohne Urne,
manchmal in Särgen und sporadisch in anatomischer Lage deponiert. Die
urnenlosen, flachen Brandgräber stimmen manchmal hinsichtlich Ausmaßen
und Orientierung mit den Körpergräbern überein. In großen Brandgräbern
wurden in mehreren Fällen viele -- in einem Fall bis zu 18 -- Individuen
untergebracht. Wie bei Hügelgräbern treten auch bei Flachgräbern
Steinkonstruktionen in Form von Kisten und Pflastern auf, wobei
gelegentlich der Eindruck entsteht, die Steinsetzungen seien bewusst im
Sinne eins Musters oder Symbols ausgelegt worden\footnote{\textcite{dabrowski_aeltere_2004},
  77-80.}.

Der Übergang zur Mittelbronzezeit war in Polen durch neue Einflüsse aus
der Nordischen Bronzezeit ab 1700calBC im Nordwesten und der Hügelgräber
Kultur nach 1600calBC im Südwesten und Südosten geprägt. Trotz der
Unterschiede zwischen diesen Kontexten scheinen sie doch einer
gemeinsamen kulturellen Sphäre zuzugehören. In beiden waren Hügelgräber
-- oft Steinhügel -- und Metallhorte wichtige Kulturphänomene. Ostpolen
war weiterhin von Vertretern der in sich heterogenen Trzciniec Kultur
besiedelt. Manche Bestattungsplätze waren auch über die Transformation
von Mittel zu Spätbronzezeit hinweg kontinuierlich belegt -- in ältere
Grabhügel wurden häufig Nachbestattungen eingebracht. Hügel- und
Flachgräber weisen insgesamt weitreichende, strukturelle Ähnlichkeiten
auf und konnten beide als Einzel- oder Kollektivgräber ausgeführt sein.
In Fortsetzung der Traditionen aus Schnurkeramik und Mierzanowice Kultur
wurde auch hier vor allem in Hügelgräbern bestattet, daneben bestand
allerdings eine Vielfalt unterschiedlicher Phänomene, die die
Heterogenität dieses Kulturraumes widerspiegeln\footnote{\textcite{dabrowski_aeltere_2004},
  73 \& 80-81.; \textcite{czebreszuk_bronze_2013}, 772 \& 775.}.

Die Spätbronzezeit in Polen war von der Lausitzer Kultur und deren
Expansion dominiert. Die Lausitzer Kultur ist die nordöstliche
Ausprägung der Urnenfelder Kultur. Sie lässt sich nach 1400calBC in
Schlesien und Großpolen erstmals archäologisch fassen und brachte
langanhaltende Stabilierung der Siedlungsaktivitäten in großen Teilen
Polens mit sich. Sie dauerte bis 400calBC, ab 800calBC freilich stark
von der Hallstatt Kultur beeinflusst. Kleinpolen geriet ab 1300calBC in
den Einfluss der Lausitzer Kultur, dabei scheint Migration von Siedlern
aus Schlesien eine wichtige Rolle gespielt zu haben. Nordostpolen
beschritt auch hier einen Sonderweg: Die Veränderung durch die Lausitzer
Kultur ist schwieriger fassbar. Die übliche Bestattungsform in der
Spätbronzezeit ist auch in Polen damit die Brandbestattung in Urnen auf
großen Gräberfeldern. Im Einzugsgebiet des San in Südostpolen trat die
Tarnobrzeg Gruppe auf, die unter Einflüssen aus Steppenraum und
Karpathenbecken, eine distinkte kulturelle Sphäre bildete\footnote{\textcite{czebreszuk_bronze_2013},
  772-773 \& 775-776.}.

\hypertarget{suddeutschland}{%
\subsection{Süddeutschland}\label{suddeutschland}}

Die kulturhistorische Entwicklung Deutschlands in der Bronzezeit ist
komplex und erlaubt die Unterscheidung etlicher Gruppen, Stile und
Kulturkomplexe. Wesentlich zum Verständnis sind seine geographische
Gliederung und intensive Interdependenzen mit angrenzenden Phänomenen,
die sich als Ergebnis seiner Lage in Zentraleuropa zu allen
Himmelsrichtungen ergeben. Geomorphologisch kann Deutschland von Süd
nach Nord grob in Folgende Regionen untergliedert werden: Die
(Bayrischen) Alpen und zugehörige Vorgebirgszonen, die süddeutsche
Schichtstufenlandschaft, die Mittelgebirge und schließlich das
norddeutsche Flachland mit Küsten und Inseln in Nord- und Ostsee. Die
großen Flusssysteme von Donau, Rhein, Weser, Elbe und Oder stellen
wichtige Kommunikationskanäle dar, die sich im Verlauf der gesamten
Vorgeschichte als Verbindungen und Grenzen in verschiedenen
Austauschsystemen verhalten haben. Süddeutschland stand nach Osten in
unmittelbarem Kontakt zu Regionen in den heutigen Grenzen von Böhmen,
Mähren, Österreich und Ungarn. Nach Süden bestand Kontakt mit den
Alpenregionen der heutigen Schweiz und Norditaliens, nach Osten mit
Frankreich. Die Entwicklungen in Norddeutschland lassen sich am besten
über seine Verbindungen zum Benelux Raum und der Nordischen Bronzezeit
in Dänemark und Südschweden verstehen. Ostdeutschland bildete mit Polen
eine Sphäre intensiver Interaktion. Das gebräuchliche chronologische
System im Süden und bis in zu den Mittelgebirgen ist die für ganz
Zentraleuropa relevante Phasengliederung nach Reinecke (s.o.), während
in Norddeutschland die Periodenunterteilung der Nordischen Bronzezeit
nach Montelius (s.u.) zur Anwendung kommt. Eine übergeordnete
Dreigliederung in früh, mittel und spät ergibt sich aus einer
vereinfachten Betrachtung der Bestattungssitten: Frühbronzezeit meint
einen Zeitraum vom Ende des 3. Jahrtausends bis 1600calBC in dem flache
Hockergräber überwiegen, Mittelbronzezeit das Fenter 1600-1300calBC mit
Körperbestattungen in Grabhügeln und Spätbronzezeit, die
Urnenfelderzeit, den Zeitraum 1300-800calBC\footnote{\textcite{jockenhovel_germany_2013},
  723-725.}.

In der folgenden Zusammenstellung wird -- entsprechend der in dieser
Arbeit vorgenommenen Regionengliederung (siehe Kapitel
\ref{data-prep-and-segmentation}) -- Deutschland zweigeteilt betrachtet.
Als gedankliche Grenzlinie dient der Main. Süddeutschland meint damit
vorrangig das Areal in den heutigen Grenzen von Bayern und
Baden-Württemberg.

In der Frühbronzezeit existierten mehrere lokal begrenzte
Kultureinheiten in Süddeutschland, die als Inseln in einer meist noch
spätneolithischen Umgebung entstanden. Im Süden und Südosten von Bayern
fand sich die Straubing-Gruppe, deren Verbreitungsgebiet sich auch nach
Österreich fortsetzte (s.o.). Westlich grenzte sie an die Ries Gruppe
an. Ausgehend vom Oberrhein fanden sich in Baden-Württemberg, jeweils
nördlich anneinander anschließend, die Singen Gruppe, die
Hochrhein-Oberrhein Gruppe, die Neckar Gruppe und schließlich die
Adlerberg Gruppe am nördlichen Oberrhein und der Untermainebene. Die
übliche Bestattungsform in diesen Kontexten waren Flachgräberfelder mit
angehockten Körperbestattungen. Die Orientierung der Toten ist
geschlechtsabhängig und folgte der Glockenbechertradition. Manche Gräber
sind mit einem Holzsarg oder einer abdeckenden Steinpackung ausgebaut.
Metallbeigaben sind selten und weitestgehend auf Kupferzierrat, Nadeln
und Dolchklingen beschränkt. Knochen und Muschelschmuck wurden dagegen
oft beigegeben\footnote{\textcite{jockenhovel_germany_2013}, 726-727.}.

Die Mittelbronzezeit war in Zentraleuropa eine Phase nachhaltiger
Innovation. Schwerter und Speere kamen auf und verbreiteten sich
schnell. Zweischneidige Rasierklingen, Pinzetten, Messer und Sicheln
erweiterten das Metallwerkzeuginventar. Pferd und Wagen gewannen als
Transportmittel wesentlich an Bedeutung. In der Mittelbronzezeit
entstand und dominierte besonders in Süddeutschland, aber darüber hinaus
in ganz Zentraleuropa, die Tumulus- oder Hügelgräberkultur. Der
Übergangsprozess dahin lief regional unterschiedlich ab, letztendlich
erfasste das Phänomen jedoch einen bemerkenswert großen Raum. Die
rund-ovalen Hügel wurden aus Erde, Sand, Grassoden, Steinen oder einer
Kombination dieser Materialen errichtet. Je nach lokaler Verfügbarkeit
von Baumaterialien unterscheidet sich auch ihre Architektur. Die Hügel
waren oft von einer Steinsetzung, einem Graben oder -- besonders in
Westfalen und den Niederlanden -- Pfostensetzungen eingehegt. Sie kommen
in der Regel nicht einzeln vor, sondern clustern in kleineren bis sehr
großen Gruppen, die gemeinsam ein landschaftsprägendens Gräberfeld mit
oft dutzenden Hügeln bilden. Jeder Hügel gehörte einer kleinen
Familiengruppe, wobei die Anlage ursprünglich meist über einer
Zentralbestattung angelegt wurde. Spätere Bestattungen wurden in den
vorhanden Hügel eingetieft und liegen deswegen meist höher als das
Ursprungsgrab. Zwischen den Hügeln einer Gruppe wurden gelegentlich
Flachgräber angelegt. Die Bestattungen wurden zunächst meist als
Körpergräber ausgeführt, der Anteil von Brandgräbern nahm im Laufe der
Mittelbronzezeit jedoch immer mehr zu. In der Regel wurde der Leichnam
ausgestreckt in Nord-Süd- oder Ost-West-Orientierung deponiert und die
Grabgrube mit Steinsetzungen oder Holzplanken ausgebaut. Die Ausstattung
der Toten ist geschlechtsabhängig und scheint die persönlich Ausstattung
im Leben widerzuspiegeln: Männer wurden mit Waffen wie Schwert, Dolch,
Axt oder Lanzenspitze sowie Schmuck in Form von Nadeln oder Armreifen
versehen, Frauen mit einer reichen Auswahl von Trachtbestandteilen.
Bernsteinperlen aus dem Baltikum erfreuten sich großer Beliebtheit in
Süddeutschland: In Württemberg und Südbayern enthält der archäologische
Befund einzelne Gräber und Horte mit tausenden Perlen. Die
Zusammenstellung des Grabbeigabeninventars ist das wesentliche Merkmal
nach dem Lokalgruppen der Süddeutschen Mittelbronzezeit wie unter
anderem die Alb Gruppe, die Hagenau Gruppe oder die Rhein-Main Gruppe
definiert werden\footnote{\textcite{jockenhovel_germany_2013}, 727-730.}.

Auch in Süddeutschland war die Brandbestattung die wesentliche
Bestattungsform der Spätbronzezeit. Die Deponierung des Leichenbrands in
einer Urne war ab 1100calBC (Ha A2) die universelle Praxis. Die Urnen
wurden zusammen mit anderen Keramikgefäßen -- manchmal Teile eines
zusammengehörigen Services -- in einfache oder mit einer Steinkiste
ausgebaute Grabgruben gelegt. Mit der Einführung der Brandbestattung
ging die Aufgabe des Hügelbaus einher. Vereinzelt wurden jedoch noch
weiter Körpergräber angelegt, die mit einer reichen Beigabenausstattung
die Traditionen der Hügelgräberzeit fortführen. Obgleich die
Spätbronzezeit in Süddeutschland mit größerer kultureller
Standardisierung als die Mittelbronzezeit assoziiert werden kann ist sie
doch bei weitem kein völlig einheitliches oder homogenes Phänomen. Sie
lässt sich ausgehend von Bz D bis Ha B2/3 in mindestens fünf Phasen
gliedern und reicht bis in die Eisenzeit hinein. Zwar sind viele
Metallartefaktkategorien weit verbreitet, Schmuck -- besonders Fibeln --
und Keramik zeigen dagegen starke regionale Variabilität. In
Süddeutschland kann die südliche Bayrische Gruppe im Alpenvorland, die
Fränkisch-Pfälzische Gruppe in Ostbayern und weiter westlich die
Untermainisch-Schwäbische Gruppe unterschieden werden. Auch innerhalb
des Urnenfelderbestattungsrituals gibt es Varianten, die sich vor allem
durch die Beigabenmenge und Auswahl ausdrücken: Die Mehrzahl der Gräber
enthält nur Keramik, manche daneben auch einige wenige Nadeln und
Schmuckgegenstände, reichere dann Messer, Rasierklingen und einfache
Waffen wie ein Bogen mit Pfeilen. Die herausragend reichen Bestattungen
sind mit größeren Waffen wie Schwertern und Speerspitzen, sowie
bronzenem Drinkgeschirr, Wagenteilen und hochwertigem Bronze- und
Goldschmuck ausgestattet. In Südwestdeutschland sind sie häufig mit
einer nord-süd orientierten Kammer aus Holz oder Stein ausgebaut. Diese
Gräber gehören meist zu erwachsenen Männern, denen seitens der
Archäologischen Forschungstradition eine Führungsposition in ihrer
lokalen Gruppe zugesprochen wird. Manche sind als Körperbestattungen
ausgeführt und enthalten außergewöhnliche Beigaben wie Zeremonialwägen,
Rohmetall und Bronzegewichte. Frauengräber dieser Art sind selten,
gelegentlich treten jedoch Gräber auf, in denen die sowohl Überreste
eines Mannes als auch einer Frau beigesetzt sind. Frauengräber sind mit
Schmuck und Trachtbestandteilen ausgestattet und allgemein beigabenärmer
als Männergräber. Geschlecht und Alter korrelieren grundsätzlich mit der
Größe von Grab und Urne. Kindergräber sind üblicherweises mit einer
feminin assoziierten Beigabenauswahl versehen\footnote{\textcite{jockenhovel_germany_2013},
  730-733.}.

\autocites{falkenstein_development_2012}{falkenstein_zum_2017}

\hypertarget{norddeutschland}{%
\subsection{Norddeutschland}\label{norddeutschland}}

Norddeutschland dient hier als vereinfachter Begriff in Abgrenzung zu
Süddeutschland und umfasst eigentlich Nord-, Ost- und Mitteldeutschland.
Diese Großregionen sind Teil unterschiedlicher Einflusssphären (s.o.)
und durchlaufen in der Bronzezeit unterschiedliche kulturelle
Entwicklungen. Norddeutschland (im Schwerpunkt Schleswig-Holstein,
Niedersachsen und Mecklenburg-Vorpommern) wird stark aus Skandinavien
(s.u.), Ostdeutschland (im Schwerpunkt Sachsen, Sachsen-Anhalt und
Brandenburg) von Polen (s.o.) und Mitteldeutschland (im Schwerpunkt
Saarland, Rheinland-Pfalz, Nordrhein-Westfalen, Hessen, Thüringen) aus
dem Süden (s.o.) und Westen (s.u.) beeinflusst. Norddeutschland lässt
sich damit als Teil der Nordischen Bronzezeit beschreiben und
Ostdeutschland ist Teil der beiden großen, aufeinander folgenden Kreise
Aunjetitzer sowie Lausitzer Kultur. Mittel- und Westdeutschland gehören
in vielerlei Hinsicht zur süddeutschen Sphäre und durchlaufen
entsprechend eine dazu ähnliche Entwicklung.

\hypertarget{norddeutschland-1}{%
\subsubsection{Norddeutschland}\label{norddeutschland-1}}

Aus dem Substrat der spätneolithischen Einzelgrabkultur entstand in
Norddeutschland, Südskandinavien und im Westbaltikum zwischen 2200 und
1600calBC die Nordische Frühbronzezeit. In vielerlei Hinsicht setzte sie
neolithische Traditionen fort -- offensichtlich beispielweise an der
weiten Verbreitung von hochwertigen Feuersteindolchen. Erste
Metallgegenstände gelangten aus Zentral- und Westeuropa nach Norden,
besonders aus dem unmittelbar angrenzenden Aunjetitzer Raum. Eine eigene
Metallverarbeitung emanzipierte sich schnell und eindrucksvoll, wobei im
Norddeutschen Raum Einflüsse aus Süddeutschland und der Schweiz sichtbar
bleiben\footnote{\textcite{jockenhovel_germany_2013}, 735.}.

Zur Mittelbronzezeit, nach 1600calBC, etablierte sich in Norddeutschland
die Sögel-Wohlde Kultur, deren Verbreitungsgebiet sich vom Osten der
Niederlande über Westphalen bis nach Jütland erstreckte. Sie zeichnet
sich durch Körpergräber in Grabhügeln aus. Im Gegensatz zur Situation in
der zeitgleich südlich davon vorherrschenden Hügelgräberkultur wurden
allerdings nur Männer mit diese Bestattungsform bedacht. Die
Beigabenauswahl umfasst Kurzschwerter, Dolche, Randleistenbeile,
Pfeilspitzen, Nadeln und gelegentlich kleine, goldene Spiralringe. Die
Sögel-Wohlde Kultur ging nördlich der Elbe und im heutigen
Schleswig-Holstein in Montelius Perioden II bis III (1450-1250calBC und
1250-1100calBC) in eine vielfältige und dynamische Kulturlandschaft
über. In diesem Kontext wurden große Grabhügel mit Steinkistengräbern
errichtet, die noch heute Landschaftsprägend wirken. Wie in
Zentraleuropa enthalten sie mehrere Gräber, wobei häufig ein Mann und
eine Frau zusammen oder nacheinander in eine Grabkammer eingebracht
wurden. Ein klassische Deutung versteht die Hügel jeweils als
Familiengrabstätte eines einzelnen Hofes in einer weitestgehend
egalitären Gesellschaft. Anhand typischer Waffenkombinationen lassen
sich Lokalgruppen wie die Westholstein Gruppe (Schwert + Speerspitze),
die Segeberg Gruppe (Schwert + Absatzbeil) und die Westmecklenburg
Gruppe (Schwert + Absatzbeil + Dolch) unterscheiden. Unter den
Grabhügeln fallen einzelne als sogenannte Trachthügel durch eine
besonders reiche Ausstattung -- z.B. mit gegossenen Bronzegefäßen -- und
gute Erhaltungssituation auf\footnote{\textcite{jockenhovel_germany_2013},
  735-736.}.

Im Laufe von Montelius Periode III wurde die Körperbestattung in
Norddeutschland langsam zugunsten der Brandbestattung aufgegeben -- ab
Periode IV war letztere die absolute Regel. Zeitlich korrelierte der
Übergang mit Importen von Bronzegefäßen und Kesselwägen aus dem
Böhmischen und Mährischen Raum. Diese Artefakte sind verknüpft mit dem
andersartigen und sicher religiös aufgeladenen
Urnenfelder-Symbolinventar, was eine kausale Verbindung der Phänomene
nahelegt und Hinweise auf den Ursprung des Urnenfelderphänomens gibt.
Die Norddeutschen Urnenfelder können wie ihre Süddeutschen Pendants
mehrere hundert Urnengräber umfassen, wurden oft um ältere Grabanlagen
herum angelegt und bis in die frühe Eisenzeit genutzt. Das
Beigabeninventar ist klein und reduziert auf Kleinwerkzeuge und
Hygieneausstattung wie Rasiermesser, Pinzetten oder Nadeln. Größere
Objekte wie Schwerter treten in einigen Fällen in Miniaturform auf. Oft
wurden Bronzegegenstände in Horten -- Totenschätzen -- außerhalb der
Gräber deponiert. Unter den Urnen fallen Sonderformen wie Haus- oder
Gesichtsurnen auf\footnote{\textcite{jockenhovel_germany_2013}, 736-737.}.

Die Nordische Spätbronzezeit, also nach üblichem Gebrauch Montelius
Perioden IV bis VI, begann mit einer langen Phase der Konsolidierung.
Sie wirkte nun stärker nach außen -- Artefakte der nordischen Bronzezeit
wie Plattenfibeln oder einschneidige, mit Bootsymbolik verzierte
Rasiermesser tauchen in Gräbern in Niedersachsen und Holland bis in die
Niederrhein Region und Pommern auf. In Periode V wurden vereinzelt große
und reich ausgestattete Brandgräber angelegt -- sogenannte
\emph{Königsgräber}. Auch in Periode VI wurde in Urnengräbern bestattet.
Sie leitet über zur eisenzeitlichen Jastorf Kultur\footnote{\textcite{jockenhovel_germany_2013},
  737-738.}.

\hypertarget{ostdeutschland}{%
\subsubsection{Ostdeutschland}\label{ostdeutschland}}

In Ost- und Mitteldeutschland, aber auch weit darüber hinaus über
Schlesien, Großpolen, Böhmen, Mähren, der Südwestslowakei und dem
nördliche Teil Niederösterreich hinweg, siedelten in der Frühbronzezeit
in einem Zeitfenster von 2300/2200 bis 1600/1500calBC Vertreter des
Aunjetitzer Kulturkomplexes. Die Wurzeln der Aunjetitzer Kultur -- die
Proto-Aunjetitzer Phase -- liegen im ausgehenden Spätneolithikum und
greifen Elemente von Schnurkeramik und Glockenbecher Kultur auf. Der
übliche Bestattungsmodus in der Aunjetitzer Kultur waren Flachgräber mit
angehockten Körperbestattungen. Sowohl Männer als auch Frauen sind
Nord-Süd orientiert mit Blick nach Osten. Mehrheitlich sind die Gräber
einfache Gruben, gelegentlich wurden sie allerdings auch mit einer
Steinkiste oder einem Baumstammsarg ausgebaut. Manchmal sind in einer
Grabgrube mehrere Tote beigesetzt. Auffällig sind einige Fälle von
Kinderbestattungen in großen Vorratsgefäßen. In der fortgeschrittenen
Aunjetitzer Kultur, etwa ab dem Übergang zum 2. Jahrtausend, wurden
einfache und kleine Kupfergegenstände zur häufigsten Beigabenkategorie.
Die Beigabenmenge und -vielfalt ist insgesamt gering, sieht man von den
wenigen, sehr reich ausgestatteten, großen und weithin sichtbaren
Grabhügeln der Leubingen Gruppe am Dreiländereck Sachsen, Sachsenanhalt
und Thüringen ab, die traditionell als \emph{Fürstengräber} bezeichnet
werden. Sie enthalten nur eine einzige, fast immer männliche Bestattung
aber ein breites und hochqualitatives Beigabeninventar inkluvsive
signifikanter Mengen Goldschmuck. Diese außergewöhnlichen Bestattungen
sind in der Aunjetitzer Kultur fast ohne Vergleich. Sie weisen auf eine
lokal begrenzte aber ausgeprägte soziale Differenzierung hin, die sich
vielleicht durch eine Beherrschung von Handelswegen zwischen Donau und
Baltikum, Salzgewinnung oder Kupfererzabbau im Harzvorland erklären
lassen könnte. Für letzteren existiert allerdings kein archäologischer
Nachweis. Vom Ende der Aunjetitzer Kultur am Beginn der Mittelbronzezeit
sind fast keine Bestattungsbefunde bekannt\footnote{\textcite{jockenhovel_germany_2013},
  725-726.}.

Ostdeutschland gehörte in der Mittelbronzezeit zum Verbreitungsgebiet
der in ganz Ostmitteleuropa von Deutschland über Polen, Böhmen, Mähren
und bis in die Slowakei dominanten Lausitzer Kultur. Sie entstand aus
einem Substrat aus Aunjetitzer Kultur und verschiedenen Lokalgruppen,
die bis ins 2. Jahrtausend an einer spätneolithischen Tradition
festgehalten hatten. Die Lausitzer Kultur selbst ist kein homogenes
Kulturphänomen -- in ihr lassen sich Keramikphasen (Prä-Lausitz Phase I:
Bz B-C, Phase II Bz C-D, Phase III Bz D-Ha A1, Phase IV Ha A2-Ha C1) und
deutlich distinkte, regionale Gruppen unterscheiden. Die Prä-Lausitz
Gruppe in Ostdeutschland, Schlesien und Großpolen war zunächst stark von
der frühen, zentraleuropäischen Urnenfelderkultur geprägt, löste sich
davon aber zu Beginn der Spätbronzezeit. Das große Verbreitungsareal
dieser Westlichen Lausitzer Gruppe reicht bis weit nach
Mitteldeutschland hinein und umfasst die Flusssysteme von Oder, Elbe und
Weichsel. Sie lässt sich in mehrere Subgruppen wie etwa die
Saalemündungsgruppe, die Unstrut Gruppe oder die Elb-Havel Gruppe
gliedern. Die übliche Bestattungsform der Lausitzer Kultur war die
Brandbestattung in Urnen. Die Gräberfelder wurden über viele
Generationen hinweg genutzt und fallen mit oft mehreren tausende
Begräbnissen sehr groß aus. Die Grabbeigaben bestehen fast
ausschließlich aus Gefäßkeramik, davon jedoch große Mengen.
Metallobjekte sind selten und auf kleine Gegenstände wie Nadeln, Schmuck
und Gebrauchsgegenstände wie Messer und Rasierklingen beschränkt. Die
Beigabenarmut erschwert Schlüsse auf soziale Unterschiede innerhalb der
bestattenden Gesellschaft. Besonders in der Elbe-Saale Region kommen
vereinzelt Waffengräber vor\footnote{\textcite{jockenhovel_germany_2013},
  734-735.}.

\hypertarget{mitteldeutschland}{%
\subsubsection{Mitteldeutschland}\label{mitteldeutschland}}

Der Raum südlich der Nordischen Bronzezeit, westlich der Aunjetitzer
Kultur und nördlich der Rhein-Main-Linie lässt sich in der
Frühbronzezeit keinem kohärenten Kulturphänomen zuordnen. Zwischen
Niederrhein und Elbe traten verschiedene Varianten der spätneolithischen
Riesenbecher Gruppe auf. Ihre Wickelschnurkeramik zeigt Parallelen zur
Sögel-Wohlde Kultur der Nordischen Bronzezeit, andere Importe weisen auf
Verbindungen nach Süddeutschland, in den Benelux-Raum und nach
Großbritannien hin. Verbreitet war die Körperbestattung in Hügelgräbern
nach Vorbild der Einzelgrabkultur\footnote{\textcite{jockenhovel_germany_2013},
  727.}.

In der Mittelbronzezeit näherten sich Zentral- und Westdeutschland der
süddeutschen Sphäre und damit der Hügelgräberkultur an. Lokalgruppen wie
die Fulda-Werra Gruppe, die Thüringer Gruppe, die Oberpfälzische Gruppe
und -- nördlich der Mittelgebirge -- die Lüneburger Gruppe gehören zum
Kontakt- und Frauentauschnetzwerk der süddeutschen Mittelbronzezeit. Je
weiter nördlich, desto stärker ist jedoch auch die Beziehung zu Periode
II der Nordischen Bronzezeit. Weiter westlich, an Mittel- und
Niederrhein sowie in Westfalen, zerfließen wie schon in der
Frühbronzezeit die scharfen Kulturgrenzen. Starke Einflüsse aus den
Niederlanden offenbaren sich in der Errichtung von
Holzpfostenumfassungen um Grabhügel und der frühen Verbreitung von
Brandgräbern\footnote{\textcite{jockenhovel_germany_2013}, 727-730.}.

Auch in der Spätbronzezeit ist Mittel- und Westdeutschland zwischen
verschiedenen Einflusssphären aufgeteilt. Westlich des Rhein siedelten
Vertreter der Niedermainisch-Schwäbischen Gruppe und östlich des Ober-
und Mittelrheins sowie im Moselgebiet und dem Saarland des
Rhin-Suisse-France orientale (RSFO) Kulturkomplex. Die Areale östlich
davon gehören bereits zur Peripherie der Lausitzer Kultur. Nördlich der
Mittelgebirge, zwischen Niederrhein Saale und Elbe, zeigen Keramik und
Bronzeartefakte weniger distinkte Unterschiede und erlauben so keine
Gruppenuntergliederung. Ab 1000calBC geriet diese Region zunehmend unter
den Einfluss der späten Nordischen Bronzezeit (Periode IV und V). Die
lokale Spätbronzezeit dauerte bis 600calBC, dann gehörte sie auch zum
Verbreitungsgebiet der eisenzeitlichen Jastorf Kultur. Wie in den
umgebenden Regionen ist auch in der mitteldeutschen Spätbronzezeit
Brandbestattung in Urnen üblich. Die Urnen wurden in alte Grabmonumente
wie Hügel oder Langbetten eingebracht oder leicht überhügelten Gruben
deponiert. Im Gegensatz zur Entwicklung in Süddeutschland wurden diese
Tradition bis in die Eisenzeit fortgesetzt\footnote{\textcite{jockenhovel_germany_2013},
  730-733.}.

\hypertarget{nordostfrankreich}{%
\subsection{Nordostfrankreich}\label{nordostfrankreich}}

Das bronzezeitliche Frankreich lässt sich in drei geographische und
kulturelle Regionen gliedern: Die Atlantikküste, die starke Impulse von
den Nordseeanrainern, Großbritannien und der Iberischen Halbinsel
erfuhr, Südfrankreich, das besonders von den Entwicklungen im
Westmediterranen Raum beeinflusst wurde und (Nord)ostfrankreich
ausgehend vom Pariser Becken. Die Bronzezeit in Frankreich dauert von
2300calBC bis 800calBC, wobei eine Aufteilung in Frühbronzezeit
(2300-1650calBC), Mittelbronzezeit (1650-1350calBC) und Spätbronzezeit
(1350-800calBC) üblich ist\footnote{\textcite{mordant_bronze_2013}, 571.}.

In Nordostfrankreich (in etwa die modernen, administrativen Regionen
Ile-de-France, Hauts-de-France, Grand-Est und Bourgogne-Franche-Comté)
begegnen sich in der Frühbronzezeit eine westliche Einflusssphäre aus
der heutigen Normandie und Bretagne, eine noch weiter nordöstlich
entlang der Nordsee gelegene Sphäre aus dem heutigen Benelux-Raum und
die Rhone Kultur aus dem Süden. Im Nordwesten Frankreichs wurden riesige
Grabhügel mit 40-50m Durchmesser und bis zu 5-6m Höhe errichtet. Diese
monumentalen Anlagen sind in der Regel nur mit einer einzigen, zentralen
Bestattung in einer großen Grabkammer versehen -- meist Männer und nur
in selten Fällen Frauen oder Kinder. Zusammen mit den außerordentlich
reichen Beigaben (Äxte, Hellebarden, Gold, Silber, Bernstein, Fayence)
sind sie Anzeiger für eine deutliche vertikale Gliederung der
Gesellschaft mit einzelnen, herausragenden Führungspersonen. Auch in
Nordostfrankreich gibt es frühbronzezeitliche Grabhügel, jedoch ist die
Erhaltungssituation erheblich schlechter als im Nordwesten. In diesen
Hügeln wurden sowohl Brand- als auch Körperbestattungen deponiert. Die
Urnen weisen Ähnlichkeiten zu Urnen aus Südengland und Flandern auf.
Obgleich die Umfassungen der Hügel Durchmesser von bis zu 100m erreichen
konnten, enthalten sie nur wenige oder keine Beigaben. Neben Grabhügeln
gab es in Nordostfrankreich zeitgleich auch Brandgräberfelder mit
einfachen Urnenbestattungen. Im südlichen Teil, im Einzugsbereich der
Rhone Kultur, in Burgund und Franche-Comté, kommen kleinere Grabhügel
mit 6-8m Durchmesser und 1m Höhe vor. Daneben wurden Neolithische
Megalithikanlagen sowie Höhlen weiter als Kollektivbestattungsplätze
genutzt. Ebenfalls in Burgund treten Einzelgräberfelder mit
Körperbestattungen in gestreckter und angehockter Lage auf. Insgesamt
ist die Beigabenauswahl in diesen Kontexten limitiert aber deutlich
geschlechtsspezifisch: Dolche, Äxte und Nadeln für Männer, Schmuck nur
für Frauen\footnote{\textcite{mordant_bronze_2013}, 571-572 \& 581.}.

In der Mittelbronzezeit, im Kontext einer massiven Zunahme von Quantität
und Qualität der Metallverarbeitung und Erfindungen wie dem Absatzbeil
und Schwertern, formt sich in Nord- und Westfrankreich eine große,
intensiv vernetzte Kulturregion, die auch Flandern und den Süden
Großbritanniens einschloss. In Bretagne und Normandie war die Errichtung
von herausragenden Individualgrabhügel in Fortsetzung einer Entwicklung,
die schon seit der fortgeschrittenen Frühbronzezeit zur Verkleinerung
der Hügel geführt hatte, weiter rückläufig und wurde schließlich
eingestellt. Darüber hinaus sind die Bestattungspraktiken in diesem Raum
weitestgehend ein Desiderat. Im Nordosten wurden die
Brandbestattungsfelder weitergeführt, die schon in der Frühbronzezeit in
diesem Raum aufgetreten waren. Sie enthalten wenige Beigaben, kein
Metall und höchstens ein Keramikgefäß als Urne. Südlich davon, in
Ostfrankreich, lässt sich eine starke Expansion der östlichen Tumulus
Kultur beobachten, die schließlich bis hinein ins Pariser Becken und ins
Loire Tal wirkte. Die übliche Bestattungform in diesem Kulturkreis war
das Hügelgrab mit mehreren Beisetzungen. Dabei konnten sowohl Körper-
als auch Brandbestattungen eingebracht werden -- letztere gewannen im
Laufe der Mittelbronzezeit an Bedeutung. Die Gräber sind reich mit
Beigaben versehen und zeigen hier eine klare Differenzierung nach
Geschlecht: Männer wurden mit Dolchen, Äxten und Pfeilspitzen, Frauen
mit Nadeln, Perlen und Armreifen ausgestattet. Das Ritual weist Bezüge
zu den kontemporären Praktiken in Süddeutschland auf\footnote{\textcite{mordant_bronze_2013},
  572-574 \& 581-582.}.

Am Übergang zur Spätbronzezeit konsolidierte sich die kulturelle
Ost-West Spaltung Frankreichs. Entlang der Atlantikküste ausgehend von
der Iderischen Halbinsel bis in den Beneluxraum hinein bestand der schon
zuvor etablierte, atlantische Kulturkomplex fort. Auch in Ostfrankreich
lassen sich die Entwicklungen vor dem Hintergrund der bereits in der
Mittelbronzezeit nachvollziehbaren Prozessen verstehen. In einem großen
Areal zwischen und jeweils jenseits von Rhein und Pariser Becken
dominierte der Rhin-Suisse-France orientale (RSFO) Kulturkomplex.
Wiederrum war die Situation in Nordostfrankreich geprägt von zwei
Einflusssphären. Neben traditionelleren Grabformen, wie etwa
Körperbestattungen in Steinkisten, wurde in Ostfrankreich im RSFO Raum
zunehmend Brandbestattung auf Urnenfeldern praktiziert. Auch im
Nordosten war die Brandgrabsitte zunehmend präsent und wurden ab dem 12.
Jahrhundert fast universell. Im westlichen Teil hielt sich die Tradition
sehr einfacher Brandgräber und karger Beigaben, die schon in der
Mittelbronzezeit Verbreitung gefunden hatte. Neben flachen Einzelgräbern
wurden für gesellschaftlich herausragende Individuen -- meist Männer --
auch Grabhügel errichtet. Sie sind mit einer Vielzahl von Beigaben
versehen, manchmal mit einem Schwert oder Wägeausrüstung. Hügelgräber
gewannen in Ostfrankreich zum Ende der Bronzezeit im 9. Jahrhundert
wieder an Bewandtnis\footnote{\textcite{mordant_bronze_2013}, 574-575 \&
  582-583.}.

\hypertarget{sudskandinavien}{%
\subsection{Südskandinavien}\label{sudskandinavien}}

Das wichtigste chronologische Instrument der skandinavischen Bronzezeit
ist die -- freilich weiterentwickelte -- Periodengliederung nach Oskar
Montelius. Der Begriff Frühbronzezeit wird für die Perioden I bis III
verwandt, die etwa das Zeitfenster von Reineckes Phasen A2 bis Ha A
abdecken. Die Bronzezeit beginnt nach dieser Terminiologie in
Skandinavien also deutlich später als in Zentraleuropa etwa um ab
1800calBC. Spätbronzeit umfasst die Perioden IV bis VI, die von Ha B1
bis C reichen. Südskandinavien meint Dänemark sowie die schwedischen
Provinzen Scania und Blekinge. Im Vergleich zu nördlicheren Teilen
Skandinaviens bieten sich hier naturgeographisch günstigere Bedingungen
für Ackerbau, was die Region in der Bronzezeit in die Rolle eines
kulturellen Zentrums für Skandinavien versetzt. Im Vergleich zum
restlichen Europa fällt die Nordische Bronzezeit vor allem durch ihre
außergewöhnlich intensive und stilistisch einzigartige
Metallverarbeitung, die Erhaltung vieler tausend Grabhügel und eine
auffällige, weit verbreitete Form der Felskunst auf\footnote{\textcite{thrane_scandinavia_2013},
  746-750.}.

Aus der Nordischen Bronzezeit sind viele Gräber erhalten -- besonders
aus den Perioden II bis IV --, die es erlauben die Entwicklung der
Bestattungssitten gut nachzuzeichnen. Bestattungen in einigen Grabhügeln
Zentraljütlands sind unter Feuchtbodenbedingungen erhalten und damit ein
hervorragendes archäologisches Archiv. Körperbestattungen (meist) in
Grabhügeln waren die Regel, bis in Periode II sporadisch einzelne
Brandbestattungen -- ebenfalls meist in Hügel eingebracht -- auftraten.
Ab Periode III war die Brandbestattung universell, sieht man von der
Situation auf der Insel Gotland ab. Der Übergang zur Bronzezeit in
Südskandinavien vollzog sich in verschiedenen Lebensbereichen langsam.
Neben dem Hausbau, wo weiter die schon im Neolithikum gebräuchlichen
Langhäuser errichtet wurden, und der Flintproduktion, zeigt sich das
auch in der Grabanlage, die an alte Tradition anknüpfte. In Dänemark und
Südschweden wurden die zu bestattenden Leichname in der Frühbronzezeit
in Baumsärgen -- meist ausgehöhlte Eichenstämme -- deponiert. Die Toten
wurden gestreckt auf den Rücken gelegt und mit dem Kopf nach Westen
orientiert. In Periode III änderte sich das hin zu einer Nord-Süd
Orientierung. Oft wurde der Tote innerhalb des Sarges auf eine
Ochsenhaut oder eine Wolldecke gebettet. Über manchen Gräbern wurden
Totenhäuser errichtet und manche Grabhügel bedecken ein Langhaus. In
Periode IV vollzog sich eine Wende hin zu kleineren Grabgruben --
Brandbestattungen brauchen weniger Platz. Große Beigaben wie Schwerter,
die zuvor mit den Toten abgelegt worden waren, entfallen damit
gleichermaßen. Im Laufe der Perioden IV bis VI wurde der Leichenbrand
zunehmend in Urnen deponiert, aber Steinkisten und einfache Erdgruben
kommen dennoch parallel weiter vor. Die Beigabenmenge nimmt im Zuge
dieser Entwicklung weiter ab: Die meisten Brandbestattungen sind nicht
mit Metallartefakten ausgestattet\footnote{\textcite{thrane_scandinavia_2013},
  754-756.}.

Das herausragendste Merkmal der Bestattungskultur der Nordischen
Bronzezeit sind ihre Grabhügel. Aus der Bronzezeit sind schätzungsweise
100.000 Erdhügel erhalten, die meisten in Südskandinavien und aus den
Perioden II bis III. Viele sind durch landwirtschaftliche Aktivität
gefährdet. In der Größe variieren sie zwischen sehr klein (5m
Durchmesser, 0,5m Höhe) und sehr groß (35m Durchmesser, 6m Höhe), wobei
sie im mit im Durchschnitt 25m Durchmesser und 2,5m Höhe insgesamt
beachtlich ausfallen. Neben diesen Erdhügeln gibt es auch etwa 30.000
Steinhügel, vor allem in den Regionen nördlich von Südskandinavien. Die
größten Exemplare in dieser Kategorie messen mehr als 70m im Durchmesser
und sind mehr als 10m hoch. Sie gehören damit zu den größten
vorgeschichtlichen Bauwerken Europas. In der südskandinavischen
Frühbronzezeit wurden Gräber häufig mit einer kleinen Steinpackung
überhügelt und auf diesem Steinhügel anschließend ein Erdhügel aus
Grassoden aufgebaut. In Dänemark gibt es auch einige spätbronzezeitliche
Steinhügel. Während die frühbronzezeitlichen Hügel häufig auf
natürlichen Erhebungen und Geländerücken errichtet wurden, finden sich
die späteren Hügel tendentiell eher in Niederungslagen. Abgesehen von
einigen Sonderformen wie Langbetten, Hügeln mit einer zum Plateau
abgeflachten Spitze und schiffsförmigen Steinsetzungen sind die Hügel
äußerlich größenunabhängig sehr ähnlich. Die Anlage der Gräber und
Steinsetzungen in ihrem Inneren unterscheidet sich jedoch deutlich von
Hügel zu Hügel. Ein Hügel kann in ein oder mehreren Bauphasen errichtet
und für spätere Bestattungen erweitert worden sein. Oft findet sich
jedoch auch in sehr großen Hügeln nur eine einzige Bestattung.
Neolithische Anlagen konnten in der Bronzezeit Weiternutzung oder Ausbau
erfahren haben, während bronzezeitliche Hügel selbst mitunter bis weit
in die Eisenzeit hinein als Grabanlagen genutzt wurden.
Spätbronzezeitliche Urnen wurden häufig in bestehende Hügel eingebracht,
sodass viele Hügel Gräber aus zwei deutlich getrennten Phasen
enthalten\footnote{\textcite{thrane_scandinavia_2013}, 752-754.}.

Trotz der vielen erhaltenen Grabanlagen ist die Erforschung der
bronzezeitlichen Bevölkerung auf Grundlage dieser Grabbefunde ein
Desiderat. Die Erhaltungssituation der Knochen ist insgesamt schlecht.
Zwischen Männern und Frauen sind keine systematischen Unterschiede beim
Bestattungsbrauch erkennbaren, abgesehen von den etwas zahlreicheren
Grabbeigaben in Männergräbern. Jedoch zeigt der Vergleich der
Geschlechtsbestimmung von Brandgräbern aufgrund
physisch-anthropologischer Kriterien gegenüber derjenigen aufgrund von
Beigaben erstaunliche Abweichungen, die die diachrone Relevanz dieses
Ergebnisses in Frage stellen. Die Beigabenmenge und -qualität scheint
immerhin Aufschluss über die soziale Gliederung zu geben, da sich
jenseits von einigen wenigen, absolut herausragenden ``Fürstengräbern''
auch regelmäßige Ausstattungsklassen andeuten: Gräber mit Goldbeigaben
und Waffen lassen sich von einfacheren mit verzierten Messern,
Rasiermessern und Pinzetten unterscheiden. Die Mehrzahl der Gräber
enthält kein oder nur ein Metallobjekt. Der Kontrast zwischen Arm und
Reich ist in der Spätbronzezeit besonders akzentuiert. Kindergräber sind
sehr selten -- erst in der Spätbronzezeit finden sich vereinzelt in
Urnen neben den Überresten eines Erwachsenen auch die eines Kindes.
Nicht alle Bestattungen sind überhügelt oder in Hügel eingebracht, die
Flachgräber verhalten sich aber abgesehen davon ebenso wie die
Hügelgräber. In der Spätbronzeit wurden in Südskandinaviens auch
Flachgräberfelder angelegt, je weiter südlich, desto größer\footnote{\textcite{thrane_scandinavia_2013},
  756-758.}.

\hypertarget{benelux}{%
\subsection{Benelux}\label{benelux}}

Im Benelux-Areal, das hier neben den Niederlanden, Belgien und Luxemburg
auch Teile Nordostfrankreichs, Frieslands und des Rheinlands
einschließen soll, begegnen sich in der Bronzezeit drei Einflussspähren:
Die Nordische, die Kontinentale und die Atlantische Bronzezeit. Die
lokale, kulturelle Entwicklung ist in dieser Konsequenz kleinteilig und
abwechselungsreich. Entscheidenden Einfluss darauf hatte die
geographische Gliederung des Areals, wobei zwei wesentliche Dichotomien
zu beachten sind: Der Gegensatz von Fluss- und Küstenniederungen
gegenüber Pleistozänem Hochland und die Trennung in Areale nördlich und
südlich der Mündungsgebiete von Maas, Rhein und Ijssel. Im
Spätneolithikum und der Bronzezeit gehören die Areale nördlich und
östlich dieser Flüsse dem Austauschnetzwerk der Nordischen Bronzezeit
zu. Südlich und Westlich der Flüsse überwiegt der Einfluss aus Nord- und
Westfrankreich sowie Großbritannien. Das Kalksteinplateau in
Südostbelgien einschließlich der Ardennen lässt sich der kontinentalen
Sphäre zurechnen\footnote{\textcite{fokkens_bronze_2013}, 550-551.}.

Für das Spätneolithikum bis 2500calBC lässt sich die Situation
vereinfacht folgendermaßen darstellen: Die späte Vlardingen-Kultur
besiedelte die nördlichen Niederlande und die Stein Gruppe die
Maas-Niederung bis zur belgischen Grenze. In den pleistozänen Höhenlagen
fand sich die Einzelgrabkultur, die die Trichterbecherkultur an dieser
Position abgelöst hatte. Ab 2500calBC war das gesamte Areal bis in die
Ardennen Teil des Glockenbecherkomplexes, wobei sich die vormaligen
kulturellen und geographischen Einheiten auch hier in Subgruppen
auszudrücken scheinen. Diese Regionalität ĺöste sich nicht auf, sondern
wurde am Ende der Frühbronzezeit im Kontext der späten
Glockenbecherkultur in der Verteilung von Leitformen wie
Wickelschnurkeramik und Riesenbecher erneut sichtbar. Ab 1850calBC, in
der Mittelbronzezeit, bildeten sich im Nordosten, Westen und Süden der
Niederlande neue Keramikstile heraus: Im Norden -- nördlich und östlich
von Ijssel und Vechte -- Elp Keramik, südlich der Ijssel Hilversum
Keramik sowie später Drakenstein Keramik mit starken Einflüssen aus
Südengland und Nordwestfrankreich, im Westen Hoogkarspel Keramik. Ab
1200 war der Benelux-Raum Teil des Urnenfelder-Phänomens. Insbesonder
dank der für Urnen verwandten Gefäße lassen sich aber wiederrum
innerhalb desselben deutlich Keramikstile unterscheiden: Im Nordosten
die Ems Gruppe, im Süden der Niederlande Einflüsse aus dem Norddeutschen
Raum und im Süden Belgiens eine starke Orientierung an der
Rhin-Suisse-France Oriental (RSFO) Tradition. Im Benelux Raum dauert die
Bestattung auf Urnenfeldern bis zum Ende der frühen Eisenzeit und damit
länger als in anderen Regionen Europas an\footnote{\textcite{fokkens_bronze_2013},
  552-553.}.

In großen Teilen des Benelux Gebiets wurden ab 2900calBC Grabhügel
errichtet, wobei diese Tradition in einzelnen Regionen, wie etwa
Nordbelgien, erst ab 2600 bzw. 2000calBC zur Regel wurde. Sie hielt dann
bis 1400calBC an. Im Norden und Osten des Benelux Raums waren bis
1200calBC Körperbestattungen, ab der Mittelbronzezeit in gestreckter
Rückenlage, die Regel. Ab 1200 überwiegt die Brandbestattung in
Urnenfeldertradition. Südlich der Maas findet ein anderer Prozess statt:
Brandbestattungen wurden hier bereits im Spätneolithikum im
Glockenbecherkontext praktiziert. Schon in der Mittelbronzezeit, also
mehrere Jahrhunderte vor der Entstehung des Urnenfelderphänomens, wurde
die Brandbestattung die vorherrschende Bestattungssitte. Im Westen der
Niederlande vollzieht sich wiederrum eine andere, lokale Entwicklung:
Nach 1600calBC wurden in Westfriesland keine Grabhügel mehr angelegt.
Auch Urnenfelder kommen hier nicht vor. Da die Region besiedelt war und
der Forschungsstand als gut gelten darf, muss davon ausgegangen werden,
dass hier ein abweichender Bestattungsritus praktiziert wurde, der
archäologisch nicht oder nur schwer zu erfassbar ist\footnote{\textcite{fokkens_bronze_2013},
  557-558.}.

Verglichen mit den Hügeln der Nordischen Bronzezeit fielen die Grabhügel
im Benelux Raum klein aus. Selten übersteigt ihr Durchmesser 15m und sie
sind sämtlich weniger als 1,5m hoch. Um den Hügel herum wurden
Begrenzungs- und Einhegeanlagen in Form von dicht oder locker gepackten
Pfostensetzungen und/oder flachen Gräben errichtet. Diese Anlagen zeigen
eine große Variabilität -- Regionalgruppen deuten sich nicht an. Die
Parameter Beigabenqualität und -quantität, Hügelgröße und Komplexität
der Einhegeanlage scheinen voneinander unabhängig zu sein. Grabhügel
bildeten landschaftliche Bezugspunkte, in deren Nähe überproportional
häufig Siedlungen anlegt, weitere Hügel errichtet oder in die weitere
Gräber eingebracht wurden. Auch Urnenfelder wurden häufig um ältere
Grabhügel herum angelegt. Zwischen den einzelnen Bestattungsereignissen
in einem Hügel konnten lange Zeiträume vergehen, was darauf hindeutet,
dass nur Mitglieder einer sozialen Elite in Grabhügeln beigesetzt
wurden. Die Beigabenarmut vieler Bestattungen, besonders südlich der
Maas, stellt diese Deutung jedoch in Frage. Nördlich des Rhein und im
Nordwesten der Niederlande trat in der Mittelbronzezeit eine
standardisierte Form der Männerbestattung auf. Die Leichname wurden als
Rückenstrecker deponiert und mit einem Rapier, einem Randleistenbeil und
manchmal einer Speerspitze, Arm- oder Haarringen, Pinzetten, einer
Rasierklinge und Pfeilspitzen ausgestattet. Nicht zuletzt aufgrund der
hohen Standardisierung der Beigaben ist die häufig vorgenommene
Assoziation dieser Bestattungen mit dem Begriff ``Fürstengrab''
zweifelhaft\footnote{\textcite{fokkens_bronze_2013}, 558-561.}.

In der Spätbronzezeit ab 1200calBC bis 800calBC wandelte sich die
vormalige Praxis, nur einen kleinen Teil der Verstorbenen archäologisch
fassbar beizusetzen. Stattdessen wurden Urnenfelder mit einer großen
Zahl von Bestattungen errichtet. Die Urnen wurden zunächst in große
Langbetten eingebracht, nach 1000calBC dann in vereinheitliche, kleine
Hügel mit einem flachen Umfassungsgraben und einer Rampe auf der
Südostseite. Obgleich sehr große Urnenfelder existieren ist die Mehrzahl
überschaubar und deutet auf eine Population hin, die sich aus drei oder
vier Familien gespeißt haben könnte\footnote{\textcite{fokkens_bronze_2013},
  561-562.}.

\hypertarget{england}{%
\subsection{England}\label{england}}

Die Bronzezeit in Großbritannien und Irland erstreckte sich über einen
Zeitraum von 2500-800/600calBC. Das schließt jedoch auch das lokale
Chalcolithikum ein, das bis 2150calBC reicht. Die Innovation der
Bronzemetallurgie verbreitete sich, erst einmal entdeckt, äußert
schnell. Die Frühbronzezeit dauerte nach britischer Terminologie von
2150-1500calBC, die Mittelbronzezeit von 1500-1150calBC und die
Spätbronzezeit schließlich von 1150calBC bis 800/600calBC. Die Insellage
hebt die Region deutlich von den oben betrachteten Fällen ab und ist
Grund für den Sonderweg, den die Entwicklung in Großbritannien und
Irland im Vergleich zum Festland ging. Gleichermaßen bestanden aber auch
vielfältige und tiefgreifende Verbindungen insbesondere im Kontext der
Atlantischen Bronzezeit nach Nordwestfrankreich und ebenfalls entlang
der in die Nordsee entwässernden Flüsse bis nach Zentraleuropa hinein.
Sowohl Irland als auch Großbritannien sind durch ihre lange,
abwechslungsreiche Küstenlinie geprägt. Großbritannien ist naturräumlich
in landwirtschaftlich gut nutzbare Niederungslagen im Süden und Osten,
in England, gegenüber unwirtlicheren Hochebenen im Westen und besonders
im Norden, also in Wales und Schottland gegliedert. Doch auch in Wales
und Schottland stechen einzelne Regionen durch hohes ackerbauliches
Potential hervor. Irlands Küste ist vielerorts hoch und schroff. Sie
umschließt nieder gelegenes Land, dessen Potential für Landwirtschaft
ebenfalls im Süden und Osten am höchsten ist\footnote{\textcite{roberts_britain_2013},
  531-533.}.

Im Gegensatz zu Frankreich, wo Kupfermetallurgie schon im 4. Jahrtausend
bekannt war, traten die frühesten Kupferartefakte in Großbritannien und
Irland erst ein Jahrtausend später um 2500/2400calBC auf. Sie waren Teil
eines Innovationspakets aus Zentraleuropa, das neben
Glockenbecherkeramik, steinernen Armschutzplatten, gestielten und mit
Widerhaken versehenen Pfeilspitzen auch Kupferdolche und goldene
Körpchenanhänger umfasste. In Großbritannien bildete sich dieses
Ereignis in einer neuen lokalen Glockenbecher Tradition ab, die sich
durch Einzelkörperbestattungen auszeichnete. In Irland führte der
Glockenbechereinfluss nicht zu einer so tiefgreifenden Veränderung der
Bestattungssitte, allerdings wurde auch hier Glockenbecherkeramik Teil
von Handlungen im Kontext des Totenrituals: Sie wurde an älteren
Grabanlagen deponiert und in neue Brandgräber in Grabhügeln eingebracht.
Sowohl in Großbritannien als auch in Irland war die bronzezeitliche
Landschaft geprägt von intentional positionierten und aufeinander
ausgerichteten Monumentalanalagen. Das schließt die neuen Grabhügel für
Glockenbecherbestattungen in Großbritannien und Keilgräber in Irland
ebenso ein wie eine Vielzahl unterschiedlicher, nichtfunktionaler Erd-,
Holz- oder Steinanlagen und neolithische Henges und Megalithanlagen. Wie
in Skandinavien entstand in Großbritannien Felskunst, die ebenfalls der
komplexen Rituallandschaft zugerechnet werden muss. Die Grabhügel und
Monumentalanlagen in Süd- und Nordostengland, Zentralschottland sowie
Ostirland und dem Orkney Archipel sind vergleichsweise größer als in
anderen Regionen, strukturell aber ähnlich\footnote{\textcite{roberts_britain_2013},
  533-535.}.

Im ausgehenden 3. und beginnenden 2. Jahrtausend überwogen in Südengland
Körperbestattungen in Grabhügelgruppen nach dem Modell der Wessex
Kultur. Die Gräber sind teilweise reich und mit exotischen Beigaben
ausgestattet. Außerhalb von Südengland herrscht eine große regionale
Variabilität von Bestattungssitten in Großbritannien und Irland: In
Nordengland hatten Bestattungen in Höhlen einige Bedeutung, in
Ostengland Moorbestattungen. In großen Teilen Westschottlands und
Irlands wurden Flachgräberfelder angelegt. Die Gräber sind einfache
Gruben oder mit einer Steinkiste ausgebaut. In Irland wurden zudem auch
ältere Keilgräber wiederbelegt und herausragend große Grabhügel neu
errichtet. Auch hinsichtlich der Beigaben gab es klare regionale
Vorlieben für einzelne Artefaktkategorien: Gagatcolliers in weiblichen
Bestattungen im Norden Großbritanniens, Bernsteincolliers im Süden,
sowie lokal Bronze- oder Feuersteindolche. In den verschiedenen
Kontexten wurde häufig sowohl Brand- als auch Körperbestattung
praktiziert, wobei die Verbrennung der Verstorbenen ab dem Beginn des 2.
Jahrtausends allgemein zu überwiegen scheint. Grabanlagen wurden häufig
über lange Zeiträume genutzt oder deutlich nach deren ursprünglicher
Anlage wieder neu belegt. Auch Ritualanlagen wie Holz-, Stein- und
Menhirkreise sowie Ring Cairns sind regelmäßig mit Bestattungen
assoziiert\footnote{\textcite{roberts_britain_2013}, 535-536.}.

Mitte des 2. Jahrtausends vollzogen sich sowohl in Großbritannien als
auch in Irland, ganz besonders jedoch im Süden und Osten Englands, eine
Reihe von Veränderungen. Die archäologische Quellenlage verschiebt sich
in diesem Zeitfenster zuungunsten von Grab- und Monumentalanlagen.
Stattdessen nimmt die Informationsdichte hinsichtlich Siedlungen und
Subsitenz zu. Jetzt häufiger befestigte Siedlungen mit kreisförmigem
Aufriss, Rundhäuser und regelmäßige, rechtwinklige Ackersysteme mit
begrenzenden Gräben und Zäunen begannen die Kulturlandschaft zu
dominieren. Wilde Pflanzen und Tiere verloren an Bedeutung in der
Subsistenz, stattdessen ist mit der Einführung von Brunnen sowie neuen
Getreiden und Hülsenfrüchten eine Intensivierung der Landwirtschaft zu
beobachten. Stein, Flint und andere organische und inorganische
Werkstoffe wurden als wichtigste Träger der materiellen Kultur von Gold
und Bronze verdrängt. Letztere sind archäologisch allerdings auch
deswegen viel sichtbarer, da sich eine ausgeprägte Deponierungstradition
etablierte. Diese bildet sich jedoch nicht im stark fragmentierten und
regional heterogenen Bestattungsbefund ab. Ab der Mitte des 2. bis zum
Anfang des 1. Jahrtausends war die Brandbestattung die am weitesten
verbreitete Bestattungsform. In vielen Regionen Englands wurden
siedlungsnahe Flachgräberfelder angelegt. Die Beigabenauswahl war fast
vollständig auf Gefäßkeramik beschränkt. In Irland wurden Urnengräber
angelegt, parallel allerdings auch weiter bis ins erste Jahrtausend
Grabhügel errichtet. Die älteren Tradition wurden langsam durch eine
sehr flexible Praxis ersetzt, nach der der Leichenbrand zusammen mit
unverzierten, groben Keramikgefäßen in Flachgräberfeldern, Grabhügeln,
Gräben und sogar Siedlungen beigesetzt wurde. Neben den eigentlichen
Brandbestattungen kommen auch Pseudogräber verhältnismäßig häufig vor.
Sie wurden unter anderem auf Gräberfeldern deponiert und enthalten große
Mengen verbranntes Getreide. In Schottland wurden die Brandbestattungen
meist in ältere Monumente eingebracht. Im frühen ersten Jahrtausend
wurde vielerorts kein archäologisch fassbares Bestattungsritual mehr
praktiziert. Die Bestattungsbefunde beschränken sich auf kleine Mengen
verbrannter menschlicher Knochen, die in Siedlungen, Gräben oder auf
Äckern sporadisch auftreten\footnote{\textcite{roberts_britain_2013},
  537-542.}.

\hypertarget{zusammenfassung-und-beobachtungen}{%
\subsection{Zusammenfassung und
Beobachtungen}\label{zusammenfassung-und-beobachtungen}}

Nach verbreiter Lehrmeinung sind Brandbestattungen in erster Linie ein
Phänomen der Spätbronzezeit. Ein fast universelles Phänomen sind
Flachgräberfelder mit einzeln beigesetzen Urnen. Aus dieser Beobachtung
heraus wird die Spätbronzezeit in Zentral-, Nord- und Westeuropa auch
als \emph{Urnenfelderzeit} bezeichnet. Tatsächlich kommen
Brandbestattungen schon erheblich früher vor.

In Ungarn wird die Brandbestattung schon in der Frühbronzezeit von den
Nagyrév- und Kisapostag Gruppen und in der Mittelbronzezeit von der
Vatya Kultur praktiziert. Auch in Großbritannien treten
Brandbestattungen schon früh und über einen langen Zeitraum parallel zu
Körperbestattungen auf. In der Mittelbronzezeit wird die Verbrennung
dort die dominante Bestattungssitte. Auch in Zentral- Nord und
Südwesteuropa treten Kremationen in geringem Anteil lange vor der
lokalen Spätbronzezeit auf\footnote{\textcite{harding_european_2000},
  111.}.

Grundsätzlich gilt, dass kohärente Bestattungsplätze und Gräberfelder in
einer Belegungsperiode jeweils einheitlich eine Bestattungsform
praktizieren. Das zeigt sich besonders in der Urnenfelderzeit, wo eine
Vielzahl großer und weitreichend untersuchter Gräberfelder in
Zentraleuropa, im Mediterranen Raum, in Frankreich und in Skandinavien
abgesehen von verschwindend wenigen Ausnahmen exklusiv mit Kremationen
belegt sind. Dazu gehören zum Beispiel die Gräberfelder Moravičany
(Mähren) mit 1260 oder Vollmarshausen (Hessen) mit 252 erfassten
Bestattungen. In der Frühen und Mittleren Bronzezeit ist birituelle
Belegung noch erheblich häufiger: Auf dem Tumuluszeitlichen Platz Dolný
Peter (Slowakei) verhalten sich Brand- zu Körperbestattungen in einem
Verhältnis 5:50, in Streda nad Bodrogom (Slowakei) beträgt das
Verhältnis 34:24, wobei weiterhin neun Kenotaphe erfasst wurden. Auf dem
größten und archäologisch wichtigsten Gräberfeld der Mittelbronzezeit
Zentraleuropas in Pitten (Niederösterreich) dominieren Kremationen mit
147:74. Ebenso gibt es aber auch in der Frühbronzezeit Gräberfelder mit
großer Einheitlichkeit wie Gemeinlebarn F (Niederösterreich) wo unter
den 258 erfassten Bestattungen nur eine einzige mit einem
Verbrennungsritual begesetzt wurde und in der Spätbronzezeit
Gräberfelder mit biritueller Belegung wie Przeczyce (Schlesien) mit
einem Verhältnis von 132:727\footnote{\textcite{harding_european_2000},
  112.}.

Von besonderem archäologischen Interessen sind eben jene Kontexte, wo
verschiedene Rituale in größter räumlicher und -- soweit erfassbar --
zeitlicher Nähe zueinander durchgeführt wurden. Systematische
Unterschiede hinsichtlich Beigabenreichtum, Geschlechterschwerpunkt oder
horizontalstratigraphischer Aufteilung von Gräberfeldern bei biritueller
Belegung\ldots{} \textbf{Mal nachsehen!}. Sowohl für das
urnenflederzeitliche Vollmarshausen als auch das frühbronzezeitliche
Gemeinlebarn F deuten sich eine horizontalstratigraphische Trennung nach
Familiengruppen an, deren Nachweis allerdings erst mit genetischen
Mitteln erfolgen könnte, die zum Zeitpunkt der Untersuchung noch nicht
zur Verfügung standen oder im Falle der Brandbestattungen von
Vollmarshausen wahrscheinlich erhaltungsbedingt ausgeschlossen werden
müssen\footnote{\textcite{harding_european_2000}, 114.}.

In Kontakt- und Übergangsbereichen der Bestattungssitten kam es an
verschiedenen Punkten zu überraschenden Überschneidungen der
Ritualausführung. In Periode III der skandinavischen Bronzeit wurde in
Dänemark Leichenbrand in Sarg- und Kistengräbern beigesetzt, die zuvor
für Körperbestattungen verwendet worden waren. In der Champagnen finden
sich Brandbestattungen in Grabgruben, die ausreichend Platz für einen
unverbrannten Körper geboten hätten. Für ein aunjetitzerzeitliches
Gräberfeld in Jeßnitz (Sachsen-Anhalt) rekonstruieren die Ausgräber ein
Ritual, das sekundäre Feuereinwirkung auf schon in Särgen deponierte
Körperbestattungen eingeschlossen hätte. \textbf{Ausbauen!} \footnote{\textcite{harding_european_2000},
  113.}

\newpage
\pby[title={Literatur},segment=\therefsegment,heading=subbibintoc]

\hypertarget{data-analysis}{%
\chapter{Datenauswertung}\label{data-analysis}}

\hypertarget{software-und-daten}{%
\section{Software und Daten}\label{software-und-daten}}

Die vorliegende Arbeit wurde in fünf verschiedenen Teilprojekten
entwickelt und hat mindestens vier Hilfsprojekte hervorgebracht oder
inspiriert:

\textbf{neomod\_textdev}\footnote{\url{https://www.github.com/nevrome/neomod_textdev}}
-- Textproduktion. Der Text der Masterarbeit wurde in R
Markdown\footnote{\url{https://rmarkdown.rstudio.com/} {[}31.07.2018{]}}
mit im bookdown Framework\footnote{\textcite{xie_bookdown_2016}};
\autocite{xie_bookdown_2018}; \url{https://bookdown.org/}
{[}31.07.2018{]}{]} verfasst. Da ausschließlich das Rendern mittels
Pandoc\^{}{[}\url{https://pandoc.org/} {[}31.07.2018{]} in
LaTeX\footnote{\url{https://www.latex-project.org/} {[}31.07.2018{]}}
ins PDF Format vorgesehen war, enthält die Textvorlage auch vereinzelt
LaTeX Ausdrücke. Jeder Commit löst dank Continous Integration mit
Travis\footnote{\url{https://travis-ci.com/} {[}31.07.2018{]}} ein
automatisches Rendern des Texts aus.

\textbf{neomod\_prepresentation}\footnote{\url{https://www.github.com/nevrome/neomod_prepresentation}}
-- Präsentationen über die Inhalte der Masterarbeit. Vor- während und
nach der Arbeit wurden mehrere Präsentation über Planung,
Arbeitsfortschritt und Ergebnisse zusammengestellt. Die Präsentationen
sind jeweils in R Markdown konstruiert, unterscheiden sich aber je
nachdem, ob ein Rendern in HTML oder PDF vorgesehen war.

\textbf{neomod\_analysis}\footnote{\url{https://www.github.com/nevrome/neomod_analysis}}
-- Datenanalyse. Sowohl die Realweltdaten als auch die Daten aus der
Simulation wurden mit R ausgewertet. Dieses Projekt hat bewusst nicht
die Form eines R Pakets, sondern setzt sich aus vielen einzelnen,
teilweise redundanten R Skripten zusammen. Hier werden auch
Ergebnisdaten und Abbildungen gespeichert. Erstere wurden aufgrund ihres
Volumens weitestgehend nicht mit Versionskontrolle dokumentiert und
liegen entsprechend nur in den lokalen Systemen vor, in denen sie
erzeugt wurden. Sie müssen bei Bedarf generiert oder -- im Fall von
Quelldaten -- heruntergeladen werden.

\textbf{popgenerator}\footnote{\textcite{schmid_popgenerator_2018};
  \url{https://www.github.com/nevrome/popgenerator}} --
Populationsgenerator. R Paket zur Konstruktion von Populationsgraphen.

\textbf{gluesless}\footnote{\textcite{clemens_schmid_gluesless_2018};
  \url{https://www.github.com/nevrome/gluesless}} -- Expansionsmodell.
C++ Programm zur Simulation der Ausbreitung von Ideen in einem
Populationsgraphen, wie er von popgenerator erzeugt wird.

\textbf{c14bazAAR}\footnote{\textcite{schmid_c14bazaar_2018};
  \url{https://www.github.com/nevrome/c14bazAAR}} --
\textsuperscript{14}C-Datenbeschaffung. R Paket zum strukturierten
Download von \textsuperscript{14}C-Daten aus verschiedenen
Quelldatenbanken -- unter anderem der hier verarbeiteten Radon-B
Datenbank.

\textbf{neimann1995}\footnote{\url{https://www.github.com/nevrome/neimann1995})}
-- Reproduktion eines Artikels. Verständnisübung entlang eines der
wesentlichen Artikel in der theoretischen Vorbereitung dieser Arbeit.,

\textbf{rdoxygen}\footnote{\url{https://www.github.com/nevrome/rdoxygen}}
-- Doxygen Dokumentation. R Paket um Doxygen Dokumentation für Source
Code in R Paketen zu erstellen.,

\textbf{txtstorage}\footnote{\url{https://www.github.com/nevrome/txtstorage}}
-- Textdatenspeicher. R Paket zur Verwaltung von Austauschdateien mit
einfachen Analyseergebnissen. Dient vor allem dazu, Zähldaten dynamisch
in den Text der Arbeit einzubinden.

Alle diese Projekte wurden und werden unabhängig voneinander mit der
Versionskontrollsoftware Git\footnote{\url{https://git-scm.com/}
  {[}31.07.2018{]}} überwacht, die den Arbeitsfortschritt in vielen
hundert einzeln kommentierten Veränderungspaketen -- ``Commits'' --
dokumentiert. Der Entstehungsprozess ist damit weitreichend
nachvollziehbar, sieht man von Vorüberlegungen und Gesprächen ab, die
keine konkreten Ergebnisse gezeitigt haben. Nach Abschluss von Korrektur
und Revision der Arbeit, werden alle Projekte über die Cloud Plattform
Github \footnote{\url{https://github.com/} {[}31.07.2018{]}} zugänglich
gemacht werden. Die im Text verarbeitete und darüber hinaus gesammelte
Literatur ist in drei thematisch getrennte Sammlungen gegliedert und
über das zotero Webportal einsehbar:

\textbf{cultural\_evolution}\footnote{\url{https://www.zotero.org/groups/2086516/cultural_evolution}}
-- Literatursammlung zu Cultural Evolution und ihren vielen Subthemen
wie Memetik, Cultural Transmission oder Sozial Learning. Geht weit über
eine rein archäologische Perspektive hinaus, umfasst aber gleichzeitig
archäologische Fallstudien ohne großen theoretischen Selbstanspruch.

\textbf{bronze\_age\_burials}\footnote{\url{https://www.zotero.org/groups/2199051/bronze_age_burials}}
-- Archäologische Literatur zur Theorie der Thanatoarchäologie und zur
kulturhistorischen Entwicklung in der Bronzezeit.

\textbf{software\_packages}\footnote{\url{https://www.zotero.org/groups/2211203/software_packages}}
-- Referenzen von wissenschaftlicher Software, die für Datenverarbeitung
sowie Text- und Abbildungsvorbereitung zum Einsatz gekommen ist. Vor
allem R Pakete und C++ Bibliotheken.

Die gesamte Datenanalyse wurde in der Statistikprogrammiersprache
R\footnote{\textcite{RCoreTeamLanguageEnvironmentStatistical2016}}
implementiert. Dabei kam neben Funktionen aus Basispaketen auch eine
große Anzahl von Community-Paketen zum Einsatz, inklusiver mehrerer
selbst entwickelter. Aufgrund der sonst unangemessen großen Menge an
Referenzen, werden im folgenden nur die Pakete genannt, deren Funktionen
tatsächlich unmittelbar aufgerufen wurden und nicht deren oft
umfangreiche Sammlung an Abhängigkeiten. Die Zusammenstellung umfasst
jedoch auch Pakete, die im Laufe der Entwicklung intensiv zum Einsatz
kamen, dann aber aufgrund inhaltlicher oder technischer Veränderungen
ersetzt werden mussten. Über die Entwicklungszeit dieser Arbeit haben
sich viele der verwendeten Pakete ebenfalls weiterentwickelt -- die
angebene Versionsnummer bezieht sich auf diejenige, mit der die in
Version 1.0 dieser Arbeit abgedruckten Ergebnisse erstellt wurden.
Folgende Pakete kamen zum Einsatz zur Text- und Literaturverarbeitung
(bookdown\footnote{\textcite{xie_bookdown_2016};
  \textcite{xie_bookdown_2018}}, citr\footnote{\textcite{aust_citr_2017}},
knitr\footnote{\textcite{xie_dynamic_2015}; \textcite{xie_knitr_2014};
  \textcite{xie_knitr_2018}}, markdown\footnote{\textcite{allaire_markdown_2017}},
rmarkdown\footnote{\textcite{allaire_rmarkdown_2018}}), zur
Datenbeschaffung (c14bazAAR, rnaturalearth\footnote{\textcite{south_rnaturalearth_2017}}),
zur allgemeinen Datenmanipulation (broom\footnote{\textcite{robinson_broom_2018}},
dplyr\footnote{\textcite{wickham_dplyr_2018}}, forcats\footnote{\textcite{wickham_forcats_2018}},
kableExtra\footnote{\textcite{zhu_kableextra_2018}}, pbapply\footnote{\textcite{solymos_pbapply_2018}},
plyr\footnote{\textcite{wickham_split-apply-combine_2011}},
purrr\footnote{\textcite{henry_purrr_2018}}, readr\footnote{\textcite{wickham_readr_2017}},
stringi\footnote{\textcite{gagolewski_r_2018}}, stringr\footnote{\textcite{wickham_stringr_2018}},
tibble\footnote{\textcite{muller_tibble_2018}}, tidyr\footnote{\textcite{wickham_tidyr_2018}},
reshape, reshape2), zur Graphikerstellung (cowplot\footnote{\textcite{wilke_cowplot_2018}},
ggplot2\footnote{\textcite{wickham_ggplot2_2016}}, gridExtra\footnote{\textcite{auguie_gridextra_2017}},
png\footnote{\textcite{urbanek_png_2013}}), für geographische Analysen
(raster\footnote{\textcite{hijmans_raster_2017}}, sf\footnote{\textcite{pebesma_sf_2018}},
sp), für statistische Analysen und Spezialdatenverarbeitung
(Bchron\footnote{\textcite{haslett_simple_2008}}, car\footnote{\textcite{fox_r_2011}},
vegan\footnote{\textcite{oksanen_vegan_2018}}), zur Erstellung von
WebApps und Interaktiven Präsentationen im Shiny Framework
(shiny\footnote{\textcite{chang_shiny_2018}}) und zur allgemeinen Arbeit
und Paketentwicklung in R (devtools\footnote{\textcite{wickham_devtools_2018}},
magrittr\footnote{\textcite{bache_magrittr_2014}}, pryr\footnote{\textcite{wickham_pryr_2018}},
Rcpp\footnote{\textcite{eddelbuettel_extending_2017};
  \textcite{eddelbuettel_rcpp_2011};
  \textcite{eddelbuettel_seamless_2013}}, rlang\footnote{\textcite{henry_rlang_2018}},
roxygen2\footnote{\textcite{wickham_roxygen2_2017}}, testthat\footnote{\textcite{wickham_testthat_2011}}).

Die Expansionssimultion gluesless ist in C++\footnote{\textcite{standard-cpp-foundation_international_2017}}
umgesetzt, um auf dessen höhere Geschwindigkeit und bessere Werkzeuge
für objektorientiertes Programmieren zurückgreifen zu können. Zur
Abbildung des Populationsgraphen kam zunächst die Boost Graph
Library\footnote{\url{https://www.boost.org/doc/libs/1_67_0/libs/graph}
  {[}01.08.2018{]}; \textcite{siek_boost_2002}} (BGL) zum Einsatz, wurde
dann aber aufgrund von Performance-Problemen durch die C++ Bibliothek
des Stanford Network Analysis Project\footnote{\url{https://snap.stanford.edu/}
  {[}01.08.2018{]}; \textcite{leskovec2016snap}} (SNAP) abgelöst.

Die für die Kartengestaltung benötigten Raumdaten, also
Landmasse-Außengrenzen, Administrative Ländergrenzen sowie Flüsse und
Seen, stammen aus dem Natural Earth Projekt\footnote{\url{https://www.naturalearthdata.com}
  {[}02.08.2018{]}}. Verwendet wurden Daten des mittleren
Auflösungssniveaus, das eine Maßstabsperspektive von 1:50.000.000
wiedergeben soll. Die Daten wurden mittels des R Pakets
rnaturalearth\footnote{\textcite{south_rnaturalearth_2017}} direkt in R
heruntergeladen.

\hypertarget{radonb-dataset}{%
\section{Datensatz Radon-B}\label{radonb-dataset}}

Radon-B\footnote{\textcite{kneisel_radon-b_2013}} ist eine öffentlich
verfügbare Datenbank, die einzelne Radiokohlenstoffdatierungen --
\textsuperscript{14}C-Daten -- aus der Bronze- und Frühen Eisenzeit in
Europa sammelt. Sie konzentriert sich auf ein Zeitfenster zwischen 2300
bis 500calBC ab, enthält jedoch auch Daten jenseits dieses Limits. Neben
Radon-B steht mit ihrer Schwesterdatenbank
Radon\autocite{martin_hinz_radon_2012} eine strukturell äquivalente
Sammlung mit einem Schwerpunkt auf neolithischen Daten zur Verfügung.
Jedes Datum ist mit Kerndaten und Metainformationen verknüpft (siehe
Tabelle \ref{tab:radonbparams}). Die Informationen wurden aus einzelnen
Publikationen zusammengetragen und sind teilweise unvollständig,
inkonsistent oder fehlerhaft (siehe auch Kapitel
\ref{source-criticism}).

\begin{table}

\caption{\label{tab:radonbparams}Parameter, die in Radon-B für jedes Datum vorliegen.}
\centering
\fontsize{8}{10}\selectfont
\begin{tabu} to \linewidth {>{\bfseries\raggedleft\arraybackslash}p{0.5em}>{\raggedright\arraybackslash}p{25em}}
\toprule
 & Parameter\\
\midrule
1 & \textbf{Lab Code + Lab Nr.}\newline \textit{z.B. Ua-25144, OxA-1602, HAR-4341}\newline Die allgemeine, individuelle Kennnummer, die sich aus einem Kürzel des Labors, das die Messung durchgeführt hat, und einer fortlaufenden, laborspezifischen Prozessnummer zusammensetzt.\\
\addlinespace \hline \addlinespace
2 & \textbf{BP (Before Present)}\newline Das \textsuperscript{14}C-Alter, das mit der Messung ermittelt wurde in Jahren vor 1950 nach Christus [uncalBP].\\
\addlinespace \hline \addlinespace
3 & \textbf{Std (Standard deviation)}\newline Die messbedingte Standardabweichung des \textsuperscript{14}C-Alters.\\
\addlinespace \hline \addlinespace
4 & \textbf{$\delta$\textsuperscript{13}C}\newline Ein Maß für das Isotopenverhältnis des stabilen Isotops \textsuperscript{13}C / \textsuperscript{12}C zwischen der Probe und einem Standard in Promille [\textperthousand].\\
\addlinespace \hline \addlinespace
5 & \textbf{$\delta$\textsuperscript{13}C Std}\newline Die Standardabweichung des $\delta$\textsuperscript{13}C-Werts.\\
\addlinespace \hline \addlinespace
6 & \textbf{Sample Material}\newline \textit{z.B. charcoal, bone, seed}\newline Oberkategorie des Probenmaterials.\\
\addlinespace \hline \addlinespace
7 & \textbf{Sample Material Comment}\newline \textit{z.B. hazel, oak, barley, boar}\newline Nähere Kategorisierung und Artenzuordnung des Probenmaterials.\\
\addlinespace \hline \addlinespace
8 & \textbf{Feature Type}\newline \textit{z.B. settlement (house), rockshelter,     Grave (cremation)}\newline Befund bzw. Fundplatzkategorie, aus dem das Probenmaterial stammt.\\
\addlinespace \hline \addlinespace
9 & \textbf{Feature}\newline \textit{z.B. House I, from a mass of burnt debris...}\newline Bezeichnung des Befunds in der Grabungsdokumentations des Fundplatzes.\\
\addlinespace \hline \addlinespace
10 & \textbf{Culture}\newline \textit{z.B. Late Bronze Age, Únětice, Nordic Bronze Age}\newline Allgemeine, archäologische Kultur- oder Phasenzuordnung des Probenkontexts.\\
\addlinespace \hline \addlinespace
11 & \textbf{Phase}\newline \textit{z.B. Nagyrév Group, Mierzanowice, Period III}\newline Präzisere Kultur- oder Kontextansprache.\\
\addlinespace \hline \addlinespace
12 & \textbf{Site}\newline \textit{z.B. La Croix-Saint-Ouen, Stedten, Byneset}\newline Bezeichnung des Fundplatzes, aus dem die Probe stammt.\\
\addlinespace \hline \addlinespace
13 & \textbf{Country}\newline \textit{z.B. Germany, France, Poland}\newline Land in dem der Fundplatz liegt.\\
\addlinespace \hline \addlinespace
14 & \textbf{Country Subdivision}\newline \textit{z.B. Baden-Württemberg, Surrey, Greater Poland}\newline Zugehörige administrative Region innerhalb des Landes.\\
\addlinespace \hline \addlinespace
15 & \textbf{Literature}\newline Literaturreferenz auf die Publikation aus der die Informationen über das Datum entnommen wurden.\\
\addlinespace \hline \addlinespace
16 & \textbf{Comment}\newline Freitextkommentarfeld mit Zusatzinformationen zu dem einzelen Datum.\\
\bottomrule
\end{tabu}
\end{table}

\hypertarget{data-prep-and-segmentation}{%
\subsection{Datenvorbereitung und
Gliederung}\label{data-prep-and-segmentation}}

Eine hinsichtlich der Variablenauswahl etwas reduzierte\footnote{Für
  einen Überblick, welche Variablen heruntergeladen und wie umbenannt
  werden:
  \url{https://github.com/ISAAKiel/c14bazAAR/blob/master/data-raw/variable_reference.csv}},
aber hier ausreichende Version von Radon-B wurde mittels des R Pakets
c14bazAAR direkt in R bezogen. Dieser Ausgangsdatensatz enthielt alle zu
diesem Zeitpunkt {[}15.07.2018{]} öffentlichen Einträge: 11.048 Daten
von 2.766 Fundplätzen aus 48 Ländern. Der erste
Datenverarbeitungsschritt war das Entfernen aller Daten ohne
Altersinformation und aller Daten außerhalb der theoretischen Reichweite
der Kalibrationskurve (71-46401calBP) (10956 Daten verblieben). Zur
Kalibration wurde das R Paket Bchron\footnote{\textcite{haslett_simple_2008}}
und die darin enthaltenen Version es IntCal13 Datensatzes\footnote{\textcite{reimer_intcal13_2013}}
verwendet. Bchron berechnet das kalibrierte Alter mittels Numerischer
Integration\footnote{\url{https://github.com/andrewcparnell/Bchron/blob/master/R/BchronCalibrate.R}
  {[}02.08.2018{]}} und liefert für jedes Datum eine normierte
Wahrscheinlichkeitskurve. Alter mit Wahrscheinlichkeiten unterhalb eines
Schwellwerts von \(1\mathrm{e}{-6}\) wurden abgeschnitten und Alter
innerhalb des \(2\sigma\) Wahrscheinlichkeitsbereichs gesondert
markiert. Die so erhaltenen, unterschiedlich wahrscheinlichen,
kalibrierten Alter für jedes einzelne Datum wurden ab hier von calBP in
calBC umgerechnet, um üblichen archäologischen Konventionen und dem
allgemeinen Sprachgebrauch zu entsprechen. Um ein Subset des so
vorbereiteten Gesamtdatensatzes zu erzeugen, das die Anforderungen der
Fragestellung erfüllt, wurde er auf all jene Daten reduziert, die in
ihrem \(2\sigma\) Bereich mindestens ein Alter im Zeitfenster
800-2200calBC (1401 Jahre) vorweisen können (7543 Daten). Radon-B stellt
in der Variable \emph{Feature Type} (\emph{sitetype} in c14bazAAR)
teilweise kategorisierte Informationen zum Befundkontext jedes Datums
zur Verfügung: Fragestellungsrelevant sind die Kategorien
\emph{cemetery}, \emph{Grave}, \emph{Grave (mound)}, \emph{Grave (mound)
inhumation}, \emph{Grave (mound) cremation}, \emph{Grave (flat)},
\emph{Grave (flat) inhumation}, \emph{Grave (flat) cremation},
\emph{Grave (cremation)} und \emph{Grave (inhumation)} (2361 Daten).
Statt der Variablen \emph{Feature Type} wurden mittels Pattern Matching
zwei neue Variablen mit jeweils drei Werten geschaffen:
\emph{burial\_type} mit den Kategorien \emph{inhumation},
\emph{cremation} und \emph{unknown} sowie \emph{burial\_construction}
mit den Kategorien \emph{mound}, \emph{flat} und \emph{unknown}. Die
Fragestellung erfordert es auch, alle Daten ohne Raumbezug, also ohne
Koordinateninformation, zu entfernen (2336 Daten). Nach diesen
Arbeitsschritten lässt sich der Hauptausgangsdatensatz als Tabelle mit
2336 Zeilen und 15 Spalten beschreiben, darunter die hier wesentlichen
mit Angaben zu Labornummer, Koordinaten, Dichteverteilung des
kalibrierten Alters und \emph{burial\_type} sowie
\emph{burial\_construction}. Eine Karte der so vorbereiteten Gräberdaten
zeigt die hohe Heterogenität der Datendichte in verschiedenen Regionen
Europas (siehe Abbildung \ref{fig:general-map}.

\begin{figure}
\includegraphics{../neomod_analysis/figures_plots/general_maps/general_map} \caption[Übersichtskarte der \textsuperscript{14}C Daten an bronzezeitlichen Gräbern in Europa]{Übersichtskarte der \textsuperscript{14}C Daten an bronzezeitlichen Gräbern in Europa. Die Karte zeigt Daten aus einem  Zeitfenster von 2200 bis 800calBC. Einzelne Daten liegen außerhalb des gewählten Kartenausschnitts. Jedes Datum ist nach seinen Kontextinformationen hinsichtlich der Variablen \textit{burial type} und \textit{burial construction} in Form und Farbe markiert.}\label{fig:general-map}
\end{figure}

Abbildung \ref{fig:general-map-research-area} zeigt das
Untersuchungsareals dieser Arbeit. Es folgt keinen natürlichen oder
kulturellen Grenzen, sondern wurde rein künstlich in Anbetracht der
räumlichen Verteilung der zusammengestellten \textsuperscript{14}C-Daten
festgelegt. Auf Grundlage visueller Analyse der Punktdichte schien es
angemessen, ein rechteckiges Areal aufzuspannen. Die Projektion, die
dieser Festlegung, allen Kartierungen und auch der Regionengliederung
zugrunde liegt ist bewusst mit EPSG:102013\footnote{\url{https://epsg.io/102013}
  {[}02.08.2018{]}} -- Europe Albers Equal Area Conics gewählt, da diese
auch auf kontinentalem Maßstab und bei Landmassen in betonter Ost-West
Ausdehnung ein hohes Maß an Flächentreue gewährleistet\footnote{\textcite{snyder_map_1987},
  98-99.}. Das ist eine wichtige Eigenschaft für die Definition von
vergleichbaren, räumlichen Untersuchungseinheiten. 1894 der 2336 oben
ausgewählten Daten stammen aus dem Rechteckareal.

\begin{figure}
\includegraphics{../neomod_analysis/figures_plots/general_maps/general_map_research_area} \caption[Karte mit den Grenzen des Untersuchungsareals]{Karte mit den Grenzen des Untersuchungsareals. Wie Abbildung \ref{fig:general-map}.}\label{fig:general-map-research-area}
\end{figure}

Der nach oben beschriebenem Vorgehen zusammengestellte Arbeitsdatensatz
umfasst also 1894 Einträge aus der Radon-B Datenbank. Die effektive
Anzahl an \textsuperscript{14}C Daten, die diese Einträge wiedergeben,
ist jedoch geringer: Eine Zählung der Labornummern ergibt 1831
individuelle Werte. Diese Diskrepanz ergibt sich aus Einträgen mit
keiner (\emph{n/a-n/a}) oder unvollständiger (z.B. \emph{MAMS-n/a},
\emph{Gd-n/a}, \emph{Ke-n/a}) Labornummer sowie Daten die mehrfach in
die Datenbank eingegeben wurden (z.B. \emph{OxA-29003},
\emph{GrN-10754}, \emph{BRAMS-1217}). Letzteres betrifft 46 Einträge in
dieser Datenauswahl, die ein und dasselbe \textsuperscript{14}C Datum
zwei- oder mehrfach repräsentieren. Die Anzahl an Gräbern, die durch die
Einträge repräsentiert werden ist noch geringer: Für 498 Einträge gilt,
dass die ihnen zugehörige Kombination aus Fundplatz und Befund von
mindestens einem weiteren Eintrag abgedeckt wird. Der wichtigste Grund
dafür ist, dass für ein Grab häufig mehrere \textsuperscript{14}C Daten
in Auftrag gegeben werden. Die Abweichungen zwischen Einträgen und
\textsuperscript{14}C Daten sowie \textsuperscript{14}C Daten und
Gräbern scheinen also auf den ersten Blick schwerwiegend zu sein.
Nichtsdestoweniger wurde in einem ersten Durchlauf der Berechnungen von
einer Korrektur abgesehen, und tatsächlich waren die Auswirkungen dieses
Versäumnisses auf die relative zeitliche und räumliche Entwicklung der
Hauptuntersuchungsparameter erstaunlich gering. Die Über- und
Unterbetonung der Verhältnisse durch beide Fehler zeigte keine übermäßig
starke Tendenz hinsichtlich der Variablen \emph{burial type} (Absolute
Werteverteilung innerhalb der Dubletten: cremation: 170, inhumation:
104, unknown: 224) und \emph{burial construction} (flat: 82, mound: 86,
unknown: 330) und auch zeitlich und räumlich waren die Abweichungen
scheinbar weitestgehend zufällig verteilt. In dieser Datenkombination
also eher ein statistisches Rauschen. Da allerdings keine Garantie
besteht, dass das auch für andere Datenkombinationen in Zukunft gelten
wird, schien es sinnvoll einen Algorithmus zu entwickeln, um die Fehler
zumindest teilweise auszugleichen.

Erstere Abweichung zwischen der Menge an Einträgen und den tatsächlich
vorhanden \textsuperscript{14}C Daten ist durch unvollständige
Datenpublikation und Fehleingabe bedingt. Sie wird sich mit der
stückweisen Verbesserung des Radon-B Datensatzes in Zukunft hoffentlich
selbst lösen. Da sich die Einträge jenseits der Labornummer häufig
unterscheiden, bleibt im Augenblick nur die Diskussion von Einzelfällen
oder die Inkaufnahme von geringfügigem Datenverlust bei einer
automatisierten Lösung. Um die reproduzierbare Natur dieser Arbeit nicht
in Frage zu stellen, kamen Werkzeuge aus dem c14bazAAR Paket zum
Einsatz, die Einträge mit äquivalenter Labornummer automatisch
zusammenführen. Abweichende Einträge in den Ausgangsdaten werden dabei
als unbekannte Werte behandelt. Von den 1848 oben zusammengestellten
Einträgen blieben 1.848 erhalten.

Die zweite Mengenabweichung zwischen \textsuperscript{14} Daten und
Gräbern ist schwerwiegender, da sie immerhin nahezu ein Drittel der
Einträge betrifft und keine Aussicht besteht, dass sich dieses Problem
mit einer Verbesserung der Datenlage lösen wird: Sie ist Teil der
Semantik des Datensatzes. Da für einzelne Gräber mehrere (bis zu 8)
\textsuperscript{14}C Daten vorliegen, muss für diese jeweils ein
individuelles chronologisches Modell definiert werden, dass alle Daten
vereint. Für einen großen Teil der Gräber könnte zwar angenommen werden,
dass die Datierungen sich tatsächlich nur auf ein einzelnes, zu einem
bestimmten Zeitpunkt in einen geschlossenen Befund eingebrachtes
Individuum beziehen, das geht allerdings nicht aus den in Radon-B
enthaltenen Metainformationen hervor. Stattdessen muss in Betracht
gezogen werden, dass auch Kollektivgräber mit langer Belegungszeit und
vielen einzelnen Bestattungen mit nur einer Befundbezeichnung
charakterisiert wurden. Die Befundangabe für manche Einträge ist sehr
unpräzise (z.B. \emph{Kollektivgrab}, \emph{Einzelgrab}, \emph{from ring
ditch}) und es ist nicht ersichtlich, ob die Daten tatsächlich von einer
einzelnen Bestattung stammen. Die Herausforderung besteht also darin,
auf Grundlage der vorhandenen Daten einerseits das übermäßige Gewicht
zeitlich scharf umgrenzter Gräber mit einzelnen, mehrfach datierten
Bestattungen zu mindern, und andererseits der diachronen Entwicklung in
über lange Zeit genutzten Grabanlagen gerecht zu werden. Um das zu
erreichen wurden die 486 nach der oben durchgeführten Entfernung der
Labornummer-Dubletten verbliebenen Mehrfacheinträge in einem ersten
Schritt weiter auf jene Befundtermini reduziert, die tatsächlich einen
einzelnen Grabbefund meinen könnten. Das sind vor allem jene 252 mit
numerischen Zeichen (z.B. \emph{Bef. 530 Doppelbestattung}, \emph{Grab,
Bef. 35635}, \emph{Objekt 461}), weswegen die Auswahl auf sie beschränkt
wurde. Innerhalb dieser Auswahl wurden nach Fundplatz und Befund
gegliederte Gruppen angelegt und deren kalibrierte Dichteverteilungen
zusammengeführt. Notwendig wäre dafür eigentlich ein individuelles
chronologisches Modell für jede dieser Datengruppen. Stattdessen wurden
die einzelnen Dichteverteilungen addiert und auf die das Gesamtmaximum
bezogen normiert. Die Information, ob ein Alter zum \(2\sigma\) Bereich
eines Datums gehört, wurde immer dann als wahr angenommen, wenn es im
\(2\sigma\) mindestens eines Datums liegt. Aus der Perspektive der
\textsuperscript{14}C Datenverarbeitung ist dieses Vorgehen nicht
korrekt, angesichts der zugrundeliegenden Fragestellung und der
Herausforderungen des Datensatzes jedoch sinnvoll: Jeder Eintrag im
Datensatz soll einen Ort und einen Zeitraum definieren, in dem die mit
ihm assoziierten Angaben für die Primärvariablen \emph{burial\_type} und
\emph{burial\_construction} auftraten. Durch die Reduktion der Daten auf
einen einzelnen Eintrag wird die Überbetonung dieser Information
vermieden. Gleichzeitig wird aber auch der mitunter langen
Belegungsdauer eines durch den einzelnen Eintrag repräsentierten
Grabmonuments Rechnung getragen. Eine Verbesserung der Metainformationen
zu jedem Datum (z.B. relativchronologische Position zu anderen Daten des
selben Grabes) würde eine wesentliche Verbesserung dieses Algorithmus
ermöglichen. Nach der vorgenommenen Reduktion verblieben 1704 jeweils
befundspezifische Einträge.

Innerhalb des Untersuchungsareals wurden künstliche Regionen abgegrenzt,
um die zeitliche und räumliche Entwicklung der Variablen
\emph{burial\_type} und \emph{burial\_construction} in sinnvollen und
der verfügbaren Datenmenge angemessenen Einheiten beobachten zu können
(siehe Abbildung \ref{fig:general-map-research-area-regions}). Der
Prozess der Erstellung dieser Regionen war semiautomatisch und darauf
angelegt kulturelle Makroregionen der Bronzezeit zumindest
näherungsweise abzubilden. Dafür stand mir auch eine unpublizierte,
händisch entworfene Regionengliederung von Jutta Kneisel und Oliver
Nakoinz als Vorlage zur Verfügung. In den Grenzen des
Untersuchungsareals wurde ein Raster von Punkten angelegt, die jeweils
als Zentrum einer der geplant runden Regionen dienen sollten. Dieses
Raster wurde manuell so angepasst, bis es sich den Zentrumspunkten
wesentlicher geographischer, kultureller und/oder
forschungsgeschichtlicher Einheiten annährte. Die Distanz zwischen den
Zentren beträgt in dieser Konfiguration 400km (im Bezugssystem der
EPSG:102013 Projektion). In einem weiteren Schritt wurden kreisförmige
Regionen um die Zentrumspunkte aufgebaut. Der Kreisradius wurde nach
Augenmaß mit 240km so gewählt, dass möglichst alle bekannten
\textsuperscript{14}C Daten (also damit auch Gräber) in mindestens einer
Region verortet sind. Das Überlappen von Regionen wurde dabei in Kauf
genommen. Andere Regionendefinitionen anhand alternativer geometrischer
Formen (Rechtecke, Hexagone), nach der Dichteverteilung von Fundpunkten,
anhand sich zeitlich wandelnder, archäologisch erfasster, kultureller
Einheiten sind denkbar und sollten bei zunehmender Datenverfügbarkeit in
Zukunft evaluiert werden. Das gilt auch hinsichtlich der Größe der
Einheiten, die aufgrund der diachron geringen Datenmenge sehr groß
gewählt werden mussten. Nur aus den acht Regionen, die in Abbildung
\ref{fig:general-map-research-area-regions} definiert werden, sind
ausreichend \textsuperscript{14}C Daten an Gräbern bekannt, um eine
nähere Betrachtung zu rechtfertigen. Der Schwellwert dafür wurde mit 60
Gräbern jedoch sehr niedrig angelegt um das effektive Untersuchungsareal
nicht noch weiter verkleinern zu müssen. Die Benennung der Regionen war
an den modernen, administrativen Einheiten orientiert, die die Kreise im
wesentlichen einschließen (siehe Kapitel \ref{representativity} für eine
nähere Beschreibung der geographischen Ausdehnung der Regionen). Ihre im
folgenden stets eingehaltene, geographische Reihenfolge von Südost nach
Nordwest soll die Lesbarkeit- und Interpretierbarkeit von Abbildungen
erhöhen. Mit angegeben ist die Menge an Gräbern pro Region:
\emph{Southeastern Central Europe} (70), \emph{Poland} (134),
\emph{Southern Germany} (213), \emph{Northeastern France} (64),
\emph{Northern Germany} (475), \emph{Southern Scandinavia} (209),
\emph{Benelux} (284), \emph{England} (113). Durch die Regionengliederung
verringerte sich das effektive Untersuchungsareal weiter. Von den 1704
Gräbern im Rechteckareal verblieben 1562. Das ist der Ausgangsdatensatz
auf dem alle folgenden Beobachtungen beruhen.

\begin{figure}
\includegraphics{../neomod_analysis/figures_plots/general_maps/general_map_research_area_regions} \caption[Karte mit künstlichen Untersuchungsregionen]{Karte mit künstlichen Untersuchungsregionen. Wie Abbildung \ref{fig:general-map-research-area}. Die Regionen sind farblich markiert.}\label{fig:general-map-research-area-regions}
\end{figure}

\hypertarget{descriptive-data-analysis}{%
\subsection{Deskriptive Analyse}\label{descriptive-data-analysis}}

Aus dem Areal der kreisförmigen, artifiziellen Großregionen, die für
diese Arbeit festgelegt wurden (siehe Abbildung
\ref{fig:general-map-research-area-regions}) liegen in Radon-B
Informationen zu mindestens 1562 Gräbern auf Grundlage von 1701
\textsuperscript{14}C Daten vor (zur Datenauswahl und -vorbereitung
siehe Kapitel \ref{data-prep-and-segmentation}). Geht man davon aus,
dass die Eingaben in Radon-B korrekt sind, dann stammen die
\textsuperscript{14}C Daten von 454 Fundplätzen. Zu den Daten sind 41
verschiedene Periodenbegriffe und 25 archäologische Kulturen
dokumentiert, diese Information ist jedoch aufgrund der Datensituation
sinnvoll auswertbar (siehe Kapitel \ref{source-criticism}). 1160 Daten
wurden an Probenmaterial von Knochen und Zähnen (169 davon verbrannt),
367 von Holz und Holzkohle, ein kleiner Teil (20) von sonstigen
Materialien wie Nüssen, Harz oder Pech gemessen. Für die restlichen 154
Daten ist keine Materialangabe hinterlegt. Hinsichtlich der Variablen
\emph{burial\_type} und \emph{burial\_construction} gelten die in
Tabelle \ref{tab:dprcrosstab} dargestellte Verhältnisse.

\begin{table}[!h]

\caption{\label{tab:dprcrosstab}Kreuztabelle}
\centering
\fontsize{8}{10}\selectfont
\begin{tabu} to \linewidth {>{\raggedright}X>{\raggedleft}X>{\raggedleft}X>{\raggedleft}X}
\toprule
  & flat & mound & unknown\\
\midrule
cremation & 66 & 96 & 241\\
\addlinespace
inhumation & 291 & 95 & 224\\
\addlinespace
unknown & 12 & 117 & 559\\
\bottomrule
\end{tabu}
\end{table}

Von besonderem Interesse für die vorliegende Arbeit ist eine diachrone
Perspektive in der Bestattungssittenentwicklung. Schon eine Kartierung
der Gräber in Zeitschritten von 200 Jahren (siehe Abbildung
@ref\{fig:general-map-research-area-timeslices\}) offenbart generelle
Trends hinsichtlich \emph{burial\_type} und \emph{burial\_construction}
in Früh-, Mittel- und Spätbronzezeit: Flachgrab -- Hügelgrab --
Flachgrab und Körpergrab -- Brandgrab. Diese Beobachtung kann mittels
des erstellten Datensatzes erheblich präzisiert sowie räumlich- und
zeitlich explizit gemacht werden.

\begin{landscape}
\begin{figure}
\includegraphics{../neomod_analysis/figures_plots/general_maps/general_map_research_area_timeslices} \caption[Einzelkarten nach Zeitschritten]{Einzelkarten nach Zeitschritten. Chronologische und räumliche Entwicklung der \textsuperscript{14}C Datenverteilung. Einzelabbildungen wie Abbildung \ref{fig:general-map-research-area}, jeweils aber nur mit den Daten, deren $2\sigma$ Wahrscheinlichkeitsbereich das entsprechende calBC Datum (2200calBC, 2000calBC, 1800calBC, ...) überspannt. Die Transparenz bzw. Opazität spiegelt die normierte Wahrscheinlichkeit des Auftretens eines Datums für genau diesen Zeitpunkt wieder.}\label{fig:general-map-research-area-timeslices}
\end{figure}
\end{landscape}

Entscheidend ist hierfür nicht unbedingt, wann und wo exakt welche Art
Grab angelegt wurde. Stattdessen soll aus dieser Information eine
quantitative Beschreibung zur Verbreitung und Dominanz von Ideen
extrahiert werden. Zur Erstellung dieses Proxies wurde die Annahme
getroffen, dass eine Idee zu einem bestimmten Zeitpunkt in einer Region
dann als anwesend gewertet werden muss, wenn ein Grab in dieser Region
existiert, dessen \(2\sigma\) Wahrscheinlichkeitsbereich ermittelt aus
einem oder mehreren \textsuperscript{14}C Daten diesen Zeitpunkt
umfasst. Ein Beispiel: Die Idee ``Körperbestattung'' muss 1447calBC
anwesend gewesen sein, da dieses Jahr im \(2\sigma\) Bereich des
\textsuperscript{14}C Datums NZA-32497 liegt, das für die
Körperbestattung I2639/25217 vom Fundplatz Amesbury Down angefertigt
wurde. Liegen mehrere Daten aus einer Region für ein Jahr vor, dann kann
das Auftreten der verschiedenen Ausprägungen von \emph{burial\_type} und
\emph{burial\_construction} gezählt werden. Tut man das für alle Jahre
mit allen Ausprägungen, dann ergeben sich sechs aufschlussreiche
Zeitreihen, die sich jahrweise sinnvoll zur Gesamtzahl der Beobachtungen
addieren (für \emph{burial\_type} siehe Abbildung
\ref{fig:development-amount-regions-burial-type}), für
\emph{burial\_construction} Abbildung
\ref{fig:development-amount-regions-burial-construction}). Aus den
Ausprägungsmengen lässt sich auch das jeweilige Verhältnis der Ideen in
jedem Jahr in jeder Region berechnen (für \emph{burial\_type} Abbildung
\ref{fig:development-proportions-regions-burial-type}), für
\emph{burial\_construction} Abbildung
\ref{fig:development-proportions-regions-burial-construction}). Dabei
wurden die Gräber, für die keine Information zu den Primärvariablen
vorliegt (\emph{unknown}) ignoriert. Die Entwicklung der Verhältnisse
ist die für die Fragestellung dieser Arbeit interessantere Perspektive.
Eine Betrachtung der Stichprobengröße aus der die Proportionen
abgeleitet wurden, ist aber unumgänglich um die Aussagekraft in einer
Region und in einem Zeitfenster beurteilen zu können. Für manche
Regionen und Zeiträume liegen sehr wenige Gräber vor. Die Ergebnisse
aller folgenden Analysen müssen entsprechend mit Vorsicht gelesen
werden. Sie könnten sich bei zunehmender Datenmenge verändern.
Nichtsdestoweniger erlauben die Abbildungen
\ref{fig:development-proportions-regions-burial-type}) und
\ref{fig:development-proportions-regions-burial-construction} schon
jetzt bemerkenswerte, wenn auch teilweise objektiv falsche
Interpretationen (für eine Bewertung siehe \ref{representativity}):

\begin{figure}[!t]

{\centering \includegraphics[width=.9\textwidth,]{../neomod_analysis/figures_plots/development/development_amount_regions_burial_type} 

}

\caption[Entwicklung der Datenmenge für \textit{burial type} in den Regionen]{Entwicklung der Datenmenge für \textit{burial type} in den Regionen. Für jedes Jahr wird die Menge an Daten bestimmt, zu deren $2\sigma$ Wahrscheinlichkeitsbereich es gehört. Die absolute Datenmenge wird getrennt nach den Ausprägungen der Variable \textit{burial type} aufaddiert und nach deren Verteilung abgebildet. Die Y-Achse ist auf maximal 80 sich überschneidende Daten limitiert um zu vermeiden, dass die übergoße Datenmenge aus \textit{Northern Germany} die Angaben für die anderen Regionen unlesbar macht.}\label{fig:development-amount-regions-burial-type}
\end{figure}

\begin{figure}[!t]

{\centering \includegraphics[width=.9\textwidth,]{../neomod_analysis/figures_plots/development/development_amount_regions_burial_construction} 

}

\caption[Entwicklung der Datenmenge für \textit{burial construction} in den Regionen]{Entwicklung der Datenmenge für \textit{burial construction} in den Regionen. Wie Abbildung \ref{fig:development-amount-regions-burial-type}, allerdings für die Variable \textit{burial construction}. Die Gesamtdatenmenge ist die selbe, die innere Verteilung aufschlussreich unterschiedlich im Vergleich zu \textit{burial type}.}\label{fig:development-amount-regions-burial-construction}
\end{figure}

\begin{figure}[!t]

{\centering \includegraphics[width=.9\textwidth,]{../neomod_analysis/figures_plots/development/development_proportions_regions_burial_type} 

}

\caption[Entwicklung der Ideenproportionen für \textit{burial type} in den Regionen]{Entwicklung der Ideenproportionen für \textit{burial type} in den Regionen. Im Gegensatz zu den Abbildungen \ref{fig:development-amount-regions-burial-type} und \ref{fig:development-amount-regions-burial-construction}, die die absolute Datenmengen wiedergeben, zeigen die Abbildungen \ref{fig:development-proportions-regions-burial-type} und \ref{fig:development-proportions-regions-burial-construction} die relative, proportionale Entwicklung der Varianten in den Primärvariablen. Leerstellen markieren Zeitfenster und Regionen, aus denen keine Informationen vorliegen.}\label{fig:development-proportions-regions-burial-type}
\end{figure}

\begin{figure}[!t]

{\centering \includegraphics[width=.9\textwidth,]{../neomod_analysis/figures_plots/development/development_proportions_regions_burial_construction} 

}

\caption[Entwicklung der Ideenproportionen für \textit{burial construction} in den Regionen]{Entwicklung der Ideenproportionen für \textit{burial construction} in den Regionen. Wie Abbildung \ref{fig:development-proportions-regions-burial-type}. Hier die Proportionen der Varianten \textit{mound} und \textit{flat} der Variablen \textit{burial construction}.}\label{fig:development-proportions-regions-burial-construction}
\end{figure}

Definitiv gab es im Laufe der Bronzezeit in Europa einen Trend weg von
der Körperbestattung hin zur Brandbestattung. Um 2200 waren
Brandbestattungen in Polen, Süddeutschland, Nordostfrankreich und
Norddeutschland nahezu unbekannt. Im Nordwesten, in Großbritannien und
im Benelux Raum, und im Südosten, in Österreich und Tschechien, gab es
jedoch frühe Brandbestattungstraditionen. Diese Kontexte könnten als
Ursprungsgebiete des später omnipräsenten Phänomens diskutiert werden.
Während Brandbestattungen sowohl in Österreich und Tschechien als auch
im Benelux Gebiet im Laufe der Frühbronzezeit an Bedeutung verloren und
erst in der Mittelbronzezeit wieder gewannen, stieg ihr Anteil in
England stetig. Körper- und Brandbestattung hielten sich hier lange die
Waage. Ähnlich verhielt es sich in Südskandinavien, wo der Anteil an
Brandbestattungen bis zur Spätbronzezeit allerdings wesentlich geringer
ausfiel. In Polen und Norddeutschland setzte sich Brandbestattung mit
dem Beginn er Spätbronzezeit relativ plötzlich und vollständig durch. In
Nordostfrankreich und Norddeutschland vollzieht sich dieser Wandel schon
in der Mittelbronzezeit. In beiden Kontexten spielten Körperbestattungen
auch danach eine wesentliche Rolle.

Hinsichtlich der Frage nach Grabüberhügelung ist das Bild erheblich
heterogener. In Österreich und Tschechien waren Flachgräber bis in die
Spätbronzezeit die dominante Bestattungsform, Überhügelung gewann aber
ab der Mittelbronzezeit an Relevanz. In Polen hielten sich Flach- und
Hügelgrab bis in die Mittelbronzezeit die Waage, dann setzten sich
Flachgräber durch. Die Datenmenge aus Süddeutschland und
Nordostfrankreich ist außerordentlich gering: Glaubt man der Stichprobe,
dann vollzog sich in Süddeutschland am Beginn der Mittelbronzezeit ein
plötzlicher, radikaler Wechsel von der Bestattung in Flach- zu
Hügelgräbern. In Nordostfrankreich hätte es in der Bronzezeit keine
Flachgräber gegeben. Norddeutschland durchlief eine Entwicklung von der
Dominanz von Flachgräbern in der Frühbronzezeit, einer kurzen Phase
zunehmender Überhügelung in der Mittelbronzezeit gefolgt von erneuter
Dominanz der Flachgrabsitte in der Spätbronzezeit. Eine ähnliche
Entwicklung deutet sich in Südskandinavien an: Flachgräber überwogen
deutlich, wurden über Jahrhunderte aber zunehmend -- fast vollständig --
von Hügeln ersetzt, bis die Beisetzung in Flachgräbern am Ende der
Bronzezeit wieder häufiger wurde. Im Benelux Raum waren Hügelgräber
durchgehend dominant, in der Früh- und Spätbronzezeit traten Flachgräber
jedoch ebenfalls in signifikantem Umfang auf. In England waren
Flachgräber ein kurzes Phänomen in der Frühbronzezeit, das später nicht
wieder auftrat.

\hypertarget{source-criticism}{%
\subsection{Quellenkritik}\label{source-criticism}}

Die Verwendung des Radon-B Datensatzes im Kontext von Methode und
Fragestellungen dieser Arbeit ist aus mehreren Gründen problematisch.
Eine Quellenkritik muss dabei auf verschiedenen Ebenen ansetzen: Bei der
technischen und inhaltlichen Umsetzung der Datenbank, bei der
Repräsentativität der im Datensatz vertretenen Stichprobe und
schließlich bei der Frage, ob diese Art Daten für eine Betrachtung von
Kulturvorgängen im Allgemeinen und im Rahmen der Cultural Evolution
Theorie und Terminologie geeignet ist.

\hypertarget{dateneingabe}{%
\subsubsection{Dateneingabe}\label{dateneingabe}}

Die Radon-B Datenbank hat inhaltliche Unzulänglichkeiten, die sich vor
allem aus inkonsistenter Dateneingabe infolge mangelnder Vorgaben in
Freitextfeldern ergeben. Eine systematische Lösung dieser Probleme wäre
sehr aufwändig, da sie die individuelle, nicht automatisierbare
Reevaluation eines großen Teils der Einträge erfordern würde.

\begin{itemize}
\tightlist
\item
  Viele Einträge sind unvollständig. Die Unvollständigkeiten rühren
  sicher teilweise aus Mängeln in den ausgewerteten Publikationen: Der
  \(\\delta\)\textsuperscript{13}C Wert, Material und Spezies des
  beprobten Überrests oder der kulturhistorische Kontext sind mitunter
  nicht bekannt oder nicht publiziert.
\item
  Mehrere eigentlich kategorisierbare Kontextvariablen sind mit einer
  inkosistenten Kategorienauswahl versehen oder nicht sinnvoll
  hierarchisiert. Während für manche Variablen (Feature Type, Sample
  Material) eine Kategorisierung bewusst festgelegt worden zu sein
  scheint, die dann nur in wenigen Fällen durch freie Einträge erweitert
  wurde, scheint bei anderen (Culture, Phase) keine klare Vorgabe zu
  bestehen, welche semantischen Inhalte in welcher Struktur dort
  eingefügt werden sollen.
\item
  Viele Einträge sind mehrsprachig -- vor allem Englisch und Deutsch --
  wobei in ein und dem selben Datensatz mitunter mehrere Sprachen für
  einzelne Felder auftreten.
\item
  Die Koordinateninformationen sind übermäßig genau, wenn man in
  Betracht zieht, dass sie in der Regel nur den Fundplatz und keine
  Befunde auf demselben verorten. Die scheinbare Genauigkeit reicht
  häufig in den Zentimeter-Bereich.
\end{itemize}

\hypertarget{representativity}{%
\subsubsection{Repräsentativität}\label{representativity}}

Die Abbildungen \ref{fig:general-map},
\ref{fig:development-amount-regions-burial-type} und
\ref{fig:development-amount-regions-burial-construction} zeigen, dass
der Datensatz für einzelne Regionen und Zeiträume verhältnismäßig viele,
für die Mehrzahl jedoch sehr wenige Daten enthält. Diese
Ungleichverteilung der Daten hat viele verschiedene Gründe.

\begin{itemize}
\tightlist
\item
  Die Datenaufnahme in Radon-B ist von Schwerpunkten,
  Forschungsinteressen und Projektfinanzierung der beteiligten
  Wissenschaftler abhängig. Daten wurden zeitlich und räumlich
  bedarfsweise aus der Literatur gesammelt.
\item
  Die Verfügbarkeit von \textsuperscript{14}C Daten in der Literatur ist
  wiederrum stark daran gebunden, ob ein modernes Forschungsprojekt mit
  Konzentration auf eine Region und ein Zeitfenster durchgeführt wurde.
  Neben der zufälligen Verteilung der Interessensgebiete der Forschenden
  spielt hier auch die politische Rahmensituation -- etwa die lang
  andauernde Trennung Europas in West und Ost -- eine entscheidende
  Rolle, die die Arbeit in manchen Räumen erleichtert oder kompliziert.
\item
  Auch im Untersuchungsgebiet dieser Arbeit begegnen sich
  unterschiedliche Forschungstraditionen, die \textsuperscript{14}C
  Daten in der Vergangenheit mehr oder weniger essentiell für die
  Konstruktion einer absolutchronologischen Einschätzung gehalten haben.
  \textsuperscript{14}C Daten sind in verschiedenen metallzeitlichen
  Kontexten der Genauigkeit relativchronologischer Einordnungen -- etwa
  auf Grundlage von Fibeltypologie -- unterlegen und werden deswegen nur
  sporadisch zur Schaffung von absolutchronologischen Einhängepunkten
  genutzt. In Feuchtbodenkontexten und bei Verfügbarkeit der
  erforderlichen Hölzer wird Dendrochronologische Datierung bevorzugt
  angewandt. Die verfügbaren \textsuperscript{14}C Kalibrationskurven
  bilden in manchen Zeiträumen Plateaus aus, was die erreichbare
  Datierungsgenauigkeit der Daten signifikant einschränkt. Ist ein
  solcher Effekt bekannt, dann werden erst gar keine Daten in Auftrag
  gegeben.
\item
  \textsuperscript{14}C Datierung erfordert eine zwar kleine, aber
  hinreichend genau kontextualisierbare Menge organischen Fundmaterials.
  Aufgrund von Altholz- und Altwassereffekt werden kurzlebige
  Probenmaterialien wie beispielsweise Knochen von terrestrischen
  Lebewesen oder verkohlte Getreidekörner bevorzugt. In vielen
  Mineralbodenkontexten ist kein ausreichend gutes, datierbares Material
  vorhanden. Das kann ganze Großregionen betreffen, wenn etwa infolge
  kalkarmer Böden generell schlechte Knochenerhaltung vorherrscht.
\item
  Aus sehr wenigen Regionen und Zeiten der europäischen Bronzezeit sind
  so viele Gräber dokumentiert, dass davon ausgegangen werden kann, eine
  signifikante Stuchprobe der Bestattungskultur der Gesamtpopulation
  erforscht zu haben. Stattdessen ist in mehreren Kontexten
  offensichtlich, dass nur die Bestattungen einzelner sozialen Gruppen,
  eines Geschlechts oder einer ethnischen Gruppe erfasst wurden: Die
  Menge und Natur der bekannten Gräber kann schlicht nicht für alle
  Verstorbenen repräsentativ sein. In diesem Fall ist anzunehmen, dass
  weitere, abweichende Bestattungsrituale praktiziert wurden, die keine
  oder archäologisch nur schwer fassbare Überreste hinterlassen haben.
  Diese Rituale sind in den \textsuperscript{14}C Daten aus Radon-B
  nicht abgebildet.
\end{itemize}

Hinsichtlich der Primärvariablen dieser Arbeit \emph{burial\_type} und
\emph{burial\_construction} ergeben sich weitere, besondere
Effekte\footnote{\textcite{harding_european_2000}, 84-85.}.

\begin{itemize}
\tightlist
\item
  Die Störung und Beraubung von Gräbern war ein in Geschichte und
  Vorgeschichte weit verbreitetes Phänomen. Hügelgräber wurden dabei
  aufgrund ihrer guten Sichtbarkeit üblicherweise stärker angegriffen
  als Flachgräber und könnten infolge dessen in Radon-B
  unterrepräsentiert sein.
\item
  Es ist davon auszugehen, dass ein großer Teil der in der Bronzezeit
  errichteten Hügelgräber durch landwirtschaftliche Aktivität --
  langjähriges Überpflügen -- zerstört wurde. Auch das ist Grund für
  eine Unterrepräsentation im gesamten archäologischen Befund.
\end{itemize}

Die Repräsentativität der relativen Entwicklung der Primärvariablen
(siehe Abbildung \ref{fig:development-proportions-regions-burial-type}
und \ref{fig:development-proportions-regions-burial-construction}) kann
zumindest oberflächlich geprüft werden, indem man die oben vorgestellten
Ergebnisse auf Grundlage des Radon-B Datensatzes mit einer Auswertung
der relativen Aussagen aus der Literatur vergleicht, wie sie in Kapitel
\ref{regions-archaeological-overview} zusammengefasst werden. Für
Abbildung \ref{general-map-regions-countries} wurden die Gräber und die
künstlichen Regionen zur besseren Orientierung auf die modernen
Ländergrenzen projiziert, auf die sich Kapitel
\ref{regions-archaeological-overview} bezieht. Zudem wurden die
Literaturangaben in eine stark simplifizierende Modellabbildung
\ref{fig:development-proportions-regions-pseudoquant} verarbeitet, die
den Anteil der Bestattungsformen in den künstlichen Regionen wiedergibt.
Angaben wie ``In der Mittelbronzezeit dominiert in Süddeutschland die
Bestattung in Hügelgräbern'' wurden klassifiziert um pseudoquantitativ
visualierbar zu werden. Von den 5 Klassen wurde die Zuordnung zu 0 --
Merkmal ist nicht vorhanden -- und 4 -- Merkmal ist absolut dominant --
nicht vorgenommen, da selbst bei extrem regelhaften
Bestattungstraditionen in einem archäologischen Kontext stets Ausreißer
auftreten. Außerdem sollte damit auch den berechtigten Unsicherheiten
hinsichtlich der Repräsentativität des archäologischen Befunds Rechnung
getragen werden. Die Angaben in dieser Abbildung sind extrem
approximativ -- es bleibt zu hoffen, dass eine echte, quantitative
Auswertung irgendwann möglich sein wird, wenn mehr Daten in
standardisierter Form vorliegen.

Der Vergleich zwischen dem in dieser Arbeit errechneten Proxie der
Bestattungssittenentwicklung und den jeweils vorherrschenden,
archäologischen Narrativen zeigt insgesamt viele Defizite des
Datensatzes auf. Für Regionen mit relativ großer Datenverfügbarkeit sind
die Nachhersagen besser -- häufig durchaus korrekt -- während die
Angaben für die besonders Informationsarmenregionen in mehreren Fällen
schlicht eine falsche Tendenz anzeigen:

\begin{figure}
\includegraphics{../neomod_analysis/figures_plots/general_maps/general_map_regions_countries} \caption[Karte mit Regionen und modernen Ländergrenzen]{Karte mit Regionen und modernen Ländergrenzen. Wie Abbildung \ref{fig:general-map-research-area-regions}, hier allerdings nicht auf dem Hintergrund der größten europäischen Flüsse sondern vor modernen, administrativen Grenzen.}\label{fig:general-map-regions-countries}
\end{figure}

\begin{figure}[!t]

{\centering \includegraphics[width=.9\textwidth,]{../neomod_analysis/figures_plots/development/development_pseudoquant} 

}

\caption[Pseudoquantitative Auswertung der archäologischen Beobachtungen]{Pseudoquantitative Auswertung der archäologischen Beobachtungen. Visualierung der in Kapitel \ref{regions-archaeological-overview} zusammengestellten, archäologischen Informationen für die Dominanz der wesentlichen Bestattungstraditionen hinsichtlich der Variablen \textit{burial type} und \textit{burial construction}. Effektiv wurden drei Klassen zur subjektiven Klassifizierung der Ergebnisse verwendet: Idee nicht oder schwach vorhanden, Idee wird in ähnlichem Umfang wie ihre Gegenvariante praktiziert, Idee dominiert.}\label{fig:development-proportions-regions-pseudoquant}
\end{figure}

\hypertarget{osterreich-tschechien-slowakei-und-ungarn-southeastern-central-europe-70}{%
\paragraph{\texorpdfstring{Österreich, Tschechien, Slowakei und Ungarn
-- \emph{Southeastern Central Europe}
(70)}{Österreich, Tschechien, Slowakei und Ungarn -- Southeastern Central Europe (70)}}\label{osterreich-tschechien-slowakei-und-ungarn-southeastern-central-europe-70}}

Die künstliche Untersuchungsregion \emph{Southeastern Central Europe}
umfasst den Osten Österreichs (Oberösterreich, Niederösterreich,
Steiermark, Burgenland), den Nordwesten Ungarns, die Westslowakei,
Mähren und große Teile Böhmens. Die Dichte der \textsuperscript{14}C
Daten an Gräbern aus Radon-B ist gering -- aus den verschiedenen
Regionen liegen zwar jeweils einzelne Daten vor, große Areale müssen
jedoch ohne Evidenz auskommen. Diachron ist das Datenaufkommen
durchgehend niedrig und erreicht in der Mittelbronzezeit einen
Tiefpunkt. Während für die \emph{burial type} Variable eine konstanter
Informationsfluss vorhanden ist, nimmt die Anzahl an Gräbern ohne
Angaben zu \emph{burial construction} in der Spätbronzezeit derart zu,
dass überhaupt keine Aussage mehr für sie abgeleitet werden kann. Um die
relative Entwicklung innerhalb der Variaben \emph{burial type} und
\emph{burial construction}, wie sie mithilfe des Radon-B Datensatzes auf
Grundlage von \textsuperscript{14}C Daten nachhergesagt wurde, burteilen
zu können, muss zunächst ein Resumee aus den Angaben in Kapitel
\ref{regions-archaeological-overview} gezogen werden. Das ist dadurch
erschwert, dass die Region vielfältige geographische und kulturelle
Räume schneidet. Brand- und Körperbestattung bestanden im Südosten
Mitteleuropas schon in der Frühbronzezeit nebeneinander und auch über
die Expansion der Grabhügelsitte in der Mittelbronzezeit blieben die
lokalen Tradition in dieser Hinsicht stabil -- mit insgesamt einer
leichten Zunahme der Brandgräber. Die Urnenfelderzeit brachte den
Übergang zur absoluten Dominanz der Brandbestattung. Die Entwicklung der
\emph{burial type} Variable im Radon-B Proxy bildet diese Entwicklung
grundsätzlich korrekt ab. Die Informationen aus \textsuperscript{14}C
Daten zu \emph{burial construction} zeichnen die Relevanz von
Flachgräbern in der Frühbronzezeit korrekt nach, letztere werden dann
aber in der Mittelbronzezeit gegenüber der Anlage von Grabhügeln
überbetont. In der Spätbronzezeit waren Bestattungen auf
Flachgräberfeldern häufig, obgleich auch Hügel der Grabhügelkultur
weiter belegt wurden. Dieser Umstand kann -- sicher aufgrund der
geringen Datenmenge -- nicht korrekt aus Radon-B abgeleitet werden.

\hypertarget{polen-poland-134}{%
\paragraph{\texorpdfstring{Polen -- \emph{Poland}
(134)}{Polen -- Poland (134)}}\label{polen-poland-134}}

Die Region \emph{Poland} deckt Landesmitte und Westen des heutigen
Polens vollständig ab. Die \textsuperscript{14}C Daten aus dieser Region
stammen allerdings fast ausschließlich aus den Woiwodschaften
Kajuwien-Pommern, Lodsch und Niederschlesien. Die Datenmenge ist
insgesamt gering, mit deutlich mehr Daten aus der Frühbronzezeit bis
etwa 1700calBC. Aus Mittel- und Spätbronzezeit sind nahezu keine
Informationen vorhanden. Das gilt sowohl für die \emph{burial\_type} als
auch die \emph{burial\_construction} Variable, wobei erstere immerhin
geringfügig besser dokumentiert ist. Die relative Entwicklung von
\emph{burial type} gibt die \emph{realen} -- also dem klassischen,
archäologischen Narrativ entsprechend -- Verhältnisse näherungsweise
wieder: Körperbestattungen waren bis in die Spätbronzezeit in Westpolen
tatsächlich deutlich häufiger, während Brandbestattungen erst mit der
Urnenfelderzeit dominant auftraten. Auch zuvor waren immer wieder
Brandgräber angelegt worden, doch grundsätzlich ist die über Radon-B
erfasste Tendenz korrekt. \emph{burial construction} wird dagegen
insofern falsch wiedergegeben, als das Hügelgräber auch in der
Mittelbronzezeit eine wichtige Rolle spielten, die der Datensatz nicht
würdigt.

\hypertarget{suddeutschland-southern-germany-213}{%
\paragraph{\texorpdfstring{Süddeutschland -- \emph{Southern Germany}
(213)}{Süddeutschland -- Southern Germany (213)}}\label{suddeutschland-southern-germany-213}}

Die künstliche Kreisregion \emph{Southern Germany} umfasst Bayern
vollständig, schließt aber auch größte Teile Baden-Württembergs und
Thüringens sowie ausgedehnte Flächen in Mähren und Westösterreich ein.
Aus letzteren Regionen liegen keine \textsuperscript{14}C Daten vor. Die
Mehrzahl der vorhandenen Daten stammt aus der Oberpfalz, Niederbayern
und Schwaben sowie (in Württemberg) dem Regierungsbezirken Tübingen. Für
die Frühbronzezeit sind bemerkenswert viele Daten aus Süddeutschland in
Radon-B erhalten, diese Menge nimmt aber im Laufe der Mittelbronzezeit
ab und lässt die Spätbronzezeit beinahe ohne datierte Gräber zurück. Die
\emph{burial type} Variable ist recht gut erschlossen -- obgleich
unverhältnismäßig viel mehr der frühen Körpergräber als
urnenfelderzeitliche Brandbestattungen erfasst wurden. Diese
Informationsfülle steht in krassem Kontrast zur Situation hinsichtlich
der \emph{burial construction} Variable, für die über die gesamte
Bronzezeit nahezu keine Informationen aus diesem Areal bereit stehen.
Betrachtet man die aus Radon-B errechnete, relative Entwicklung und
vergleicht sie mit der archäologischen Literatur, dann zeigt sich für
\emph{burial type} eine gute Übereinstimmung. Körperbestattung
dominierte lange, wurde dann aber ab der Mittelbronzezeit und massiv in
der Spätbronzezeit von der Brandbestattung verdrängt. Die errechneten
Proportionen für \emph{burial construction} könnten leicht als aufgrund
der verschwindend geringen Datenmenge völlig wertlos abgetan werden,
tatsächlich gibt der Proxie jedoch für Früh- und Mittelbronzezeit
immerhin die richtige Tendenz an: Die Dominanz von Flachgräbern wurde
von Hügelbestattungen abgelöst. Zur Spätbronzezeit lässt der Datensatz
keine Aussagen zu.

\hypertarget{nordostfrankreich-northeastern-france-64}{%
\paragraph{\texorpdfstring{Nordostfrankreich -- \emph{Northeastern
France}
(64)}{Nordostfrankreich -- Northeastern France (64)}}\label{nordostfrankreich-northeastern-france-64}}

\emph{Northeastern France} bezeichnet eine Kreisregion, die die
französischen Verwaltungsregionen Grand-Est und Bourgogne-Franche-Comté,
den Osten Baden-Württembergs, Rheinland-Pfalz und das Saarland, sowie
Luxemburg und Wallonien in Belgien umfasst. Nur aus Württemberg und in
Richtung des Pariser Beckens stehen hier Akkumulationen von
\textsuperscript{14}C Daten zur Verfügung. Die Datenmenge ist insgesamt
gering, in der Früh- und beginnenden Mittelbronzezeit jedoch gravierend
niedrig. Für \emph{burial type} sind immerhin nach diesem Zeitfenster
eingeschränkt Aussagen möglich, \emph{burial construction} dagegen muss
obskur bleiben. Die Nachhersage der relativen Entwicklung dieser
Variablen ist entsprechend fehlerbehaftet. Immerhin die Parallelexistenz
von Brand- und Körperbestattung in Früh- und Mittelbronzezeit wird
korrekt wiedergegeben. Die existierte allerdings schon in der
Frühbronzezeit und entwickelte sich sicher nuancierter als der Proxie
vermuten lässt. Entegegen der Ableitung aus den \textsuperscript{14}C
Verhältnissen waren Flachgräber in \emph{Northeastern France} durchaus
verbreitet.

\hypertarget{norddeutschland-northern-germany-475}{%
\paragraph{\texorpdfstring{Norddeutschland -- \emph{Northern Germany}
(475)}{Norddeutschland -- Northern Germany (475)}}\label{norddeutschland-northern-germany-475}}

Zur künstlichen Region \emph{Northern Germany} gehören alle Neuen
Bundesländer in Ostdeutschland, aber auch große Teile Niedersachsens und
Schleswig-Holsteins. Die \textsuperscript{14}C Daten dieser Region
stammen überwiegend aus Sachsen-Anhalt und angrenzenden Regionen in
Thüringen, Sachsen und Niedersachsen. Auch aus dem Raum nördlich von
Bremen sind einge Daten in Radon-B dokumentiert. In der Datenbank sind
aus keiner anderen Region sind so viele datierte Gräber enthalten wie
aus Norddeutschland. Der überwiegende Teil davon fällt in die
Frühbronzezeit, während die Menge in der Mittelbronzezeit ab etwa
1800calBC für einige Jahrhunderte bis 1400calBC auffällig einbricht.
Erst für die Spätbronzezeit ist wieder eine signifikante Anzahl an Daten
belegt. Die Metainformationen zu \emph{burial type} und \emph{burial
construction} zeichnen diese Entwicklung beide in etwa gleichem Umfang
nach. Da in Norddeutschland Einflüsse aus allen vier Himmelsrichtungen
zusammenkommen und maßgeblich zur Ausformung der lokalen Kulturphänomene
beitragen, ist es auch hier schwierig die Gesamtentwicklung hinsichtlich
der Primärvariablen äbzuschätzen. \emph{burial site} wird durch den
\textsuperscript{14}C Proxy insofern richtig wiedergegeben, als dass
Körperbestattungen in der Frühbronzezeit dominierten und besonders im
Norden ab der Mittelbronzezeit von Brandbestattungen abgelöst werden. In
der Spätbronzezeit war Brandbestattung deutlich dominanter, als der
Datensatz glauben lässt. Hinsichlich \emph{burial construction} erfasst
der aktuelle Radon-B Bestand die grundsätzliche Dominanz von
Flachgräbern gut, spricht ihnen jedoch in für Früh- und Mittelbronzezeit
mehr Relevanz zu als in der archäologischen Beurteilung des
Zusammenhangs üblich.

\hypertarget{sudskandinavien-southern-scandinavia-209}{%
\paragraph{\texorpdfstring{Südskandinavien -- \emph{Southern
Scandinavia}
(209)}{Südskandinavien -- Southern Scandinavia (209)}}\label{sudskandinavien-southern-scandinavia-209}}

Die Region \emph{Southern Scandinavia} schließt ganz Dänemark mit
Jütland, Fünen, Seeland und allen kleineren Inseln sowie große Teile des
Osten Südschwedens, und Küstenregionen in Schleswig-Holstein und
Mecklenburg-Vorpommern ein. Die \textsuperscript{14}C Daten sind sehr
gut auf Jütland, Fünen und die schwedische Skåne Region verteilt. Da die
Bronzezeit in Skandinavien erheblich später als in Mitteleuropa beginnt
und an einem anderen, chronologischen System gemessen wird, lassen sich
die diachronenen Entwicklungen nicht einfach mittels des dreigliedrigen
Gerüsts aus Früh-, Mittel- und Spätbronzezeit vergleichen. Die Mehrzahl
der Daten aus Skandinavien stammt aus dem Zeitraum ab Periode IB.
Insofern ist die Datendichte für fast die gesamte Bronzezeit gut -- auch
hinsichtlich der beiden Variablen \emph{burial type} und \emph{burial
construction}. Das Spätneolithikum ab 2200calBC ist dagegen ein
Desiderat -- obgleich auf höherem Niveau als beispielsweise die
Frühbronzezeit in \emph{Northeastern France}. Die Einpassung in
Abbildung \ref{fig:development-proportions-regions-pseudoquant} ist
schwierig, da jener Zeitraum, der aufgrund der Situtation in
Mitteleuropa mit Frühbronzezeit beschriftet ist, eigentlich dem lokalen
Spätneolithikum entspricht. Nimmt man die daraus folgende, gedankliche
Verschiebung vor -- die Zentraleuropäische Mittelbronzezeit enspricht
etwa der Nordischen Frühbronzezeit -- und vergleicht dann den Proxie der
relativen Grabformentwicklung mit den archäologischen Beobachtungen,
dann findet man eine solide Übereinstimmung: Körpergräber waren in
Periode I und II (etwa 1800-1300calBC) die Regel, aber auch
Brandbestattungen traten vereinzelt auf. Ab Periode III gewannen
Brandbestattungen die Oberhand. In der Skandinavischen Frühbronzezeit
(also Mittelbronzezeit in Abbildung
\ref{fig:development-proportions-regions-pseudoquant}) waren
Bestattungen in Grabhügeln die absolute Regel. In der Spätbronzezeit
traten Flachgräberfelder hinzu.

\hypertarget{benelux-benelux-284}{%
\paragraph{\texorpdfstring{Benelux -- \emph{Benelux}
(284)}{Benelux -- Benelux (284)}}\label{benelux-benelux-284}}

Zur \emph{Benelux} Region gehören hier die Niederlande, der Norden
Belgiens und große Teile von Niedersachen und Nordrheinwestfalen. Aus
all diesen Regionen liegen \textsuperscript{14}C Daten an Gräbern in
Radon-B vor, die Datendichte in Belgien und den Niederlanden ist jedoch
deutlich höher. Wie in \emph{Southern Scandinavia} ist auch in der
\emph{Benelux} Region die Frühbronzezeit im Gegensatz zu Mittel- und
Spätbronzeit unterrepräsentiert -- wenn auch in etwas schwächerer Form.
Während die Informationsdichte der \emph{burial type} Variable diesem
Verlauf folgt, ist \emph{burial construction} in der Spätbronzezeit
nahezu undokumentiert. Wie in der \emph{Southern Germany} Region
zeichnet sich auch hier eine gravierende Differenz zwischen der
Informationsdichte beider Variablen ab. Betrachtet man die relative
Entwicklung der Variablen wie sie sich aus den Gräberdaten ableiten
lässt im Vergleich zur archäologischen Lehrmeinung, so ergeben sich hier
im Einzelnen sowohl Übereinstimmung als auch Abweichungen. Die
Untersuchungsregion schließt auch hier unterschiedliche Kulturphänomene
mit teilweise diametral unterschiedlichen Bestattungssitten ein, dennoch
ist eine allgemeine Abschätzung der Verhältnisse möglich. Im Norden und
Osten des Benelux Raums waren bis zur Spätbronzezeit Körperbestattungen
die Regel, während südlich der Maas Brandbestattungen schon früher
dominieren. Der Radon-B Proxie überbetont die Dominanz der
Brandbestattung im Gesamtareal scheinbar. Die Relevanz von
Hügelbestattungen in Früh- und Mittelbronzezeit gibt er dagegen korrekt
wieder, und auch die Mischung zwischen Hügel- und Flachgräbern im
Urnenfelderkontext, die sich häufig aus der Weiternutzung bestehender
Monumentalanlagen ergibt, ist sichtbar.

\hypertarget{england-england-113}{%
\paragraph{\texorpdfstring{England -- \emph{England}
(113)}{England -- England (113)}}\label{england-england-113}}

Die \emph{England} Region erstreckt sich über England bis auf die Höhe
von York und Liverpool, inkorporiert Teile von Wales und lässt dagegen
Cornwall and Devon im Südwesten außen vor. Tatsächlich berührt sie auch
einen schmalen Küstenstreifen am Festland, aus dem aber keine
\textsuperscript{14}C Daten vorliegen. Die vorhandenen Daten
konzentrieren sich im Südosten Britanniens, sind dort jedoch gut
verteilt. Die diachrone Datenverfügbarkeit ist in Früh- und
Mittelbronzezeit relativ gut, lässt dann allerdings in der
Spätbronzezeit nach. \emph{burial type} und \emph{burial construction}
verhalten sich entsprechend. Die aus den \textsuperscript{14}C Daten
abgeleitete relative Entwicklung gibt für \emph{burial type} die
richtige Tendenz an: Brandgräber gewannen im Laufe der Bronzezeit immer
weiter an Bedeutung, während Körperbestattung noch lange parallel lange
praktiziert wurden. Der Datensatz betont die Dominanz der Brandgräber
jedoch zu sehr. Hinsichtlich der Variablen \emph{burial construction}
entspricht die grundsätzliche Dominanz von Hügelgräbern durchaus dem
archäologischen Narrativ, nichtsdestoweniger traten daneben -- besonders
in der Spätbronzezeit -- auch Flachgräberfelder auf, die der Datensatz
zu Unrecht ausschließt.

\hypertarget{datenbedeutung}{%
\subsubsection{Datenbedeutung}\label{datenbedeutung}}

\begin{itemize}
\tightlist
\item
  Unterschied: Selbst Hügel anlegen vs.~Grab in vorhandenen Hügel
  einbringen
\end{itemize}

\hypertarget{simulation}{%
\section{Simulation}\label{simulation}}

\hypertarget{simulation-theorie}{%
\subsection{Theoretische Grundlagen und
Funktionalität}\label{simulation-theorie}}

Wie werden warum welche Gegebenheiten der realen Welt im Modell
abgebildet?

Obgleich das die Situation in manchen kulturhistorischen Zusammenhängen
nur unzureichend abbildet, wird hier der Einfachheit davon ausgegangen,
dass vertikale Beziehungen immer in einer 2:1 Relation von Eltern zu
Kind bestehen. Die Eltern müssen älter sein als das Kind, also einen
früheren Geburtszeitpunkt haben, dürfen aber zu diesem Zeitpunkt auch
nicht zu alt -- also tot -- sein. Tatsächlich müsste für die Zuordnung
von Eltern zu Kindern hier eine Vielzahl von Parametern beachtet werden:
Ein Zeitfenster der gemeinsamen Fruchtbarkeit der Eltern, Partnertreue,
Altersäquivalenz der Partner oder eine theoretische Maximalproduktion
von Kindern pro weiblichem Individuum, um nur einige zu nennen. Diesen
Anforderungen kann man stets nur unvollständig gerecht werden, die
Realwelt lässt sich jedoch mit einer Simulation, die die
Populationserzeugung generativ voranschreiten lässt besser abbilden als
mit dem hier gewählten Ansatz. Nach anfänglichen Versuchen musste darauf
jedoch

Ein Filtern nach diesen Kriterien ist aber bei großen Populationszahlen
ein zeit- und rechenaufwändiger Vorgang und wurde nach anfänglichen
Versuchen ebenso verworfen wie weitere Kriterien wie Letztere hätte eine
Unterscheidung nach dem Geschlecht erfordert.

\hypertarget{implementierung-und-algorithmen}{%
\subsection{Implementierung und
Algorithmen}\label{implementierung-und-algorithmen}}

Die Simulationssoftware besteht aus zwei speziell für diese Anwendung
entwickelten Modulen: Der Populationsgenerator popgenerator und die
Expansionssimulation gluesless. Die Module sind aufeinander abgestimmt
und in den verwendeten Versionen funktional auf die Fragestellungen
dieser Arbeit zugeschnitten. Beide ließen sich jedoch relativ leicht für
einen breiteren Anwendungsbereich öffnen, wenn dafür in Zukunft
Notwendigkeit bestehen sollte. Die im folgenden ausgearbeitete
Beschreibungen beziehen sich entsprechend jeweils speziell auf Version
1.0 der Module, die für die Berechnung in dieser Arbeit zum Einsatz
kamen. Im Gegensatz zu den Ausführungen im Kontext von Datenvorbereitung
und -analyse in Kapitel @ref(\#radonb-dataset), wo darauf zugunsten des
Leseflusses bewusst verzichtet wurde, werden hier nun die wesentlichen
Funktionen und Klassen namentlich genannt. Das soll es erleichtern, die
grundsätzlicher Architektur der Software zu verstehen.

\hypertarget{populationsgenerator}{%
\subsubsection{Populationsgenerator}\label{populationsgenerator}}

Der Populationsgenerator ist als R Paket implementiert. Er ist in drei
Submodule gegliedert: \texttt{population\_generator},
\texttt{relations\_generator} und \texttt{ideas\_generator}, die
nacheinander aufgerufen werden können -- die Interfacefunktion
\texttt{prepare\_pops\_rels\_ideas} automatisiert das. Jedes dieser
Module erweitert einen Eingabedatensatz sukzessive um nondeterministisch
generierte Daten hinsichtlich der zeitübergreifenden Gesamtpopulation,
der Beziehungen innerhalb dieser Population und der Verteilung zweier
Ideen darin zu einem hypothetischen Nullzeitpunkt. Der Eingabedatensatz
-- das \texttt{models\_grid} -- muss in Form eines \texttt{data.frames},
also der in R üblichen Datenstruktur für tabellierte Daten, vorliegen.
Jede Zeile in diesem Datensatz enthält zunächst die Parameter -- später
auch die Ausgabedaten -- für eine Population, ihre Relationen und Ideen.
Komplexe Parameter sind dabei in \texttt{list\ columns} gespeichert.
Diese erlauben es, fast beliebige Datenstrukturen in den Zellen eines
\texttt{data.frames} zu schachteln. Tabelle \ref{tab:param-popgenerator}
beschreibt alle Parameter kurz und umreißt ihren theoretischen
Wertebereich. Die tatsächlich für die Simulation relevanten Werte werden
in Kapitel \ref{simulation-parameters} diskutiert. Die Erzeugung von
Populationen und Relationen ist soweit wie möglich vektorisiert, also
ohne Schleifen oder schleifenersetzende Strukturen (\texttt{apply},
\texttt{purrr::map}) programmiert, um die Berechnungsdauer zu
minimieren. Dieser Schritt ist eine Konzession an die Technik und führte
zu einem höheren Abstraktionsgrad der Populationsnetzwerke als
ursprünglich geplant.

\begin{table*}

\caption{\label{tab:param-popgenerator}Parameter des popgenerator Moduls}
\centering
\fontsize{8}{10}\selectfont
\begin{tabu} to \linewidth {>{\raggedright\arraybackslash}p{14em}>{\raggedright\arraybackslash}p{20em}>{\raggedright\arraybackslash}p{20em}}
\toprule
Paramter & Beschreibung & Theoretischer Wertebereich\\
\midrule
\textit{Allgemeine Parameter}\newline \textbf{timeframe}\newline \textit{integer vector} & Eine Liste der Kalenderjahre über die sich die Simulation erstrecken soll. Das kann eine künstliche Sequenz sein, oder sich an dem Realweltphänomen orientieren, das mit der Simulation erforscht werden soll. & Beliebige, unterbrechungsfreie Sequenz von sukzessive aufeinander folgenden Integerwerten:\newline     \texttt{-2200:-800}, \texttt{0:500}, \texttt{seq(-1000, 1000)}\\
\addlinespace \hline \addlinespace
\textit{Populationsparamter}\newline \textbf{unit\_amount}\newline \textit{integer} & Anzahl der Gruppen, in die die Population untergliedert sein soll. & Beliebiger positiver Integerwert.\newline     \texttt{1}, \texttt{8}, \texttt{100}\\
\addlinespace \hline \addlinespace
\textit{Populationsparamter}\newline \textbf{unit\_names}\newline \textit{list of factors} & Eine Liste der Gruppennamen, wahlweise mit einer Definition ihrer inneren Reihenfolge. & Liste mit Namen. Die Anzahl muss \textit{unit\_amount} entsprechen.\newline     \texttt{list(factor("regionA", levels = regionen), factor("regionB", levels = regionen))}\\
\addlinespace \hline \addlinespace
\textit{Populationsparamter}\newline \textbf{unit\_size\_functions}\newline \textit{list of functions} & Funktionen, die die Populationsgröße für jede Gruppe in Abhängigkeit von der Simulationszeit definieren. & Liste mit Funktionen. Die Anzahl muss \textit{unit\_amount} entsprechen.\newline     \texttt{list('regionA' = function(t) \{100\}, 'regionB' = function(t) \{100 + 10 * cos(t * 0.1)\})}\\
\addlinespace \hline \addlinespace
\textit{Populationsparamter}\newline \textbf{age\_distribution\_function}\newline \textit{function} & Funktion, die die durchschnittliche Altersverteilung der Menschen in der Population beschreibt. & Eine Funktion, die einen Wert für jeden Eingabewert aus \textit{age\_range} zurückgibt.\newline     \texttt{function(x) \{1 / (1 + 0.0004 * 0.7\^\ (-7 * log(x)))\}}\\
\addlinespace \hline \addlinespace
\textit{Populationsparamter}\newline \textbf{age\_range}\newline \textit{integer vector} & Altersfenster, auf das die Altersverteilungsfunktion angewandt wird. & Beliebige, unterbrechungsfreie Sequenz von sukzessive aufeinander folgenden Integerwerten im Bereich der menschlichen Lebenserwartung:\newline     \texttt{1:70}, \texttt{1:120}\\
\addlinespace \hline \addlinespace
\textit{Beziehungsparameter}\newline \textbf{amount\_friends}\newline \textit{integer} & Menge an horizontalen Beziehungen, die ein Individuum aufbaut. & Beliebiger positiver Integerwert oder 0.\newline     \texttt{0}, \texttt{5}, \texttt{100}\\
\addlinespace \hline \addlinespace
\textit{Beziehungsparameter}\newline \textbf{unit\_interaction\_matrix}\newline \textit{integer matrix} & Matrix die definiert, welche Gruppe in welchem Umfang mit welcher anderen Gruppe interagiert. Diese Tabelle kann verschiedene -- räumliche, kulturelle, wirtschaftliche -- Distanzen ausdrücken. & Kreuztabelle in Form einer Integermatrix. Die Werte können beliebige positive Integerwert oder 0 sein.\newline     \texttt{matrix(c(0, 1, 1, 0), nrow = 2, ncol = 2)}\\
\addlinespace \hline \addlinespace
\textit{Beziehungsparameter}\newline \textbf{cross\_unit\_proportion\_child\_of}\newline \textit{double} & Anteil der vertikalen Beziehungen, die nicht innerhalb einer Gruppe bestehen, sondern über Gruppengrenzen hinweg reichen. & Double Wert zwischen 0 und 1.\newline     \texttt{0}, \texttt{0.02}, \texttt{0.7}\\
\addlinespace \hline \addlinespace
\textit{Beziehungsparameter}\newline \textbf{cross\_unit\_proportion\_friend}\newline \textit{double} & Anteil der horizontalen Beziehungen, die über Gruppengrenzen hinweg reichen. & Doublewert zwischen 0 und 1.\newline     \texttt{0}, \texttt{0.02}, \texttt{0.7}\\
\addlinespace \hline \addlinespace
\textit{Beziehungsparameter}\newline \textbf{weight\_child\_of}\newline \textit{integer} & Stärke einer vertikalen Beziehung. Die Beziehungsstärke hat Einfluss darauf, ob eine Idee beim Versuch von einem Individuum zum anderen zu springen Erfolg hat. & Beliebiger positiver Integerwert oder 0.\newline     \texttt{0}, \texttt{5}, \texttt{100}\\
\addlinespace \hline \addlinespace
\textit{Beziehungsparameter}\newline \textbf{weight\_friend}\newline \textit{integer} & Stärke einer horizontalen Beziehung. & Beliebiger positiver Integerwert oder 0.\newline     \texttt{0}, \texttt{5}, \texttt{100}\\
\addlinespace \hline \addlinespace
\textit{Ideenparameter}\newline \textbf{names}\newline \textit{character vector} & Namen der Ideen. & Vektor mit Namen.\newline     \texttt{c('ideaA', 'ideaB')}\\
\addlinespace \hline \addlinespace
\textit{Ideenparameter}\newline \textbf{start\_distribution}\newline \textit{data.frame} & Proportionaler Anteil der Ideen in jeder Region zum Startzeitpunkt der Simulation. Eine Tabelle mit einer Zeile für jede Gruppe und einer Spalte für jede Idee. & \textit{data.frame} mit Proportionen pro Gruppe und Idee. In den Zellen ist der Anteil der Idee in dieser Gruppe als Doublewert zwischen 0 und 1 angegeben. Die Zeilensumme muss 1 sein.\newline     \texttt{data.frame(ideaA = c(0.2, 0.5), ideaB = c(0.8, 0.5))}\\
\addlinespace \hline \addlinespace
\textit{Ideenparameter}\newline \textbf{strength}\newline \textit{double vector} & Stärke der Ideen. & Integervektor mit Werten zwischen ...\newline     \texttt{c(1,1)}\\
\bottomrule
\end{tabu}
\end{table*}

\texttt{population\_generator} dient dazu eine große Menge von
Individuen zu generieren, die gemeinsam eine generationenüberschreitende
Population bilden. Menschen sind nur durch ihre Lebenszeit und ihre
Gruppenzugehörigkeit definiert und damit -- wie in Kapitel
\ref{simulation-theorie} beschrieben -- sehr einfach modelliert. Die
zeitliche Auflösung des popgenerator Moduls ist jahrweise und damit an
die Erfordernisse der Fragestellung dieser Arbei angepasst. Für die
Erzeugung von Populationen, also der Verarbeitung der Daten im
\texttt{models\_grid}, werden zunächst die Populationsparameter aus
jeder Zeile in Instanzen der \texttt{S4}-Konfigurationsklasse
\texttt{population\_settings} überführt. Diese und die folgenden
Transformation in solche Konfigurationsobjekte erleichtern die
Datenweitergabe innerhalb des Pakets. Aus jedem einzelnen
\texttt{population\_settings} Objekt wird eine Population geschaffen. Da
jede Population aus einer oder vielen Gruppen besteht, die biologische
Vererbungsgruppen wie Familien oder Clans repräsentieren und jeweils
eine individuelle Größenentwicklung durchlaufen können, wird für jede
Gruppe in der Population ein Konfigurationsobjekt der Klasse
\texttt{unit\_settings} zusammengestellt. Zur Erzeugung einer Gruppe
wird dieses Objekt an die Funktion \texttt{generate\_unit} übergeben,
die die Hauptlast der Menschengenerierung trägt. In ihr wird zunächst
das Integral unter der \texttt{unit\_size\_function} im um einen
Bufferbereich erweiterten Untersuchungszeitfenster berechnet, um zu
ermitteln, wie viele Menschen-Jahr Kombinationen erforderlich sind, um
die vorgegebene Populationsgrößenentwicklung abzubilden. Aus
\texttt{age\_distribution\_function} und \texttt{age\_range} lässt sich
die durchschnittliche Lebenserwartung der Menschen errechnen, die sich
aus den Eingabeparameters ergibt. Beide Informationen zusammen
ermöglichen es, die Anzahl an Menschen zu bestimmen, die insgesamt
erforderlich ist, um die Größenentwicklung näherungsweise aufzubauen. Um
diese Anzahl an Menschen den Erfordernissen der Populationsentwicklung
entsprechend auf dem Zeitstrahl in mehr oder wenigen dichten Gruppen
anzuordnen, wird eine regelmäßiges Sequenz von Geburtsfenstern
abgegrenzt. Die Länge eines Geburstsfensters entspricht der mittleren
Lebenserwartung. Um die oben errechnete Gesamtzahl der Menschen auf die
Fenster aufzuteilen, wird wiederum das Integral der
\texttt{age\_distribution\_function} in jedem Fenster ermittelt und in
Verhältnis zur Gesamtsumme dieser Integrale gesetzt. Damit steht für
jedes Geburtsfenster ein Faktor bereit, um den Anteil der
Gesamtpopulation zu berechnen, der in diesem Zeitfenster existiert. Mit
dieser Information können die entsprechenden Menschen mit der
\texttt{generate\_humans} Funktion zufällig generiert werden. Das
Beispiele in Abbildung \ref{fig:popgen-sizedev-example} belegt die
Funktionalität dieses Ansatzes, illustriert aber auch den Umfang der
Abweichungen von Ergebnispopulationsgröße und Vorgabefunktion.
\textbf{TODO: Diskussion}. Zum Abschluss der Berechnungen in
\texttt{population\_generator} werden die Gruppenpopulationen zur
Gesamtpopulation zusammengeführt. Diese liegt in der Form eines
\texttt{data.frames} vor, wobei jede Zeile ein Indiviuum repräsentiert.
Jedes Individuum erhält eine eindeutige ID und bringt Informationen zu
seiner Lebensdauer, seinem Geburts- und Sterbezeitpunkt sowie seiner
Gruppenzugehörigkeit mit. Das Sortierkriterium im Gesamtdatensatz ist
das Geburtsjahr. Ein solcher Ergebnisdatensatz wird für jedes Modell,
also jede Zeile, im Eingabedatensatz \texttt{models\_grid} erzeugt und
kann entsprechend in der \texttt{list\ column} \emph{populations} dort
hinzugefügt werden.

\begin{figure}
\includegraphics{../neomod_analysis/figures_plots/popgenerator_examples/create_unit_population_size_development_comparison} \caption[huhu]{huhu}\label{fig:popgen-sizedev-example}
\end{figure}

Das Modul \texttt{relations\_generator} erweitert diesen von
\texttt{population\_generator} modifizierten Eingabedatensatz. Es dient
dazu, die vorhandene Gesamtpopulation sinnvoll inneinander zu verknüpfen
um ein diachrones, soziales Netzwerk zu schaffen. Dafür wird zunächst
für jedes Modell ein Objekt der Konfigurationsklasse
\texttt{relations\_settings} instanziiert, das neben der in
\texttt{relations\_generator} generierten Population auch die
Beziehungsparameter enthält, die vorgeben, welche Eigenschaften das
Netzwerk besitzen soll. Dessen eigentliche Erzeugung ist ein
vierteiliger Prozess: Vertikale und Horizontale Beziehungen, die sich
als Kanten im Netzwerk zwischen den Knoten der Individuen ausdrücken,
werden getrennt voneinander aber jeweils innerhalb der oben erzeugten
Gruppen hergestellt. Anschließend wird ein Teil der vorhandenen
Beziehungen so umgelenkt, dass er die Gruppengrenzen überschreitet und
somit die Gesamtpopulation verschränkt. Abschließend werden die
Beziehungen je nach Typ mit einem Kantengewicht versehen. Die Erzeugung
der vertikalen Beziehungen mit \texttt{generate\_vertical\_relations}
funktioniert gruppenweise, verbindet ein jüngeres Individuum mit -- wenn
im entsprechenden Zeitfenster vorhanden -- zwei älteren und orientiert
sich dabei nicht an deren realem Alter sondern an Indexdistanzen (für
eine Erklärung der Hintergründe dieser Lösung siehe Kapitel
\ref{simulation-theorie}). Eltern werden zufällig aus der Perspektive
der Kinder gewählt, indem zunächst ein Indexbereich -- ein Bereich
zwischen zwei individuellen IDs -- festgelegt wird, aus dem die
potentiellen Eltern stammen können. Zwar kann sicher angenommen werden,
dass ein Kind eine niederere ID besitzen muss als seine Eltern, doch
darüber hinaus ist der Umgang mit den Indizes rein approximativ: Die
durchschnittliche Geburtsjahrdistanz eines Indexschrittes hängt von der
Populationsgröße und -entwicklung ab. Um eine Distanz in Jahren in eine
Distanz in Indexschritten umzuwandeln, muss die zeitlich lokale,
mittlere Indexdistanz ermittelt werden. Das angezielte
Altersdistanzfenster dafür wurde zwischen 40 und 15 Jahren vor dem
Geburtsjahr des Kindes festgelegt. Die für jedes Kind individuelle und
effektiv zufällige Auswahl der Eltern erfolgt also aus einem Pool von
Individuuen, deren Index zwischen jenen liegt, die durchschnittlich die
lokale 40 Jahresgrenze über- oder die durchschnittlich 15 Jahresgrenze
unterschreiten. \texttt{generate\_horizontal\_relations} zur Erzeugung
der horizontalen Beziehungen funktioniert nach dem selben technischen
Prinzip wie \texttt{generate\_vertical\_relations}. Hier werden
allerdings abhängig vom Wert von \texttt{amounts\_friends} mitunter
wesentlich mehr Beziehungen hergestellt und das Bezugsfenster ist mit
einer Altersdistanz von 50 Jahren in beide Richtungen ausgehend vom
Geburtszeitpunkt des jeweiligen Individuums deutlich breiter. Sind die
vertikalen und horizontalen Beziehungen gruppenintern etabliert, dann
werden mit \texttt{modify\_relations\_cross\_unit} einige dieser
Verbindungen zugunsten von gruppenübergreifenden Beziehungen aufgelöst.
Die Eingabeparamter \texttt{cross\_unit\_proportion\_child\_of} und
\texttt{cross\_unit\_proportion\_friend} sind entscheidend dafür, in
welchem Umfang das für die beiden Beziehungstypen passiert. Davon
abhängig werden mehr oder weniger Beziehungen für eine Modifikation
zufällig ausgewählt. Diese besteht darin, dass eines der beiden
Individuen durch ein anderes aus einer anderen Gruppe, aber dem selben
Geburtsfenster ersetzt wird. Welche andere Gruppe gewählt wird, wird
über eine Zufallsentscheidung auf Grundlage der
\texttt{unit\_interaction\_matrix} festgelegt. In einem letzten Schritt
innerhalb des \texttt{relations\_generator} Submoduls werden die
Beziehungen nach den Eingabevariablen \texttt{weight\_child\_of} und
\texttt{weight\_friend} mit einem Gewichtswert versehen. Die vier
Teilschritte dienen gemeinsam dazu einen \texttt{data.frame} zu
schaffen, der sinnvolle Beziehungen zwischen Individuen der
Gesamtpopulation dokumentiert. Die \texttt{data.frames} für jedes Modell
werden in der \texttt{list\ column} \emph{relations} an den
Eingabedatensatz \texttt{models\_grid} angehängt.

Das letzte und einfachste Submodul des popgenerator Pakets,
\texttt{idea\_generator}, \ldots{}

popgenerator stellt einige dedizierte Exportfunktionen bereit, die vor
allem die Übergabe des generierten Populationsnetzwerks an die
Expansionssimulation gluesless ermöglichen sollen. Der Austausch erfolgt
über verschiedene, speziell formatierte Textdateien, die von
popgenerator ins Dateisystem abgelegt und von gluesless gelesen werden.

\hypertarget{expansionssimulation}{%
\subsubsection{Expansionssimulation}\label{expansionssimulation}}

Das C++ Programm gluesless simuliert die Expansion von Ideen in einem
Populationsgraphen wie er mittels des popgenerators erzeugt werden kann.
gluesless macht sich die objektorientierte Natur von C++ zu Abbildung
der Simulationswelt in vier Klassen zunutze: \texttt{Networkland},
\texttt{Aether}, \texttt{Timeline}, \texttt{Idea}. Die Klassenmethoden
greifen außerdem auf mehrere globale Hilfsfunktionen zurück. gluesless
kann drei Eingabeparameter verarbeiten: Der Pfade zu einer pajek-Datei
(.paj), die das Populationsnetzwerk beschreibt, der Pfad zu einer
speziell formatierten Textdatei mit der Ideenverteilung zum
Nullzeitpunkt und der Pfad der Ausgabetextdatei. Mit diesen Parametern
kann es einfach auf der Kommandozeile aufgerufen werden.

Wird das Programm mit den korrekten Eingaben gestartet, dann wird die
\texttt{main} Methode ausgeführt und zunächst jeweils eine Instanz der
Klassen \texttt{Networkland}, \texttt{Aether} und \texttt{Timeline}
angelegt. \texttt{Networkland} repräsentiert die Netzwerkwelt in der die
Ideen leben und interagieren. Ihr Hauptbestandteil ist ein Zeiger auf
ein Objekt der Klasse \texttt{TUndirNet}\footnote{\url{https://snap.stanford.edu/snap/doc/snapuser-ref/d8/da8/classTUndirNet.html}
  {[}13.08.2018{]}} aus der SNAP Bibliothek, das dazu dient, den
Populationsgraphen als sehr einfaches, ungerichtetes Netzwerk zu
speichern und sehr schnell zugänglich zu machen. Die Hauptaufgabe der
Klassenmethoden von \texttt{Networkland} ist es, ein bedarfsgerechtes
Interface zu diesem Netzwerkdatentyp bereit zu stellen. Es kann
verschiedene Fragen beantworten: z.B. ``Existiert ein bestimmter Knoten
im Netzwerk?'', ``Welche Nachbarn hat ein Knoten?'', ``Welchen
Gewichtswert hat eine bestimmte Beziehung?''. Außerdem erlaubt es die
Manipulation des Netzwerks, indem Knoten gelöscht werden können. Der
\texttt{Aether} ist die gedankliche Einheit, die die Netzwerkwelt und
alle Ideen umschließt. Er ist die Kapsel, die den aktuellen Zustand von
Welt und Agenten abbildet. Dafür besitzt er einen Zeiger auf die im
Programmablauf angelegte Instanz der \texttt{Networkland} Klasse und
einen Vektor mit Zeigern auf Instanzen der \texttt{Idea} Klasse. Neben
Funktionen, die Informationen zum aktuellen Zustand von Netzwerk und
Ideen zurückgeben, besitzt der \texttt{Aether} auch die \texttt{develop}
Methode. Sie steuert unter welchen Bedingungen und in welcher
Reihenfolge Ideen am Übergang von einem Zeitschritt zum nächsten agieren
dürfen. Die Basiskonfiguration sieht eine zufällige Abfolge vor. Die
\texttt{Timeline} Klasse umschließt wiederrum gedanklich den
\texttt{Aether} und besitzt dafür einen Zeiger auf die Instanz dieser
Klasse. Sie dient dazu, in jedem Zeitschritt diagnostische Werte zum
Zustand des Aethers abzugreifen und in Vektoren aufgelistet vorzuhalten.
Dazu gehört zum Beispiel die verbleibende Größe des Netzwerks. Auch
\texttt{Timeline} verfügt über eine \texttt{develop} Methode, die
einerseits die \texttt{develop} Methode im Aether anstößt und
andererseits die Messung der diagnostischen Werte auslöst. Es ist diese
\texttt{develop} Methode die im Hauptprogrammablauf in einer
while-Schleife so lange immer wieder aufgerufen wird, bis die
Ideenexpansion endet.

Die Hauptlast der Expansionssimulation tragen Methoden in der
\texttt{Idea} Klasse. Das ist in Programmstruktur- und semantik
sinnvoll, da Ideen als aktiv handelnde Agenten modelliert werden sollen:
Jede Form von Aktivität soll von ihnen ausgehen, während etwa die
Netzwerkwelt, die Menschen und ihre Beziehungen abbildet, nur als
passive Landschaft verstanden wird (siehe Kapitel
\ref{simulation-theorie} für eine Erläuterung dieser Perspektive). Ideen
besitzen einen Namen, einen Zeiger auf die Instanz der
\texttt{Networkland} Klasse in der sie leben und zwei Vektoren, die die
IDs der Netzwerkknoten speichern, auf denen sie aktuell sitzen und auf
denen sie zum Zeitpunkt deren Todes saßen. Um letzteres zu verstehen,
muss man den Algorithmus in der \texttt{expand} Methode betrachten, die
die Ausbreitung der Ideen steuert. Zu Programmbeginn besetzen alle Ideen
die ihnen mittels einer Eingabedatei zugewiesenen Startknoten im
Netzwerk. In jedem Zeitschritt der Simulation darf nun jede Idee einmal
handeln -- \texttt{Aether::develop} legt die Reihenfolge dabei fest. Der
erste Schritt der Idee ist es, eine Liste aller Nachbarknoten zu den von
ihr okkupierten zu erstellen. Die Idee versucht all diese Nachbarknoten
einzunehmen, muss dafür allerdings verschiedene Hindernisse überwinden,
die sich auf die Zufallsentscheidung ob sie Erfolg hat auswirken. Da in
diesem Kontext viele Abfragen von Beziehungs- und Knoteninformationen im
Netzwerk durchgeführt werden müssen, findet die Berechnung dieser
Entscheidungen in einem Subnetzwerk statt, dass sich auf den aktuellen
Dominanzbereich einer Idee und den unmittelbaren Nachbarn beschränkt.
Für jeden Nachbarn werden drei Informationen abgefragt: Die Anzahl an
Verbindungen zu Knoten, die die Idee schon hält, das maximale
Kantengewicht unter all diesen Beziehungen und ob der Knoten selbst
bereits von einer anderen Idee besetzt wird. Grundsätzlich wird für die
Entscheidug ob ein Nachbarknoten der Idee zugesprochen wird gegen das
Kantengewicht gewürfelt. Die Wahrscheinlichkeit wird erhöht, wenn
mehrere Verbindungen zu Knoten der Idee bestehen und deutlich
verringert, wenn der Nachbarknoten bereits Teil des Einflussgebiets
einer anderen Idee ist. Nachdem auf dieser Grundlage für jeden
Nachbarknoten entschieden wurde, ob er zu der aktuell handelnden Idee
gehören wird, verlagert die Idee ihre Existenz auf diese Nachbarknoten.
Ihre bisherigen Knoten sterben, das heißt sie werden aus dem Netzwerk
gelöscht. Nur ihre Bezeichnung wird als Eroberung der Idee gespeichert.
Das Löschen der alten Knoten führt automatisch dazu, dass die Ideen
entlang der impliziten Zeitachse des Populationsnetzwerks voran
schreiten und bildet gleichermaßen semantisch den Forschungskontext ab:
Ideen zur Bestattungsformen drücken sich im Tod ihrer Träger aus. Dieser
Algorithmus wird für jede Ideen in jedem Zeitschritt ausgeführt.

Die Expansionssimulation endet, wenn sich die Größe des Netzwerks von
einem Zeitschritt zum nächsten nicht mehr ändert, also wenn die Ideen
alle Knoten in ihrer Reichweite erobert haben. Naturgemäß führt das
dazu, dass nicht alle Knoten im Netzwerk überhaupt von einer Idee
erobert werden. Je nachdem, wie hoch die Hürden für eine erfolgreiche
Knoteneroberung angesetzt sind, verbleiben mehr oder weniger Knoten
(siehe Kapitel \ref{simulation-theorie}. Die \texttt{Timeline} Klasse
stellt schließlich die Exportfunktion \texttt{export\_to\_text\_file}
bereit, die am Ende der \texttt{main} Methode ausgeführt wird. Diese
transformiert und speichert die diagnostischen Daten und den Endzustand
der Ideen hinsichtlich der von ihnen eingenommenen Netzwerkknoten in
eine menschenlesbare Datei, die später zur Auswertung wieder in R
eingelesen werden kann.

\hypertarget{simulation-parameters}{%
\subsection{Parameter und
Simulationsverhalten}\label{simulation-parameters}}

\hypertarget{kulturelle-und-raumliche-distanz}{%
\section{Kulturelle und Räumliche
Distanz}\label{kulturelle-und-raumliche-distanz}}

\hypertarget{fragestellung-und-methode}{%
\subsection{Fragestellung und Methode}\label{fragestellung-und-methode}}

\[d_{ij}^2 = \sum_{k = 1}^{n} (p_{ik} - p_{jk})^2\]

\begin{itemize}
\tightlist
\item
  \(d_{ij}^2\): Squared Euclidean distance between two groups \(i\) and
  \(j\)
\item
  \(k\): Variant counter
\item
  \(n\): Total amount of variants in a population
\item
  \(p_{ik}\): Relative frequency of the \(k\)'th variant in population
  \(i\)
\item
  \(p_{jk}\): Relative frequency of the \(k\)'th variant in population
  \(j\)
\end{itemize}

SED überbetont Unterschiede

Korrelieren die beiden Kulturdistanten?

Mantel Test: Was das, und warum?

Toblers Gesetz

Im Fall der großen, künstlichen und sich gegenseitig sogar überlappenden
Regionen, die in dieser Arbeit als Untersuchungseinheiten festgelegt
wurden, kann das nur mithilfe von Distanzklassen umgesetzt werden.
Realitätsnähere oder auch nur kontinuierlich skalierte Distanzen
erfordern eine Festlegung von Punkten

\hypertarget{analyse-und-ergebnisse}{%
\subsection{Analyse und Ergebnisse}\label{analyse-und-ergebnisse}}

Die Analyse ist zweigeteilt\ldots{}

\hypertarget{kulturelle-distanz}{%
\subsubsection{Kulturelle Distanz}\label{kulturelle-distanz}}

Um die kulturelle Distanz zwischen den Regionen basierend auf den
Variablen \emph{burial\_type} und \emph{burial construction} zu
berechnen, wurden die in Kapitel \ref{descriptive-data-analysis}
vorgestellten Datensätze zur relativen Entwicklung der Varianten
weiterverwendet. Diese enthalten zu jedem Jahr im
Untersuchungszeitfenster 2200-800calBC Proxyinformationen zum relativen
Anteil der verschiedenen Bestattungsformen in den in Kapitel
\ref{data-prep-and-segmentation} etablierten, künstlichen
Untersuchungsregionen. Auf dieser Grundlage lässt sich jahrweise die SED
zwischen jeder Region berechnen. Die Abbildungen
\ref{fig:sed-dev-matrix-bt} und \ref{fig:sed-dev-matrix-bc} zeigen den
Verlauf der SED-Entwicklung für jede Regionenbeziehung in einer
Plotmatrix.

Die Mehrzahl der Kurven in Abbildung \ref{fig:sed-dev-matrix-bt} für
\emph{burial type} entspricht den Erwartungen: Zu Beginn der Bronzezeit
ist Körperbestattung im gesamten Untersuchungsareal weit verbreitet,
also ist die Euklidische Distanz 2200calBC in fast jeder
Regionenbeziehung gering. Nur der Benelux-Raum, in dem der
Brandbestattungsanteil schon in der Frühbronzezeit groß ist, schert aus.
Ab der Mittel-, besonders aber am Übergang zur Spätbronzezeit, vollzog
sich in ganz Europa der Übergang zur Dominanz der Brandbestattung. Die
SED-Werte steigen über diesen Zeitraum durchweg an und zeichnen
deutliche Peaks oder Entwicklungen mit mehreren, aufeinanderfolgenden
Maxima. Das ist Konsequenz dessen, dass der Übergangsprozess in jeder
Region tatsächlich und so wie er in Radon-B abgebildet ist
unterschiedlich und gegenüber anderen Räumen zeitversetzt ablief. Am
Ende der Spätbronzezeit nähert sich die durchschnittliche SED wieder
Null an, da dann in fast ganz Europa die Brandbestattung vorherrschte.
Abbildung \ref{fig:sed-dev-bt} stellt den selben Sachverhalt in einer an
Abbildung \ref{fig:development-amount-regions-burial-type} orientierten
Form dar. Das erleichtert es, die Rolle einzelner Regionen im
Gesamtkontext zu beurteilen. Die in Radon-B dargestellte, frühe und
vollständige Übernahme des Brandbestattungsritus im Benelux-Raum hebt
ihn deutlich von den anderen Regionen ab. England dagegen, mit seinem
langsamen, graduellen Übergang von Körper- zu Brandbestattung fällt
durch durchgehend mäßige Distanzen zu allen anderen Regionen auf. Sie
kommt ohne deutliche Minima und Maxima aus. Die Situation für
\emph{burial construction} ist, wie aus den Abbildungen
\ref{fig:sed-dev-matrix-bc} und \ref{fig:sed-dev-bc} ersichtlich,
komplizierter. Hier ist kein allgemeiner Trend zu erkennen, sondern eine
Vielzahl an lokalen und zeitlich begrenzten Phänomenen. Das liegt auch
daran, wie der aus Radon-B entwickelte Proxy die Entwicklungen der
realen Welt abbildet (siehe Kapitel \ref{representativity}). Zur
Mittelbronzezeit sollte die Hügelgräberkultur eine grundsätzliche
Reduktion der SED für fast alle Regionenbeziehungen bewirken -- dieser
Effekt ist tatsächlich leicht, aber bei weitem nicht so akzentuiert wie
wünschenwert sichtbar.

\begin{figure}
\includegraphics{../neomod_analysis/figures_plots/sed/regions_regions_squared_euclidian_distance_burial_type} \caption[Plotmatrix der SEDs für \textit{burial type}]{Plotmatrix der SEDs für \textit{burial type}. Matrix mit $8*8=64$ Einzelplots. Jeder Plot zeigt die Entwicklung der Squared Euclidian Distance (SED) für die Variable \textit{burial type} über den Zeitraum von 2200 bis 800calBC für jeweils eine Regionenbeziehung. Über der grauen Linie der echten Datenentwicklung liegt jeweils eine stärkere in der Farbe der Spaltenregion, die den Verlauf eines dynamisch eingepassten Splines (loess, span = 0.3) zeigt.}\label{fig:sed-dev-matrix-bt}
\end{figure}

\begin{figure}
\includegraphics{../neomod_analysis/figures_plots/sed/regions_squared_euclidian_distance_burial_type} \caption[Entwicklung der \textit{burial type} SEDs für jede Region]{Entwicklung der \textit{burial type} SEDs für jede Region. Alternative Darstellung der Plotmatrix in \ref{fig:sed-dev-matrix-bt}.}\label{fig:sed-dev-bt}
\end{figure}

\begin{figure}
\includegraphics{../neomod_analysis/figures_plots/sed/regions_regions_squared_euclidian_distance_burial_construction} \caption[Plotmatrix der SEDs für \textit{burial construction}]{Plotmatrix der SEDs für \textit{burial construction}. Wie Abbildung \ref{fig:sed-dev-matrix-bt}.}\label{fig:sed-dev-matrix-bc}
\end{figure}

\begin{figure}
\includegraphics{../neomod_analysis/figures_plots/sed/regions_squared_euclidian_distance_burial_construction} \caption[Entwicklung der \textit{burial construction} SEDs für jede Region]{Entwicklung der \textit{burial construction} SEDs für jede Region. Alternative Darstellung der Plotmatrix in \ref{fig:sed-dev-matrix-bc}.}\label{fig:sed-dev-bc}
\end{figure}

Das arithmetische Mittel der SEDs zweier Regionen ist ein Maß für deren
diachrone Gesamtähnlichkeit. Die Abbildungen
\ref{fig:sed-mean-matrix-bt} und \ref{fig:sed-mean-matrix-bc}
visualiseren diesen Wert in Form einer Matrix. Für \emph{burial type}
ist eine grundsätzliche Ähnlichkeit der Zentral- und Nordeuropäischen
Regionen (\emph{Southeastern Central Europe}, \emph{Poland},
\emph{Southern Germany}, \emph{Northeastern France}, \emph{Northern
Germany}, \emph{Southern Scandinavia}) zu erkennen, während der
Benelux-Raum und Großbritannien davon distanziert scheinen,
untereinander aber große Ähnlichkeit aufweisen. Die größten
Übereinstimmungen bestehen in Zentraleuropa zwischen \emph{Southern
Scandinavia} und \emph{Southern Germany}, zwischen \emph{Poland} und
\emph{Southern Germany} sowie zwischen \emph{Northeastern France} und
\emph{Northern Germany}. Das spricht für ein hinsichtlich dieser
Variable ausgeprägtes Austauschnetzwerk nördlich der Alpen. Für
\emph{burial construction} gelten dagegen andere Bedingungen. Im durch
diese Variable definierten Interaktionsnetzwerk sind die mittleren
Distanzen in Zentraleuropa viel größer. Stattdessen bilden Nord- und
Nordwesteuropa (\emph{Southern Scandinavia}, \emph{Benelux},
\emph{England}) ein relativ kohärente Einheit. Besonders \emph{Benelux}
und \emph{England} verhalten sich hinsichtlich der Entwicklung von
Flach- und Hügelgrabsitte bemerkenswert ähnlich.

\begin{figure}
\includegraphics{../neomod_analysis/figures_plots/sed/regions_regions_mean_squared_euclidian_distance_burial_type} \caption[Matrix der durchschnittlichen SEDs für \textit{burial type}]{Matrix der durchschnittlichen SEDs für \textit{burial type}. Die Farbintensität bildet die Werteverteilung ab.}\label{fig:sed-mean-matrix-bt}
\end{figure}

\begin{figure}
\includegraphics{../neomod_analysis/figures_plots/sed/regions_regions_mean_squared_euclidian_distance_burial_construction} \caption[Matrix der durchschnittlichen SEDs für \textit{burial construction}]{Matrix der durchschnittlichen SEDs für \textit{burial construction}. Wie Abbildung \ref{fig:sed-dev-matrix-bc}.}\label{fig:sed-mean-matrix-bc}
\end{figure}

Vor dem Hintergrund dieser Beobachtungen ist es besonders interessant,
die kulturellen Netzwerke, die sich aus den Distanzen von \emph{burial
type} und \emph{burial construction} aufspannen, zu vergleichen.
Abbildung \ref{fig:mantel-bt-bc} zeigt den Raum, der sich aus den beiden
SED-Wertereihen ergibt und die Anordung der
\(\frac{8*8}{2}-\frac{8}{2}=28\) Regionenbeziehungen darin. Sie erlaubt
es für jedes 200-Jahre-Zeitfenster abzuschätzen, ob die beiden Variablen
korrelieren. Der Manteltest zeigt an, ob eine solche Korrelation
statistisch signifikant ist. Die Abbildung belegt, dass die
Distanznetzwerke von \emph{burial type} und \emph{burial construction}
nicht äquivalent sind. Auch eine signifikante Negativkorrelation besteht
nicht. Der Grad der Übereinstimmung zweier Regionen in der Frage nach
Körper- und Brandbestattung ist also nicht geeignet, die Ähnlichkeit der
Entwicklungen dieser Regionen hinsichtlich der Dominanz von Flach- oder
Hügelgräbern nachherzusagen. Einzig für das Zeitfenster am Übergang zur
Mittelbronzezeit von 1800 bis 1600calBC wird das 5\%-Signifikanzniveau
des Manteltests beinahe unterschritten und damit eine statistisch
belastbare Beziehung angedeutet.

\begin{landscape}
\begin{figure}
\includegraphics{../neomod_analysis/figures_plots/sed/squared_euclidian_distance_burial_type_vs_burial_construction} \caption[\textit{burial type} SED aufgetragen gegen \textit{burial construction} SED für verschiedene Zeitfenster]{\textit{burial type} SED aufgetragen gegen \textit{burial construction} SED für verschiedene Zeitfenster.}\label{fig:mantel-bt-bc}
\end{figure}
\end{landscape}

\hypertarget{kulturelle-und-raumliche-distanz-1}{%
\subsubsection{Kulturelle und Räumliche
Distanz}\label{kulturelle-und-raumliche-distanz-1}}

Um die kulturelle mit der räumlichen Distanz zu vergleichen und auf
Korrelation zu untersuchen, musste zunächst ein Maß für die räumliche
Distanz festgelegt werden. Dafür wurden Distanzklassen definiert (siehe
Abbildung \ref{fig:map-regions-distance-classes}) indem die
geographischen Distanzen zwischen den Zentrumspunkten der Region
normalisiert und bewusst so aufgeteilt wurden, dass eine sinnvolle,
vierstufige Einteilung entstand.

\begin{figure}
\includegraphics{../neomod_analysis/figures_plots/general_maps/general_map_distance_network} \caption[Karte des klassifizierten, räumlichen Distanznetzwerks]{Karte des klassifizierten, räumlichen Distanznetzwerks. Die Kantenstärke repräsentiert das Kantengewicht: Breite Linien repräsentierten enge Beziehungen mit niedrigem Distanz-Wert. Die Kanten des Netzwerks sind bogenförmig um Überschneidungungen zu vermeiden, die es erschweren würden das Gewicht der einzelnen Verbindungen abzulesen.}\label{fig:map-regions-distance-classes}
\end{figure}

So wie das räumliche Distanznetzwerk aller Regionenbeziehungen visuliert
werden kann, so lassen sich auch die beiden kulturellen Netzwerke
darstellen (siehe Abbildung \ref{fig:map-sed-bt} und
\ref{fig:map-sed-bc}). Im Gegensatz zu den im Untersuchungsnetzwerk
weitestgehend statischen Distanzen im räumlichen Netzwerk verändern sich
die Entfernungen im \emph{burial type} und \emph{burial construction}
Netzwerk im Laufe der Bronzezeit deutlich. Deswegen ist hier erneut ein
Trennung in Zeitfenster sinnvoll.

\begin{landscape}
\begin{figure}
\includegraphics{../neomod_analysis/figures_plots/sed/sed_map_research_area_timeslices_burial_type} \caption[Einzelkarten des \textit{burial type} SED Netzwerks für verschiedene Zeitfenster]{Einzelkarten des \textit{burial type} SED Netzwerks für verschiedene Zeitfenster.}\label{fig:map-sed-bt}
\end{figure}
\end{landscape}

\begin{landscape}
\begin{figure}
\includegraphics{../neomod_analysis/figures_plots/sed/sed_map_research_area_timeslices_burial_construction} \caption[Einzelkarten des \textit{burial construction} SED Netzwerks für verschiedene Zeitfenster]{Einzelkarten des \textit{burial construction} SED Netzwerks für verschiedene Zeitfenster.}\label{fig:map-sed-bc}
\end{figure}
\end{landscape}

Um die Korrelation der Kulturnetzwerke mit dem räumlichen
Distanznetzwerk zu prüfen, eignet sich -- ähnlich wie in Abbildung
\ref{fig:mantel-bt-bc} -- die Projektion der 28 Regionenbeziehungen in
den SED-Raumdistanz Raum (siehe Abbildung \ref{fig:mantel-bt-spatial}
und \ref{fig:mantel-bc-spatial}). Die ordinale Skalierung der
Raumdistanzklassen erschwert das einerseits, erlaubt andererseits aber
auch das Anlegen diagnostischer Boxplots zur Veranschaulichung der
Werteverteilung. Sie hat auch zufolge, dass für den Manteltest auf
Korrelation nicht wie oben Pearsons Korrelationskoeffizient, sondern der
Spearmansche Rangkorrelationskoeffizient berechnet werden muss. Für die
Beziehung zwischen dem SED Netzwerk basierend auf \emph{burial type} und
der räumlichen Distanz ergeben sich aus Abbildung
\ref{fig:mantel-bt-spatial} mehrere Beobachtungen: Erstaunlicherweise
scheint die kulturelle Distanz grundsätzlich nicht mit der räumlichen zu
korrelieren. Ein signifikanter Zusammenhang zwischen den Distanzmaßen
offenbart sich nur für das frühbronzezeitliche Zeitfenster 2000 bis
1800calBC. Das bedeutet, dass direkt aneinander angrenzende Großregionen
nur in der Frühbronzezeit, also lange vor dem universellen Umstieg zur
urnenfelderzeitlichen Brandbestattung, tendenziell eher gleiche
Entwicklungen durchlaufen als weiter voneinander entfernte. Das ist ein
Indiz dafür, dass in der Frühbronzezeit mehr unmittelbarer gedanklicher
Austausch zwischen Nachbarregionen stattgefunden hat als in den
anschließenden Perioden, in denen die hier untersuchten Großräume
stärker voneinander isoliert waren. In der Mittelbronzezeit durchlaufen
die SED-Entwicklungen der \emph{burial type} Variable die schon in
Abbildung \ref{sed-dev-matrix-bt} beobachteten Spitzen. Die Korrelation
zwischen kultureller und räumlicher Distanz ist über diese Peaks hinweg
gering und kippt zwischen 1400 und 1200calBC, also am Übergang zur
Spätbronzezeit, sogar ins Negative -- wenn auch nicht signifikant. Die
kulturelle Distanz zwischen mehreren geographisch weit voneinander
entfernten Regionen ist dann geringer als zwischen mehreren direkt
aneinander angrenzenden. Betrachtet man das SED Netzwerk auf Grundlage
von \emph{burial construction} und untersucht es mittels Abbildung
\ref{fig:mantel-bc-spatial} auf räumliche Korrelation, kommt man zu
grundsätzlich ähnlichen Schlüssen wie im \emph{burial type} Kontext. Die
Distanzwerte sind hier insgesamt größer und die SED Unterschiede
innerhalb der einzelnen Raumdistanzklassen deutlich akzentuierter,
dennoch ist das Zeitfenster der mit Abstand größten, räumlichen
Korrelation auch hier die Frühbronzezeit. Signifikant ist diese nur für
das Zeitfenster zwischen 1800 und 1600calBC. Negativkorrelation deutet
sich für \emph{burial construction} nicht an, doch auch hier nähert sich
der Korrelationskoeffizient in der Spätbronzezeit Null an.

\begin{landscape}
\begin{figure}
\includegraphics{../neomod_analysis/figures_plots/sed/squared_euclidian_distance_vs_spatial_distance_burial_type} \caption[\textit{burial type} SED aufgetragen gegen die räumlichen Distanzklassen für verschiedene Zeitfenster]{\textit{burial type} SED aufgetragen gegen die räumlichen Distanzklassen für verschiedene Zeitfenster.}\label{fig:mantel-bt-spatial}
\end{figure}
\end{landscape}

\begin{landscape}
\begin{figure}
\includegraphics{../neomod_analysis/figures_plots/sed/squared_euclidian_distance_vs_spatial_distance_burial_construction} \caption[\textit{burial type} SED aufgetragen gegen die räumlichen Distanzklassen für verschiedene Zeitfenster]{\textit{burial type} SED aufgetragen gegen die räumlichen Distanzklassen für verschiedene Zeitfenster.}\label{fig:mantel-bc-spatial}
\end{figure}
\end{landscape}

\hypertarget{simulation-1}{%
\subsection{Simulation}\label{simulation-1}}

\hypertarget{kausale-interaktionsbeziehungen}{%
\section{Kausale
Interaktionsbeziehungen}\label{kausale-interaktionsbeziehungen}}

\hypertarget{fragestellung-und-methode-1}{%
\subsection{Fragestellung und
Methode}\label{fragestellung-und-methode-1}}

\hypertarget{analyse-und-ergebnisse-1}{%
\subsection{Analyse und Ergebnisse}\label{analyse-und-ergebnisse-1}}

\hypertarget{simulation-2}{%
\subsection{Simulation}\label{simulation-2}}

\newpage
\pby[title={Literatur},segment=\therefsegment,heading=subbibintoc]

\hypertarget{zusammenfassung-und-abschlieende-gedanken}{%
\chapter{Zusammenfassung und Abschließende
Gedanken}\label{zusammenfassung-und-abschlieende-gedanken}}

\hypertarget{abbildungen}{%
\chapter{Abbildungen}\label{abbildungen}}

\printbibliography


\end{document}
